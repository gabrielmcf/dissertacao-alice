%% abtex2-modelo-trabalho-academico.tex, v-1.9 laurocesar
%% Copyright 2012-2013 by abnTeX2 group at http://abntex2.googlecode.com/ 
%%
%% This work may be distributed and/or modified under the
%% conditions of the LaTeX Project Public License, either version 1.3
%% of this license or (at your option) any later version.
%% The latest version of this license is in
%%   http://www.latex-project.org/lppl.txt
%% and version 1.3 or later is part of all distributions of LaTeX
%% version 2005/12/01 or later.
%%
%% This work has the LPPL maintenance status `maintained'.
%% 
%% The Current Maintainer of this work is the abnTeX2 team, led
%% by Lauro César Araujo. Further information are available on 
%% http://abntex2.googlecode.com/
%%
%% This work consists of the files abntex2-modelo-trabalho-academico.tex,
%% abntex2-modelo-include-comandos and abntex2-modelo-references.bib
%%

% ------------------------------------------------------------------------
% ------------------------------------------------------------------------
% abnTeX2: Modelo de Trabalho Academico (tese de doutorado, dissertacao de
% mestrado e trabalhos monograficos em geral) em conformidade com 
% ABNT NBR 14724:2011: Informacao e documentacao - Trabalhos academicos -
% Apresentacao
% ------------------------------------------------------------------------
% ------------------------------------------------------------------------

\documentclass[
	% -- opções da classe memoir --
	12pt,				% tamanho da fonte
	openright,			% capítulos começam em pág ímpar (insere página vazia caso preciso)
	oneside,			% para impressão em verso e anverso. Oposto a oneside
	a4paper,			% tamanho do papel. 
	% -- opções da classe abntex2 --
	%chapter=TITLE,		% títulos de capítulos convertidos em letras maiúsculas
	%section=TITLE,		% títulos de seções convertidos em letras maiúsculas
	%subsection=TITLE,	% títulos de subseções convertidos em letras maiúsculas
	%subsubsection=TITLE,% títulos de subsubseções convertidos em letras maiúsculas
	% -- opções do pacote babel --
	english,			% idioma adicional para hifenização
	french,				% idioma adicional para hifenização
	spanish,			% idioma adicional para hifenização
	brazil				% o último idioma é o principal do documento
	]{abntex2}

% ---
% PACOTES
% ---

% ---
% Pacotes fundamentais 
% ---
\usepackage{lmodern}			% Usa a fonte Latin Modern			
\usepackage[T1]{fontenc}		% Selecao de codigos de fonte.
\usepackage[utf8]{inputenc}		% Codificacao do documento (conversão automática dos acentos)
\usepackage{lastpage}			% Usado pela Ficha catalográfica
\usepackage{indentfirst}		% Indenta o primeiro parágrafo de cada seção.
\usepackage{color}				% Controle das cores
\usepackage{graphicx}			% Inclusão de gráficos
\usepackage{microtype} 			% para melhorias de justificação
% ---
		
% ---
% Pacotes adicionais, usados apenas no âmbito do Modelo Canônico do abnteX2
% ---
\usepackage{lipsum}				% para geração de dummy text
% ---

% ---
% Pacotes de citações
% ---
\usepackage[brazilian,hyperpageref]{backref}	 % Paginas com as citações na bibl
\usepackage[alf]{abntex2cite}	% Citações padrão ABNT

% --- 
% CONFIGURAÇÕES DE PACOTES
% --- 

% ---
% Configurações do pacote backref
% Usado sem a opção hyperpageref de backref
\renewcommand{\backrefpagesname}{Citado na(s) página(s):~}
% Texto padrão antes do número das páginas
\renewcommand{\backref}{}
% Define os textos da citação
\renewcommand*{\backrefalt}[4]{
	\ifcase #1 %
		Nenhuma citação no texto.%
	\or
		Citado na página #2.%
	\else
		Citado #1 vezes nas páginas #2.%
	\fi}%
% ---

% Conteudo padrao da Folha de Rosto
\makeatletter
\renewcommand{\folhaderostocontent}{
  \begin{center}

    %\vspace*{1cm}
    {\ABNTEXchapterfont\large\imprimirautor}
	
    \vspace*{\fill}\vspace*{\fill}
    \begin{center}
      \ABNTEXchapterfont\bfseries\Large\imprimirtitulo
    \end{center}
    \vspace*{\fill}
	
    \abntex@ifnotempty{\imprimirpreambulo}{%
      \hspace{.45\textwidth}
      \begin{minipage}{.5\textwidth}
      	\SingleSpacing
         \imprimirpreambulo
       \end{minipage}%
       \vspace*{\fill}
    }%

	{Linha de pesquisa: Filosofia Moderna}


    {\large\imprimirorientadorRotulo~\imprimirorientador\par}
    \abntex@ifnotempty{\imprimircoorientador}{%
       {\large\imprimircoorientadorRotulo~\imprimircoorientador}%
    }%
    
	%{\abntex@ifnotempty{\imprimirinstituicao}{\imprimirinstituicao\vspace*{\fill}}}    
    {Departamento de Filosofia da Faculdade de Filosofia e Ciências Humanas da UFMG}
    \vspace*{\fill}

    {\large\imprimirlocal}
    \par
    {\large\imprimirdata}
    \vspace*{1cm}

  \end{center}
}
\makeatother


\newcommand{\conferir}{\textbf{(conferir)}}

\newcommand{\Hum}{\textit{Humano}\xspace}


% ---
% Informações de dados para CAPA e FOLHA DE ROSTO
% ---
\titulo{Tentativas e tentações naturalistas na filosofia de Nietzsche}
\autor{Alice Medrado}
\local{Belo Horizonte}
\data{FAFICH/UFMG, 2014}
\orientador{Prof. Dr. Rogério Antônio Lopes}
\instituicao{%
  Universidade Federal de Minas Gerais -- UFMG
  \par
  Faculdade de Filosofia e Ciências Humanas -- FAFICH
  \par
  Programa de Pós-Graduação}
\tipotrabalho{Dissertação (Mestrado)}
% O preambulo deve conter o tipo do trabalho, o objetivo, 
% o nome da instituição e a área de concentração 
\preambulo{Dissertação apresentada ao Programa de Pós-graduação em Filosofia da Universidade Federal de Minas Gerais, como parte dos requisitos para obtenção do título de Mestre em Filosofia.	}
% ---


% ---
% Configurações de aparência do PDF final

% alterando o aspecto da cor azul
\definecolor{blue}{RGB}{41,5,195}

% informações do PDF
\makeatletter
\hypersetup{
     	%pagebackref=true,
		pdftitle={\@title}, 
		pdfauthor={\@author},
    	pdfsubject={\imprimirpreambulo},
	    pdfcreator={Gabriel Fonseca},
		%pdfkeywords={abnt}{latex}{abntex}{abntex2}{trabalho acadêmico}, 
		colorlinks=true,       		% false: boxed links; true: colored links
    	linkcolor=black,          	% color of internal links
    	citecolor=black,        		% color of links to bibliography
    	filecolor=black,      		% color of file links
		urlcolor=black,
		bookmarksdepth=4
}
\makeatother
% --- 

% --- 
% Espaçamentos entre linhas e parágrafos 
% --- 

% O tamanho do parágrafo é dado por:
\setlength{\parindent}{1.3cm}

% Controle do espaçamento entre um parágrafo e outro:
\setlength{\parskip}{0.2cm}  % tente também \onelineskip

% ---
% compila o indice
% ---
\makeindex
% ---

% ----
% Início do documento
% ----
\begin{document}

% Retira espaço extra obsoleto entre as frases.
\frenchspacing 

% ----------------------------------------------------------
% ELEMENTOS PRÉ-TEXTUAIS
% ----------------------------------------------------------
% \pretextual

% ---
% Capa
% ---
\imprimircapa
% ---

% ---
% Folha de rosto
% (o * indica que haverá a ficha bibliográfica)
% ---
\imprimirfolhaderosto
% ---

% ---
% Inserir a ficha bibliografica
% ---

% Isto é um exemplo de Ficha Catalográfica, ou ``Dados internacionais de
% catalogação-na-publicação''. Você pode utilizar este modelo como referência. 
% Porém, provavelmente a biblioteca da sua universidade lhe fornecerá um PDF
% com a ficha catalográfica definitiva após a defesa do trabalho. Quando estiver
% com o documento, salve-o como PDF no diretório do seu projeto e substitua todo
% o conteúdo de implementação deste arquivo pelo comando abaixo:
%
% \begin{fichacatalografica}
%     \includepdf{fig_ficha_catalografica.pdf}
% \end{fichacatalografica}
%\begin{fichacatalografica}
%	\vspace*{\fill}					% Posição vertical
%	\hrule							% Linha horizontal
%	\begin{center}					% Minipage Centralizado
%	\begin{minipage}[c]{12.5cm}		% Largura
%	
%	\imprimirautor
%	
%	\hspace{0.5cm} \imprimirtitulo  / \imprimirautor. --
%	\imprimirlocal, \imprimirdata-
%	
%	\hspace{0.5cm} \pageref{LastPage} p. : il. (algumas color.) ; 30 cm.\\
%	
%	\hspace{0.5cm} \imprimirorientadorRotulo~\imprimirorientador\\
%	
%	\hspace{0.5cm}
%	\parbox[t]{\textwidth}{\imprimirtipotrabalho~--~\imprimirinstituicao,
%	\imprimirdata.}\\
%	
%	\hspace{0.5cm}
%		1. Palavra-chave1.
%		2. Palavra-chave2.
%		I. Orientador.
%		II. Universidade xxx.
%		III. Faculdade de xxx.
%		IV. Título\\ 			
%	
%	\hspace{8.75cm} CDU 02:141:005.7\\
%	
%	\end{minipage}
%	\end{center}
%	\hrule
%\end{fichacatalografica}

% ---
% Inserir folha de aprovação
% ---

% Isto é um exemplo de Folha de aprovação, elemento obrigatório da NBR
% 14724/2011 (seção 4.2.1.3). Você pode utilizar este modelo até a aprovação
% do trabalho. Após isso, substitua todo o conteúdo deste arquivo por uma
% imagem da página assinada pela banca com o comando abaixo:
%
% \includepdf{folhadeaprovacao_final.pdf}
%
\begin{folhadeaprovacao}

  \begin{center}
    {\ABNTEXchapterfont\large\imprimirautor}

    \vspace*{\fill}\vspace*{\fill}
    \begin{center}
      \ABNTEXchapterfont\bfseries\Large\imprimirtitulo
    \end{center}
    \vspace*{\fill}
    
    \hspace{.45\textwidth}
    \begin{minipage}{.5\textwidth}
        \imprimirpreambulo
    \end{minipage}%
    \vspace*{\fill}
   \end{center}
        
   %Trabalho aprovado. \imprimirlocal, janeiro de 2014:

   \assinatura{\textbf{\imprimirorientador} \\ FAFICH/UFMG - Orientador} 
   \assinatura{\textbf{Profª. Drª. Telma de Souza Birchal} \\ FAFICH/UFMG}
   \assinatura{\textbf{Prof. Dr. Olímpio José Pimenta Neto} \\ IFAC/UFOP}
   %\assinatura{\textbf{Professor} \\ Convidado 3}
   %\assinatura{\textbf{Professor} \\ Convidado 4}
      
   \begin{center}
    \vspace*{0.5cm}
    {\large\imprimirlocal}
    \par
    {\large\imprimirdata}
    \vspace*{1cm}
  \end{center}
  
\end{folhadeaprovacao}
% ---

% ---
% Dedicatória
% ---
%\begin{dedicatoria}
%   \vspace*{\fill}
%   \centering
%   \noindent
%   \textit{ Aqui vai a dedicatória pra vida, pros deuses e pros amores.} \vspace*{\fill}
%\end{dedicatoria}
% ---

% ---
% Agradecimentos
% ---
\begin{agradecimentos}
Sou grata, primeiramente, ao professor Rogério Lopes pela orientação absolutamente generosa e engrandecedora, pela interlocução arejada e instigante, pelo prazer de sua convivência. A convivência intelectual com o Rogério e a referência de sua pesquisa delinearam os traços fundamentais deste trabalho, e significaram sempre uma oportunidade de cultivo e crescimento. Agradeço ao professor Olímpio Pimenta, que trouxe contribuições valiosas para este trabalho. Nossas conversas sempre me deixaram com a impressão de ter perdido algum pré-conceito e ganhado alguma poesia; o Olímpio é para mim um exemplo de vida filosófica vigorosa. Agradeço à professora Telma Birchal, cuja participação em minha formação me é de grande estima, cujo trabalho sempre admirei e que me fez muito feliz ao aceitar o convite para participar da banca examinadora. Agradeço às professoras Iracema Macedo e Míriam Campolina, que me iniciaram na pesquisa filosófica e que sempre serão para mim modelos de vigor e rigor. Agradeço ao professor Marco Brusotti pela oportunidade de discutir alguns pontos deste trabalho em sua forma ainda embrionária, para grande benefício meu; à sua cordial disposição, minha gratidão. Agradeço de coração a todo o Grupo Nietzsche da UFMG, que me propiciou o mais salubre dos ambientes de pesquisa, convivência intelectual e vital (Ana Marta, Anderson, Daniel, Oscar, Sílvia, Vítor, Wander e William, um abraço especial para vocês!).  Agradeço o apoio da CAPES/CNPq. 

	Gratidão inestimável a meus pais, Ellen e Wallen, que sempre estimularam em mim a paixão pelo conhecimento, e me ofereceram os meios afetivos, simbólicos e materiais para persegui-la. À Rachel, prova viva de que a vida é tão generosa que me deu mais uma mãe. A toda a minha família prodigiosamente multiplicada pela vida – vó Fija, vó Irenith, vó Cecy, Cissa, Lu, Fabim, João, Teca e Dri, vocês são o melhor de tudo. À Martha Ulhôa, pelas afinidades e por ter contribuído com a revisão de alguns pontos deste trabalho. Ao meu amigo José Paulo Neto, companheiro de estudos, grande interlocutor, ouvinte benévolo nos momentos de entusiasmo e de escassez de entusiasmo. Valeu, Zé! Aos amigos Juliana Jardim e Cleuber Amaro, pequenos mananciais ambulantes de força e bom humor. Ao amigo Mário Geraldo, manancial poético. A Júlio Ruas, pelos anos de companheirismo intelectual e vital. Ao amigo Jonas Guerzoni, por manter sempre teso o arco da conversa. Aos meus queridos Nathália Valentini e Michel Menezes, que no momento mais difícil me ofereceram ajuda e muito mais que um porto seguro. Muito amor. Ao muito querido Alexandre Zuba, que me presenteou com algumas artes inspiradas neste trabalho e com muitas outras artes. Ao querido amigo Gabriel Fonseca, que cuidou da diagramação do texto. Agradeço especialmente a Afrânio Biscardi, Leo Assunção e Pedro Corgozinho, que iniciaram comigo a caminhada pela filosofia, e me deram o sentido exato de por que filosofia começa com \textit{filia}, afinidade, e nunca prescinde de um bom agonismo.

\end{agradecimentos}
% ---

% ---
% Epígrafe
% ---
\begin{epigrafe}
    \vspace*{\fill}
    \hfill
    \begin{minipage}{.7\textwidth}
	\begin{flushright}
\textit{“Conhece-te a ti mesmo” é toda ciência.} -- Apenas no final do conhecimento de todas as coisas o homem terá conhecido a si mesmo. Pois as coisas são apenas as fronteiras do homem. 

NIETZSCHE, Aurora, §48.
	\end{flushright}
    \end{minipage}
\end{epigrafe}
% ---

% ---
% RESUMOS
% ---

% resumo em português
\setlength{\absparsep}{18pt} % ajusta o espaçamento dos parágrafos do resumo
\begin{resumo}
A presente dissertação percorre algumas questões levantadas no debate sobre possíveis interpretações “naturalistas” da obra de Nietzsche. Esse debate tem revisitado alguns aspectos da filosofia nietzschiana que dizem respeito principalmente à valorização de uma disposição de espírito cientificamente cultivada e ao uso de alguns resultados científicos advindos das ciências naturais e humanas na abordagem de problemas tradicionais da filosofia, bem como às ocasionais intervenções do filósofo no debate científico de sua época. Tomando tais aspectos como fio condutor, perseguimos a questão sobre os usos do pensamento científico na tarefa de crítica cultural e construção de uma filosofia imoralista. Dentre os escritos nietzschianos, selecionamos uma amostra constituída por \textit{Humano, demasiado humano} e \textit{Além de Bem e Mal}, dois livros especialmente programáticos em que esses pontos “naturalistas” servem como recurso a diferentes experimentos filosóficos, e que representam dois modelos possíveis de vida filosófica na modernidade.

 \textbf{Palavras-chaves}: Palavras chave.
\end{resumo}

% resumo em inglês
\begin{resumo}[Abstract]
 \begin{otherlanguage*}{english}
Master thesis approaching a few questions brought up in the debate on possible “naturalist” interpretations of Nietzsche's \textit{oeuvre}. Revisited aspects of nietzschean philosophy are the positive evaluation of a scientifically cultivated disposition of mind, and the resource to results delivered by the natural and human sciences as a means to approach traditional philosophic questions, as well as Nietzsche's occasional interventions in the scientific debate of his time. Taking such aspects as a guiding line, we pursue the question of the role of scientific thought in the task of cultural criticism and construction of an immoralist philosophy. From Nietzsche's \textit{oeuvre}, we selected the texts \textit{Human, all too human} and \textit{Beyond Good and Evil}, which represent two different models of philosophic life possible in modernity, two particularly programmatic books in which such “naturalist” topics are used in different philosophic experiments.

   \vspace{\onelineskip}
 
   \noindent 
   \textbf{Key-words}: key words
 \end{otherlanguage*}
\end{resumo}



% ---
% inserir lista de abreviaturas e siglas
% ---
\begin{siglas}
\item[GT/NT] O Nascimento da Tragédia
\item[PHG/FT] A filosofia na Época Trágica dos Gregos
\item[SE/SE] Terceira Consideração Intempestiva: Schopenhauer como Educador
\item[MA/HH] Humano, demasiado humano
\item[MA/HH II – VM/OS] Humano, demasiado humano (vol. 2): Opiniões e Sentenças Diversas
\item[MA/HH II – WS/AS] Humano, demasiado humano (vol. 2): O Andarilho e Sua Sombra
\item[M/A] Aurora
\item[FW/GC] A Gaia Ciência
\item[Z/Z] Assim falou Zaratustra
\item[JGB/ABM] Além de Bem e Mal
\item[GM/GM] Genealogia da Moral
\item[EH/EH] Ecce Homo
\item[WM/VP] A Vontade de Poder

\end{siglas}
% ---


% ---
% inserir o sumario
% ---
\pdfbookmark[0]{\contentsname}{toc}
\tableofcontents*
\cleardoublepage
% ---



% ----------------------------------------------------------
% ELEMENTOS TEXTUAIS
% ----------------------------------------------------------
\textual

% ----------------------------------------------------------
% Introdução
% ----------------------------------------------------------
\chapter*{Introdução}
\addcontentsline{toc}{chapter}{Introdução} 

O grande fascínio de se pesquisar a obra nietzschiana se deve a que nada ali responde por uma mera curiosidade: cada tema é tomado do modo mais interessado e interessante, o entrelaçamento entre os temas faz com que a obra alcance um máximo de organicidade. Este fato tão fascinante coloca também as maiores dificuldades para a pesquisa, que necessariamente implica um recorte e uma simplificação. Apesar desses riscos, acreditamos que pode-se afirmar com segurança que a obra nietzschiana se destina a uma busca nunca arrefecida pelas condições optimais em que a cultura pode atingir o máximo vigor, e é pautada pela intuição de que essas condições remetem à superação do ponto de vista moral. Nossa pesquisa percorre uma das linhas com que Nietzsche teceu sua filosofia tão múltipla e tão vibrante: o ingrediente científico, de certa forma negligenciado num primeiro momento da recepção nietzschiana, cuja retomada mais recentemente tem impulsionado novos debates e interpretações da obra. Não se encontra neste trabalho, no entanto, uma contribuição original ao debate sobre a relação de Nietzsche com suas fontes científicas, debate que se beneficiou muito das contribuições de grandes pesquisadores como Jörg Salaquarda, George Stack, John Richardson, Gregory Moore, Wilson Frezzatti Jr. e Rogério Lopes. O que se encontra aqui é uma pesquisa sobre os usos nietzschianos do pensamento científico no interior de dois programas de vida filosófica: \textit{Humano, demasiado humano}, que edita uma proposta de estreita associação entre ciência e filosofia, anunciada como um programa de “filosofia histórica” – e \textit{Além de Bem e Mal}, que reedita e confere novas tensões ao programa de \textit{Humano}. Dedicamos os capítulos \ref{cap1} e \ref{cap3} à análise de cada um desses livros, perseguindo a questão de como a visada e os resultados científicos se ligam às linhas gerais dos programas filosóficos que eles anunciam. No capítulo \ref{cap2} tratamos das principais inovações, experimentos e tensionamentos que vêm à tona entre um livro e outro. O que trazemos aqui, portanto, são duas “fotografias” de diferentes momentos da obra, momentos de fundação de um plano filosófico. Desenvolvemos uma pequena narrativa que viabiliza a comparação entre esses dois momentos. Fazemos referência ao debate sobre as fontes na medida do imprescindível, em momentos em que perder de vista as interlocuções de Nietzsche comprometeria o acesso às primeiras camadas do texto, ao que mais visivelmente é “fotografado”. Acreditamos que o que é “fotografado” aqui são duas tentativas, duas respostas nietzschianas à questão de como a vida filosófica é possível numa era de conquistas científicas, no limiar da passagem para uma cultura pós-metafísica e extramoral. Buscamos visualizar os experimentos que responderiam a duas grandes questões: o que o filósofo vê pelas lentes da ciência, como a ciência estrutura um método e uma disciplina do pensamento, como a filosofia nietzschiana incorpora uma visada científica? E como o filósofo tomou auxílio na ciência para realização da grande tarefa que se colocou – a tarefa de superação da visão moral de homem e mundo?

\chapter{A virada científica de \textit{Humano, demasiado humano}}
\label{cap1}


Nossa investigação sobre o lugar da ciência na filosofia de Nietzsche começa por \textit{Humano, demasiado humano}, um livro consensualmente tido por ponto de inflexão na obra do filósofo. As primeiras seções do livro deixam claro que este se destina a lançar um novo programa filosófico, a que Nietzsche chama \textit{filosofia histórica}, programa que teria suas principais linhas de força no exercício de uma estreita aproximação entre as ciências e a filosofia; na rejeição da metafísica como forma de abordagem dos problemas filosóficos e como fonte de criação da cultura; na suspeita sistemática frente a toda forma de especulação; na aposta na ciência como referência e meio de cultivo de uma reforma dos afetos, pautada pela moderação. Para melhor compreender a novidade dos rumos em que Nietzsche lança sua filosofia com essa obra, acreditamos que seria necessário fazer uma breve leitura em contraste de suas obras de juventude, a partir da qual tem-se a impressão imediata de que os pontos nevrálgicos de \Hum, como apontamos, operam uma inversão exata do tratamento dado às principais questões com que o filósofo se ocupou anteriormente. Quer dizer, por um lado é possível notar a continuidade de algumas questões às quais Nietzsche, desde o início, dedicou um lugar central – pode-se dizer que essas questões, em linhas gerais, têm por objeto os processos de vida e morte das culturas, bem como os modos de vida que tais culturas são capazes de engendrar, sendo que ambos cultura e modos de vida particulares dependeriam de certos arranjos entre investimentos de caráter cognitivo, afetivo, artístico, por vezes religioso, etc. No entanto, numa leitura comparativa entre a obra de juventude e o trajeto inaugurado por \Hum, fica patente o esforço nietzschiano no sentido de perseguir esse objeto por meios radicalmente diferentes, o que significa conferir uma nova tensão às questões, restabelecer estratégias retóricas e investigativas, arriscar novas respostas.

	É notável, acima de tudo, que Nietzsche tenha de fato se dedicado a um programa de \textit{filosofar histórico} alguns anos depois de expressar seu receio de que o desenvolvimento sem precedentes do sentido histórico na modernidade pudesse secar as fontes de criatividade da cultura (preocupação que aparece reiteradamente nas “extemporâneas” – \textit{Da Utilidade e Desvantagem da História para a Vida e Schopenhauer como Educador}, em especial).

	Um leitor que começasse seu percurso pela obra de Nietzsche a partir de \Hum, estando por isto familiarizado com a crítica contundente que o filósofo lança ao pensamento metafísico já nas primeiras seções desse livro, ficaria no mínimo confuso ao ver que em sua obra de juventude Nietzsche associa a atividade filosófica enquanto tal à atividade especulativa da metafísica. Em vários escritos anteriores a \Hum o lugar que Nietzsche confere à filosofia seria o da criação de ficções totalizantes, destinadas a cumprir o papel de edificar e estimular o engajamento na vida cultural, uma vez que o conteúdo dessas ficções seria capaz de criar uma justificação geral da existência, reconciliando o homem com seus aspectos terríveis (cf. \textit{O Nascimento da Tragédia}) – com isto, a filosofia é pensada sempre na vizinhança da metafísica. Nesse momento, Nietzsche acreditaria que as ficções ou poesias conceituais de caráter totalizante, confeccionadas pela filosofia especulativa, cumpririam ainda o papel de lidar com os impulsos cognitivos que não encontram satisfação no âmbito meramente científico, de forma que a filosofia poderia com isto sanar a vontade de conhecimento a todo custo promovida pela ciência
	\footnote{Ideia herdada do programa filosófico de F.A. Lange, em \textit{A História do Materialismo}, e de certa forma realizada pela  metafísica da vontade de Schopenhauer. (Cf. LOPES, 2008), e expressa em \textit{A filosofia da época trágica dos gregos}: “a ciência debruça-se sobre tudo que é passível de ser conhecido, pretendendo, com cega avidez, conhecer tudo a qualquer custo; o pensar filosófico, ao contrário, põe-se sempre a caminho das coisas que são mais dignas de serem conhecids, dos grandes e relevantes conhecimentos. (…) 'Isto é grande', diz ela, e, com isso, eleva o homem sobre a cega e incontida ânsia de seu impulso ao conhecimento. Por meio do conceito de grandeza, ela amansa esse impulso: sobretudo por considerar alcançável e alcançado o maior de todos os conhecimentos, a saber, aquele pertinente à essência e ao núcleo das coisas.” (PHG/FT, §4).}.
	 Ao que tudo indica, nesse primeiro momento de sua atividade como pensador, Nietzsche entende que a filosofia ocupa o lugar de uma criadora de hipóteses, metáforas e imagens em geral, capazes de responder não só a um impulso estético, mas também a uma curiosidade pelo caráter último da existência – âmbito de investigação banido definitivamente do campo da ciência moderna após o trabalho crítico de Kant. Na medida em que oferecem um tipo de satisfação a esses impulsos – com formulações do tipo “tudo é água” ou “o mundo é vontade” – tais produtos filosóficos têm um sentido que é, novamente, edificante, pois liberam o homem da pressão desses interesses cognitivos, permitindo a ele voltar seus esforços à atividade cultural e às necessidades práticas, em geral.

	Seria preciso um contato mais aprofundado com o contexto de composição das primeiras obras de Nietzsche para se entender seu receio em relação à satisfação ilimitada dos impulsos cognitivos. Neste sentido, a pesquisa nietzschiana tem se beneficiado dos crescentes trabalhos apontando a influência decisiva do contato do filósofo com a obra de Albert Friedrich Lange em seus anos de formação (década de 1860, cf. LOPES, 2008). Através da \textit{História do Materialismo} de Lange, Nietzsche teria acessado a ideia de que os impulsos cognitivos correspondem a uma parcela menor do conjunto de disposições psicofísicas atuantes no homem e na cultura, e por consequência, a satisfação desses impulsos cumpriria um papel menor no engajamento com a vida cultural. Em sua avaliação do materialismo, Lange credita a essa tradição o estabelecimento de uma disciplina do pensamento à qual se deve a conquista de um método rigoroso capaz de pautar o desenvolvimento da ciência; no entanto, Lange não é um entusiasta da visão de mundo materialista, pois duvida que ela possa ter apelo o bastante para manter o homem engajado nas funções da vida comum. É verdade que Nietzsche esboçou uma tentativa de desafiá-lo neste ponto, quando se dedicou a um projeto de reconstrução da personalidade de Demócrito; dessa maneira, Nietzsche ambicionava reconstituir uma visão de mundo materialista estimulante o bastante para rivalizar com as concorrentes idealistas. O projeto no entanto foi abandonado, e o jovem filósofo passou a se dedicar à composição de \textit{O Nascimento da Tragédia}, obra que aposta não no atomismo materialista, mas num arranjo entre a filosofia schopenhaueriana, o drama musical de Wagner e as ficções conceituais de Dioniso e Apolo – arranjo com que ambicionava engatilhar um renascimento cultural na Alemanha. Encontra-se no livro a ideia de que as diferentes culturas seriam resultado de diferentes combinações entre os impulsos estéticos apolíneo e dionisíaco e um terceiro impulso, o impulso lógico-teórico; a partir daí, a modernidade é criticada por expressar um desenvolvimento excessivo do impulso lógico-teórico, uma “hipertrofia do lógico”. Nesta obra e nos demais escritos de juventude, fica claro que Nietzsche compartilha com Lange um certo receio de que certo modo de pensamento cético, conquanto tenha sido uma grande conquista para a atividade científica, poderia ter um efeito paralisante sobre a capacidade de ação, em geral
\footnote{Na verdade, este parece ser um receio de que Nietzsche nunca se livra completamente. Em diferentes momentos de sua obra, aparecerá a preocupação em construir algo como um “ceticismo da força”, isto é, um ceticismo que não tenha um efeito enfraquecedor sobre a vontade, os instintos e todos os outros motores da ação (que, em grande medida, são fatores não cognitivos).}. 
É justamente essa “paralisia” que Nietzsche diagnostica no modo de se fazer História em seu tempo, por exemplo; embora o desenvolvimento do sentido histórico seja visto por Nietzsche como uma das maiores conquistas de seu século, ele suspeita do historicismo científico que, enquanto prática “desinteressada”, resultaria apenas num colecionismo inócuo de “fatos” históricos, sem apelo para a transformação da cultura
\footnote{Cf. PAOLIELLO, 2009. Tomando como referência a obra \textit{Da Utilidade e Desvantagem da História pra vida}, Paoliello aponta que: “O historicismo seria então para Nietzsche uma força nociva na medida em que tudo o que ele ensinava era a resignação idólatra diante de supostos fatos. (…) O resultado, definiu Nietzsche, era uma história que não gerava  mais cultura, mas apenas ensinava o respeito à cultura já produzida, que graças ao historiador cientista, em seu valor, era convertida ela mesma em um fato incontestável e insuperável.”  P. 15}.

	Segundo Lopes, o encaminhamento que Nietzsche dá à sua obra de juventude seria sintomático de um comprometimento com a chamada “tese da inevitabilidade antropológica da metafísica”, ideia que teria suas raízes em três grandes interlocutores de nosso filósofo: Kant, Lange e Schopenhauer – pensadores que lidam de forma ligeiramente diferente com a concepção de que o homem tem algumas necessidades éticas, espirituais e culturais que não podem ser satisfeitas no rigor da atividade científica, devendo ser buscadas nas construções metafísicas. O diálogo de Nietzsche com seus grandes interlocutores filosóficos, a este respeito, está minuciosamente analisado e documentado na obra de Lopes, que não reproduziremos neste trabalho. No momento, interessa-nos apenas manifestar nossa filiação à tese de Lopes, por entendermos que a visão de juventude de Nietzsche sobre o papel das criações metafísicas na cultura é o principal fio que se rompe com a nova direção dada à sua filosofia a partir de \Hum, e também que, a partir de então, Nietzsche se ocupará em grande medida com o projeto de criação de uma cultura pós-metafísica.

	Apesar das marcantes dissonâncias entre \textit{O Nascimento da Tragédia} e \Hum, acreditamos encontrar uma constante dentre as preocupações nietzschianas. Como sugerimos acima, acreditamos que, no que diz respeito a seu campo de investigação, a obra de Nietzsche, como um todo, pode ser lida como um estudo dos processos de vida e morte das culturas – neste ponto estamos de acordo com estudiosos como Giacoia Jr., que aponta o trabalho de crítica radical às “formas superiores” da cultura ocidental ao longo da obra de Nietzsche (Cf. GIACOIA JR., 1997). Se esta atividade de crítica cultural delimita o campo de estudos do filósofo, do ponto de vista normativo Nietzsche se mostra um ativista incansável no combate à visão moral de mundo, por suspeitar desde sempre que o avanço da visão moral sobre as várias áreas da vida humana tende a trazer como efeito algum grau de embotamento intelectual, afetivo/pulsional, cultural. Isto é expresso, em termos nietzschianos, como a suspeita de que “precisamente a moral seria culpada de que jamais se alcançasse o \textit{supremo brilho e potência} do tipo homem” (GM/GM, Pr. 6).

	Supomos, portanto, que a obra de Nietzsche é perpassada por um fio condutor que a inscreve no campo da filosofia da cultura, e que esse percurso nietzschiano é investido da tônica de um imoralista
\footnote{Nietzsche atribui a si mesmo o termo “imoralista”, principalmente na obra madura. O que queremos apontar, adiante, é que essa intenção “imoralista” já se faz notar nas obras anteriores. Para adiantar um pouco do que Nietzsche entende por imoralismo, e como o justifica, tomemos um pequeno trecho de \textit{Além de Bem e Mal}: “Que me perdoem a descoberta de que até agora todas as filosofias morais foram enfadonhas e soporíferas – e de que nada prejudicou mais 'a virtude', a meus olhos, do que o \textit{enfado} de seus advogados (...)” (JGB/ABM, §228). Se a moral é criticada, aqui, por prejudicar “a virtude”, deixa-se entrever um ponto de vista normativo a partir do qual o filósofo realiza sua crítica: enquanto imoralista, Nietzsche não deixa em momento algum de analisar, avaliar, e \textit{prescrever} alguns valores, virtudes e “deveres” (em sentido não categórico). Ainda, em \textit{Aurora}: “Não nego, como é evidente – a menos que eu seja um tolo –, que muitas ações consideradas imorais devem ser evitadas e combatidas; do mesmo modo, que muitas consideradas morais devem ser praticadas e promovidas – mas acho que, num caso e no outro, \textit{por razões outras que as de até agora}.” (M/A, §103). Por isto, ao longo deste trabalho falaremos de “visão não moral da normatividade”, um termo não utilizado por Nietzsche, mas abrangente o bastante para compreender as diversas ocorrências em que ele critica a “visão moral de mundo”, “visão moral da vida”, e ainda a moralização dos valores estéticos, do comportamento, etc. Aqui, não tomamos “normatividade” em sentido estrito, porque mesmo em seus momentos mais prescritivos Nietzsche está longe de afirmar uma \textit{norma} ou \textit{regra} universal; se o adotamos é porque nos parece que ele pode falar (sem o peso de “moral”) de questões referentes aos valores, que a filosofia nietzschiana sempre tem em vista. Temos em mente algo como a distinção contemporânea entre moral e ética; aplicar esta distinção à interpretação da filosofia nietzschiana é algo sugerido, por exemplo, por Maudemarie Clark, que entende a moral como “uma interpretação particular da vida ética”, apontando a possibilidade de “uma versão não-moral da vida ética” (Clark, 2000, p. 105. Tradução livre.). Em \textit{Genealogia da Moral}, Nietzsche situa o aparecimento dessa interpretação moral da vida num segundo estágio do processo civilizatório: no primeiro estágio, chamado moralidade de costumes (\textit{Sittlichkeit der Sitte}), o foco estaria em memorizar as regras que delimitam o que é socialmente aceito ou proibido, o que era feito por recurso a terríveis castigos físicos; o surgimento da moral (\textit{Moral}) marca o estágio em que esse tipo de regulação social passa a ser feito por recurso a dispositivos psicológicos, como o sentimento de culpa, dispositivos com os quais desenvolve-se uma versão internalizada da regra social. Ver também: Leiter, 1995. Voltaremos a este ponto em maior detalhe mais adiante. }. 
É fundamentalmente enquanto imoralista que Nietzsche tenta compreender e intervir na cultura, objetivo que persegue através de diferentes experimentos. Com isto, nos contrapomos a certas leituras, que marcaram em especial o primeiro momento de recepção da obra de Nietzsche, segundo as quais seus escritos devem ser lidos como poesia sofística em frequente contradição, ou como um apanhado desordenado de aforismos que podem ser lidos aleatoriamente sem prejuízo de – ou talvez, sem preocupação com – o sentido de cada livro e da obra como um todo. Esse tipo de leitura foi defendida, por exemplo, por Arthur Danto: 

\begin{citacao}
Os livros de Nietzsche dão a aparência de ter sido reunidos ao invés de compostos. Eles são feitos, em sua maioria, de aforismos contundentes, e de ensaios que raramente ultrapassam umas poucas páginas... E qualquer um dos aforismos ou ensaios poderia ser colocado tanto em um volume quanto em outro sem afetar muito a unidade ou estrutura. E os livros eles mesmos… não exibem qualquer estrutura especial enquanto corpus. Nenhum deles pressupõe uma familiaridade com o outro... Seus escritos podem ser lidos praticamente em qualquer ordem, sem qualquer impedimento significativo à compreensão de suas ideias. (DANTO, 1965, p. 19).
 \end{citacao} 

	O procedimento que adotaremos neste trabalho se guia por prerrogativas exatamente inversas àquelas de Danto, e segue as pistas dadas pelo próprio Nietzsche ao longo de sua obra, especialmente nas obras finais, em que se mostra mais autoconsciente de sua forma particular de expressão filosófica. Ao nos referirmos às “pistas” de leitura deixadas por Nietzsche temos em mente, por exemplo, a advertência do filósofo de que uma boa leitura de \textit{Genealogia da Moral} depende de familiaridade com sua obra anterior: “Se este livro resultar incompreensível para alguém, ou dissonante aos seus ouvidos, a culpa, quero crer, não será necessariamente minha. Ele é bastante claro, supondo-se – e eu suponho – que se tenha lido minhas obras anteriores, com alguma aplicação na leitura: elas realmente não são fáceis.” (GM, p.8) Conquanto possamos desconfiar das tentativas \textit{a posteriori} de Nietzsche no sentido de conferir unidade à sua obra, nossa estratégia, consistirá em considerar o contexto do surgimento de cada obra e, no interior de cada uma delas, estarmos atentos à ordem de apresentação, argumentos, formas de expressão e estratégias retóricas empregadas em cada aforismo, tentando captar também as ocasionais referências a obras anteriores; partimos do pressuposto de que, dentre o propósito maior de combater a visão moral, cada livro formula um \textit{alvo} específico, e uma estratégia de se atingir esse alvo.

	Se o sentido da pesquisa nietzschiana é chegar a formas (culturais) menos moralizantes de compreender, conhecer, sentir e avaliar, o meio como esse fim é perseguido ao longo de sua profícua obra é a da \textit{experimentação}
\footnote{O termo alemão \textit{Versurch} –  com sentido de experimento, ensaio, tentativa – tem muitas e importantes ocorrências na obra nietzschiana. Lembrando que o termo remete tanto aos procedimentos científicos quanto à forma experimental como Montaigne e Emmerson, dois grandes interlocutores de Nietzsche, desenvolveram suas filosofias; \textit{Versurch} é a tradução alemã de \textit{Essay}. O termo ganha destaque num aforismo de \textit{Além de Bem e Mal}, em que é usado para caracterizar os filósofos cujos surgimento Nietzsche projeta no futuro: “esses filósofos do futuro bem poderiam, ou mesmo mal poderiam, ser chamados de \textit{tentadores}. Esta denominação mesma é, afinal, apenas uma tentativa e, se quiserem, uma tentação.” (JGB/ABM, §42). O próprio Nietzsche certamente se inclui como um antecipador desse tipo de filósofo. Cf. LAMPERT, 2001, p. 76.}. 
Cada livro pode ser lido como um todo orgânico, com um certo recorte e ordenação de temas, abordagens, argumentos e estratégias retóricas, que expressam uma \textit{forma} possível, um \textit{experimento} de como se chegar a uma normatividade não-moral. Se cada livro realiza esse experimento de forma relativamente autônoma, completa e compreensível, a obra como um todo expressa um percurso que a cada etapa pressupõe e recupera os avanços já obtidos nas tentativas anteriores; portanto, talvez Nietzsche tenha razão em sugerir que sua própria obra fosse vista como uma \textit{escada} (Cf. MA/HH, Pr., §7). Em alguns casos, se há dissonância é porque se perdeu de vista o contraponto de alguma obra anterior: no desenvolvimento da obra, o filósofo dialoga, expande e oferece alternativas aos caminhos que já haviam sido abertos, retoma intuições que haviam sido deixadas de lado, apresenta autocrítica, e, em vários momentos, utiliza uma retórica que pressupõe a familiaridade do leitor com o estilo e o conteúdo do que já havia sido trabalhado anteriormente. Isto não quer dizer que não haja tensões, até mesmo insolúveis, entre os diferentes caminhos teóricos abertos por Nietzsche em seus experimentos, mas ao que nos parece, a filosofia nietzschiana foi lida exageradamente como um apanhado de contradições.

	Segundo nossa leitura, \textit{O Nascimento da Tragédia} serve, igualmente, ao propósito de combate à visão moral, com a particularidade de que esse livro expressa uma forma e um caminho que não se repetem no restante da obra do filósofo. \textit{NT} se caracteriza por buscar nos gregos, especialmente na poesia trágica e na filosofia pré-socrática, um \textit{exemplo} de que uma vida menos moral já foi possível, de forma que eles ainda teriam a oferecer os \textit{elementos} que permitiriam avançar, na modernidade, o projeto de (re)construção de uma cultura não moralizante. Na tentativa de autocrítica em \textit{Ecce Homo}, quinze anos após o lançamento de seu primeiro livro, Nietzsche lamenta que este tenha se tornado conhecido justamente pelo que nele é “erro”: “por sua aplicação ao \textit{wagnerismo}, como se este fosse um sintoma de ascensão.” (EH/EH, iv, §2). \Hum pode ser lido como uma tentativa de corrigir esse erro, o erro de ter confiado no wagnerismo como uma via pra se chegar aos fins imoralistas desde sempre pretendidos pelo filósofo. Decepcionado com o festival de inauguração do teatro de Bayreuth, Nietzsche parece se dar conta de que, ao contrário do que esperava e acima de tudo, a obra de Wagner teria o efeito de dar novo fôlego à visão moral de mundo cristã, e de servir a um modo de produção e consumo artísticos “filisteus”
\footnote{Sob a expressão “filisteus da cultura” ou “cultura filisteia”, Nietzsche critica a acepção burguesa de cultura enquanto consumo passivo de bens artísticos, religiosos e científicos, aplicados à função de mero “embelezamento” e “entretenimento”. Para Nietzsche, esta seria uma expressão apequenada da cultura, entendida como modo de produção da vida, modo de sentir, fazer e expressar que é pervasivo e capaz de conferir sentido e vigor até os mínimos detalhes da vida cotidiana. Nietzsche apresenta uma crítica ampla à cultura filisteia já em \textit{Schopenhauer como Educador}, portanto cerca de 2 anos antes da inauguração de Bayreuth; apontamos aqui alguns pontos dessa crítica que, a nosso ver, se repetem na obra de Nietzsche, mesmo depois de ele se distanciar desse ambiente de justificação e retórica schopenhauerianas: “For there exists a species of misemployed and appropriated culture you have only to look around you! (…). Among these forces is, first of all, the greed of the money-makers, which requires the assistance of culture and by way of thanks assists culture in return, but at the same time, of course, would like to dictate its standards and objectives. (…). The goal would then be to create as many current human beings as possible, in the sense in which one speaks of a coin as being current; and, according to this conception, the more of these current human beings it possesses the happier a nation will be. (…) A man is allowed only as much culture as it is in the interest of general money-making and world commerce he should possess, but this amount is likewise demanded of him. (…) Secondly, there is the greed of the state, which likewise desires the greatest possible dissemination and universalization of culture and has in its hands the most effective instruments for satisfying this desire. Whenever one now speaks of the 'cultural state', one sees it as facing the task of releasing the spiritual energies of a generation to the extent that will serve the interests of existing institutions: but only to this extent; as a forest river is partially diverted with dams and breakwaters so as to operate a mill with the diminished driving-power thus produced – while the river's full driving-power would rather endanger the mill than operate it (…) Thirdly, culture is promoted by all those who are conscious of possessing an ugly or boring content and want to conceal the fact with so-called '\textit{beautiful form}'. Under the presupposition that what is inside is usually judged by what is outside, the observer is to be constrained to a false assessment of the content through externalities, through words, gestures, decoration, display, ceremoniousness. It sometimes seems to me that modern men bore one another to a boundless extent and that they finally feel the need to make themselves interesting with the aid of all the arts.” (SE/SE, vii, p. 164-166).}, 
que em nada expressam o ideal de vida cultural que Nietzsche vinha desenvolvendo.

	Curiosamente, o nome de Wagner não é citado, sequer uma vez, no primeiro volume de \textit{HH}. Talvez isto se deva a que, finalmente livre da ascendência do grande músico, Nietzsche tenha tido condições de desenvolver uma intuição que expressou de forma ainda tímida em \textit{NT}: de que, por vezes, os artistas estão a serviço dos filósofos, e são, por assim dizer, seus porta-vozes. Se em \textit{NT} essa ideia se expressa na crítica à dupla Sócrates-Eurípedes, por sua vez \textit{HH} toma como alvo a associação Schopenhauer-Wagner. Portanto, mais do que Wagner, Schopenhauer parece ser o alvo de \textit{HH}. No que diz respeito ao ambiente schopenhaueriano anterior, presente em NT, Nietzsche reconhecerá posteriormente: “Talvez me censurem muita 'arte' neste ponto, muita sutil falsificação de moeda: que eu, por exemplo, de maneira consciente-caprichosa fechei os olhos à cega vontade de moral de Schopenhauer, num tempo em que já era clarividente o bastante acerca da moral”
\footnote{Esta autocrítica de Nietzsche aparece, significativamente, no prefácio acrescentado a \textit{HH} em 1886, e é, portanto, um exemplo da preocupação do filósofo em fazer compreender a razão das diferenças entre este e seu primeiro livro. Que a palavra “arte” apareça aqui entre aspas é indicativo do espírito irônico da passagem, sugerindo a associação entre “arte” e “autoengano”, “simplificação”, etc., tema que Nietzsche perseguia quando escreveu esse prefácio. Cf. JGB/ABM §192, §291.}.
A se fiar no próprio Nietzsche, o fato de que Schopenhauer tenha deixado de ser um aliado e passado a ser o alvo da crítica de \Hum não se deve a qualquer “revelação” posterior, sobre, por exemplo, o aspecto moral da ética de negação da vontade schopenhaueriana, ou sobre as muitas inconsistências de sua metafísica da vontade, que têm sido apontadas por vários leitores ao longo das décadas, já na época em que Nietzsche era ainda um estudante de filologia. Ao que tudo indica, a ruptura com Schopenhauer se deve a um esgotamento do uso sempre instrumental a que Nietzsche submeteu a filosofia de seu mestre. Para dar a medida do quanto a relação de Nietzsche com Schopenhauer se deu de forma instrumental, basta lembrar que na obra dedicada a este filósofo, \textit{Schopenhauer como Educador}, nenhuma doutrina fundamental da filosofia schopenhaueriana é analisada, se chega a ser mencionada; nas cerca de 70 páginas desse ensaio, Schopenhauer é textualmente citado apenas uma ou duas vezes, dentre elas, significativamente, numa passagem sobre a vida heroica, a única alternativa, para o filósofo, de uma vida ética que não se atém à prática ascética, uma alternativa que estava longe de constituir o centro da ética schopenhaueriana
\footnote{“(...) but he may console himself with the words of his great teacher, Schopenhauer: 'A happy life is impossible: the highest that man can attain to is a heroic one. He leads it who, in whatever shape or form, struggles against great difficulties for something that is to the benefit of all and in the end is victorius, but who is ill-rewarded for it or not rewarded at all. Then, when he has done, he is turned to stone, like the prince in Gozzi's \textit{Recorvo}, but stands in a noble posture and with generous gestures. He is remembered and is celebrated as a hero; his will, mortified a whole life long by effort and labour, ill success and the world's ingratitude, is extinguished in Nirvana.'”. (SE/SE, iv, p. 152.)  Nietzsche cita um trecho de \textit{Parerga e Paraliponem}}. 
Nesse escrito, Nietzsche se utiliza da filosofia schopenhaueriana principalmente como referência para a composição de um certo ambiente retórico, razão por que o próprio Nietzsche não se detém na apresentação das doutrinas schopenhauerianas, mas insiste na importância da “primeira impressão, por assim dizer uma impressão fisiológica” produzida pelo contato com a filosofia de Schopenhauer
\footnote{“I am describing nothing but the first, as it were physiological, impression Schopenhauer produced upon me, that magical outpouring of the inner strength of one natural creature on to another that follows the first and most fleeting encounter; and when I subsequently analyse that impression I discover it to be compounded of three elements, the elements of his honesty, his cheerfulness and his steadfastness.” (SE/SE, ii, p. 136.)}.

	O estrondoso sucesso do festival de Bayreuth e a gélida recepção de \textit{NT} podem ter sugerido a Nietzsche que a filosofia de Schopenhauer, enquanto instrumento, foi mais efetiva num campo muito diferente daquele que nosso filósofo tinha em vista. A guinada anti-schopenhaueriana de \textit{HH} pode ser entendida, então, como uma saída frente à frustração dos efeitos que Nietzsche esperava obter com \textit{NT}: o festival wagneriano, afinal, não expressou o ideal de cultura nele elaborado e, exceto pela polêmica com o filólogo Wilamowitz, a recepção acadêmica do livro foi marcada por um silêncio sepulcral. 

	 \Hum expressa, portanto, uma busca por novos aliados, que Nietzsche encontra, por exemplo, na leitura dos utilitaristas ingleses, uma interlocução que se deve em grande medida a sua recente amizade com Paul Rée. Os próprios “espíritos livres”, a que Nietzsche dedica o livro, são uma ficção conceitual que expressa sua ânsia por estar em nova companhia, que possibilitasse “manter a alma alegre em meio a muitos males (doença, solidão, exílio, acedia, inatividade)” (MA/HH, Pr., §2). A tentativa de se livrar da ascendência schopenhaueriana se expressará, também, em termos do experimento de estilo que Nietzsche leva a cabo com este livro, não tanto pelo desenvolvimento da forma aforismática, mas principalmente por se desvencilhar do ambiente romântico da filosofia schopenhaueriana, que em \textit{HH} dá lugar à aproximação com o iluminismo, a uma retórica da sobriedade e da moderação. O que está em questão aqui não diz respeito apenas a uma nova abordagem aos problemas da filosofia da cultura, mas a preocupação, de fato, com um novo modo de vida. 

	Os elementos de que Nietzsche se utiliza nesta guinada apontam, na verdade, para uma retomada daquilo que compunha o ambiente intelectual nietzschiano pré-Wagner, que inclui, além de seus estudos sobre o helenismo tardio, o rigor científico conquistado em sua formação como filólogo, suas várias leituras em ciências naturais e no que hoje chamamos “ciências humanas” (\textit{Geisteswissenschaften})
\footnote{Essas leituras que Nietzsche realizou em seus anos de formação são abordadas de forma muito informativa em LOPES (2008). Outras referências relevantes sobre o tema são: MOORE, BROBJER (2001); PORTER, (2000).}. 
Já que é apenas com este livro que Nietzsche traz a público esse lado mais científico de sua formação, \Hum tem sido lido como uma obra “positivista”. Em geral, o efeito causado pelo livro é tão \textit{sui generis} que alguns pesquisadores têm sugerido considerá-lo como único representante do \textit{interregno} entre suas obras de juventude e sua “filosofia madura”, ou seja, como único livro a se encaixar de fato na alcunha de período do “positivismo cético”
\footnote{Essa ideia é aventada por ALMEIDA (2005, p. 88), HUSSAIN (2004, nota 98), e em seguida atenuada ou descartada por ambos, na medida em que passam a uma consideração mais nuançada do cientificismo de \textit{HH}, e da continuidade de algumas de suas linhas na obra posterior. Nossa impressão é de que, se esses autores chegam a aventar a ideia, isto se deve a uma certa estranheza que costuma advir como efeito da leitura de \textit{Humano}. Dito isto, não nos ateremos à discussão sobre a divisão canônica da obra de Nietzsche em 3 períodos; para o presente trabalho, parece mais interessante considerar \textit{cada} livro como uma unidade de sentido, que tem um alvo específico e, por consequência, uma forma particular de composição. Isto não significa, é claro, que deixaremos de levar em conta a cronologia das obras.}.
Talvez por isto o livro tenha recebido muito menos atenção dos comentadores que seu irmão mais velho, \textit{O Nascimento da Tragédia}, e certamente foi menos lido que as obras de “maturidade” (\textit{Zaratustra}, \textit{Genealogia da Moral}, etc.). 

	Até hoje, Nietzsche é amplamente lembrado como crítico da moderna hipertrofia dos impulsos lógico-teóricos. \Hum, por outro lado, força o leitor a uma visão nuançada dessa questão, apontando para a ideia de que o desenvolvimento dos impulsos lógico-teóricos seria um ponto sem volta da civilização ocidental, de forma que se aposta na ciência inclusive como meio possível de disciplinamento desses impulsos\footnote{É interessante notar que o fato aparentemente contraditório de o filósofo ter avançado um projeto de “filosofia histórica” em \textit{Humano}, alguns anos depois de ter criticado a “hipertrofia” do sentido histórico na modernidade (especialmente na \textit{Segunda Consideração Intempestiva – Da Utilidade e Desvantagem da História pra Vida}) foi mencionado pelo próprio Nietzsche na autocrítica acrescentada ao segundo volume de \textit{Humano}, em 1886: “e o que disse contra a 'enfermidade histórica', disse como alguém que de forma lenta e laboriosa aprendeu a dela se curar, e que absolutamente não se dispunha a renunciar à 'história' porque havia sofrido com ela.” (MA/HH II, Pr. §1.)}.

	Nas últimas duas décadas, ao menos, leituras naturalistas têm ganhado espaço na pesquisa nietzschiana, seguindo a esteira de trabalhos como o de Jörg Salaquarda, e, mais recentemente, o caminho aberto por Maudemarie Clark em seu livro seminal \textit{Nietzsche on Truth and Philosophy}. Essa via de interpretação da filosofia nietzschiana é motivada, em grande medida, por uma insatisfação com a forma como a recepção pós-modernista ou desconstrucionista de Nietzsche tratou alguns pontos de sua filosofia. Com exceção de Foucault\footnote{Que, aliás, não cabe confortavelmente sob o rótulo de “pós-modernista” ou “desconstrutivista”.}, a maior parte dessa tradição pós-modernista teria passado em branco sobre a questão do interesse e do diálogo de Nietzsche com as ciências e, ainda, sobre o lugar que as ciências ocupariam num projeto de pesquisa da verdade. A partir de então, diferentes pesquisadores têm buscado reconstruir o aspecto naturalista da obra de Nietzsche, e como resultado encontra-se hoje todo um espectro de interpretações que vão desde a tentativa, algo grosseira, de encaixar a filosofia nietzschiana nos padrões de um naturalismo essencialista – via de interpretação representada por Brian Leiter
\footnote{Isto porque, para Leiter, haveria certos “fatos” relativos a “tipos” de indivíduos, que implicariam essencialmente num modo de necessidade causal. Esta posição é assim resumida pelo autor: “cada pessoa tem uma constituição psicofísica fixa que a define como um tipo particular de pessoa” (LEITER, 2002, p. 8.). Nossa impressão é de que, ao falar em “fatos”, Leiter perde de vista o que há de mais característico nos tipos ou personagens conceituais nietzschianos: seu aspecto de caricatura, que os aproxima antes dos tipos ideais webberianos.} 
– até experimentos bastante originais e criativos, como por exemplo aquele desenvolvido por John Richardson, que opera uma  leitura seletiva da obra de Nietzsche buscando harmonizá-la com os paradigmas da biologia evolucionista e vice-versa. 

	A tentativa um tanto enrijecida de Leiter pode ser expressão, na verdade, do desconforto do autor, e não só dele, com a dificuldade em que esbarra todo aquele que busca definir algo como um “naturalismo” nietzschiano. Leiter vê como insuficiente, por exemplo, a tentativa esboçada por Christopher Janaway de sistematizar as principais razões por que a filosofia nietzschiana seria “naturalista”. A lista de Janaway inclui: 

\begin{citacao}
Ele [Nietzsche] se opõe à metafísica transcendente, seja aquela de Platão ou do cristianismo ou de Schopenhauer. Ele rejeita as noções de alma imaterial, de controle absolutamente livre da vontade, do puro intelecto autotransparente, enfatizando por outro lado o corpo, fala da natureza animal dos seres humanos, e tenta explicar numerosos fenômenos invocando as noções de impulsos, instintos e afetos que ele localiza em nossa existência física, corporal. Os seres humanos devem ser 'traduzidos de volta em natureza', já que ao contrário nós falsificaríamos sua história, sua psicologia, e a natureza de seus valores – sobre os quais devemos conhecer algumas verdades, como um meio para a importantíssima revaloração dos valores. Isto é o naturalismo de Nietzsche em sentido amplo, que não será contestado aqui.\footnote{Todas as citações no corpo do texto de obras ainda sem tradução para o português, como a de Janaway, foram livremente traduzidas por nós. Optamos por manter os textos originais nas notas de rodapé.} (JANAWAY, 2007, p. 34.)
\end{citacao}

	Para Leiter, a lista de Janaway caracteriza menos um naturalismo “em sentido amplo” que uma lista de “coisas e lousas”, não porque discorde de que estes sejam elementos da filosofia nietzschiana relevantes ao tema do naturalismo, mas porque não estaria clara a ligação entre eles, e o que eles têm de caracteristicamente naturalista (Cf. LEITER, 2013, p. 577). A alternativa de Leiter consiste em tentar encaixar a filosofia nietzschiana em padrões correntemente aceitos de “naturalismo”, buscando elementos nas obras para corroborar a leitura de que Nietzsche está empenhado em reduzir os fenômenos morais a certos “fatos” duros, no sentido fisicalista. Do ponto de vista da história da filosofia, esta conclusão de Leiter sobre a obra de Nietzsche não parece correta; além disso, o autor talvez careça de sensibilidade filosófica para dar conta dos arranjos sofisticados entre ciência, arte, retórica, especulação filosófica, que Nietzsche experimenta ao longo de sua obra.

	Num breve e brilhante artigo, Bernard Williams explica – a nosso ver, de forma suficiente e definitiva – a dificuldade de se falar em um naturalismo nietzschiano: 
	
\begin{citacao}
A dificuldade é sistemática. Se uma psicologia moral “naturalista” tem que caracterizar a atividade moral em um vocabulário que possa ser igualmente aplicado a todo o resto da natureza, então ela está comprometida com um reducionismo fisicalista que conduz claramente a um beco sem saída. Se o caso é descrever a atividade moral em termos que podem ser aplicados a outros domínios, mas não a todos os domínios, não temos muita ideia de quais termos devem ser esses, ou quão “especial” admite-se que seja a atividade moral, em consonância com o naturalismo. Se estamos autorizados a descrever a atividade moral em quaisquer termos que pareçam suscitados por ela, então o naturalismo não exclui coisa alguma, e voltamos ao começo. O problema é que o próprio termo “naturalismo” invoca uma abordagem verticalizada, na qual se supõe que sabemos de antemão quais termos são necessários para descrever qualquer fenômeno “natural”, e somos convidados a aplicar tais termos à atividade moral. Mas nós não sabemos quais termos são esses, a menos que eles sejam (inutilmente) os termos da física, e isso leva à dificuldade. (WILLIAMS, 2011, p. 20).
\end{citacao}

	Williams, então, sugere que pela ideia de “realismo” nos aproximamos de uma melhor compreensão da associação de Nietzsche com as ciências, de sua inserção na tentativa de elaborar um discurso filosófico que considera o homem enquanto parte da natureza, e dos fins antimoralistas desse discurso:

\begin{citacao}
A abordagem de Nietzsche consiste em identificar um excesso de conteúdo moral na psicologia, apelando primeiro àquilo que um intérprete experiente, honesto, sutil, não otimista, pode entender do comportamento humano em outras áreas [não morais]. Tal intérprete pode ser dito – usando uma expressão descaradamente avaliativa – “realista”, e nós podemos dizer que aquilo a que essa abordagem nos conduz é a uma psicologia moral realista, ao invés de naturalista. O que está em questão não é a aplicação de um programa científico predefinido, mas antes uma interpretação informada de algumas experiências e atividades humanas em relação com outras. (WILLIAMS, 2011, p. 21.)
\end{citacao}

	A tarefa de deflação da visão moral de mundo, que se beneficia de vários conhecimentos científicos, bem como de certa disciplina e disposição de espírito cultivadas pelas ciências, tais quais apontadas por Williams como características de interpretação “realista”, são os traços visíveis da filosofia nietzschiana a que nos ateremos ao longo de nossa busca por seu suposto “naturalismo”. Vale lembrar que os traços apontados por Williams como característicos de um tipo de “realismo” estão em perfeita conformidade com a acepção de “naturalismo” corrente no século XIX, em que não havia ainda a exigência fisicalista. Com isto, podemos adiantar que o sentido de “naturalismo” que tende a vir à tona em nossa interpretação é bastante amplo, e está comprometido apenas a manter em vista a relevância de certa informação e conformação científica no interior da filosofia nietzschiana, uma filosofia demasiado abrangente e sofisticada para que este apoio científico seja visto como o seu único traço fundamental. 

	No caminho que seguiremos neste trabalho, não nos interessa cortar as arestas da filosofia nietzschiana para fazê-la encaixar-se definitivamente em qualquer conceito atual de naturalismo. A opção “naturalismo fisicalista”, acima de tudo, já está descartada. Nossa proposta é mapear o lugar que a ciência ocupa na obra de Nietzsche enquanto programa filosófico, e, em linhas gerais, traçar os pontos de contato entre esse programa e a acepção de naturalismo corrente no século XIX. Optamos, então, por manter em foco o debate de Nietzsche com as questões naturalistas que circulavam em seu tempo, observando como o filósofo absorve essas questões, e como elas ganham maior ou menor destaque em diferentes momentos de sua filosofia. Com isto, não pretendemos por um ponto final na discussão sobre “quão naturalista é a obra de Nietzsche”, nem sobre a possível sobrevivência, no século XXI, de algumas dessas questões e das respostas ensaiadas por Nietzsche. Em razão de uma série de limitações, não poderemos tratar em detalhe de toda a extensão dos escritos nietzschianos, mas selecionamos uma amostra que tem como ponto de partida \textit{Humano, demasiado humano}, em razão de ser este o livro com que Nietzsche pela primeira vez traz a público um programa filosófico que se define, em seus aspectos mais centrais, pela aproximação com as ciências, como temos dito. Nos próximos capítulos, partiremos para a comparação entre \Hum e \textit{Além de Bem e Mal}, um livro que foi primeiro concebido por Nietzsche como uma reescrita de \textit{HH}, o que faz supor que com \textit{ABM} o filósofo reeditaria sua aproximação com as ciências; este último livro, no entanto, apresenta conteúdos programáticos e conceituais sobremaneira estranhos a \textit{HH}, fazendo com que o leitor forçosamente repense o sentido dessa aproximação, e o sentido do suposto “naturalismo nietzschiano”. Infelizmente, não poderemos levar em consideração, de forma integral, o conjunto de escritos não publicados por Nietzsche, conhecidos como “fragmentos póstumos”. Esses fragmentos oferecem uma visão valiosa dos “bastidores” da filosofia nietzschiana, deixando entrever hipóteses, diálogos, leituras e preocupações que marcam a gênese de cada obra publicada; no entanto, as edições mais completas e prestigiadas desses textos, como as de Colli e Montinari, não foram ainda traduzidas do alemão, sendo, por ora, inacessíveis para nós.

	Até agora, tentamos apontar as opções interpretativas e linhas gerais que nos guiarão neste estudo; recapitulando, partimos da suposição de que: 1) A obra de Nietzsche como um todo se dedica ao estudo dos processos de vida e morte das culturas 2) Nietzsche tem a intenção de intervir na cultura a favor da redução de seu conteúdo moral e moralizante, tarefa na qual as informações e o modo de pensar científico têm um papel auxiliar importante 3) A filosofia nietzschiana se constrói de forma \textit{experimental}, sendo que cada livro pode ser lido como um todo orgânico que expressa um experimento particular, uma forma possível de intervenção na cultura – o sucesso e insucesso relativos dos componentes desse experimento são retomados e reavaliados nos experimentos (livros) seguintes.

	Neste capítulo, trataremos das razões programáticas pelas quais a associação entre filosofia e ciência ganha significância central no experimento desenvolvido em \Hum. Tomamos como ponto de partida a ideia de que a ciência cumpre três funções cardeais no interior do programa filosófico de \Hum: 1) enquanto meio de aquisição de virtudes epistêmicas consideradas fundamentais a um filosofar-histórico, isto é, a atividade científica como forma de cultivo de honestidade intelectual, modéstia, cautela, economia de princípios – virtudes que, para Nietzsche, não teriam sido alcançadas pelo modo tradicional de se fazer filosofia; 2) enquanto instrumento, auxílio na elaboração positiva dos “fins ecumênicos” a serem perseguidos pela cultura, tendo em mente a urgência que Nietzsche confere em \textit{HH} à necessidade de confecção de uma cultura pós-metafísica; 3) enquanto referência normativa e fonte de conhecimentos que têm função terapêutica no plano do indivíduo\footnote{Mapeamento das funções da ciência em HH proposto por Rogério Lopes. Cf. Lopes (2008). Ao falarmos, no item 3) de ciência como “referência normativa” temos em mente a ideia de que aquelas virtudes epistêmicas nomeadas acima (honestidade, cautela, moderação, etc.) são evocadas por Nietzsche também no plano de uma ética prática, de forma que sua aplicação se estende a partir de um âmbito epistemológico até o campo do cuidado de si, etc.}.

\section{O devir histórico da filosofia}

\begin{quotation}
\textit{Química dos conceitos e sentimentos.} – Em quase todos os pontos, os problemas filosóficos são novamente formulados tal como dois mil anos atrás: como pode algo se originar do seu oposto, por exemplo, o racional do irracional, o sensível do morto, o lógico do ilógico, a contemplação desinteressada do desejo cobiçoso, a vida para o próximo do egoísmo, a verdade dos erros? Até o momento, a filosofia metafísica superou essa dificuldade negando a gênese de um a partir do outro, e supondo para as coisas de mais alto valor uma origem miraculosa, diretamente do âmago e da essência da “coisa em si”. Já a filosofia histórica, que não se pode mais conceber como distinta da ciência natural, o mais novo dos métodos filosóficos, constatou em certos casos (e provavelmente chegará ao mesmo resultado em todos eles), que não há opostos, salvo no exagero habitual da concepção popular ou metafísica, e que na base dessa contraposição está um erro da razão: conforme sua explicação, a rigor não existe ação altruísta nem contemplação totalmente desinteressada; ambas são apenas sublimações, em que o elemento básico parece ter se volatilizado e somente se revela à observação mais aguda. – Tudo o que necessitamos, e que somente agora nos pode ser dado, graças ao nível atual de cada ciência, é uma química das representações e sentimentos morais, religiosos e estéticos, assim como de todas as emoções que experimentamos nas grandes e pequenas relações da cultura e da sociedade, e mesmo na solidão: e se essa química levasse à conclusão de que também nesse domínio as cores mais magníficas são obtidas de matérias vis e mesmo desprezadas? Haveria muita gente disposta a prosseguir com essas pesquisas? A humanidade gosta de afastar da mente as questões acerca da origem e dos primórdios: não é preciso estar quase desumanizado, para sentir dentro de si a tendência contrária? (MA/HH, §1)
\end{quotation}

Ao fazer a abertura de \textit{Humano} com o aforismo acima, intitulado “\textit{Química dos conceitos e sentimentos}”, Nietzsche não deixa dúvidas quanto à agenda a ser perseguida ao longo do livro e ao tom que marcará essa investigação. Se o reproduzimos aqui, é porque esse primeiro aforismo oferece uma síntese primorosa, na qual são nomeadas questões centrais e respectivas resoluções metodológicas: a rejeição do modo metafísico de investigação filosófica
\footnote{Entendemos que por “metafísica” Nietzsche se refere, em geral, tanto à posição epistemológica que insiste na possibilidade de conhecimento substantivo a priori, quanto à ontologia que expressa um compromisso com a existência de objetos que podem ser conhecidos (se é que o podem) apenas por meios \textit{a priori}. Interpretação sugerida por CLARK, DUDRICK, 2012, p. 20.}; 
a defesa de que “as coisas de mais alto valor” (criações estéticas, religiosas, etc.) sejam abordadas pelos mesmos métodos científicos aplicados às coisas comuns, próximas, ao mesmo tempo em que se destina um lugar na agenda filosófica para essas coisas “próximas” elas mesmas (isto é, às coisas que constituem os modos de vida cotidiana e que são portanto tangíveis, mundanas, observáveis); a ênfase no estudo da cultura pelo viés dos valores; o anúncio do “mais novo método filosófico”, a filosofia histórica; a busca por uma forma de compreensão da ação que evite o modelo moralizante centrado em altruísmo/egoísmo; a visada às “questões acerca da origem”, que serão retomadas, ao longo do livro, na forma de uma protogenealogia.

Buscaremos abordar estes pontos ao longo do capítulo, mas nesta seção nos ateremos à proposta do “mais novo método filosófico”, que teria como ponto de partida a associação entre filosofia e ciências empíricas, em especial a História. Manteremos em vista, também, suas implicações no programa de intervenção na cultura a que \textit{Humano} se destina.

	O alvo não nomeado dessa discussão metodológica é a filosofia transcendental kantiana – com seu “mais novo método filosófico” Nietzsche se opõe à dedução por introspecção de categorias a priori, necessárias e universais que, segundo o entendimento de leitores de sua época, constituía o \textit{modus operandi} da filosofia kantiana
\footnote{Nietzsche se insere, portanto, na querela conhecida como \textit{Methodenstreit}, que se desenrolou nas universidades alemãs por quase todo o século XIX. Essa discussão se dá no contexto de uma crise de legitimação das chamadas ciências humanas (\textit{Geisteswissenschaften}) na universidade alemã. Por defender uma especificidade metodológica para a filosofia, diferente do método das ciências empíricas, a filosofia kantiana foi em grande parte tida como capaz de resolver a questão, e estabelecer um “pacto” de convivência entre a filosofia e as diferentes ciências. A proposta de Nietzsche aqui, pelo contrário, tem claramente o sentido de aproximar a filosofia dos métodos das ciências empíricas. Muitos pesquisadores têm afirmado que por “método” Nietzsche não se refere a um procedimento específico, mas uma certa disposição de espírito voltada à cautelosa pesquisa empírica e à observação dos princípios de economia, simplicidade, coerência, etc. A \textit{Methodenstreit} foi analisada em detalhe por LOPES (2008, 2011). Ver também LEITER (2002), que tornou conhecida a expressão “naturalismo metodológico”, termo que aplica à filosofia nietzschiana.}. 
Com isto, Nietzsche nega qualquer especificidade metodológica à filosofia, em especial, nega que a filosofia deva se pautar por qualquer tipo de apriorismo; portanto, toma como ponto de partida a exigência de que o filósofo não mais desconsidere as observações empíricas sobre os diversos temas a serem abordados. Esta discussão é perpassada pela suspeita de que os fatores que condicionam nossa percepção do mundo (e, em última instância, nossa atuação sobre ele) estão em devir, não são realidades fixas, muito menos universais. 

Nossa impressão, na verdade, é que a realidade do devir é menos uma suspeita do que um axioma, cumprindo uma função estrutural na filosofia de \textit{Humano}. Nesse livro, Nietzsche está disposto a discutir muitos, quase todos, dos traços fundamentais da nossa percepção do mundo (por exemplo, os princípios de identidade e substância); quase nenhum deles escapa à acusação de que esses traços são “erros”, ou seja, ficções instrumentais inconsistentes que não correspondem a algo objetivamente real, sendo apenas condições mais ou menos imprescindíveis à nossa organização psicofísica. Mas é justamente por supor uma realidade em fluxo que Nietzsche condena todos esses traços como “artigos de fé”. A ideia é: todas as formas com que organizamos o mundo têm valor meramente condicional (transcendental, neste sentido), enquanto o devir parece ter realidade objetiva. O filósofo sugere que a temporalidade observável nos fenômenos de mudança relativa parece deixar entrever algo objetivamente mais real – quer dizer, ao compararmos nossas experiências entre si, observando o sentido de “mudança” que há entre elas, seria razoável supor que essas experiências são parte de um mundo que está em fluxo num sentido mais radical do que ordinariamente percebemos. A tese do devir radical, portanto, seria uma ampliação do alcance e intensidade do sentido de mudança.

O uso que Nietzsche faz da noção de devir, suas implicações ontológicas e epistemológicas, têm sido objeto de muita discussão. Alguns pesquisadores, como Michael Green (\textit{Nietzsche and the Transcendental Tradition}, 2002), caracterizam a atitude de Nietzsche quanto à noção de devir como uma “aposta ontológica”. Green investiga a aposta de Nietzsche enquanto antípoda daquela de Platão. Segundo Green, frente à impossibilidade do pensamento de abarcar a mudança percebida pelas sensações, Platão teria passado a ver esse mundo empírico em movimento como menos real que o Ser parmenídico, e aposta na realidade deste último, pois que sempre idêntico a si mesmo, congênito às formas do pensamento. Nietzsche aposta em Heráclito; ou antes, ele aposta numa interpretação radical, cratiliana, do mundo em fluxo sugerido por Heráclito. Na verdade, ao falar em “fluxo”, em “devir”, já estaríamos no terreno da metáfora – sobremaneira, a realidade a que Crátilo aponta seria indizível, não seria estável o bastante para que lhe penduremos um nome, seria fugidia a todos os conceitos e categorias com que o pensamento opera. A aposta de Nietzsche pode parecer um tanto estranha, portanto
\footnote{Segundo Green, a necessidade de se “apostar” num mundo parmenídico ou heraclítico decorreria, portanto, de um suposto impasse entre o aparato conceitual intelectivo (que opera fundamentalmente com o conceito de objetos estáveis, idênticos a si mesmos e incondicionados) e a percepção sensível (que nos apresenta os objetos sensíveis como condicionados, sujeitos a mudança e, segundo Spir, não idênticos a si mesmos, uma vez que são percebidos por estímulos variados e diferentes entre si, por exemplo quando percebo uma mesa ao mesmo tempo como “sólida” e “marrom”. Spir apontaria, portanto, a uma espécie de heraclitismo da sensação, conclusão que é assim explicada por Green: “The sensations [\textit{Sinnesempfindungen}] and the inner states of the cognizing subject [the contrast is between, for example, colour-sensations and pain] form the entire cognizable world, the world of experience, which is conditioned in all its parts. Therefore what the old Heraclitus taught is true: the world of experience is to be compared to a river in which new waves continuously displace the earlier ones and which doesn’t remain completely identical to itself for even two successive instances (1:277).” GREEN, 2002, p. 61.}.

O que nos interessa na leitura de Green não é tanto o quadro teórico em que se apoia para tratar a questão da aposta no devir, nem as conclusões epistemológicas a que ele chega, mas a tipificação de diferentes \textit{apostadores}, no caso, Nietzsche e Platão. Se há motivos para se apostar tanto na estabilidade do Ser, quanto no devir radical, é de se supor que o peso final da aposta recai sobre uma afinidade pessoal do filósofo com certa visão de mundo, sendo que essa visão será refletida também na sua compreensão das condições de conhecimento. Se apostamos, com Platão, no eterno Ser parmenídico, adotamos uma posição que deixa espaço para a possibilidade de conhecimento \textit{a priori}; se apostamos, com Nietzsche, no fluxo heraclítico do vir-a-ser que nunca chega a ser, somos lançados a um âmbito em que só faz sentido pensar o conhecimento como proveniente da experiência, que nunca se repete identicamente, que sempre exige renovada observação.

Além desse sentido de afinidade filosófica pessoal com a visão de mundo como devir, há vários argumentos de ordem epistemológica que têm sido apontados pelos comentadores, na tentativa de entender essa tese nietzschiana. Alguns têm apontado que, via o neokantiano Afrikan Spir
\footnote{Filósofo ucraniano. Autor de \textit{Denken und Wirklichkeit} (\textit{Pensamento e Efetividade}). }, Nietzsche teve acesso a um argumento de negação da idealidade transcendental do tempo. William Mattioli (2011) por exemplo, reconstitui com muita propriedade o argumento spiriano e as considerações de Nietzsche a este respeito. De forma resumida, poderíamos dizer que o argumento tem como ponto de partida o fato de que nós não só temos representações da temporalidade, mas nossas representações, elas mesmas, nos aparecem de forma temporal, sendo que esse próprio aparecer temporal prova a existência do tempo. Nas palavras de Mattioli:

\begin{quotation}
O argumento decisivo desta passagem é o seguinte: ao dizer que meus estados de consciência e minhas representações me aparecem como sucessivos e mutáveis, sou obrigado a aceitar que essa aparência mesma possui uma realidade objetiva enquanto estado de consciência – realidade da qual não podemos abstrair a temporalidade sem contradizer radicalmente a evidência fenomenológica mais elementar do processo do representar.(MATTIOLI, 2011, p. 240).
\end{quotation}

Mais adiante, Mattioli traça uma associação entre o procedimento deste argumento e o \textit{cogito} cartesiano. Assim como, no \textit{cogito}, não se pode pensar que não se está pensando, e portanto se conclui que há a res pensante, na negação da idealidade do tempo não se pode pensar que não existe o tempo, uma vez que o pensamento ele mesmo é temporal. Mattioli explica:

\begin{quotation}
Trata-se aqui, portanto, de uma constatação em certo sentido fenomenológica de que à essência de toda \textit{cogitatio} pertence um tempo que é constitutivo da \textit{cogitatio} ela mesma e que não pode ser dela abstraído. Assim, a temporalidade não é um atributo acidental do pensamento e da representação, mas, antes, uma de suas determinações mais essenciais. O aparecer sucessivo e em constante mudança de dados sensíveis imanentes na atividade do representar é fenomenologicamente indubitável. Por conseguinte, na medida em que a sucessão e a mudança das representações possuem realidade objetiva; na medida em que, no próprio aparecer das representações, uma coisa se segue realmente à outra, não estamos autorizados a negar a realidade do tempo. Com isso, a tese kantiana da idealidade transcendental do tempo estaria refutada. (MATTIOLI, 2011, p. 241).
\end{quotation}

Assim como Green, Mattioli fala em uma “aposta ontológica” em relação ao devir. Ao que nos parece, é justamente este sentido de \textit{aposta} que deve permanecer em mente quando analisamos as ocorrências em que Nietzsche fala desse tema. Insistimos na impressão de que o filósofo parece ser levado ao axioma do devir quase que por uma afinidade íntima, provavelmente uma marca de identificação com a tradição heraclítica. Tudo o mais – os argumentos ocasionais sobre o “caos das sensações”; a função da noção de devir como conceito-limite, como tem sido apontada por alguns intérpretes; mesmo argumentos mais robustos e consistentes, como aquele que leva à negação da idealidade transcendental do tempo, que vimos acima – tudo isso certamente foi levado em conta pelo filósofo, mas provavelmente como corroboração de uma intuição a que Nietzsche sempre deu ouvidos, que aparece como pressuposto desde seus primeiros escritos. Ao falar em devir, Nietzsche raramente apresenta esses argumentos (na obra publicada), e quando o faz, dá a impressão de estar racionalizando uma decisão que já estava tomada. Isto não torna menos relevante o fato de  Nietzsche sugerir que sua aposta é corroborada pelas ciências de seu tempo, que tanto no campo da física quanto da biologia, por exemplo, se aproximavam de modelos mais dinâmicos. Além disso, há um interesse de Nietzsche pelo dinamismo da produção científica enquanto tal, na medida em que ela é movida pela constante transformação dos paradigmas teóricos e revisão de resultados
\footnote{Ver STACK, 2005, (Pr) p. xi.: “In his studies of early philosophies of science and philosophical scientists Nietzsche absorbed the phenomenon of the theoretical displacementsof orientations in the history of science. Although he does not formulate anything precisely like Kuhn’s theory of paradigm shifts, he certainly was moving in that direction”. }.

Segundo nossa narrativa, portanto, a adesão de Nietzsche à visão de mundo como devir tem origem numa afinidade filosófica com a tradição heraclítica. Esse tipo de afinidade sempre constitui um objeto de interesse mais relevante para o filósofo do que as diferentes teorias em suporte de uma ou outra visão. Neste sentido, estamos de acordo com pesquisadores como John Richardson, que afirma que: 

\begin{quotation}
“(...) não há dúvidas de que o maior interesse de Nietzsche é sobre nossas visões sobre o tempo, e não sobre a natureza do tempo ele mesmo. Ele pretende oferecer uma descrição e crítica dessas visões. E a propósito dessas visões ele está menos interessado em nossas crenças sobre o tempo que em nossas avaliações do tempo e do devir, e em como nosso sentimos sobre eles. O que importa especialmente não é uma teoria sobre se o mundo é ser ou devir, mas qual dessas opções é valorizada, qual delas se deseja que o mundo seja. De fato o que se deseja que elas sejam no nível do corpo, dos impulsos, e não conscientemente. (…) Além disto – deslocando-se ainda mais do foco na “verdade sobre o tempo” – ele examina essas avaliações não primariamente para descrevê-las mas para julgá-las, e frequentemente por padrões não-epistêmicos. Ele quer menos entender os diferentes tipos de pessoas a favor do devir ou do ser do que acessar e “ranqueá-las”, em particular por seus graus de força. Ele promove, acima de tudo, como o mais forte, o tipo “Dionisíaco”, que vê e quer um mundo do devir. Ou, talvez, ele meramente nos mostra que \textit{ele} é o tipo de pessoa que favorece o devir em detrimento do ser – i.e. meramente expressa uma atitude sem afirmá-la. Tudo isto está muito longe da meta de uma teoria do tempo.” (RICHARDSON, 2006, p. 209)
\end{quotation}

Contudo, a preponderância do interesse de Nietzsche sobre os impactos normativos de uma visão de mundo como ser ou como devir não implica em que o filósofo tenha sumamente desconsiderado os aspectos epistemológicos e as diferentes teorias sobre o tempo, o que o próprio Richardson reconhece mais adiante: “Mas eu afirmo que uma certa teoria do tempo, contudo, está na raiz desses diagnósticos e avaliações de nossas visões sobre o tempo. Algumas dessas avaliações são \textit{epistêmicas} – acessam as diferentes visões pelo viés de sua verdade.” (RICHARDSON, 2006, p. 209). A escolha pelo devir, que sem dúvidas teve impacto na forma nietzschiana de filosofar como um todo, parece datar da época de formação do jovem Nietzsche
\footnote{Uma vez que a ideia já estava presente, por exemplo, no escrito de juventude “Fado e História”, de 1862.}, mas a partir daí, ele certamente encontrou trunfos teóricos importantes para lhe dar suporte: um deles é a negação da idealidade transcendental do tempo, mencionada acima, e o outro é a perspectiva evolucionista da biologia, a qual abordaremos na próxima seção.

Essa aposta na ontologia do devir pode levantar a suspeita de que a filosofia nietzschiana, afinal, não consegue evitar as mesmas armadilhas metafísicas que criticava já no primeiro aforismo de \textit{HH}, citado acima, e que são um tema de toda sua obra seguinte. Mesmo que haja aqui um resquício de pensamento metafísico, em algum sentido, é preciso reconhecer que Nietzsche tenta levar sua discussão para além da distinção noumenon/fenômeno. Ao falar em “devir” como uma realidade fundamental, Nietzsche não está dando um nome para a coisa-em-si kantiana; pelo menos, isto está longe da compreensão que expressa a respeito de sua própria filosofia. Segundo a leitura que Nietzsche faz da filosofia kantiana – que, em grande medida, se refere igualmente aos neokantianos Spir e Schopenhauer\footnote{Tem sido notado que Schopenhauer e Spir se inserem de forma heterodoxa na tradição kantiana. Esses dois filósofos, no entanto, certamente se consideravam herdeiros da filosofia kantiana, e foram lidos por Nietzsche como tal.} – a coisa-em-si opera uma duplicação do mundo, ao dividi-lo entre fenômeno (condicionado por formas necessárias e universais da razão e da percepção) e \textit{noumenon} (enquanto resíduo incondicionado, não-relacional, idêntico a si mesmo e incognoscível a que se chama “coisa-em-si”). 

Se esta é uma boa leitura de Kant e seus herdeiros mais próximos é uma questão que deixamos aos pesquisadores atuais da filosofia kantiana
\footnote{ores do próprio Nietzsche têm apontado várias imprecisões no seu entendimento das nuances da filosofia kantiana: “Nietzsche quite often refers to 'categories of reason' when he means categories of the understanding, questions of the 'truth' of categories such as 'unity' when he is actually concerned with the validity of their ontological reference to natural entities, and sometimes misleadingly refers to the 'phenomenal' flux that lies \textit{behind} the conditioned phenomena which, in Kant's sense, are all we know.” STACK, 2005, p. 21. }.
 O que nos interessa aqui é que esse é o entendimento que Nietzsche tem do kantismo, em contraposição ao qual desenvolve sua filosofia do devir. Para nós, está muito claro que um certo “agnosticismo” de Nietzsche em relação à coisa-em-si faz com que este conceito seja ostensivamente apontado como irrelevante em \textit{HH}; é a conclusão que parece se impor a partir de passagens como esta: 

\begin{quotation}
“(...) A \textit{prova científica} de qualquer mundo metafísico já é tão \textit{difícil}, talvez, que a humanidade não mais se livrará de alguma desconfiança em relação a ela. E quando temos desconfiança em relação à metafísica, de modo geral as consequências são as mesmas que resultariam se ela fosse diretamente refutada e não mais nos fosse lícito acreditar nela.” (MA/HH, §21). 
\end{quotation}

É de se supor, portanto, que a noção de devir não se candidata a ocupar o posto de “mundo metafísico”, visto que este último é tido como inócuo; por outro lado, o estar-em-devir da nossa constituição é sempre invocado como fator altamente relevante para sua compreensão. Enquanto a coisa-em-si permaneceria incognoscível, como conceito limite, para Nietzsche o devir é relativamente apreensível através dos fenômenos de mudança, abordados pelas ciências históricas. Se por um lado a coisa-em-si é entendida como o “incondicionado, e portanto também incondicionante” (MA/HH, §16), por outro lado o devir é apontado como \textit{condição} fundamental do que existe. O mundo enquanto “devir”, portanto, não se comporta como coisa-em-si, por definição um aspecto não-relacional do mundo que é cabalmente incognoscível. O devir pode ser interpretado. Todas as interpretações dependem fatalmente de simplificação, estando sempre sujeitas a falhas ou  inconsistências internas; ao que parece, “simplificação”, “esquematização”, “recorte”, tudo isto conta como “erro” no vocabulário nietzschiano. Todas as interpretações, portanto, tendem a \textit{falsificar} o mundo em alguma medida. A ciência é uma dessas interpretações, mas uma interpretação capaz de avançar em objetividade porque atua criticamente em relação às categorias que utiliza e porque colide um grande número de experiências. 

Nietzsche, é claro, se insere no contexto do século XIX, que ficou conhecido como o século da História – imprimir um sentido histórico ao estudo da natureza, como fez a biologia evolucionista, é ainda hoje visto como uma grande façanha desse século. Se no primeiro aforismo do livro já estava anunciado o “mais novo método filosófico”, isto é, a “filosofia histórica”, o aforismo subsequente desenvolve o tema de como a aquisição de sentido histórico é uma conquista capaz de corrigir o que seria um erro renitente dos filósofos: “Falta de sentido histórico é o defeito hereditário de todos os filósofos; inadvertidamente, muitos chegam a tomar a configuração mais recente do homem, tal como surgiu sob a pressão de certas religiões e mesmo de certos eventos políticos, como a forma fixa de que se deve partir.”. Na sequência, o filósofo recorre à história da “evolução humana” como meio de se chegar a uma visão mais sóbria do homem atual, e lança um veredicto programático: “Mas tudo veio a ser; \textit{não existem fatos eternos}: assim como não existem verdades absolutas. - Portanto, o \textit{filosofar histórico} é doravante necessário, e com ele a virtude da modéstia.” (MA/HH, §2). Nietzsche sugere que sua intuição do mundo como devir é corroborada pelas diversas ciências, na medida em que elas mesmas se encaminham para modelos cada vez mais dinâmicos, deixando de ver, por exemplo, a constituição biológica do homem e dos outros animais como um dado fixo, cujo sentido se esgotaria na morfologia que apresentam atualmente. O tom taxativo da afirmação “não existem fatos eternos” não deixa dúvidas quanto à função axiomática da noção de devir; no entanto, a possibilidade de pesquisa científica de “verdades absolutas” sobre esse mundo em devir está riscada.

Ao que tudo indica, o conhecimento desse mundo, enquanto devir, não faz qualquer referência a um \textit{em-si}. O filósofo aposta que, futuramente, o conhecimento científico poderá solapar definitivamente as malhas conceituais em que nos enredamos com a distinção fenômeno/coisa-em-si (cf. MA/HH, §16), uma vez que tiver desvelado a gênese dessa distinção. Nietzsche parece encaminhar sua filosofia para um quadro conceitual em que seria mais apropriado dizer que, pelas formas com que habitualmente observamos e categorizamos o mundo, habitamos a \textit{superfície} de algo que é, em maior grau, o devir radical. Se falamos em “formas habituais” de se representar esse mundo é porque para Nietzsche as categorias que aplicamos nessas representações carecem da universalidade e necessariedade das categorias kantianas: são vistas como determinadas pela história evolutiva (biológica e social) da espécie, e incorporadas em maior ou menor grau conforme sua utilidade à organização psicofísica dos organismos. A expressão é do próprio Nietzsche: 

\begin{quotation}
Desse mundo da representação, somente em pequena medida a ciência rigorosa pode nos libertar – algo que também não seria desejável –, desde que é incapaz de romper de modo essencial o domínio de \textit{hábitos ancestrais de sentimento}; mas pode, de maneira bastante lenta e gradual, iluminar a história da gênese desse mundo como representação – e, ao menos por instantes, nos elevar acima de todo evento. (MA/HH, §16, itálico nosso). 
\end{quotation}

Se a ciência pode “nos elevar acima de todo evento” é porque consegue deixar entrever algo que está na contramão das formas fundamentais com que representamos o mundo (identidade, substância, etc.), e da nossa capacidade de perceber o mundo apenas enquanto “eventos”  independentes entre si, na forma do devir relativo.		

	Se conseguimos estabelecer o ponto até aqui, deve parecer compreensível que nos encaminhamos para uma interpretação que atribui à filosofia nietzschiana um tipo de realismo moderado ou quase-realismo – entendido como ponto de vista que supõe uma realidade mente-independente, que no entanto não se comporta como coisa-em-si, visto que se trata de uma realidade passível de observação e interpretação, sendo a ciência tida por capaz de progressivamente depurar os erros que tendem a incidir sobre as interpretações. A “depuração” científica desses “erros” interpretativos seria feita por meio de uma abordagem crítica dos pressupostos cognitivos envolvidos na interpretação, e da observação de valores epistêmicos cultivados pela comunidade científica: simplicidade, economia, coerência, confrontação de dados, continuidade de esforços, etc. A aposta na realidade do devir implica numa aposta de que quanto mais dinâmico um modelo científico se mostrar, maior sua elegibilidade a um maior \textit{grau} de verdade, restando o horizonte do devir absoluto, a que é impossível corresponder, como reserva falibilista para as teorias. 
	
	Na próxima seção, nos ateremos em maior detalhe à chamada “teoria do erro” nietzschiana, seus pressupostos e implicações epistemológicas e filosóficas em geral. No momento, o que está em jogo é compreender o programa cultural que Nietzsche julga ser necessário dada a aposta na realidade do devir e o cenário de um desenvolvimento científico que torna essa realidade cada vez mais descritível e manipulável. Isto é, até agora apresentamos os traços gerais da argumentação de Nietzsche, segundo a qual o ocidente se encaminha para uma visão cientificamente informada de mundo que tende a criar modelos cada vez mais dinâmicos, de modo que o filósofo passa a entender como uma necessidade premente o investimento numa nova forma de filosofar, a que chama \textit{filosofar histórico}.
	
	Ao que nos parece, o programa da \textit{filosofia histórica} visa pelo menos dois grandes efeitos sobre a cultura: 1. criar um tipo de relativismo que nos permitiria dispensar certos investimentos culturais até então tidos como necessários, mas que passam a ser apontados como um atavismo e um entrave à renovação da cultura, e 2. munidos da positividade prática da ciência, investir em novas metas e objetivos culturais, a que Nietzsche chama “objetivos ecumênicos da humanidade”. Esses dois pontos dependeriam de um tipo específico de investimento científico e filosófico: o estudo das “condições da cultura”, ou seja, uma visão dos fatores históricos que levaram à criação dos valores e instituições de que dispomos, bem como das mudanças que podem fazer com que tais valores e instituições não mais sejam vistos como necessários. 

As necessidades de que Nietzsche quer se desfazer, é claro, são as chamadas “necessidades metafísicas”, amplamente entendidas: seja como necessidade de criação de uma imagem totalizante do mundo, supostamente capaz de apaziguar e reconciliar o homem com a vida prática; seja qualquer necessidade de duplicação do mundo, tanto aquela presente na distinção noumenon/fenômeno, quanto na versão que inscreve o homem numa teleologia de vida após a morte, no âmbito de um mundo metafísico essencialmente diferente do mundo da experiência (necessidade em que se apoiam muitas das grandes religiões, inclusive a cristã, é claro); ou ainda, a necessidade de ascetismo como única via ética possível frente à perplexidade causada pelo sofrimento e pela morte.

	Nietzsche de fato coloca todas essas diferentes necessidades num mesmo plano, e as trata como igualmente passíveis de serem dispensadas. Isto não parece se dever a qualquer falta de acuidade na formulação da questão, mas à suspeita de que essas diferentes necessidades podem ter tido alguns fatores genéticos em comum, de modo que Humano apresenta hipóteses acerca dessa gênese em muitos de seus aforismos. Tomemos por exemplo, este trecho do aforismo intitulado \textit{Inocuidade da metafísica no futuro}:
	
	\begin{quotation}
	Logo que a religião, a arte e a moral tiverem sua gênese descrita de maneira tal que possam ser inteiramente explicadas, sem que se recorra à hipótese de intervenções metafísicas no início e no curso do trajeto, acabará o mais forte interesse no problema puramente teórico da “coisa em si” e do “fenômeno”. (…). (MA/HH, §10)\footnote{No aforismo 5 de \textit{HH}, avança-se a ideia de que essa duplicação do mundo teria origem na experiência do sonho, que em tempos ancestrais foi interpretada como uma forma de acesso a um “\textit{segundo mundo real}”, inclusive habitado pelos mortos, que aparecem no sonho. O aforismo apresenta a ideia de que toda metafísica tem origem por via dessa duplicação original, que se liga à experiência do sonho, de forma que através dela se teria chegado à duplicação corpo/alma, e talvez até mesmo na crença nos deuses. Como há um ranço da noção de progresso e darwinismo social no livro, no aforismo 12 desenvolve-se a ideia de que esse tipo de crença no sonho está relacionado aos estados “primitivos” da humanidade. Há aqui, e no livro como um todo, um certo traço positivista, em que ecoa a tese mais conhecida dos \textit{Opúsculos de Filosofia} de Auguste Comte, segundo a qual a humanidade teria passado pelos “estágios” evolutivos: teológico, metafísico e positivo. A ascensão de uma cultura cientificista, nesse contexto, tende a ser vista como um desdobramento necessário da história. Esta visão da história aparece residualmente em \textit{HH}, e aliás está em conflito com outros compromissos nietzschianos. Nietzsche parece tratar esse problema de forma mais refletida nas obras posteriores, desenvolvendo uma crítica interessante à noção de “progresso”, por exemplo.}.
	\end{quotation}

A tríade “religião, moral e arte” é recorrente no livro (MA/HH §1, §3, §4, §10, entre outros), sugerindo que essas áreas da cultura ocidental dependeram, até agora, de investimentos fundamentalmente metafísicos. A associação entre essas três áreas da cultura gira em torno de algumas diferentes ideias: em primeiro lugar, a ideia de que tradicionalmente construimos uma compreensão da religião, da moral e da arte em termos metafísicos, enxergando-as como campos algo “especiais” da atividade humana, brotando diretamente da “coisa-em-si”, talvez. Há ainda, a ideia de que essas áreas compartilhariam de uma certa visão de mundo, metafisicamente fundada. Esta ideia tem a ver com a tese já subentendida no trecho citado acima, e que será completamente desenvolvida nas obras subsequentes de Nietzsche: a de que o pensamento metafísico opera fundamentalmente com uma duplicação do mundo, na qual o mundo da experiência ordinária é visto como menos real, menos verdadeiro e pior que o suposto outro mundo, embora este permaneça ordinariamente intangível e incognoscível. Nietzsche critica esse tipo de duplicação do mundo apontando-o como sintoma de insatisfação, cansaço, negação. No campo da arte, por exemplo, essa duplicação de mundo estaria presente na ideia de que o gênio artístico tem um tipo de acesso privilegiado à realidade última do mundo; os esforços de naturalização do “gênio” artístico se concentram grandemente no combate a esta ideia (cf. MA/HH, §155). Além disto, em \textit{Humano} é recorrente a ideia de que essas formas de expressão humana – religião, moral e arte, enquanto apoiadas numa forma metafísica de pensar – operam pela criação de discursos alegóricos sobre o mundo, ao passo que em \textit{Humano} se encontra todo um esforço do filósofo no sentido de criticar a alegoria, obviamente contrária ao espírito de sobriedade e cientificismo do livro, como se pode ver neste aforismo:

\begin{quotation}
A metafísica dá para o livro da natureza uma explicação, digamos, \textit{pneumática}, como a Igreja e seus eruditos faziam outrora com a Bíblia. É preciso grande inteligência para aplicar à natureza o mesmo tipo de rigorosa arte interpretativa que os filólogos de hoje criaram para todos os livros: com a intenção de meramente compreender o que quer dizer o texto, e não de farejar, ou mesmo pressupor, um \textit{duplo} sentido. Mas como, mesmo em relação aos livros, a má exegese não está de modo algum superada, e como na melhor sociedade culta ainda encontramos frequentemente resíduos de interpretação alegórica e mística, assim também ocorre no tocante à natureza – e mesmo pior ainda. (MA/HH, §8).
\end{quotation}

Tendo isso em vista, talvez não pareça surpreendente a afirmação de que o combate de Nietzsche a tais “necessidades metafísicas” se deve à percepção de que elas simplesmente não mais representam uma possibilidade para um certo grupo que então começava a surgir na Europa, os “espíritos livres” a quem o livro é dedicado. Por espíritos livres Nietzsche se refere, é claro, a algo como um “tipo ideal”, eficaz o bastante para apontar que as supostas necessidades metafísicas deixaram de ter apelo prático para uma parcela significativa da população, dadas as novas condições de vida que despontavam no século XIX. Está anunciado em \textit{Humano} um fenômeno sobre o qual Nietzsche se expressará de forma memorável em \textit{A Gaia Ciência}: a “morte de Deus”
\footnote{“O maior acontecimento recente – o fato de que “Deus está morto”, de que a crença no Deus cristão perdeu o crédito – já começa a lançar suas primeiras sombras sobre a Europa. Ao menos para aqueles cujo olhar, cuja \textit{suspeita} no olhar é forte e refinada o bastante para esse espetáculo, algum sol parece ter se posto, alguma velha e profunda confiança parece ter se transformado em dúvida: para eles o nosso velho mundo deve parecer cada dia mais crepuscular, mais desconfiado, mais estranho, “mais velho”. (…) De fato, nós, filósofos e 'espíritos livres', ante a notícia de que 'o velho Deus morreu' nos sentimos como iluminados por uma nova aurora; nosso coração transborda de gratidão, espanto, pressentimento, expectativa – enfim o horizonte nos aparece novamente livre, embora não esteja limpo, enfim os nossos barcos podem novamente zarpar ao encontro de todo perigo, novamente é permitida toda ousadia de quem busca o conhecimento, o mar, o nosso mar, está novamente aberto, e provavelmente nunca houve tanto 'mar aberto'.” (FW/GC, §343). Ver também: FW/GC §108, §125, §153.}.
 Segundo o diagnóstico apresentado no parágrafo 109 de \textit{Humano}, o homem moderno, ambientado numa cultura científica, padeceria de sofrimentos outros, para os quais não há consolação metafísica que se aplique.

Este fato é visto como de certa forma um desdobramento “natural” da história europeia; não haveria exatamente um problema em dispensar essas necessidades, uma vez que se tem uma visão relativizada sobre sua vigência; a possibilidade de que a cultura se livre dessas necessidades, na verdade, é vista como uma oportunidade de renovação e melhoria: 

\begin{quotation}
Mas deveríamos também aprender, afinal, que as necessidades que a religião satisfez e que a filosofia deve agora satisfazer não são imutáveis; podem ser \textit{enfraquecidas e eliminadas}. Pensemos, por exemplo, na miséria cristã da alma, no lamento sobre a corrupção interior, na preocupação com a salvação – conceitos oriundos apenas de erros da razão, merecedores não de satisfação, mas de destruição. Uma filosofia pode ser útil \textit{satisfazendo} também essas necessidades, ou \textit{descartando-as}; pois são necessidades aprendidas, temporalmente limitadas, que repousam em pressupostos contrários aos da ciência. (MA/HH, §27).
\end{quotation}

É de se esperar, claro, que esse fenômeno de secularização do pensamento e de reorganização das necessidades vitais e culturais tenha impacto sobre a forma como entendemos a religião, a moral e a arte, áreas que tradicionalmente tiveram um lugar de destaque na cultura. Vem à tona, então, a preocupação sobre como uma cultura pós-metafísica – uma cultura marcadamente cética, cientificista, embalada pela agitação febril que marca o surgimento das sociedades urbanizadas e industrializadas da modernidade – será capaz de se colocar novos grandes objetivos culturais, a que Nietzsche se refere como “objetivos ecumênicos”. Essa preocupação é assim formulada por Nietzsche:

\begin{quotation}
Uma desvantagem essencial trazida pelo fim das convicções metafísicas é que o indivíduo atenta demasiadamente para seu curto período de vida e não sente maior estímulo para trabalhar em instituições duráveis, projetadas para séculos; ele próprio quer colher a fruta da árvore que planta, e portanto não gosta mais de plantar árvores que exigem um cuidado regular durante séculos, destinadas a sombrear várias sequências de gerações. Pois as convicções metafísicas levam a crer que nelas se encontra o fundamento último e definitivo sobre o qual se terá de assentar e construir todo o futuro da humanidade; o indivíduo promove sua salvação quando, por exemplo, funda uma igreja ou um mosteiro, ele acha que isto lhe será creditado e recompensado na eterna vida futura da alma, que é uma obra pela eterna salvação da alma. Pode a ciência despertar uma tal crença nos seus resultados? O fato é que ela requer a dúvida e a desconfiança, como os seus mais fiéis aliados; apesar disso, com o tempo a soma de verdades intocáveis, isto é, sobreviventes a todas as tormentas do ceticismo, a toda decomposição, pode se tornar tão grande (na dietética da saúde, por exemplo), que com base nisso haja a decisão de empreender obras 'eternas'. (…). (MA/HH, §22).
\end{quotation}

O tom do aforismo não deixa dúvidas de que Nietzsche entende que este é um momento de crise, que a formulação desses novos grandes objetivos culturais ainda não está definida. No aforismo seguinte, intitulado \textit{A era da comparação} (\textit{MA/HH}, §23), o filósofo expressa a ideia de que a Europa moderna vive uma agitação sem precedentes, em que se pode emular, algo aleatoriamente, os diversos costumes já existentes na história, em que a mobilidade geográfica dissolve o sentimento de pertença a um local e a uma tradição, de forma que o homem moderno  constantemente compara seu modo de vida com os outros tantos possíveis; segundo Nietzsche, “A intensificação do sentimento estético escolherá definitivamente entre as tantas formas que se oferecem à comparação”, e mais à frente sugere que operar uma seleção desses diversos hábitos morais é uma “tarefa” que se impõe a sua época (\textit{MA/HH}, §23). Embora Nietzsche não seja taxativo sobre quais escolhas exatamente deverão ser feitas, ele aponta alguns pontos que supostamente poderiam ser abraçados como “objetivos ecumênicos”. De imediato, há a aposta, é claro, no desenvolvimento de uma cultura científica ela mesma, e, por consequência, no desenvolvimento dos valores em geral associados à ciência (sobriedade, cautela no julgamento, algum tipo de desconfiança cética em relação às práticas e valores, simplicidade nas formas de sentir e compreender o mundo, etc.). Há ainda a aposta de que essa cultura cientificista pode levar à intensificação de um modo de vida centrado no cultivo individual, fomentado, por exemplo, em informações cada vez mais acuradas sobre a “dietética da saúde”.

Mas isto não é tudo. Quer dizer, para além da aposta na ciência como fonte de conhecimentos e atitudes que poderão engendrar um novo modo de vida na modernidade, há uma preocupação de Nietzsche com a aplicação positiva da ciência, no sentido de estabelecer planos, cientificamente controlados, para os rumos da humanidade. No aforismo 33, por exemplo, desenvolve-se a ideia de que, de um ponto de vista supra-histórico, tem-se que enfrentar a conclusão de que “no conjunto a humanidade não tem objetivo \textit{nenhum}, e por isso, considerando todo o seu percurso, o homem não pode nela encontrar consolo e apoio, mas sim desespero.” (\textit{MA/HH}, §33). A positividade prática da ciência abriria uma possibilidade sem precedentes de estabelecer e executar grandes objetivos conjuntos. A este respeito, isto é, sobre quais seriam esses objetivos, \textit{Humano} contém indicações ainda mais vagas; há vários aforismos que dão testemunho do interesse de Nietzsche num tipo de planificação positiva, mas também de seu receio quanto às possíveis consequências nefastas desse tipo de intervenção:

\begin{quotation}
Após o fim da crença de que um deus dirige os destinos do mundo e, não obstante as aparentes sinuosidades no caminho da humanidade, a conduz magnificamente à sua meta, os próprios homens devem estabelecer para si objetivos ecumênicos, que abranjam a Terra inteira. (…) Em todo caso, para que a humanidade não se destrua com um tal governo global consciente, deve-se antes obter, como critério científico para objetivos ecumênicos, \textit{um conhecimento das condições da cultura} que até agora não foi atingido. Esta é a imensa tarefa dos grandes espíritos do próximo século. (\textit{MA/HH}, §25).
\end{quotation}

Recapitulando, então, o ponto dos objetivos ecumênicos: embora o tom do livro sugira que o momento é de indefinição, cautela, preparação, Nietzsche faz uma aposta geral e uma prescrição particular a este respeito. A aposta geral é de que qualquer avanço na perseguição desses objetivos dependerá de um forte investimento na ciência, enquanto produtora de conhecimentos indispensáveis à formulação dos objetivos eles mesmos, mas também enquanto referência para o cultivo de uma certa disposição de espírito favorável à construção desses objetivos. A prescrição particular é de que o momento pode ser favorável para aqueles que buscam construir um modo de vida individual, como o tipo “espírito livre”, que poderá incorporar em sua vida pessoal tanto esses conhecimentos quanto essa atitude geral de pesquisa, cautela, sobriedade, etc.

	Mas além disto, acreditamos que Nietzsche se adianta em relação a pelo menos um objetivo específico, para o qual ensaia ele mesmo a aplicação de alguns conhecimentos científicos e seus possíveis impactos sobre a cultura – este seria o objetivo de construir formas menos vingativas e punitivas de julgar a ação. Trataremos desse “objetivo ecumênico” na próxima seção.
	
\section{A vida embebida em erro e a tese da irresponsabilidade radical}

\begin{quotation}
\textit{Questões fundamentais de metafísica.} – Quando algum dia se escrever a história da gênese do pensamento, nela também se encontrará, sob uma nova luz, a seguinte frase de um lógico eminente: “A originária lei universal do sujeito cognoscente consiste na necessidade interior de reconhecer cada objeto em si, em sua própria essência, como um objeto idêntico a si mesmo, portanto existente por si mesmo e, no fundo, sempre igual e imutável, em suma, como uma substância”. Também essa lei, aí denominada “originária”, veio a ser – um dia será mostrado como gradualmente surge essa tendência nos organismos inferiores: como os estúpidos olhos de toupeira dessas organizações vêem apenas a mesma coisa no início; como depois, ao se tornarem mais perceptíveis os diferentes estímulos de prazer e desprazer, substâncias distintas são gradualmente diferenciadas, mas cada uma com \textit{um} atributo, isto é, uma única relação com tal organismo. – O primeiro nível do [pensamento] lógico é o juízo, cuja essência consiste, segundo os melhores lógicos, na crença. Na base de toda crença está a \textit{sensação do agradável ou do doloroso} em referência ao sujeito que sente. Uma terceira e nova sensação, resultado das duas precedentes, é o juízo em sua forma inferior. – A nós, seres orgânicos, nada interessa originalmente numa coisa, exceto sua relação conosco no tocante ao prazer e à dor. Entre os momentos em que nos tornamos conscientes dessa relação, entre os estados do sentir, há os de repouso, os de não sentir: então o mundo e cada coisa não têm interesse para nós, não notamos mudança neles (como ainda hoje alguém bastante interessado em algo não nota que um outro passa ao lado). Para uma planta, todas as coisas são normalmente quietas, eternas, cada coisa igual a si mesma. Do período dos organismos inferiores o homem herdou a crença de que há \textit{coisas iguais} (só a experiência cultivada pela mais alta ciência contradiz essa tese). A crença primeira de todo ser orgânico, desde o princípio, é talvez a de que todo o restante é uno e imóvel. – Nesse primeiro nível do lógico, o pensamento da \textit{causalidade} se acha bem distante: ainda hoje acreditamos, no fundo, que todas as sensações e ações sejam atos de livre-arbítrio; quando observa a si mesmo, o indivíduo que sente considera cada sensação, cada mudança, algo \textit{isolado}, isto é, incondicionado, desconexo, que emerge de nós sem ligação com o que é anterior ou posterior. Temos fome, mas primariamente não pensamos que o organismo queira ser conservado; esta sensação parece se impor \textit{sem razão e finalidade}, ela se isola e se considera \textit{arbitrária}. Portanto, a crença na liberdade da vontade é o erro original de todo ser orgânico, de existência tão antiga quanto as agitações iniciais da lógica; a crença em substâncias incondicionadas e coisas semelhantes é também um erro original e igualmente antigo de tudo o que é orgânico. Porém, na medida em que toda a metafísica se ocupou principalmente da substância e da liberdade do querer, podemos designá-la como a ciência que trata dos erros fundamentais do homem, mas como se fossem verdades fundamentais. (\textit{MA/HH}, §18)
\end{quotation}

Esse aforismo apresenta uma argumentação sofisticada, que avança por meio de uma série de hipóteses, e é parte de uma grande narrativa sobre os processos de formação de crenças, cognição e interpretação do mundo. Essa narrativa visa, em última medida, explicar como chegamos a ter uma visão das ações como moralmente boas ou más, merecedoras de punição e recompensa. 

Encontra-se no início do texto uma citação do “lógico eminente” Afrikan Spir: “'A originária lei universal do sujeito cognoscente consiste na necessidade interior de reconhecer cada objeto em si, em sua própria essência, como um objeto idêntico a si mesmo, portanto existente por si mesmo e, no fundo, sempre igual e imutável, em suma, como uma substância'”. Segundo Mattioli, o heterodoxo neokantiano Spir, em seu livro \textit{Denken und Wirklichkeit}, “propõe uma releitura da filosofia crítica que considera como único elemento realmente \textit{a priori} do pensamento, no sentido transcendental, o \textit{princípio lógico da identidade}. Ele seria o princípio de base de organização da experiência, a partir do qual seríamos capazes de estabelecer e identificar objetos empíricos estáveis, compreendidos como substâncias, a partir dos dados sempre instáveis das sensações.” (MATTIOLI, 2011, p. 231). Nietzsche, por sua vez, propõe que se leia a tese spiriana sob “uma nova luz”, de modo que o princípio lógico da identidade não mais seria visto como um \textit{a priori} do pensamento no sentido transcendental, mas como um traço estruturante da cognição que vem à tona mediante um longo processo biológico, como um mecanismo adaptativo surgido sob a pressão vital da necessidade de se organizar em meio ao devir, mecanismo que gradativamente se complexifica ao longo da história evolutiva do orgânico.

A narrativa que se delineia no aforismo acima distingue alguns estágios da evolução do aparato biológico e psicológico a partir do princípio de identidade, e aponta o desenvolvimento dos fatores envolvidos na percepção e formação de crenças a respeito do mundo. Essa narrativa tem início com um primeiro momento anterior ao lógico, quando do surgimento dos “organismos inferiores”, para os quais o mundo em devir é indiscernível. Nesse primeiro estágio, segundo Nietzsche, “os estúpidos olhos de toupeira dessas organizações vêem apenas a mesma coisa no início”
\footnote{Talvez o exemplo se tornasse mais adequado se pensássemos não em uma toupeira, mas em algum organismo mais simples, como uma ameba.}. 
O filósofo especula que este estágio seria caracterizado pelo fato de o princípio lógico de identidade ainda não estar plenamente desenvolvido, de forma que este seria também um estágio primitivo da cognição – os “organismos inferiores” ainda não percebem diferentes coisas no mundo, e portanto não se relacionam com ele de forma lógica. No entanto, por especulação, Nietzsche atribui a esses organismos algum tipo de percepção e interpretação do mundo: mesmo que esses organismos não distingam diferentes “coisas” ao seu redor, o ambiente seria percebido como unidade estável, como “a mesma coisa”.

Num segundo momento, haveria um ganho de sensibilidade por parte dos organismos, a partir do qual “substâncias distintas são gradualmente diferenciadas”. Este estágio marcaria o surgimento do lógico, enquanto forma de percepção que distingue diferentes coisas no ambiente; nesse momento ainda rudimentar de surgimento do lógico, cada coisa é percebida por “\textit{um} atributo, isto é, uma única relação com tal organismo”, ou seja, como objeto de uma relação aprazível ou dolorosa. As sensações de prazer e dor, e por consequência, a percepção de diferentes entidades discretas, seriam o suficiente para que os organismos complexificassem o processo de formação de crenças e emissão de juízos sobre o mundo; o aforismo sugere que essas crenças primitivas diriam respeito sobretudo à identificação de entidades discretas como possíveis fontes de prazer e dor, ou seja, consistiriam em se acreditar que certa coisa (igual a si mesma, estável, durável, diferente e independente das outras coisas) se caracteriza ou não por surtir efeitos prazerosos. Os juízos mais básicos, é de se imaginar, tratariam da classificação dessas coisas como “úteis”, “perigosas”, “nocivas”, “a serem perseguidas”, “a serem evitadas”, etc., conforme sua “única relação com tal organismo”. Essa etapa de surgimento do lógico seria ainda devedora daquela primeira experiência dos organismos inferiores, em que se percebia o mundo como uma totalidade estável, de forma que o filósofo especula que: “A crença primeira de todo ser orgânico, desde o princípio, é talvez a de que todo o restante é uno e imóvel.” - pois seria justamente esse sentido de unidade e estabilidade o traço fundamental a partir do qual se identifica algo como uma coisa o traço mais básico da interpretação do mundo, traço herdado por organismos que progressivamente se tornam capazes de identificar mais e mais coisas. 

Ainda segundo a narrativa nietzschiana, a própria vida interior do organismo é interpretada como coisa, cada evento interior seria também entendido como igual a si mesmo, estável, diferente, desconexo e independente de outras coisas: “Temos fome, mas primariamente não pensamos que o organismo queira ser conservado; esta sensação parece se impor sem \textit{razão e finalidade}, ela se isola e se considera arbitrária.” O fato mesmo de cada objeto ser identificado por si mesmo, como independente dos outros objetos, teria levado à crença no caráter “arbitrário” da agência. Quer dizer, nossa percepção do mundo é estruturada pela identificação de diferentes coisas, sendo que cada coisa é vista como uma \textit{substância} autônoma, desconexa e perene, e além disto tendemos a  projetar imediatamente sobre cada coisa o aspecto agradável ou nocivo que advém de nossa relação com ela – e esta estrutura naturalmente é projetada em toda interpretação dos fenômenos internos e externos, inclusive ações humanas.

A conclusão visada e apenas sugerida pelo aforismo (que será desenvolvida ao longo do livro), é de que o arbítrio passa a ser visto como fonte unívoca e unilateral dos efeitos aprazíveis e dolorosos. Essa visão da ação como “arbitrária” seria um equívoco em relação aos processos causais, ou seja, uma incompreensão dos diversos fatores que condicionam a ação: “Nesse primeiro nível do lógico, o pensamento da \textit{causalidade} se acha bem distante: ainda hoje acreditamos, no fundo, que todas as sensações e ações sejam atos de livre-arbítrio”. A ideia de Nietzsche é que o pensamento metafísico chegou a formular uma noção de livre-arbítrio porque se ocupou desse traço fundamental da percepção sensível e do processo de formação de crença sem tomar o distanciamento necessário para que se apurasse o “pensamento da \textit{causalidade}”, mas antes com uma adesão imediata à tendência natural a emitir juízos sobre cada coisa, vista como um agente arbitrário. Essa adesão imediata às crenças, própria da metafísica, ao que parece, teria oferecido legitimação à forma reificada de compreensão do mundo. O pensamento metafísico não teria dado conta do fato de que as crenças assim formadas teriam valor meramente adaptativo, sendo que seu conteúdo propriamente cognitivo seria falso, pois “contradito” pela ciência na medida em que ela refina a imagem do devir e dos processos causais. Ao legitimar esse mecanismo adaptativo epistemicamente falho, a metafísica, diferentemente da ciência, teria se fiado excessivamente na tendência a prontamente identificarmos cada coisa como aprazível/benéfica ou dolorosa/nociva, e com isto teria pavimentado o caminho para que se julgue as coisas como metafisicamente (essencialmente) boas ou más, merecedoras de castigo e punição na medida em que supostamente agem bem ou mal com base em seu próprio livre-arbítrio.

Encontra-se aqui, portanto, uma teoria sobre a origem de nossos juízos morais, que estariam centrados na crença na liberdade da vontade. Mais adiante, se desenvolverá a ideia de que é exatamente por se fundar sobre essa concepção equivocada da vontade, sobre esse “erro fundamental”, que a interpretação moral da ação tende a emitir juízos excessivamente condenatórios, e criar formas jurídicas excessivamente vingativas. Nietzsche mobiliza grandes esforços teóricos, nos dois volumes de \textit{Humano}, com vistas a desconstruir essa visão moral da ação, o que, juntamente com a construção de práticas jurídicas menos vingativas, se delineia como um “objetivo ecumênico” a ser buscado. 

A construção de novas formas de se entender e julgar a ação conta talvez como o principal compromisso normativo de Nietzsche em \textit{Humano}, enquanto projeto de criação de uma cultura pós-metafísica. A ideia é que a modernidade herdou toda uma tradição de compreender e julgar mal a ação, conferindo-lhe aspectos metafísicos e moralizantes, mas que ela teria condições, no momento, de assumir um ponto de vista mais crítico em relação aos pressupostos dessa tradição. Segundo o discurso nietzschiano, ao construirmos formas menos metafísicas e menos moralizantes de juízo sobre a ação haveria um ganho tanto em termos de vigor cultural quanto de acuidade de compreensão. A crítica de Nietzsche a essa tradição, portanto, se apoia tanto na ideia de que ela tornou-se um atavismo cultural\footnote{Embora defenda a passagem para uma cultura pós-metafísica, Nietzsche em vários momentos entende a cultura metafísica como uma etapa no processo de humanização do animal homem, portanto como um momento importante do desenvolvimento da cultura. Cf. MA/HH II – WS/AS, §44.}, quanto na ideia de que ela é \textit{errada}, se baseia em pressupostos equivocados. 

Esta parece ser uma das principais razões do entusiasmo demonstrado em \textit{Humano} pela cultura cientificista, pois aberta a possibilidade de que a ciência seja capaz de depurar os erros em que o senso comum e a filosofia se apoiaram por tanto tempo, tem-se a chance de evitar um tipo de sofrimento visto como absolutamente desnecessário e contraproducente, isto é, o sofrimento advindo de práticas condenatórias e vingativas exacerbadas. Este parece ser também o melhor exemplo de como a informação científica pode levar a uma nova forma de filosofar, substituindo, por exemplo, o vocabulário do “livre-arbítrio” e do “pecado” por uma visão do valor adaptativo de certos comportamentos, pela história dos instintos herdados, das condições sócio-históricas em que certos “crimes” vêm à tona, etc. É por ter em vista uma nova compreensão da agência humana, e das formas de juízo que se desenvolveram sobre ela, que Nietzsche faz recuar sua narrativa até os níveis mais primários da formação de crenças e juízos em geral.

Na raiz do problema da interpretação da ação, Nietzsche aponta todo um quadro de dispositivos simplificadores e falsificadores do mundo, sendo que seu diagnóstico sugere a tese de que o orgânico, enquanto tal, se estabelece a partir do autoengano sobre si mesmo e o mundo: todo organismo teria a necessidade de criar um sentido de estabilidade para se proteger e se orientar num mundo que, na verdade, é perpassado pela instabilidade do devir, portanto, todo organismo precisaria se enganar em alguma medida sobre a radicalidade da mudança a que está sujeito. Esse sentido de estabilidade, fruto de uma necessidade adaptativa dos organismos, é apontado pelo filósofo, por vezes, como a “marca de nascença” da tendência orgânica a uma visão forçosamente simplificada do mundo; essa tese constitui o pano de fundo para as diversas ocorrências de afirmações como “Toda a vida humana está profundamente embebida em inverdade” (\textit{MA/HH}, §34), ou “o que agora chamamos de mundo é o resultado de muitos erros e fantasias que surgiram gradualmente na evolução dos seres orgânicos e cresceram entremeados” (\textit{MA/HH}, §17). O descompasso entre um mundo em devir e a necessidade biológica de se \textit{organizar} teria levado os organismos à necessidade de ver o mundo, algo ficticiamente, como entidades estáveis. E essa necessidade de \textit{coisificação} do mundo é apontada por Nietzsche como a “origem baixa” de todo nosso aparato cognitivo. A partir desse erro original, estaríamos sujeitos a nos enganar sistematicamente, inclusive a respeito da ação. 

A tentativa de reformar nosso modo de compreender e julgar a ação, portanto, dependeria de um trabalho crítico sobre os traços mais básicos de nossa organização psicofísica, condicionantes dos processos de formação de crenças. Tendo isto em vista, percebe-se logo o que há de “ecumênico” nesse objetivo de combate à visão moral da ação. Este é um ponto, a um tempo, específico e imenso. Como combater uma visão que se apoia nos dispositivos mais entranhados da nossa percepção? E se essa visão é tão entranhada, com referência a quê se poderia dizer que ela é falsa, que é um “erro fundamental”? Quais são esses erros? Sob que ponto de vista o filósofo pode considerá-los como “erros”? E como a ciência seria capaz de nos libertar deles?

Vale apontar que, embora haja em \textit{Humano} todo um elogio ao conhecimento empírico, nele não se encontra qualquer endosso à ideia, de ascendência empiricista, segundo a qual a mente humana seria, a princípio, uma \textit{tabula rasa}. Para Nietzsche, sempre vemos o mundo segundo formas herdadas, que antecedem e condicionam nossa experiência. Mas uma vez que, justamente, é a história empírica, biológica e cultural dos indivíduos e, em última instância, da espécie, o que delinearia os traços da nossa percepção de mundo, esses traços não são vistos por Nietzsche como estritamente \textit{a priori}, como traços universais e necessários a todo ser racional. É significativo que o aforismo em foco (\textit{MA/HH}, §18) tenha início com uma citação de Afrikan Spir, e portanto com um recurso à tradição transcendental. Logo em seguida, no entanto, Nietzsche marca a diferença de sua filosofia, ao entender que também a “lei transcendental” de que fala Spir veio a ser através de um processo empírico de evolução dos organismos. Se Nietzsche tivesse optado por seguir exclusivamente a esteira da filosofia transcendental, o tema do erro dificilmente emergiria de forma tão conspícua em sua filosofia, pois, no interior dessa tradição, os traços estruturantes do pensamento são tidos por universais e necessários, o que garante sua legitimidade e confiabilidade. Uma outra opção seria se filiar estritamente às teses de evolucionistas como Herbert Spencer, para quem o sucesso adaptativo desses traços é uma prova de seu acerto epistêmico; segundo Spencer, portanto, seria legítimo supor que, se os organismos se conservaram por ver o mundo como um apanhado de unidades estáveis, o mundo de fato deve ser estruturado assim. 

Por outro lado, é notável o tom de desconfiança de Nietzsche em relação às formas de representação de que dispomos: elas seriam “hábitos” de que, ele espera, a ciência pode talvez nos “libertar”; a afirmação de que a libertação total desses hábitos “não seria desejável” sugere, é claro, que se trata de hábitos extremamente incorporados, que foram tecidos nos níveis mais básicos de nossa constituição psicofísica, cumprindo uma função vital. Como não poderia deixar de ser, a própria ciência opera no interior do domínio desses hábitos. Esses “hábitos ancestrais” ou “erros fundamentais”, isto é, as formas mais básicas empregadas na percepção e conceitualização que fazemos do mundo seriam: unidade, identidade, substância, liberdade da vontade – sendo que elas estariam intimamente ligadas entre si, num arranjo responsável pela pintura que fazemos do mundo. 

	Nietzsche deixa clara sua posição em relação a esses artigos de fé: por mais úteis que eles tenham sido, por mais fundamental que seja o papel que eles desempenham na organização de nossa relação com o mundo, não há qualquer garantia de que eles sejam traços necessários em sentido absoluto, isto é, podemos imaginar a possibilidade de um processo evolutivo diferente, em que outros traços exerceriam funções estruturantes do orgânico. Acima de tudo, para Nietzsche, não há qualquer garantia de que esses traços sejam cognitivamente avalizados. 
	
As reiteradas afirmações de que a pintura que fazemos do mundo se apoia em “erros” tem dividido comentadores, que buscam entender exatamente \textit{o que} é falsificado pelas nossas representações, e \textit{por que} Nietzsche parece estar tão seguro do caráter enganador dessas representações. Grande parte dos comentadores entende que só faz sentido falar em \textit{falsificação} num quadro conceitual que envolva de alguma maneira a noção de coisa-em-si, isto é, para estes intérpretes, é porque a coisa-em-si permanece como reserva de realidade intocada pelas nossas percepções e juízos que estes estariam fadados a carecer de objetividade última, enfim, nossas percepções e juízos seriam falsos pela impossibilidade de corresponderem à coisa-em-si; esta linha de interpretação se construiu em grande medida na esteira do livro de Maudemarie Clark, \textit{Nietzsche on Truth and Philosophy} (1990). 

	Clark mobiliza um arsenal conceitual extremamente intrincado, tomado de empréstimo à filosofia analítica, para defender a tese de que a chamada “teoria do erro” nietzschiana só se justifica por um compromisso filosófico com a noção de coisa-em-si, sendo que o filósofo teria  abdicado desse compromisso em suas obras de “maturidade” (a partir de \textit{Zaratustra}). A intérprete chega mesmo a postular a ideia de que Nietzsche teria incorrido num incrível lapso filosófico, pois teria rechaçado o conceito de coisa-em-si como autocontraditório por volta de 1886 – quando da redação de \textit{Além de Bem e Mal} e do quinto e último livro de \textit{A Gaia Ciência} – mas no entanto não teria se dado conta imediatamente da consequência filosófica da rejeição desse conceito. A consequência esperada por Clark, o abandono da teoria do erro, só teria sido percebida por Nietzsche a partir da redação de \textit{Genealogia da Moral}, em 1887.
	
Embora o livro de Clark tenha sido importante por abrir toda uma seara de interpretações nietzschianas, não tardou muito até que vários autores levantassem contundentes objeções à sua tese. No estado atual da discussão, chegou-se à conclusão mais ou menos consensual de que não está claro que Nietzsche tenha reorganizado sua filosofia a partir de uma suposta rejeição do conceito de coisa-em-si, uma vez que tal mudança não é anunciada, se é que ela é anunciada, com o alarde que Nietzsche geralmente imprime a suas inovações conceituais. Como mencionamos acima, nós próprios estamos convencidos de que Nietzsche já aponta a inocuidade do conceito de coisa-em-si em \textit{HH}, de forma que não faz sentido supor que este conceito (visto como inócuo) seria a principal referência no tratamento dado ao tema do erro, tão importante no livro
\footnote{O tema da inocuidade do conceito de coisa-em-si, e seu estatuto problemático na filosofia transcendental é recorrente entre os filósofos neokantianos a que Nietzsche teve acesso, como notado por George Stack: “Kant cannot legitimately refer to the existence of things-in-themselves on the basis of his own principles because this would be to claim a transcendental, negative 'knowledge' that is prohibited by his general theory. Even as a limit-concept or \textit{Grenzbegriff} the thing-in-itself is refined away into a vanishing unreality. Lange believes that the conceptual dissolution of the idea of things-in-themselves leads to the granting of a gain in reality to phenomena, since a phenomenon  'embraces everything that we can call 'real''.” (STACK, 2005, p. 23.)}. 
Essa suposição atribuiria ao filósofo uma contradição performativa grotesca. Além disto e mais importante, foram apontadas várias evidências textuais de que as noções de “erro”, “falsificação”, etc., continuam a desempenhar um papel operatório importante na filosofia do último Nietzsche (especialmente em \textit{Crepúsculo dos Ídolos}, que data do último ano de atividade do filósofo, 1888)
\footnote{Cf. HAN-PILE (2011); HUSSAIN (2004); BORNEDAL (2010); ACAMPORA (2006).}. 
Insistimos ainda que já em \textit{Humano} há várias passagens que sugerem que a distinção fenômeno/coisa-em-si, ela mesma, pode ser incluída na lista dos “erros fundamentais” em que se apoia o pensamento – e se o conceito mesmo de coisa-em-si pode ser um erro, que sentido haveria em dizer que todo erro se deve a uma falha de correspondência com a coisa-em-si?
\footnote{“Talvez reconheçamos então que a coisa em si é digna de uma gargalhada homérica: que ela \textit{parecia} ser tanto, até mesmo tudo, e na realidade está vazia, vazia de significado.” (MA/HH, §16). Aliás, a crítica ao conceito de “incondicionado” que aparece em MA/HH, §18, citado acima, se estenderia a sua aplicação nas teorias do livre-arbítrio, lembrando que “incondicionado” é o termo usado por Afrikan Spir para se referir à coisa-em-si. Para Kant, é justamente porque a coisa-em-si, enquanto oposto do mundo fenomênico, seria incondicionada, que ela poderia abarcar a possibilidade da liberdade. Essa consequência normativa do uso kantiano do conceito de coisa-em-si é seguramente uma das fortes razões para que Nietzsche busque desenvolver sua filosofia em um outro quadro conceitual. Ver ainda, Fragmento Póstumo de quando da redação de \textit{HH}: “Während Schopenhauer von der Welt der Erscheinung aussagt, dass sie in ihren Schriftzügen das Wesen des Dinges an sich zu erkennen gebe, haben strengere Logiker jeden Zusammenhang zwischen dem Unbedingten, der metaphysischen Welt und der uns bekannten Welt geleugnet: so dass in der Erscheinung eben durchaus nicht das Ding an sich erschiene. Von beiden Seiten scheint mir übersehen, dass es verschiedne irrthümliche Grundauffassungen des Intellectes sind, welche den Grund abgeben, weshalb Ding an sich und Erscheinung in einem unausfüllbaren Gegensatz zu stehen scheinen: wir haben die Erscheinung eben mit Irrthümern so umsponnen, ja sie so mit ihnen durchwebt, dass niemand mehr die Erscheinungswelt von ihnen getrennt denken kann. Also: die üblen, von Anfang an vererbten unlogischen Gewohnheiten des Intellectes haben erst die ganze Kluft zwischen Ding an sich und Erscheinung aufgerissen; diese Kluft besteht nur insofern unser Intellect und seine Irrthümer bestehen. Schopenhauer hinwiederum hat alle characteristischen Züge unserer Welt der Erscheinung — d.h. der aus intellectuellen Irrthümern herausgesponnenen und uns angeerbten Vorstellung von der Welt — zusammengelesen und statt den Intellect als Schuldigen anzuklagen, das Wesen der Dinge als Ursache dieses thatsächlichen Weltcharacters angeschuldigt. — Mit beiden Auffassungen wird eine Entstehungsgeschichte des Denkens in entscheidender Weise fertig werden: deren Resultat vielleicht auf diesen Satz hinauslaufen dürfte: das was wir jetzt die Welt nennen, ist das Resultat einer Menge von Irrthümern welche in der gesammten Entwicklung der organischen Wesen allmählich entstanden, in einander verwachsen und uns jetzt als aufgesammelter Schatz der ganzen Vergangenheit vererbt werden. Von dieser Welt als Vorstellung vermag uns die strenge Wissenschaft thatsächlich nur in geringem Maasse zu lösen, insofern sie die Gewalt uralter Gewohnheiten nicht zu brechen vermag: aber sie kann die Geschichte der Entstehung dieser Welt als Vorstellung aufhellen.” FP, 23[125], 1876.}.
Enfraquecida a interpretação sugerida por Clark, permanece a questão de se criarem outras bases para a compreensão da assim chamada teoria nietzschiana do erro.

A alternativa mais recorrente consiste em se atribuir algum tipo de sensualismo à filosofia nietzschiana. Segundo esta leitura, Nietzsche, aproximando-se de uma ideia expressa no \textit{Teeteto}, entenderia que os sentidos nunca enganam, de forma que a possibilidade de erro recai inteiramente nos juízos que fazemos a partir das percepções sensoriais. Portanto, nesta leitura a teoria do erro se reduziria à ideia de que as percepções sensoriais são falsificadas pelos juízos, ou pela linguagem. Por este viés, a teoria do erro pode ser discutida independentemente do suposto compromisso de Nietzsche com o conceito de coisa-em-si. Nadeem Hussain, a seu modo, é um dos representantes desta linha de interpretação, que se apoia numa ginástica conceitual quase tão extenuante quanto aquela realizada por Clark, mas que parece ter levado, sim, a resultados mais plausíveis que a opção concorrente
\footnote{Cf. HUSSAIN (2004).}. 
Há ainda a contribuição bastante sóbria de intérpretes como George Stack, que têm apontado diversas passagens do texto nietzschiano em que se encontra a ideia de que há um processo simplificador envolvido já na percepção sensível, uma vez que o filósofo sugere que haveria uma pré-seleção inconsciente dos diversos estímulos a que o sujeito se expõe na experiência ordinária, de forma que apenas uma pequena parcela deles chegaria a se tornar uma sensação
\footnote{“Nietzsche argues that the sense-organs are themselves 'organs of abstraction' since the senses select out of a chaos of impressions what is of interest to us or what our senses are capable of responding to because of their limited range. What he says about the function of our sensory system points in the direction of the contemporary theory of the senses as transducers that transform the stimuli by which they are affected.” STACK, 2005, p. 27.}. 
Para não nos desviarmos demais do foco de nossa discussão, deixaremos de nos aprofundar nos detalhes do debate entre os comentadores e nos ateremos a uma interpretação da questão como exposta em \textit{Humano}
\footnote{Um livro que, aliás, tem sido contemplado de modo algo ligeiro nesses debates.}.

Nossa sugestão de interpretação consiste em entender as diversas ocorrências de afirmações “falsificacionistas” em \textit{Humano} como parte de uma narrativa sobre história biológica e cultural. O ponto crucial dessa narrativa é que ela aponta para a tese de que todo nosso aparato perceptivo, intelectual e conceitual apenas \textit{secundariamente} se aplica a fins cognitivos, isto é, os sentidos fisiológicos, os instrumentos e práticas intelectuais de que dispomos chegaram até nós ao longo de um processo evolutivo, tendo sido selecionados por fatores fundamentalmente pragmáticos, utilitários e adaptativos
\footnote{Cf. STACK 2005, p. 9: “The \textit{naiveté} of some thinkers, he remarks, was their failure to see that our senses and our 'categories of reason' involved 'the adjustment of the world for utilitarian ends.' They mistakenly believed that they possessed a 'criterion' of 'truth' and 'reality.' That is, they tended to 'make absolute something conditioned.' In effect, an 'anthropocentric idiosyncrasy' was taken as the measure of all things.” Mais adiante, o autor conclui de forma sintética: “The essence of things is unknown to us not because it is hidden behind the veil of the thing-in-itself, but because of the sensory, linguistic, and conceptual transformational activity of man's 'organization' and the evolved, inherited cognitive framework that serves his survival.” P. 27.}. 
Caso esse aparato perceptivo, intelectual e conceitual se destinasse a fins exclusivamente cognitivos, não haveria problema algum em corrigi-lo através de uma abordagem científica; neste caso, não faria sentido sugerir que essa libertação seria “indesejável”, como na passagem citada anteriormente: “Desse mundo da \textit{representação}, somente em pequena medida a ciência rigorosa pode nos libertar – algo que também não seria desejável –. desde que é incapaz de romper de modo essencial o domínio de hábitos ancestrais de \textit{sentimento}” (MA/HH, §16). Acrescentamos o itálico para destacar o fato de que tanto \textit{representação} (\textit{Vorstellung}) quanto \textit{sentimento} (\textit{Empfindung}) são mencionados como fatores envolvidos nos “equívocos” a que estamos expostos, e que são objeto da ciência. O que conta, aqui, é a capacidade de esses “artigos de fé”, por assim dizer, tornarem a vida mais vivível, e não sua capacidade de informar objetivamente sobre a realidade – razão por quê uma libertação completa desses artigos de fé seria impraticável.

O alvo não nomeado dessa discussão é o evolucionista Herbert Spencer
\footnote{Restam dúvidas de que Nietzsche tenha lido as obras de Spencer quando da composição de \textit{Humano}; de toda forma, nesta época ele já havia tido contato indireto com as ideias do evolucionista inglês via Lange e Spir. Não há dúvidas de que Nietzsche leu Spencer após \textit{Humano}. Cf. LOPES, 2008, p. 291, nota 244.}. 
Apesar da familiaridade com o vocabulário evolucionista spenceriano, aplicado com generosidade em \textit{Humano}, tanto no tratamento de questões mais gerais quanto de fenômenos pontuais, Nietzsche nega uma conclusão importante do pensamento spenceriano: diferentemente de Spencer, Nietzsche entende que a utilidade adaptativa de uma crença não vale como índice de acerto epistêmico. Sua conclusão: “a crença forte prova apenas a sua força, não a verdade daquilo em que se crê.” (MA/HH, §15). Ao longo de praticamente toda a obra de Nietzsche, reaparece a ideia de que uma crença equivocada pode perfeitamente exercer uma função adaptativa, e se tornar um aspecto vital de certas estruturas biológicas e sociais. Aliás, segundo o diagnóstico nietzschiano, este seria o caso da maioria de nossas crenças.

Nietzsche sugere que é apenas num estágio muito recente dessa história evolutiva que se passou a criticar o valor epistêmico desses artigos de fé, sendo que o investimento nesse aspecto propriamente cognitivo de nossa relação com o mundo cumpriria ainda um papel menor na nossa estruturação: 

\begin{quotation}
“Apenas homens muito ingênuos podem acreditar que a natureza humana pode ser transformada numa natureza puramente lógica; mas, se houvesse graus de aproximação a essa meta, o que não se haveria de perder nesse caminho! Mesmo o homem mais racional precisa, de tempo em tempo, novamente de natureza, isto é, de sua \textit{ilógica relação fundamental com todas as coisas}.” (\textit{MA/HH}, §31)\footnote{Com \textit{Humano}, tem início uma narrativa que só se desenvolverá completamente na \textit{Genealogia da Moral}, segundo a qual a exigência de \textit{veracidade}, a exigência de não mentir, que a princípio cumpria um papel meramente gerencial nas comunidades humanas transforma-se, em certo momento de nossa história, num interesse genuíno pela verdade; isto teria levado a um reinvestimento em todo nosso aparato perceptivo e intelectual, que então volta-se criticamente contra si mesmo e é \textit{exaptado} para fins de pesquisa da verdade. A aplicação do conceito de \textit{exaptação} na interpretação da obra de Nietzsche é sugerida por RICHARDSON (2004); o conceito é tomado de empréstimo da biologia, e tem o sentido de conferir a uma estrutura uma função outra que aquela para a qual essa estrutura foi evolutivamente selecionada.}.
\end{quotation}

Somente se mantivermos em vista a ideia de que estamos sujeitos a nos enganar tanto por instinto, por paixão, como por hábito social, quanto pela aplicação dos conceitos mais arraigados é que, nos parece, podemos dar conta do tema do erro tal qual ele aparece em \textit{HH}. Se somos convencidos pela argumentação nietzschiana de que o homem, assim como outros organismos, se relaciona de forma fundamentalmente ilógica com todas as coisas, passamos da pergunta “de onde vem o erro?” à pergunta “como, então, é possível algum acerto?”. 

	Consideremos, por exemplo esta ocorrência: “Nossas sensações de espaço e tempo são falsas, porque, examinadas consistentemente, levam a contradições lógicas” (\textit{MA/HH}, §19). Já é possível entrever, por esta passagem, o método pelo qual se sugere que esses erros sejam trazidos à luz: através de um exame que aponte as inconsistências internas às nossas experiências, em geral, mediante o qual esses erros são levados a uma espécie de curto-circuito. Se a ciência pode ver esses “erros” como erros, isto se deve à sua capacidade de apontar tais inconsistências, mas não implica em que ela tenha uma forma de acesso privilegiado à verdade, lembrando que ela “é incapaz de romper de modo essencial o domínio de hábitos ancestrais” (MA/HH, §16). Ou seja, não há uma diferença de natureza entre a forma como a ciência compreende o mundo e como o compreendemos pelo senso comum, por exemplo; a diferença da ciência consistiria num maior trabalho crítico sobre os pressupostos e condições que atuam sobre a produção de conhecimento, de forma que com isto ela poderia, sim, evitar alguns dos “erros” em que incorrem algumas dessas outras formas de conhecimento.
	
Poderíamos recorrer a um exemplo contemporâneo para ilustrar como isto é feito. Tomemos uma experiência que circula pelos canais televisivos de divulgação científica, que tem o sentido de mostrar como somos enganados por truques de mágica: o experimento consiste na apresentação de um vídeo em que uma bolinha de plástico parece sumir da mão direita do mágico e instantaneamente se materializar na mão esquerda; em seguida, somos expostos ao mesmo vídeo, só que desta vez o rosto do mágico é cortado do enquadramento, e já que não mais nos distraímos com a figura do rosto, podemos enxergar claramente como o mágico passa a bolinha de uma mão para a outra. Segundo a narrativa científica, a capacidade de identificar rostos humanos teve um papel fundamental na evolução de seres sociais como nós, de forma que a mera presença de um rosto pode nos impedir de perceber outros objetos ou acontecimentos, menores nesse sentido, por mais que eles estejam bem ali à nossa frente. Vejamos se não é isto que está em jogo, por exemplo, neste trecho da passagem focal:

\begin{quotation}
(…) Na base de toda crença está a sensação do \textit{agradável} ou \textit{doloroso} em referência ao sujeito que sente. Uma terceira e nova sensação, resultado das duas precedentes, é o juízo em sua forma inferior. - A nós, seres orgânicos, nada interessa originalmente numa coisa, exceto sua relação conosco no tocante ao prazer e à dor. Entre os momentos em que nos tornamos conscientes dessa relação, entre os estados do sentir, há os de repouso, os de não sentir: então o mundo e cada coisa não tem interesse para nós, não notamos mudança neles (como ainda hoje alguém bastante interessado em algo não nota que um outro passa ao lado). (\textit{MA/HH}, §18).
\end{quotation}

Ainda sobre o tema do erro, é relevante notar que o filósofo emprega uma outra linha de argumentação, que não deixa de ser bastante retórica, e que consiste em levantar suspeitas sistemáticas – pode-se dizer, hiperbólicas\footnote{Incluir consideração sobre Han-Pile} – contra todas as crenças mais fortemente incorporadas. Este é um traço importante da forma nietzschiana de filosofar, que tem sido lembrada, com propriedade, como uma “escola da suspeita”. Trata-se de algo como uma argumentação reversa, que toma como ponto de partida a suspeita de que se somos constantemente constrangidos – por nossa constituição física, por hábitos, pela aplicação de categorias, e todo tipo de convenção social – a pensar por identidade, estabilidade, liberdade da vontade, etc., isto provavelmente deve ser falso. Daí a aposta na realidade do devir radical.

Lembremos mais uma vez que Nietzsche acreditava que essa sua intuição, suspeita ou aposta na realidade do devir era corroborada pelos resultados das ciências de sua época: tanto pela visada evolucionista em Biologia – que impediu definitivamente que se pensasse a natureza como algo essencialmente acabado e imutável – quanto pelas inovações no campo da Física, que vieram à tona com a crise do modelo newtoniano. Neste caso, referimo-nos à crise do modelo de descrição do movimento por impacto, representado pelo conhecido modelo das bolas de sinuca, segundo o qual cada átomo material, ao entrar em contato com outro átomo, transmitiria a este seu movimento; o modelo entra em crise devido a sua insuficiência para explicar fenômenos associados à gravidade, em que parece haver algo como “ação à distância”
\footnote{“Lange atribui a gradual revisão do conceito de “átomo” tal como estabelecido pelos antigos à atuação conjunta de duas tendências modernas: 1) o método crítico na filosofia; 2) o método experimental nas ciências. A contribuição da ciência teria sido entretanto mais decisiva para a superação de uma concepção dogmática da atomística. Com o estabelecimento por Newton da lei da gravitação tornou-se imprescindível a admissão de uma ação à distância, com o que a mecânica clássica precisou abrir mão do caráter intuitivo ligado ao modelo atomístico herdado da antiguidade. O aspecto contraditório contido na suposição de uma ação à distância não foi com isso contornado, mas simplesmente ignorado, pois se percebeu que ele não representava qualquer entrave ao progresso da pesquisa. Com isso se tornou claro que a ciência poderia prescindir para seu progresso de um fundamento último, podendo se contentar com a mera postulação de um ponto fixo. Cf. LANGE, 1866, p. 360.” LOPES (2008), p.72.}. 
Essa crise do modelo newtoniano abriu espaço para que muitas outras hipóteses fossem avançadas no campo da Física, e uma delas recebeu bastante atenção por parte de nosso filósofo e surtiu efeitos em sua filosofia: trata-se do modelo do físico R. J. Boscovich, que propôs a substituição do conceito de “matéria” pelo conceito de “força”, de forma que os fenômenos naturais não mais seriam pensados como ação de unidades substanciais, mas como expressão de pontos de força não-extensos de existência temporal
\footnote{Cf. STACK, 2005, p. 14. Stack aponta que Nietzsche entrou em contato com as teorias de Boscovich no início da década de 1870.}.

O modelo de Boscovich teria reforçado a ideia de que o conceito mesmo de “substância” ou “matéria” carece de contrapartida na realidade, sendo portanto merecedor do status de “artigo de fé” ou “erro fundamental” em que se apoiou o pensamento devido a uma necessidade biológica, como temos lembrado, de forma que, nesse modelo, Nietzsche encontrou subsídios científicos para dissolver a imagem material do mundo em puro devir.

	Esta linha de pensamento teve grande apelo para o jovem filósofo, e foi por ele revisitada ao longo de toda sua atividade intelectual. Uma implicação filosófica importante, para Nietzsche, é que com a derrocada do materialismo atomista, cai por terra também o chamado “atomismo do espírito”, isto é, a ideia de Sujeito como algo material, substancial, identitário, com a qual operam as filosofias do livre-arbítrio. 
	
	Este é um trunfo importante para a sugestão do filósofo de que nós teríamos condições de dispensar a concepção material de sujeito, e com ela todo um modelo de interpretação da ação humana embasada na noção de livre-arbítrio, por mais que essas noções estejam profundamente imbuídas em nossa constituição psicofísica e cultural, tendo se tornado bastante intuitiva. A argumentação do filósofo ressalta o fato surpreendente de como a ciência, operando no interior das formas mais primitivas, mais fortemente incorporadas, de cognição do mundo, conseguiu chegar a uma visão de mundo absolutamente contraintuitiva, uma visão “desumanizada”\footnote{Cf. STACK, 2005, p. 16.}.
	
Nietzsche reconhece, portanto, que a pesquisa científica opera no interior dos “erros fundamentais” sem os quais não existiria a própria lógica – que pressupõe objetos estáveis, idênticos a si mesmos, numericamente diferentes de outros objetos, ao invés de um mundo radicalmente em devir. Neste sentido, a ciência seria responsável, num primeiro momento, por um movimento de “humanização” do mundo, isto é, por uma projeção completa e autoconsciente, dos traços estruturantes da cognição humana sobre o mundo. Mas, diferentemente do senso comum, o pensamento científico procede por estrito controle dos pressupostos e variáveis envolvidos no processo de conhecimento, e num distanciamento que permite vê-los como meros instrumentos mais ou menos ficcionais com os quais pode-se refutar algumas crenças mal estabelecidas:

\begin{quotation}
Em todas as constatações científicas, calculamos inevitavelmente com algumas grandezas falsas: mas sendo tais grandezas no mínimo \textit{constantes}, por exemplo, nossa sensação de tempo e de espaço, os resultados da ciência adquirem perfeito rigor e segurança nas suas relações mútuas; podemos continuar a construir em cima delas – até o fim derradeiro em que a hipótese fundamental errônea, os erros constantes, entram em contradição com os resultados, por exemplo, na teoria atômica. Então ainda nos sentimos obrigados a supor uma “coisa” ou “substrato” material que é movido, enquanto todo o procedimento científico perseguiu justamente a tarefa de dissolver em movimento tudo o que tem natureza de coisa (de matéria) (…). (MA/HH, §19).
\end{quotation}

Entende-se, portanto, que essa atividade científica de esquematização do mundo através das \textit{constantes} do pensamento humano tende a levar a resultados que apontam as inconsistências internas a tais constantes, e assim se abrem pequenas brechas através das quais se delineia um mundo \textit{desumanizado}, isto é, um mundo no qual, a rigor, não estaria incluída a existência de “coisas”, “materiais”, “substâncias”, e todos esses artigos que cumprem um papel fundamental na forma como nos expressamos ordinariamente.

	Conquanto o filósofo tenha grande interesse por tais resultados da pesquisa científica de sua época, por favorecerem uma visão dinâmica de mundo e fortalecerem sua crítica à moral do livre-arbítrio, ele entende que, mais do que os próprios resultados, a filosofia se beneficia acima de tudo pelo contato com o \textit{método} científico em si, seu potencial de disciplinarização dos impulsos cognitivos e de estabelecer uma atitude de espírito condizente com a necessidade de constante reinterpretação do mundo
\footnote{As seções finais do primeiro volume de \textit{Humano} (parágrafos 629-638) recapitulam o tema da consciência metódica desenvolvida pela ciência, e seus possíveis efeitos sobre a cultura, apresentando uma argumentação densa e muito rica. Trataremos dessas seções mais adiante.}. 
Antes de mais nada, seria essa consciência metódica a peça fundamental para se enfrentar o desafio de lidar com uma visão de mundo em devir, e superar a visão da ação como produto do livre-arbítrio.

No caso da avaliação das ações, essa consciência metódica incentivaria a consideração pelo contexto em que se insere o agente, a reconstituição cautelosa das forças que atuam sobre suas possibilidades de ação. Por fim, um juízo emitido após esse processo de investigação, metodicamente rigorosa, não seria a expressão de uma “convicção”, entendida como “a crença de estar, em algum ponto do conhecimento, de posse da verdade absoluta.” (\textit{MA/HH}, §630), mas seria a expressão de uma “interpretação” ciente de seus pressupostos e limites, e assim se poderia renunciar a um fundamento último para as práticas condenatórias e premiadoras. Mas este tipo de renúncia, é claro, representa um desafio tão radical quanto renunciar a pensar o mundo por identidade.

O mundo como devir seria o inverso do mundo pensado como o reino de substâncias identitárias. Pensar o mundo através de um modelo de acontecimentos contínuos em pleno devir seria, de certa forma, uma tarefa sobre-humana, uma dificuldade que tem implicações, também, na nossa quase incapacidade de pensar a ação de outra forma que não seja como um fato isolado, o que, como foi dito, está na origem de nossa crença no livre-arbítrio e fundamenta nossa forma de avaliá-las. Para Nietzsche, mesmo que sejamos incapazes de representar a totalidade do devir, devemos mantê-lo em vista como horizonte de pesquisa, e buscar formas de compreensão mais dinâmicas e contextualizadas. Mesmo que seja impossível levar em consideração a totalidade das forças atuantes no mundo em devir, mesmo que essas forças sejam demasiado instáveis para que possamos tratá-las pelos meios idenditários com que o pensamento normalmente opera, Nietzsche sugere que podemos nos esforçar para manter em vista o caráter processual dos eventos e, por assim dizer, aos poucos refinar a imagem das diversas causas que se expressam numa ação.

	Segundo essa visão, as ações seriam remetidas ao pano de fundo de todo um processo causal que em última instância se refere à história evolutiva dos organismos, considerada numa perspectiva de longa duração. 
	
	Portanto, se chegamos a ver as ações como resultado (e parte) de um processo que de forma alguma é transparente para nós, como quer Nietzsche, abala-se a credibilidade de um modelo de avaliação da ação centrada num sujeito do livre-arbítrio, metafisicamente compreendido. Para desenvolver sua contraposição à moral do livre-arbítrio, Nietzsche encontra subsídios nas teorias de utilitaristas ingleses como John Stuart Mill e seu sucessor Herbert Spencer, autores que em alguma medida antecipam aspectos do pensamento darwinista, e que têm entre seus precursores o naturalista David Hume, por exemplo. À época de Nietzsche, as discussões desses pensadores eram uma alternativa de teoria moral não teocêntrica disponível. Seu vocabulário deflacionado, centrado nos conceitos de prazer, dor e conservação da vida, combinado à sua incorporação de modelos científicos de explicação retirados em linhas gerais da biologia evolucionista, fazia com que os moralistas ingleses se apresentassem como praticamente a única aliança possível nessa tarefa de naturalização da moral. O que essa tradição de pensadores da moral tinha a oferecer a Nietzsche, portanto, eram elementos teóricos que não remetiam a noções de Deus, autotransparência do sujeito, ou livre-arbítrio, (e, portanto, não recorriam a termos morais para explicar a moral) mas que partiam da concepção de que, como os outros animais, o homem age com vistas à conservação da vida, que nessa busca ele se orienta por uma série de instintos e hábitos dos quais raramente se torna consciente, e pauta suas ações pela possibilidade de aumentar o prazer e evitar a dor.
	
Esse quadro conceitual utilitarista – operando com noções de conservação da vida, hábito e prazer/desprazer – permite a Nietzsche desenvolver um modelo explicativo que atenderia aos princípios naturalistas de simplicidade e economia. Com essas noções simples, reduzidas, e empiricamente acessíveis, se poderia abordar uma vasta gama de fenômenos relacionados à ação humana, de forma que o modelo cumpre satisfatoriamente as exigências metódicas. Segundo esse modelo, o efeito aprazível, por exemplo, conta como motivação inicial da ação, e após um processo de \textit{coerção} reaparece como reforço de um comportamento que já se tornou habitual. Uma vez que os organismos buscam naturalmente a conservação da vida, e que o prazer pode ser visto como um índice de condições propícias à conservação, os organismos estariam legitimados em buscar o prazer, mesmo que nessa busca estejam sujeitos a erros de cálculo.

	Ao invés de se pensar a ação como resultado da deliberação racional de um sujeito constante, substancialmente idêntico a si mesmo, e autotransparente, a proposta é pensá-la como expressão de um organismo constantemente premido por diversas necessidades que ele mesmo conhece apenas parcialmente, assim como apenas parcialmente é capaz de calcular as chances de se envolver num conflito com outro organismo, ou calcular a eficácia do meio pelo qual pretende atingir sua meta. Na busca por satisfação dessas necessidades, o organismo se fia em suas próprias sensações de prazer e dor, e num mecanismo falível de cálculo racional. Essa narrativa biologicizante sobre a ação humana visa o efeito de reduzir o apelo das condenações morais, e com isto contornar uma certa cultura jurídica que, segundo o filósofo, seria marcada pela “exaltação” do sentido de vingança e punição. Veja-se, por exemplo, este aforismo intitulado \textit{O que há de inocente nas más ações}:
	
\begin{quotation}
Todas as “más” ações são motivadas pelo impulso de conservação ou, mais exatamente, pelo propósito individual de buscar o prazer e evitar o desprazer; são, assim, motivadas, mas não são más. “Causar dor em si” \textit{não existe}, salvo no cérebro dos filósofos, e tampouco “causar prazer em si” (compaixão no sentido schopenhaueriano). Na condição \textit{anterior} ao Estado, matamos o ser, homem ou macaco, que queira antes de nós apanhar uma fruta da árvore, quando temos fome e corremos para a árvore: como ainda hoje faríamos com um animal, ao andar por regiões inóspitas. - As más ações que atualmente mais nos indignam baseiam-se no erro de [imaginar] que o homem que as comete tem livre-arbítrio, ou seja, de que dependeria do seu bel-prazer não nos fazer esse mal. Esta crença no \textit{bel-prazer} suscita o ódio, o desejo de vingança, a perfídia, toda a deterioração da fantasia, ao passo que nos irritamos muito menos com um animal, por considerá-lo irresponsável. Causar sofrimento não pelo impulso de conservação, mas por represália – é consequência de um juízo errado, e por isso também inocente. O indivíduo pode, na condição que precede o Estado, tratar outros seres de maneira dura e cruel, visando \textit{intimidá-los}: para garantir sua existência, através de provas intimidantes de seu poder. Assim age o homem violento, o poderoso, o fundador original do Estado, que subjuga os mais fracos. Tem o direito de fazê-lo, como ainda hoje o Estado o possui; ou melhor: não há direito que possa impedir que o faça. Só então pode ser preparado o terreno para toda a moralidade, quando um indivíduo maior ou um indivíduo coletivo, como a sociedade, o Estado, submete os indivíduos, retirando-os de seu isolamento e os reunindo em associação. A moralidade é antecedida pela coerção, e ela mesma é ainda por algum tempo coerção à qual a pessoa se acomoda para evitar o desprazer. Depois ela se torna costume, mais tarde obediência livre, e finalmente quase instinto: então, como tudo o que há muito tempo é habitual e natural, acha-se ligada ao prazer – e se chama \textit{virtude}. (MA/HH, §99)
\end{quotation}

O ponto aqui é que, por entender mal os mecanismos da ação, o ocidente acabou por criar uma cultura “do ódio”, “da perfídia”, da “deterioração da fantasia” – sentimentos que nesse momento o filósofo considera como uma espécie de atavismo, como algo a ser superado por uma compreensão mais apurada dos fatores biológicos, psicológicos e histórico-culturais que condicionariam a ação humana. Por fim, nas últimas linhas do aforismo, Nietzsche esboça uma narrativa sobre as origens da moral a partir de fatores que não implicam em uma vocação metafísica do homem, ou um desígnio divino, nem em qualquer sentimento genuinamente moral, mas em mecanismos muito humanos de controle e condicionamento, já que toda moral teria surgido por um processo que tem início com uma “coerção” que se torna habitual e por fim prazerosa.

	Vale ressaltar que, se Nietzsche fala num \textit{cálculo} de prazer e utilidade, esse cálculo nada tem a ver com livre-arbítrio, concepção diretamente combatida pelo filósofo em diversas ocasiões. Ao pensar a razão como uma função orgânica, como um dispositivo de calcular movido pelos diversos impulsos, instintos e hábitos, Nietzsche se distancia radicalmente da ideia de um sujeito do livre-arbítrio, capaz de deliberar racionalmente de forma pura e constante, sujeito que seria objeto de absolvição ou condenação numa instância metafísica.
	
	Em certo sentido, não deixa de ser uma ética intelectualista a que se desenvolve em \textit{Humano}, pois se todas as ações humanas têm a motivação comum do prazer e o instrumento calculador da razão, há de se pensar que elas só resultam em dor quando há algum erro de cálculo, por desconhecimento do objeto ou dos meios necessários pra se atingir um objeto de prazer, ou mesmo por não se levar em conta as represálias que podem se levantar contra uma ação. Tal ética intelectualista se caracteriza antes por uma psicologia que entende que todo homem sempre acredita agir bem, no sentido de que sempre \textit{acredita} agir de modo a alcançar o prazer, atitude na qual está justificado, uma vez que o prazer seria um índice de comportamento útil à conservação da vida. A conclusão nietzschiana, um tanto engenhosa, é de que “o homem sempre age bem”, conclusão que dá título a um aforismo que reproduzimos aqui:
	
\begin{quotation}
Não acusamos a natureza de imoral quando ela nos envia uma tempestade e nos molha; por que chamamos de imoral o homem nocivo? Porque neste caso supomos uma vontade livre, operando arbitrariamente, e naquele uma necessidade. Mas tal diferenciação é um erro. Além disso, nem a ação propositadamente nociva é considerada sempre imoral; por exemplo, matamos um mosquito intencionalmente e sem hesitação, porque o seu zumbido nos desagrada; condenamos o criminoso intencionalmente e o fazemos sofrer, para proteger a nós e à sociedade. No primeiro caso é o indivíduo que, para conservar a si mesmo ou apenas evitar um desprazer, faz sofrer intencionalmente; no segundo é o Estado. Toda moral admite ações intencionalmente prejudiciais em caso de \textit{legítima defesa}: isto é, quando se trata da \textit{autoconservação}! Mas esses dois pontos de vista são \textit{suficientes} para explicar todas as más ações que os homens praticam uns contra os outros: o indivíduo quer para si o prazer ou quer afastar o desprazer, a questão é sempre, em qualquer sentido, a autoconservação. Sócrates e Platão estão certos: o que quer que o homem faça, ele sempre faz o bem, isto é: o que lhe parece bom (útil) segundo o grau de seu intelecto, segundo a eventual medida de sua racionalidade. (\textit{MA/HH}, §102).
\end{quotation}

Com este aforismo, Nietzsche aproxima sua própria versão do utilitarismo à ética socrática, proverbialmente intelectualista
\footnote{Há pelo menos um diálogo platônico que tem sido interpretado majoritariamente como expressão de uma ética não só intelectualista mas  hedonista – trata-se do \textit{Protágoras}, no qual os debatedores chegam à ideia de que as coisas prazerosas sempre são boas e se as evitamos é porque acreditamos que elas possam acarretar alguma dor ou porque buscamos um prazer maior. Aliás, os temas e a estrutura da argumentação do \textit{Protágoras} guardam várias semelhanças com \textit{Humano}, no entanto este é um tópico que não poderemos desenvolver aqui, mas num trabalho futuro.}. 
O interessante aqui é que, ao recorrer a essa ética intelectualista, Nietzsche coloca todas as ações num mesmo patamar, a princípio: todas as ações seriam igualmente uma resposta à pressão vital da busca pelo prazer e fuga da dor. A busca pelo prazer seria uma constante biológica, ao passo que as regras sociais que definem as formas moralmente aceitas de se buscar e usufruir o prazer são contingentes, historicamente elas seriam muito menos estáveis, muito mais variáveis, que essa pulsão vital pelo prazer. Além disso, todos os agentes teriam condições mais ou menos precárias de estipular o sentido de sua ação: em geral, conhecemos mal o passado (p.e., o processo seletivo, biológico ou cultural, que levou à configuração atual de nossos hábitos e instintos, que nos fazem perseguir este ou aquele tipo de objeto de prazer, ou reagir de diferentes formas aos obstáculos que se colocam a essa busca, etc.) e o futuro das ações (p.e., se elas obterão o resultado esperado, ou se elas serão interpretadas como adequadas às regras de certa comunidade, etc.)
\footnote{Cf. RICHARDSON (2006), que fala sobre como o orgânico se funda sobre uma perspectiva de autoengano sobre o passado e o futuro de seus impulsos.}. 
Nessa busca por prazer, o sujeito se expõe à possibilidade onipresente de erro, que só é mitigada pela maior ou menor capacidade de cálculo que está ao alcance de cada um. Conclui-se, nas palavras de Nietzsche: “Sem prazer não há vida; a luta pelo prazer é a luta pela vida. Se o indivíduo trava essa luta de maneira que o chamem de \textit{bom} ou de maneira que o chamem de \textit{mau}, é algo determinado pela medida e a natureza de seu intelecto.” (\textit{MA/HH}, §104).

Segundo a narrativa nietzschiana, quando uma ação em geral é considerada “má” há um descompasso entre o meio pelo qual o agente busca atingir sua meta de prazer e a configuração particular da moral vigente na sociedade em que esse agente se insere. Neste contexto, as chamadas “más ações” não deveriam ser pensadas como atos de \textit{maldade}, mas como fruto de uma pressão vital combinada a um erro intelectual, um erro de cálculo. Alia-se a este diagnóstico a ideia de que nada há de surpreendente nesse tipo de erro, uma vez que o aparato cognitivo/intelectual seria um dos dispositivos mais recentemente incorporados ao organismo humano, e responsável por uma função menor, se comparado ao restante de nossas disposições pulsionais. Uma vez que todas as ações são movidas pelo objetivo comum de otimizar o prazer e fugir da dor, o sucesso das ações dependeria de certos arranjos entre os impulsos herdados pelo agente e sua capacidade de calcular os meios de obtenção de satisfação a esses impulsos, bem como a possibilidade de que sua ação cause dor a outro agente, ou ainda a possibilidade de que ela esbarre em algum código moral, etc. Nesse sentido, chega-se à visão de que todo agente faz aquilo que \textit{pode} fazer, ou seja, aquilo que está a seu alcance. 

	Esse tipo de visão naturalizada da ação, associada a uma ética intelectualista, parece pretender surtir um efeito psicológico sobre os juízes: se percebemos uma ação como “má” e localizamos, na fonte dessa maldade, uma razão deliberativa, que seria sede de identidade metafísica do sujeito, como queria Descartes talvez, essa razão-identidade se torna um alvo imediato de nosso “ódio”, de nossa “perfídia”, etc.; por outro lado, se não mais pensamos a razão como fonte de identidade metafísica, mas como mera ferramenta de um organismo em sua multiplicidade de impulsos e constrangimentos, dificilmente conseguimos localizar um alvo para esses sentimentos condenatórios.

	O objetivo estratégico dessa narrativa é desestabilizar nossa tendência a prontamente emitir um juízo sobre as ações, louvá-las ou condená-las. Por mais entranhada que seja essa tendência a ver a ação como um ato isolado de um sujeito substancial, e a imediatamente identificá-la a comportamentos vistos como morais ou imorais, seria possível evitar a adesão imediata a essa tendência, e com isto manter um pathos da distância ao invés de emitir um juízo precipitado.

Acreditamos que é com vistas a criar esse tipo de pathos da distância que Nietzsche desenvolve aquela que talvez seja a tese mais chocante de \textit{Humano}, a tese da “irresponsabilidade radical”. A argumentação a favor dessa tese procede em primeiro lugar por uma expansão do princípio de “legítima defesa”, insistindo que, se o homem é constantemente premido a buscar o prazer e a conservação da vida, ele poderia ser inocentado até mesmo de suas “más ações”, na medida em que elas emergem igualmente desse constrangimento vital. Esse argumento – enquanto parte de toda a narrativa de naturalização da ação de que temos tratado – visa enfraquecer a tendência habitual a pensar que existe uma diferença essencial entre “boas” e “más” ações e, em última instância, se volta contra os fundamentos de todas as práticas de condenação ou premiação do comportamento. 

	Para Nietzsche, a consequência necessária desse tipo de consideração da ação seria, portanto, a “teoria da completa irresponsabilidade”, que é apresentada num aforismo conclusivo: 
	
\begin{quotation}
\textit{Irresponsabilidade e inocência}. - A total irresponsabilidade do homem por seus atos e seu ser é a gota mais amarga que o homem do conhecimento tem de engolir, se estava habituado a ver na responsabilidade e no dever a carta de nobreza de sua humanidade. Todas as suas avaliações, distinções, aversões, são assim desvalorizadas e se tornam falsas: seu sentimento mais profundo, que ele dispensava ao sofredor, ao herói, baseava-se num erro; ele já não pode louvar nem censurar, pois é absurdo louvar e censurar a natureza e a necessidade. Tal como ele ama a boa obra de arte, mas não a elogia, pois ela não pode senão ser ela mesma, tal como ele se coloca diante das plantas, deve se colocar diante dos atos humanos e de seus próprios atos. Neles pode admirar a força, a beleza, a plenitude, mas não lhes pode achar nenhum mérito: o processo químico e a luta dos elementos, a dor do doente que anseia pela cura, possuem tanto mérito quanto os embates psíquicos e as crises em que somos arrastados para lá e para cá por motivos diversos, até enfim nos decidirmos pelo mais forte – como se diz (na verdade, até o motivo mais forte decidir acerca de nós). Mas todos esses motivos, por mais elevados que sejam os nomes que lhes damos, brotaram das mesmas raízes que acreditamos conter os maus venenos; entre as boas e as más ações não há uma diferença de espécie, mas de grau, quando muito. Boas ações são más ações sublimadas; más ações são boas ações embrutecidas, bestificadas. O desejo único de autofruição do indivíduo (junto com o medo de perdê-la) satisfaz-se em todas as circunstâncias, aja o ser humano como possa, isto é, como tenha de agir: em atos de vaidade, de vingança, prazer, utilidade, maldade, astúcia, ou em atos de sacrifício, de compaixão, de conhecimento. Os graus da capacidade de julgamento decidem o rumo em que alguém é levado por esse desejo; toda sociedade, todo indivíduo guarda continuamente uma hierarquia de bens, segundo a qual determina suas ações e julga as dos outros. Mas ela muda continuamente, muitas ações são chamadas de más e são apenas estúpidas, porque o grau de inteligência humana que pode hoje ser atingido será certamente ultrapassado: então \textit{todos} os nossos atos e juízos parecerão, em retrospecto, tão limitados e precipitados como nos parecem hoje os atos e juízos de povos selvagens e atrasados. - Compreender tudo isso pode causar dores profundas, mas depois há um consolo: elas são as dores do parto. A borboleta quer romper seu casulo, ela o golpeia, ela o despedaça: então é cegada e confundida pela luz desconhecida, pelo reino da liberdade. Nos homens que são capazes dessa tristeza – poucos o serão! - será feita a primeira experiência para saber se a humanidade pode \textit{se transformar de moral em sábia}. O sol de um novo evangelho lança seu primeiro raio sobre o mais alto cume, na alma desses indivíduos: aí se acumulam as névoas, mais densas do que nunca, e lado a lado se encontram o brilho mais claro e a penumbra mais turva. Tudo é necessidade – assim diz o novo conhecimento: e ele próprio é necessidade. Tudo é inocência: e o conhecimento é a via para compreender essa inocência. Se o prazer, o egoísmo, a vaidade \textit{são necessários} para a geração dos fenômenos morais e do seu rebento mais elevado, o sentido para a verdade e a justiça no conhecimento; se o erro e o descaminho da imaginação foram o único meio pelo qual a humanidade pôde gradualmente se erguer até esse grau de autoiluminação e libertação – quem poderia desprezar esses meios? Quem poderia ficar triste, percebendo a meta a que levam esses caminhos? Tudo no âmbito da moral veio a ser, é mutável, oscilante, tudo está em fluxo, é verdade: - \textit{mas tudo se acha também numa corrente}: em direção a uma meta. Pode continuar a nos reger o hábito que herdamos de avaliar, amar, odiar erradamente, mas sob o influxo do conhecimento crescente ele se tornará mais fraco: um novo hábito, o de compreender, não amar, não odiar, abranger com o olhar, pouco a pouco se implanta em nós no mesmo chão, e daqui a milhares de anos talvez seja poderoso o bastante para dar à humanidade a força de criar o homem sábio e inocente (consciente da inocência), da mesma forma regular como hoje produz o homem tolo, injusto, consciente da culpa – \textit{que é, não o oposto, mas o precursor necessário daquele}. (MA/HH, §107).
\end{quotation}

Gostaríamos de fazer alguns apontamentos em relação a esse aforismo. Primeiramente, ressaltamos o fato de que, ciente da dificuldade da incorporação da tese da irresponsabilidade radical na cultura, Nietzsche dirige suas conclusões a um público específico, ao afirmar já nas primeiras linhas que: “A total irresponsabilidade do homem por seus atos e seu ser é a gota mais amarga que o homem do conhecimento tem de engolir”. É através desse “homem do conhecimento”, ou, segundo a expressão nietzschiana nesse homem do conhecimento “que será feita a primeira experiência” de incorporação da visão não-moral da ação. Essa visão teria por consequência o enfraquecimento das bases mesmas das práticas condenatórias e premiadoras que até o momento tiveram importância gerencial inegável nas diversas sociedades (embora Nietzsche em momento algum sugira a abolição imediata dessas práticas). Esse homem do conhecimento seria fundamentalmente capaz de um \textit{pathos} da distância, necessário à criação de uma nova forma de lidar com as práticas jurídicas, de deflacionar nelas o conteúdo sentimental relativo ao ódio e toda “exaltação” dos afetos. A sugestão nietzschiana parece ser que, gradualmente, mediante esse primeiro experimento, as sociedades seriam capazes de reformar suas formas de sentir e julgar a ação, evitando esses excessos.

	Ressaltamos a intenção claramente imoralista da argumentação nietzschiana, que se faz presente não só no combate à exaltação dos afetos envolvida nas práticas punitivas e premiadoras – sendo que essa exaltação afetiva seria o que afinal confere uma carga especificamente \textit{moral} às avaliações – mas também na abolição da ideia de que há uma diferença essencial entre boas e más ações. 
	
	Na seção anterior vimos como a insistência na realidade de um mundo em devir serve como corroboração ontológica para uma tese normativa sobre a necessidade de mudança na cultura, por via da reformulação dos chamados “objetivos ecumênicos”. Já o objetivo desta seção foi mostrar como essa visão de um mundo determinístico e radicalmente em devir, para a qual Nietzsche encontra subsídios na física e na biologia, serve de respaldo descritivo para a desconstrução da moral do livre-arbítrio, exemplo de “objetivo ecumênico” proposto nessa obra. O ambiente cético criado pelas diversas afirmações nietzschianas sobre a onipresença do erro, bem como sua aposta na possibilidade de gradual depuração dos erros, estão a serviço da proposta de modelação de uma atitude de espírito voltada à pesquisa cautelosa, por meio da qual estaríamos aptos inclusive a reformar nossa forma de julgar a ação. Chamamos atenção para o caráter experimental dessa forma de entender o mundo e a ação. 

Por fim, apontamos o fato curioso de que o filósofo pretende reconstruir um sentido de liberdade a partir de um cenário inteiramente determinista, e de uma visão do homem como incapaz de um controle absoluto sobre seu próprio ser. Se por um lado grande parte do livro é dedicada à crítica à moral do livre-arbítrio, vale lembrar que o livro em si é dedicado a um ideal de liberdade, representado pelos “espíritos livres”. A questão que se impõe, portanto é: dado esse cenário de um mundo determinístico precariamente cognoscível e gerenciável, qual seria a liberdade possível? 

\section{Ciência e libertação do espírito}

\begin{quotation}
\textit{O espírito livre, um conceito relativo}. – É chamado de espírito livre aquele que pensa de modo diverso do que se esperaria com base em sua procedência, seu meio, sua posição e função, ou com base nas opiniões que predominam em seu tempo. Ele é a exceção, os espíritos cativos são a regra; estes lhe objetam que seus princípios livres têm origem na ânsia de ser notado ou até mesmo levam à inferência de atos livres, isto é, inconciliáveis com a moral cativa. Ocasionalmente se diz também que tais ou quais princípios livres derivariam da excentricidade e da excitação mental; mas assim fala apenas a maldade que não acredita ela mesma no que diz e só quer prejudicar: pois geralmente o testemunho da maior qualidade e agudeza intelectual do espírito livre está escrito em seu próprio rosto, de modo tão claro que os espíritos cativos compreendem muito bem. Mas as duas outras explicações para o livre-pensar são honestas; de fato, muitos espíritos livres se originam de um ou de outro modo. Por isso mesmo, no entanto, as teses a que chegaram por esses caminhos podem ser mais verdadeiras e mais confiáveis que as dos espíritos atados. No conhecimento da verdade o que importa é \textit{possuí-la}, e não o impulso que nos fez buscá-la nem o caminho pelo qual foi achada. Se os espíritos livres estão certos, então aqueles cativos estão errados, pouco interessando se os primeiros chegaram à verdade pela imoralidade e os outros se apegaram à inverdade por moralidade. – De resto, não é próprio da essência do espírito livre ter opiniões mais corretas, mas sim ter se libertado da tradição, com felicidade ou com um fracasso. Normalmente, porém, ele terá ao seu lado a verdade, ou pelo menos o espírito da busca da verdade: ele exige razões; os outros, fé. (MA/HH, §225)
\end{quotation}

Encontramos aqui uma caracterização bastante clara do “espírito livre”, enquanto tipo ou personagem conceitual
\footnote{O tipo “espírito livre” é o destinatário de \textit{Humano}, e reaparece em diferentes momentos da obra nietzschiana. Ao longo de toda a sua obra, Nietzsche desenvolve uma tipologia de diferentes temperamentos, modos de vida e  atitudes valorativas – o tipo “nobre”, o “fundador de religiões”, o “último homem” figuram entre eles. Tratam-se, em geral, de algo como os “tipos ideais” webberianos. A ideia de “personagem conceitual”, é desenvolvida por Deleuze e Guattari em \textit{O que é filosofia?}, e exerce um papel importante na interpretação que fizeram da obra de Nietzsche: “O personagem conceitual não é o representante do filósofo, é mesmo o contrário: o filósofo é somente o invólucro de seu principal personagem conceitual e de todos os outros, que são intercessores, os verdadeiros sujeitos de sua filosofia. (…) Mas os personagens conceituais, em Nietzsche e alhures, não são personificações míticas, nem mesmo pessoas históricas nem sequer heróis literários ou romanescos.” DELEUZE, GUATTARI (1992), p. 87. Ver ainda: RIBEIRO (2011). Nietzsche certamente plasmou um novo conceito e mesmo toda uma “filosofia do espírito livre”, mas vale lembrar que nosso filósofo encontrou material para tanto na história da filosofia, especialmente nos pensadores “libertinos” dos séculos XVI e XVII, como Montaigne, La Rochefoucauld, entre outros. Cf. SANTOS (2011).}. 
No entanto, a afirmação que dá título ao aforismo, segundo a qual o espírito livre seria um conceito “relativo”, não é direta ou explicitamente desenvolvida no texto. Em relação a quê, ou em que sentido pode-se dizer que um espírito é livre? A primeira pista deixada no aforismo é a afirmação de que o espírito livre se expressa em contraposição àquilo que dele se espera “com base em sua procedência, seu meio, sua posição e função, ou com base nas opiniões que predominam em seu tempo.”. Em primeiro lugar, esse personagem representa, portanto, uma libertação em relação à regra vigente em seu tempo, seja ela qual for. Nietzsche nega que esse desvio da regra se deva a uma mera “excentricidade” ou “excitação intelectual”
\footnote{Por “excitação intelectual” Nietzsche provavelmente se refere a algo como “perturbação mental”.}, e o atribui antes a uma “maior qualidade e agudeza intelectual do espírito livre”. O filósofo acata, por outro lado, a opinião corrente segundo a qual o espírito livre seria alguém que anseia por ser notado e que, pelos princípios livres que adota, tende à realização de “atos livres, isto é, inconciliáveis com a moral cativa.”. Endossa, portanto, a ideia de que a liberdade de espírito se traduziria de alguma forma em liberdade prática.

Esses apontamentos sugerem que a liberdade do espírito livre residiria precisamente em sua não conciliação com a moral, ou ao menos com um tipo de moral, a “moral cativa”, caracterizada como uma atitude de adesão irrefletida à regra. Em oposição ao espírito livre, a primazia da fé na tradição que caracteriza o espírito cativo seria expressão do fato de que este não partilharia do anseio por ser notado, mas seria marcado antes pelo anseio por se assimilar à regra, ao comum; seria característica sua um certo embotamento intelectual que lhe permite apegar-se inclusive “à inverdade por moralidade”
\footnote{Lembrando que o ser moral é definido em \textit{Humano} como obediente à regra vigente, qualquer que seja.}; 
isto é, no espírito cativo, o impulso mais forte responde pelo desejo de estar em conformidade com a regra, não importando tanto a verdade de seus pressupostos, ou a procedência das opiniões vigentes. Por outro lado, o traço fundamental do espírito livre seria a primazia de uma virtude epistêmica, de forma que no desenvolvimento da obra nietzschiana ele será reiteradamente associado à honestidade ou integridade intelectual (\textit{Redlichkeit}). A situação do espírito livre envolve algum grau de inadequação à tradição; ao que parece, sua característica “agudeza intelectual” (traço supostamente “natural” de sua constituição) aplicada à pesquisa da verdade e exigência de “razões” colocam-no na contramão da exigência fundamental da moral: agir de forma genérica (não pessoal) orientando-se pelo costume. Na caracterização do espírito livre, figura acima de tudo um princípio normativo que orienta sua relação com as crenças – esse princípio repercute sobre a orientação de suas ações e delineia um modo de vida. 

É preciso sempre manter em mente o fato de que é esse personagem conceitual do espírito livre o destinatário do programa filosófico de \textit{Humano}, um programa que, como temos visto, se pauta pela investigação naturalista das crenças e valores herdados pelo homem moderno. A contrapartida prática desse programa seria uma atitude de cuidado com as “coisas próximas” e indiferença em relação aos encaminhamentos das questões tradicionais da metafísica. O espírito livre reedita na modernidade a agenda do cuidado de si
\footnote{Pode-se dizer que o tema da investigação naturalista sobre a origem de nossos impulsos, dispositivos cognitivos e valores ocupa a maior parte do primeiro volume de \textit{Humano}, enquanto o tema do cuidado de si e cultivo das “coisas próximas” ganha atenção ainda mais particularizada no segundo volume, em passagens como esta: “(...) – \textit{Todo} o resto deve ficar mais próximo de nós do que aquilo que até hoje nos foi ensinado como o mais importante; refiro-me às questões: que finalidade tem o homem? Qual seu destino após a morte? Como se concilia ele com Deus?, ou seja lá como se exprimam tais curiosidades. Todos procuram nos impelir a uma decisão em áreas onde não é necessário crer nem saber; mesmo para os maiores amantes do conhecimento é mais útil que ao redor de tudo indagável e acessível à razão se estenda um nebuloso e enganador cinturão de pântano, uma faixa do impenetrável, eternamente fluido e indeterminável. É justamente pela comparação com o domínio do obscuro, à margem da terra do saber, que cresce continuamente o valor do claro e vizinho mundo do saber. – Temos que novamente nos tornar \textit{bons vizinhos das coisas mais próximas} e não menosprezá-las como até agora fizemos, erguendo o olhar para nuvens e monstros noturnos. Foi em bosques e cavernas, em solos pantanosos e sob céus cobertos que o homem viveu por demasiado tempo, e miseravelmente, nos estágios culturais de milênios inteiros. Foi ali que \textit{aprendeu a desprezar} o tempo presente, as coisas vizinhas, a vida e a si mesmo – e nós, que habitamos as campinas mais claras da natureza e do espírito, ainda hoje recebemos no sangue, por herança, algo desse veneno do desprezo pelo que é mais próximo.” (MA/HH II – WS/AS, §16). Ainda, “(...) Considere-se, porém, que \textit{quase todas as enfermidades físicas e psíquicas} do indivíduo decorrem dessa falta: de não saber o que nos é benéfico, o que nos é prejudicial, no estabelecimento do modo de vida, na divisão do dia, no tempo e escolha dos relacionamentos, no trabalho e no ócio, no comandar e obedecer, no sentimento pela natureza e pela arte, no comer, dormir e refletir; ser \textit{insciente} e não ter olhos agudos para as coisas \textit{mínimas e mais cotidianas} – eis o que torna a Terra um 'campo do infortúnio' para tantos. Não se diga que aí, como em tudo, a causa é a \textit{desrazão} humana – há razão bastante e mais que bastante, isso sim, mas ela é \textit{mal} direcionada e \textit{artificialmente afastada} dessas coisas pequenas e mais próximas.(...)” (MA/HH II – WS/AS, §6.)}.

Segundo Oscar Santos, a filosofia do espírito livre aponta para uma conjunção 

\begin{quotation}
tanto das características biológicas e históricas do sujeito, quanto de seu empenho no cultivo de si mesmo, para que se obtenha (como \textit{modelo} normativo) uma pretendida natureza de exceção que tenha justamente na integridade intelectual [\textit{Redlichkeit}] seu impulso predominante, de modo que a busca por conhecimento seja justamente a expressão dos \textit{sentimentos} de inclinação e rechaço relativos à sua orientação afetiva mais própria; (…) o sentido prescritivo da filosofia do espírito livre se sustenta na conjunção de uma natureza rara, cujos tons mais sutis dizem respeito à sua integridade intelectual, e o trabalho de cultivo de si mesmo, capaz de fazer da busca pelo conhecimento algo mais que um mero meio, mas a própria finalidade e justificação da vida. (SANTOS, 2011, p. 140).
\end{quotation}

A libertação em relação à obrigatoriedade da regra, permitiria ao espírito livre estabelecer um projeto pessoal de experimentação e justificação de um modo de vida próprio, que tem a busca por conhecimento como seu fio condutor.

	O contexto naturalista do livro aponta ainda um outro sentido no qual se poderia dizer a “relatividade” do espírito livre. Mantenha-se em mente o trabalho filosófico de trazer à tona os dispositivos biológicos, psicológicos e sócio-históricos que condicionam a ação, trabalho que serviria de recurso à superação das morais fundadas na noção de liberdade da vontade – e vê-se que esse contexto parece vetar a possibilidade de uma liberdade \textit{absoluta}. Ou seja, a “liberdade relativa” da filosofia do espírito livre é pensada por oposição à “liberdade absoluta” das filosofias do livre-arbítrio.
	
	Nietzsche, enquanto crítico do livre-arbítrio, ainda assim desenvolveu um ideal de liberdade – um ponto que, a nosso ver, foi interpretado de modo muito interessante na obra de John Richardson. A interpretação de Richardson se destaca precisamente por levar em consideração o interesse de Nietzsche pelo cenário da biologia evolucionista, que foi usada pelo filósofo como um meio para a desconstrução da moral do livre-arbítrio, e oferece, ao contrário, uma visão da ação humana como condicionada por instintos, impulsos e hábitos historicamente desenvolvidos. Richardson aborda com muita clareza a interlocução de Nietzsche com os evolucionistas de sua época, e também desenvolve chaves de interpretação muito criativas a partir da aproximação das noções e propostas nietzschianas com aquelas da biologia evolucionista em geral. Na tentativa de esclarecer alguns paradoxos da filosofia nietzschiana por meio de uma harmonização desta com a biologia evolucionista, Richardson reconstitui uma narrativa que apresenta o homem como produto tanto de um processo de seleção natural quanto de seleção sociocultural. Enquanto a seleção natural atua sobre os impulsos do organismo, a seleção social operaria uma triagem de hábitos e valores a serem incentivados
\footnote{O termo “seleção social” não é usado por Nietzsche, mas desenvolvido como um recurso interpretativo por Richardson. }. 
Segundo a leitura de Richardson, Nietzsche aponta para a possibilidade de se estabelecer, ainda, um processo pessoal de seleção de impulsos, hábitos e valores – Richardson chama esse processo de \textit{autosseleção} (\textit{self selection}). A autosseleção marcaria o momento em que o indivíduo toma para si a tarefa de cultivar, incentivar ou se desengajar de impulsos, hábitos e valores que herdou por vias de seleção natural e social. Segundo a interpretação de Richardson, é justamente na possibilidade de autosseleção que residiria a possibilidade de \textit{liberdade}. Passemos a uma rápida reconstituição dos traços gerais dessa interpretação.

O ponto de partida está na constatação de que, na filosofia nietzschiana, o homem é pensado como um apanhado de \textit{impulsos}
\footnote{Nietzsche se refere à vida pulsional do sujeito aplicando com certa liberalidade não só o termo usualmente traduzido por “impulso” (\textit{Trieb}) mas também instinto (\textit{Instinkt}) e afeto (\textit{Affekt}). Há ainda um certo trânsito entre os termos usados para se referir ao objeto do que seria a seleção natural, isto é, os impulsos biológicos, e aqueles que seriam objeto da seleção social: hábitos e valores. A este respeito, Richardson afirma “By superimposition of social on natural selection, a new set of behavioral dispositions arises and evolves, a web of practices that is both a rewriting and an overwriting of the dispositions shaped by natural selection. As I've said, Nietzsche sometimes calls these new dispositions 'habits', reserving 'drives' for those more ingrained 'animal' instincts we inherit rather than learn. But he doesn't consistently observe this distinction, often calling learned tendencies 'drives' as well. The boundary is a permeable one for him, because he accepts a Lamarckian 'inheritance of acquired traits'. (…) Still, he treats them as less securely or solidly or deeply settled in this way than our animal inheritance; they can go as quickly as they came.” RICHARDSON, 2004, p. 83. Para uma justificação da tradução de \textit{Trieb} por \textit{impulso}, ver a nota 21 de Paulo César de Souza à sua tradução de \textit{Além de Bem e Mal} (edição indicada nas referências bibliográficas ao final deste trabalho). PCS busca justificar inclusive o fato de Nietzsche não distinguir de forma muito estrita os termos “impulso” e “instinto”, remetendo à questão da dificuldade de se estabelecer o que é do âmbito meramente biológico e o que é culturalmente incorporado.} 
que em grande parte atuam de forma inconsciente. É sobre esse material pulsional que atuam a seleção natural e social. Ao reconstruir o sentido nietzschiano de impulso, Richardson chega a uma fórmula segundo a qual os impulsos seriam:

\begin{quotation}
\textbf{disposições} para o \textit{comportamento}, i.e., as tendências causais do organismo a agir de certo modo. Na maioria dos casos, o próprio comportamento tende a implicar em algum \textit{produto} usual – de forma que podemos dizer que um impulso é também uma disposição para certo resultado. Ademais, um impulso é uma disposição \textbf{plástica} em direção a esse resultado, na medida em que tende a produzir diferentes comportamentos em diferentes circunstâncias, de forma tal que o resultado é atingido, por diferentes rotas, em todas elas. (RICHARDSON, 2004, p. 74)
\footnote{Autor de língua inglesa, Richardson se beneficia da semelhança do termo alemão \textit{Trieb} com o inglês \textit{drive}, que traduzimos por “impulso”. }.
\end{quotation}

Richardson ressalta a ideia de que o \textit{sentido} ou \textit{função} de um impulso não é sua adaptação imediata às circunstâncias presentes do organismo, mas uma resposta selecionada mediante situações passadas. Quer dizer, a configuração atual de um impulso é explicada pelo sucesso adaptativo que ele obteve no passado – nada garante que esse sucesso se repita no futuro, não é a possibilidade de adaptação a circunstâncias futuras o que o define. A respeito do impulso, Richardson chama atenção não para o fato de ele ser (supostamente) \textit{apto }à sobrevivência, mas para o fato de ele ser uma \textit{adaptação}. O mesmo poderia ser dito a respeito de nossas práticas sociais: a abordagem naturalista desenvolvida ao longo da obra faz crer que dispositivos socioculturais como a moral estão sujeitos a um processo semelhante àquele de seleção natural, portanto, os sentidos dessas práticas seriam igualmente explicados pela função gerencial que cumpriram no passado. Isto se aplica, por exemplo, ao surgimento de um senso moral de justiça a partir de um acordo (pré-moral) de não destruição mútua firmado num momento anterior à fundação do Estado, tema de que tratamos na seção anterior. Ademais, o filósofo sugere que o efeito propriamente moral da incorporação da regra é obtido mediante um processo de esquecimento das circunstâncias práticas que forçaram seu estabelecimento – daí a constante preocupação nietzschiana em acessar as origens esquecidas da moral. A este respeito, Richardson aponta que:

\begin{quotation}
Os \textit{sentidos} de nossos impulsos e práticas, o que eles realmente buscam (seu “para quê”), se encontram nessa história evolucionária. Uma vez que esta é em grande parte desconhecida e opaca para nós, nós não sabemos de fato por quê pensamos e agimos como tal. As razões não estão em nossos motivos e decisões (conscientes ou inconscientes), ao contrário do que sugere o modelo mental de teleologia. Nós precisamos da genealogia para desenterrar esses sentidos. (RICHARDSON, 2004, p. 35)
\footnote{Parece-nos um fato bastante claro que, embora o procedimento genealógico propriamente dito só seja apresentado por Nietzsche quase uma década após a publicação de Humano, com a \textit{Genealogia da Moral} (1887), a intuição sobre a necessidade de desvelar a origem e o desenvolvimento acidentado das práticas sociais, atentar para o processo de esquecimento que envolve seu funcionamento, já se encontra no programa de \textit{filosofia histórica de Humano}.}.
\end{quotation}

Uma vez que a configuração atual de nossos impulsos e práticas sociais se explica não por sua potencial aptidão presente ou futuro, mas por adaptações definidas no passado biológico e cultural da espécie, há sempre a possibilidade de que os mesmos acabem por tornar-se “atávicos”, anacrônicos\footnote{Cf. LOPES, 2013, p. 118.}. A investigação naturalista acerca das origens dos impulsos e práticas sociais serve ao propósito de rever seu sentido e valor atual, o que aponta para um fim terapêutico do programa de filosofia histórica lançada por Humano. Mas a ideia de que temos que buscar no passado a chave para uma libertação presente precisa ser melhor compreendida. Em relação a que, precisamente, é necessário se libertar? 

A interpretação proposta por Richardson ressalta a ideia nietzschiana de que o homem seria o “animal não-fixado” ou o “animal doente”
\footnote{Cf. A/AC, §14; JGB/ABM, §62; GM/GM iii, §13; GD/CI vii, §2.}, 
e avança uma leitura segundo a qual essa “doença” se deveria à relação paradoxal e por vezes antinomínica entre seleção natural ou biológica e seleção sociocultural
\footnote{Cf. RICHARDSON, 2004, p. 81.}.

	Enquanto a seleção natural expressaria uma lógica de reprodução do indivíduo, a seleção social seguiria uma lógica de grupo
\footnote{Cf. RICHARDSON, 2004, p. 85.}. 
É de seu interesse selecionar aqueles hábitos e práticas mais facilmente replicáveis pelo grupo, isto é, aqueles hábitos que tendem a ser mais \textit{imitados}. O sentido da seleção social é assimilar o indivíduo ao que é comum, e assim fortalecer o grupo. Por fim, a seleção social privilegiará acima de tudo o próprio impulso à imitação; é de seu interesse que esse impulso à concordância detenha a primazia sobre todos os outros impulsos individuais – Richardson chama esse fenômeno de assimilação prioritária dos hábitos comuns de “meta-hábito”, Nietzsche o chamará “instinto gregário” ou “instinto de rebanho”
\footnote{“(...) this social selection, by its very logic, favors a drive \textit{to copy}, i.e., a disposition to imitate others, to want to do the same as they do. This is the 'meta-habit' of learning habits by copying others; it is so basic and long-standing a product of social selection that it has become a stable drive itself.” RICHARDSON, 2004, p. 86. Em \textit{A Gaia Ciência}, tem-se a fórmula: “Moralidade é o instinto de rebanho no indivíduo.” (FW/GC, §116).}. 
	
	A moral, enquanto principal instrumento da seleção social, demanda que se inscreva na memória do indivíduo o veto aos comportamentos que não devem ser replicados no grupo. Para tal, recorre num primeiro momento às mais violentas técnicas mnemônicas, ao castigo:

\begin{quotation}
\textit{Degraus da moral}. – Moral é, primeiramente, um meio de conservar a comunidade e impedir sua ruína; depois é um meio de manter a comunidade numa certa altura e numa certa qualidade. Seus motivos são \textit{temor} e \textit{esperança}: e serão tanto mais rudes, vigorosos, grosseiros, quanto ainda for bastante forte a inclinação ao errado, unilateral, pessoal. Os mais horrendos meios de intimidação têm de ser aí empregados, enquanto outros mais suaves não surtirem efeito e essa dupla espécie de conservação não puder ser alcançada de outra forma (entre os meios mais fortes está a invenção de um Além com um eterno Inferno). Nisso tem de haver martírios da alma e carrascos para eles. Outros degraus da moral e, portanto, meios para os fins assinalados são as ordens de um deus (como a lei mosaica); outros mais, ainda mais elevados, são os mandamentos de uma noção absoluta do dever, com o “tu deves” – todos degraus ainda talhados grosseiramente, mas \textit{amplos}, porque os homens ainda não sabem pôr os pés nos mais finos, mais estreitos. Depois vem uma moral da \textit{inclinação}, do gosto, e enfim a da \textit{intelecção} – que está acima dos motivos ilusórios da moral, mas percebeu que durante largos períodos a humanidade não pôde ter outros.  (MA/HH II – WS/AS, §44).
\end{quotation}

A vida em grupo regulada pela moral implicaria, portanto, em algum tipo de sacrifício por parte do indivíduo. A moral é antes de tudo um inibidor da tendência do indivíduo a agir de modo “unilateral”, isto é, expressar egoisticamente os impulsos que lhe ocorrem – neste sentido, a moral atua na contramão da seleção natural, cuja lógica consiste justamente na expressão e reprodução dos impulsos individuais. Uma vez que na hierarquia imposta pela seleção social, o hábito mais incentivado é o meta-hábito, ou a disposição para imitação do “rebanho”, a moral sacrificará no indivíduo precisamente aquilo que ele tem de mais particular e pessoal, de forma que seu efeito tende à homogeneização dos impulsos\footnote{Cf. MA/HH, §96.}. A gradual incorporação da regra – primeiramente por temor do castigo físico, então por temor do castigo espiritual, e enfim pela cultivada disposição à concordância – implicaria em algum tipo de empobrecimento pulsional/corporal. Além disso, a seleção social exigiria algum grau de sacrifício intelectual por parte do indivíduo – ela exige que, na maioria dos casos, se renuncie à pesquisa da verdade e à busca por razões, de modo que o indivíduo se oriente prioritariamente por um princípio de fé na regra, no comum, ou na tradição; ela incute no espírito do indivíduo temor e esperança associados a questões que não podem ser decididas no plano do conhecimento propriamente epistêmico, isto é, questões relativas ao “destino da alma”, à “existência de Deus”, etc., e assim extravia parte da energia intelectual do plano de conhecimento relativo às questões práticas, ou do cultivo do conhecimento no campo das “coisas próximas”\footnote{MA/HH, §96, §225, §371. MA/HH II – WS/AS §6, §16, §44, §350.}.

Encontra-se na filosofia nietzschiana, portanto, uma terrível acusação contra a moral, pois lhe é imputada a responsabilidade por haver extraviado o homem de um projeto de “saúde”, isto é, de expressão equilibrada dos impulsos naturais, e de um projeto de pesquisa da verdade
\footnote{Cf. RICHARDSON, 2004, p. 120.}. 
Por outro lado, as linhas finais do aforismo acima sugerem que, apesar desses efeitos nocivos, o processo civilizatório instaurado por meio da moral tem como desfecho o surgimento de estágios mais refinados de regulação social, cujo funcionamento não mais seria coercitivo. Nesses estágios de “maturidade” social estaria aberto o caminho para o desenvolvimento de morais “da \textit{inclinação}, do \textit{gosto}, e enfim a da \textit{intelecção}”; seria característico desses estágios mais elevados uma percepção crítica das práticas morais, de seus “motivos ilusórios”, mas uma aceitação da necessidade prática passada desses motivos, frente à percepção de que “durante largos períodos a humanidade não pôde ter outros” (MA/HH II – WS/AS, §44).

Aceitar não o conteúdo, mas a necessidade dos motivos “ilusórios” da moral parece ser uma aceitação da necessidade de todo o processo civilizatório, o processo de seleção social dos hábitos e valores. Nietzsche certamente reconhece que esse processo de seleção social foi necessário à manutenção da vida em grupo, modo de vida que se impõe a seres sociais como nós. Além disso, esse processo teria levado ao desenvolvimento de uma série de recursos cognitivos relativos à memória, linguagem e consciência. Seu desenvolvimento permitiu aos indivíduos se lembrarem das regras de comportamento; comunicar (isto é, “tornar comum”) sentimentos, percepções e outros traços gerais da experiência, além de comandos, alertas, valores etc.; fixar conscientemente metas e objetivos que sirvam ao grupo
\footnote{“These factors – memory, consciousness, and language – transform the character of 'values'. They allow a behavioral disposition to 'aim' at its goals in new ways: foresightedly, self-consciously, and linguistically. Now the behavior makes the goal explicit to itself, sighting it in advance consciously, and naming it. It's only here that we arrive at values, as moral philosophers have known them – as goods we name, are aware of, and remember to live by. Still, there is something deceptive in this new foresight: the individual \textit{doesn't} really choose the values he/she pursues in this way; these values are still dictated by social selection, working on behalf of the herd.” RICHARDSON, 2004, p. 91. Cf. MA/HH, §59.}. 
Enfim, parece haver uma aceitação do processo civilizatório porque esse teria sido o processo que \textit{humanizou} o homem
\footnote{O processo de humanização do homem é o tema do último aforismo do segundo volume de \textit{Humano}: “Muitas cadeias foram postas no homem, para que ele desaprendesse de se comportar como um animal: e, de fato, ele se tornou mais rando, mais espiritual, mais alegre do que todos os animais. Mas ele ainda sofre por haver carregado tanto tempo essas cadeias, por haver faltado ar puro e livre movimento por tanto tempo: - mas elas são, estou sempre a repetir, aqueles pesados e convenientes erros das concepções morais, religiosas, metafísicas. (…)” \textit{MA/HH} – WS/AS, §350.}.

A afirmação de que por “longos períodos a humanidade não pôde ter outros” motivos e metas senão aqueles da moral sugere, por outro lado, que no presente se poderia sim, perseguir outros motivos e metas. O surgimento de morais “do \textit{gosto}, da \textit{inclinação} e da \textit{intelecção}” parece apontar para um momento de fortalecimento do indivíduo, e para a possibilidade de que o indivíduo passe a \textit{exaptar} – isto é, destinar a fins outros do que aqueles para os quais foram historicamente selecionados – o aparato impulsivo, cognitivo e valorativo que herdou. Ao que tudo indica, esses novos fins diriam respeito ao próprio ganho de vigor por parte do indivíduo que, num exercício de liberdade, poderia reconduzir-se a um projeto pessoal de saúde e pesquisa da verdade, modo de vida cuja alegoria seria o tipo “espírito livre”\footnote{“The self-creation or freedom that Nietzsche means lies in bringing the selective process into oneself. It lies in taking over, oneself, the selective role, so that one 'creates' or 'gives oneself' values. It involves a 'will to self-determination [\textit{Selbstbestimmung}], to self-value-setting [\textit{Selbst-Werthsetzung}], this will to a \textit{free} will' (HH.i.P.3).” RICHARDSON, 2004, p. 95.}.

Segundo a narrativa nietzschiana, a história da moral é de certa forma “redimida” pelo surgimento do “indivíduo coletivo”, que não mais age de forma “unilateral” e imediatista, mas tampouco está à mercê das arbitrariedades de grupo – é capaz de criar valores para si mesmo, e torna-se o “legislador das opiniões” (\textit{MA/HH}, §94). Se há aqui um trabalho de desvelar as origens biológicas e culturais de nossa constituição presente, esse trabalho não parece estar à disposição de uma filosofia de retorno às origens. O conhecimento da origem seria necessário à tarefa de atualização dos valores, ou serviria como fonte de material para a criação de novos valores. Isto porque os valores não se criam ex nihilo; a criação de novos valores dependeria de uma readaptação (ou exaptação, termo usado por Richardson) do material biológico e simbólico da história humana. 	

Assim, a liberdade dependeria justamente da capacidade de levar em conta os produtos dos diversos fatores que condicionam essa história, e usá-los a favor de seu próprio envigoramento. Enquanto o homem moral kantiano necessitaria pressupor uma “fenda” entre o fenomênico e o noumênico, de onde brotaria a liberdade absoluta indeterminada pelas cadeias causais da ação, o ponto de partida da liberdade do espírito livre está justamente em perceber-se como determinado, e apenas relativamente capaz remodelar a cadeia causal em que se insere
\footnote{Sobre a reformulação nietzschiana da noção de liberdade, Richardson aponta que: “We imagine it [freedom] the wrong way – as occurring in special moments of decision from a viewpoint poised on the moment, a 'first cause' undetermined from the past. We need to see that self selection is a lot like natural and social selection: it operates in an aggregate way that need not be supervised by an overarching consciousness. (…) It lies in an overall pattern of suspicion or skepticism, practiced against one's instincts over a long period. In repeatedly tracing out the many ways that one's values have been made by natural and social selection, and acting in the light of it, one stands free (in the way we can be) of those other forces, and values for oneself.” Mais adiante, o autor aponta ainda que, na filosofia nietzschiana, a liberdade é sempre pensada em termos relativos porque o filósofo não considera a possibilidade de nos libertarmos de todo e qualquer traço herdado: “It belongs to Nietzsche's naturalization of freedom that he conceives it as not absolute – there are certain \textit{limits} to the freedom we can have. This is because of the way some things are \textit{settled} in us: there are values we can't disengage. Some of these are personal idiosyncrasies, as Nietzsche describes his own views about women: these belong to his `spiritual \textit{fatum}, to what is quite \textit{unteachable} `down there'' (BGE231). Other unshakable attitudes belong to all of us, as human, as animal, even merely as alive. Such values constitute blind spots we can't manage to overcome, even if we can diagnose them. Self selection, in other words, can never be complete.” RICHARDSON, 2004, p. 103.}. 

	O projeto de autoconhecimento é sempre um projeto de conhecimento do mundo e da história.	O experimento de autosseleção, isto é, de cultivo pessoal de impulsos, hábitos e valores próprios, seria um processo de incorporação gradual, a nível da vida prática cotidiana, do conhecimento produzido sobre a herança biológica e cultural de cada um. O espírito livre conduziria um experimento de autoconhecimento no qual deve recorrer à observação de si mesmo e de seu meio, bem como aos diversos conhecimentos empíricos advindos dos campos da biologia, psicologia, da história, do estudo das culturas, etc. Neste desdobramento terapêutico do programa naturalista de Humano, a ciência aparece como fonte de informações (que podem se tornar cada vez mais particularizadas) aplicáveis às práticas de cultivo de si, isto é, conhecimentos sobre o corpo próprio; o efeito de diferentes tipos de clima e alimentação sobre a constituição física e o temperamento de cada um; sobre as funções adaptativas de impulsos à agressividade, à conciliação, à compaixão, etc.; ou ainda, informações sobre o contexto geográfico/geopolítico do surgimento das várias culturas e seus respectivos valores, costumes, crenças, etc. 

Mas acima de tudo, Nietzsche aposta no efeito terapêutico da ciência na medida em que ela seria capaz de formar uma disposição de espírito favorável à pesquisa e disciplinar os impulsos cognitivos envolvidos na produção de conhecimento. Em várias passagens do livro, expressa-se a ideia de que o maior e mais duradouro benefício do contato com a prática científica seria a aquisição de um \textit{método}, não no sentido procedimental, mas enquanto consciência metódica e exercício de “virtudes epistêmicas”: cautela, sobriedade, modéstia, moderação, justiça e, acima de tudo honestidade (\textit{Redlichkeit})
\footnote{“This 'honesty' (\textit{Redlichkeit}), which is Nietzsche's favorite virtue, is needed precisely for inquiry into the sources of our values – which is what it's both hardest and most important to be honest about. It opposes our ancient and sedimented instinct not to question our values – especially values shared widely by our group and kind. The great stress and pain involved in exposing the sources of these values demands as ally the subsidiary virtue of courage (\textit{Muth}) – which matters to Nietzsche only in this context, for the sake of that honesty.” RICHARDSON, 2004, p. 98. Em VAN TONGEREN (2012) o termo \textit{Redlichkeit} é traduzido por “probidade”; outras opções seriam “retidão” ou “integridade intelectual”. Cf. VAN TONGEREN, 2012, cap. 3. Ver ainda GIACOIA JR., 2010.}. 

	As seções finais do primeiro volume de Humano (§629-§638) são dedicadas ao tema do cultivo da consciência metódica, e de como a partir dela o espírito livre estabelece para si uma espécie de “dietética” da crença.
	
	O primeiro aforismo dessa série tem início com uma reflexão sobre a suposta herança de uma cultura da exaltação da paixão; essa cultura seria a responsável pela dificuldade de revermos as crenças que adotamos por motivos passionais, mesmo depois que o calor da paixão já passou. A supervalorização da paixão (que Nietzsche credita à cultura artística, inclusive
\footnote{Certamente, há aqui uma crítica ao romantismo, tema que não poderemos desenvolver.}) 
teria criado a impressão de obrigatoriedade de “ser fiéis aos nossos erros” (MA/HH, §629), de modo que a mudança de opinião costuma ser vista como uma perda, como algo doloroso. O filósofo então lança a questão:

\begin{quotation}
(…) Por que admiramos aquele que permanece fiel às suas convicções e desprezamos aquele que as muda: Receio que a resposta tenha de ser: porque todos pressupõem que apenas motivos de baixo interesse ou de medo pessoal provocam tal mudança. Ou seja: no fundo acreditamos que ninguém muda sua opinião enquanto ela lhe traz vantagem ou, pelo menos, enquanto não lhe causa prejuízo. Se for assim, porém, eis aí um péssimo atestado da significação \textit{intelectual} das convicções. Examinemos como se formam as convicções; e observemos se não são grandemente superestimadas: com isto se verificará que também a \textit{mudança} de convicção é sempre medida conforme um critério errado, e que até hoje tivemos o costume de sofrer demais com tais mudanças. (MA/HH, §629)
\end{quotation}

Nietzsche entende, portanto, que a psicologia da fidelidade à crença é motivada, no fundo, por uma espécie rasa de utilitarismo, devido à qual apega-se a uma crença devido a um cálculo já cristalizado de sua “vantagem” ou por comodismo, por ela parecer não implicar num prejuízo imediato, de forma que o fiel é que seria o verdadeiro merecedor da suspeita de “covardia”. Essa motivação tacanha seria o “péssimo atestado da significação \textit{intelectual}” das convicções. O quadro se agrava porque estas são definidas, no aforismo seguinte, como “a crença de estar, em algum ponto do conhecimento, de posse da verdade absoluta.” (MA/HH, §630). Nietzsche passa então a atacar os fundamentos dessa crença:

\begin{quotation}
Esta crença pressupõe, então, que existam verdades absolutas; e, igualmente, que tenham sido achados os métodos perfeitos para alcançá-las; por fim, que todo aquele que tem convicções se utilize desses métodos perfeitos. Todas as três asserções demonstram de imediato que o homem das convicções não é o do pensamento científico (…). (\textit{MA/HH}, §630).
\end{quotation}

O que essa crítica sugere, então, é que enquanto crença na posse da “verdade absoluta” (conceito em si mesmo criticado), crença mantida por razões extra-intelectuais, a convicção só se forma por métodos epistemicamente não confiáveis. As três condições pressupostas pelo homem de convicção, apontadas no aforismo acima, nunca poderiam ser satisfeitas por seres cognitivamente falíveis e finitos como nós, de forma que o homem epistemicamente virtuoso seria aquele que  proíbe a si mesmo qualquer adesão à convicção. O homem científico seria justamente aquele que consegue criar um \textit{pathos} da distância e um imperativo intelectual que o torna mais disposto a desapegar-se das crenças uma vez que seu caráter insão seja exposto, sua atitude para com as crenças não é uma atitude de convicção
\footnote{Mais adiante, o filósofo aponta um outro uso terapêutico dessa atitude distanciada do homem científico, isto é, seu efeito antídoto contra a intolerância e a dificuldade de conviver com a diferença; Nietzsche afirma: “Não foi o conflito de opiniões que tornou a história tão violenta, mas o conflito da fé nas opiniões, ou seja, das convicções.” (\textit{MA/HH}, §630).}. 

	No trato com as crenças, Nietzsche atribui ao “espírito científico” a tarefa de “amadurecer no homem a virtude da \textit{cautelosa abstenção}, o sábio comedimento” (\textit{MA/HH}, §631)
\footnote{Nesse aforismo, Nietzsche cita como exemplo de virtuose do comedimento o personagem Antonio, da peça \textit{Torquato Tasso}, de Goethe. O antagonista de Antonio é Tasso, apresentado por Nietzsche como representante das “naturezas não científicas e também passivas”. O exemplo interessa a Nietzsche por simbolizar uma espécie de “pacto de convivência” entre o pensador e o “homem de convicção”: “O homem de convicção tem o direito de não entender o homem do pensamento cauteloso, o teórico Antonio; o homem científico, por sua vez, não tem o direito de criticá-lo por isso, é indulgente para com o outro e sabe que em determinado caso este ainda se apegará a ele, como Tasso fez afinal com Antonio.” (MA/HH, §631).}. 
Ao que nos parece, o ponto crucial do programa filosófico de \textit{Humano} não consiste tanto na busca por refutação científica das crenças tradicionais, mas na aposta na ciência como capaz de criar a disciplina de pensamento necessária a um modelo de \textit{sabedoria}, um modo de vida contemplativa
\footnote{Cf. LOPES, 2008, p. 52-53.}. 
Por mais que haja um claro interesse nos resultados – quer dizer, nos conhecimentos específicos produzidos pela ciência, pois eles teriam aplicação direta na dietética do espírito livre – não são os resultados, mas os métodos científicos que criariam a disciplina estruturante desse modo de vida:

\begin{quotation}
No conjunto, os métodos científicos são um produto da pesquisa ao menos tão importante quanto qualquer outro resultado: pois o espírito científico repousa na compreensão do método, e os resultados todos da ciência não poderiam impedir um novo triunfo da superstição e do contra-senso, caso esses métodos todos se perdessem. Pessoas de espírito podem \textit{aprender} o quanto quiserem sobre os resultados da ciência: elas não possuem a instintiva desconfiança em relação aos descaminhos do pensar, que após prolongado exercício deitou raízes na alma de todo homem científico. Basta-lhes encontrar uma hipótese qualquer acerca de algo, e então se tornam fogo e flama no que diz respeito a ela, achando que com isso tudo está resolvido. Para essas pessoas, ter uma opinião significa ser fanático por ela e abrigá-la no peito como convicção. Diante de algo inexplicado, exaltam-se com a primeira idéia de sua mente que pareça uma explicação: do que sempre resultam as piores conseqüências, sobretudo no âmbito da política. – Por isso cada um, atualmente, deveria chegar a conhecer no mínimo uma ciência a fundo: então saberia o que é método e como é necessária uma extrema circunspecção. (…).  (MA/HH, §635).
\end{quotation}

Este aforismo sinaliza que o objeto a ser reeducado pela prática científica seria uma certa  disposição afetiva, que Nietzsche credita à maioria das pessoas – uma disposição ao “fanatismo”, um precipitado “tornar-se flama e fogo” no trato com qualquer hipótese. O foco da prescrição, aqui, não está em absorver o máximo de informações científicas possível, mas incorporar a disciplina científica pelo exercício de “conhecer no mínimo \textit{uma} ciência a fundo”. Há portanto, um componente ético da vida científica que constitui uma peça fundamental do programa filosófico de \textit{Humano}. É possível dizer que o espírito livre se orienta por um tipo de normatividade epistêmica, que conduz sua relação com as crenças e, em última instância, suas ações; mas há que se ressaltar que essa normatividade epistêmica seria adquirida no interior de uma \textit{comunidade} ou \textit{ethos} científico, portanto de forma semelhante à aquisição de qualquer virtude prática. O convívio com esse ethos seria um requisito indispensável na criação de certa disposição de espírito virtuosa no trato com as crenças
\footnote{Não por acaso, ao tratar do tema do método, o filósofo mobiliza todo um vocabulário pertencente ao campo semântico da virtude. LOPES (2013) defende a tese de que em \textit{Humano}, assim como nas demais obras do chamado “período intermediário”, haveria uma primazia da normatividade epistêmica sobre a normatividade prática. Ressaltamos o fato de que essa normatividade epistêmica, assim como outras formas de normatividade, seria cultivada pelo convívio com um certo \textit{ethos}. De todo modo, estamos de acordo com Lopes a respeito da ideia de que as virtudes que protagonizam o referencial normativo na filosofia do espírito livre são eminentemente virtudes \textit{epistêmicas}. Como temos lembrado, essas virtudes seriam principalmente “cautela”, “moderação”, “modéstia”, “justiça” e “honestidade”. Ao que nos parece, essa primazia das virtudes epistêmicas é conferida por Nietzsche nesse momento devido a uma aposta na possibilidade de elas operarem uma remodelação da vida prática: “Do ponto de vista descritivo, Nietzsche parece estar neste momento comprometido com uma versão moderada de cognitivismo acerca das emoções. Nossos afetos e impulsos têm um componente cognitivo sobre o qual seria certamente um exagero afirmar que temos um controle absoluto, mas igualmente despropositado afirmar que não temos controle algum. Este componente cognitivo é o elemento mais superficial e moldável de nossa vida mental e pode, portanto, ser reformado com algum grau de sucesso. Mas para que esta reforma possa ser posta em andamento é necessário que nós mesmos, enquanto agentes epistêmicos responsáveis, recusemos nossa adesão a crenças que não foram formadas segundo métodos epistemicamente confiáveis, ou seja, métodos que não correspondam aos critérios estabelecidos no interior das comunidades científicas e aceitos por nossas melhores teorias científicas. Ou seja, é preciso que tenhamos já previamente cultivado o que poderíamos chamar de virtudes epistêmicas. Este é o pressuposto normativo com o qual Nietzsche parece operar neste momento de sua obra, e com isso chegamos à segunda motivação por trás de sua crítica à moralidade e que poderíamos chamar de motivação epistêmica. Ela pode ser descrita nos seguintes termos: a adesão às intuições morais fundamentais que conformam o sistema da moralidade no sentido pejorativo é uma violação de nosso compromisso, igualmente exigente ou mais fundamental, com os valores da integridade intelectual. Aqui Nietzsche subordina claramente a normatividade prática à normatividade epistêmica e confere primazia a esta como um desdobramento ou mesmo um coroamento daquela. Se tal primazia é concedida por Nietzsche em termos estratégicos, como forma de minar por dentro a própria moralidade, é algo que não se deixa facilmente discernir.” (LOPES, 2013, p. 120). Nesse artigo, LOPES discute ainda a possível inserção de Nietzsche na tradição da ética das virtudes, tema que não poderemos desenvolver aqui. }. 
Essa disposição de espírito seria necessária para livrar o homem da superstição, da exaltação dos afetos, e de uma série de práticas insãs de formação de crença que o filósofo associa a uma disposição natural incentivada pela visão moral e metafísica de mundo. 

	Esse projeto de reforma da vida prática através de um modelo de reforma intelectual depende de uma aposta na possibilidade de, através do ataque às crenças mal formadas, atingir o núcleo afetivo que constitui o motor da adesão a elas
\footnote{Essa aposta parece figurar nos dois volumes de \textit{Humano} e também em \textit{Aurora}, isto é, nesse momento, há uma certa confiança na capacidade da refutação de uma crença levar ao desengajamento em relação à mesma. Ver \textit{Aurora}, §95. No entanto, a partir de \textit{A Gaia Ciência}, Nietzsche parece retirar parte de sua confiança no sucesso da reforma afetiva via reforma intelectual, e passa a destinar uma atenção cada vez maior à criação de novas estratégias de enfrentamento do componente afetivo que envolve as crenças.}. 
	
	Mesmo assim, não há uma resposta evidente para questões relativas ao efeito do conhecimento sobre a vida prática. Não está dado imediatamente, por exemplo, qual seria o efeito do abandono da crença numa recompensa eterna pós-vida, ou do conhecimento de nossa semelhança com os outros animais, sobre as questões “como devo viver?”, “qual é a vida boa?”. Em concordância com a aproximação entre normatividade epistêmica e normatividade prática, desenvolvida ao longo de \textit{Humano}, o tom do livro sugere uma passagem pouco turbulenta entre aspectos descritivos do conhecimento e seus efeitos no campo da normatividade prática. Ao que nos parece, essa transição pouco problemática entre aspectos descritivos e normativos se deve em parte ao fato de que o projeto é destinado a uma experimentação no âmbito da vida \textit{pessoal} de \textit{um} personagem conceitual, o espírito livre.

Esse personagem é historicamente delimitado: ele é o europeu moderno, instruído, produto da Ilustração democrática, filho pródigo da cultura moral cristã. Devido tanto a um traço natural de “agudeza intelectual” (\textit{MA/HH}, §225) quanto ao fato de haver surgido após um processo de interiorização dos impulsos hostis
\footnote{Esse processo é tema de \textit{Genealogia da Moral}, que portanto oferece uma abordagem mais detalhada sobre o assunto.}, 
o espírito livre é pensado como um personagem rico em vivência subjetiva. Sua disciplina metódica não é pensada como uma mera arma contra a adesão a crenças mal fundadas, mas como o fio condutor que o torna “capaz de fazer da busca pelo conhecimento algo mais que um mero meio, mas a própria finalidade e justificação da vida.” (SANTOS, 2011, p. 140). São essas características que o fazem destinatário da agenda filosófica de \textit{Humano}; o que há de normativo nessa agenda vale enquanto aplicado ao modo de vida espírito livre.

	Um traço endêmico a \textit{Humano} é uma certa preocupação de Nietzsche com o risco de “superexcitação” da vida afetiva e mental
\footnote{“\textit{Na vizinhança da loucura}. – A soma dos sentimentos, conhecimentos, experiências, ou seja, todo o fardo da cultura, tornou-se tão grande que há o perigo geral de uma superexcitação das forças nervosas e intelectuais; as classes cultas dos países europeus estão mesmo cabalmente neuróticas, e em quase todas as suas grandes famílias há alguém próximo da loucura. Sem dúvida, há muitos meios de encontrar a saúde atualmente; mas é necessário, antes de tudo, reduzir essa tensão do sentir, esse fardo opressor da cultura, algo que, mesmo sendo obtido com muitas perdas, nos permitirá ter a grande esperança de um \textit{novo Renascimento}. Ao cristianismo, aos filósofos, escritores e músicos devemos uma abundância de sentimentos profundamente excitados: para que eles não nos sufoquem devemos invocar o espírito da ciência, que em geral nos faz um tanto mais frios e céticos, e arrefece a torrente inflamada da fé em verdades finais e definitvas; ela se tornou tão impetuosa graças ao cristianismo, sobretudo.” (MA/HH, §44). Provavelmente, há razões de cunho biográfico para essa preocupação de Nietzsche. A partir de \textit{Aurora}, o tema do risco da superexcitação dá lugar ao tema do aumento do “sentimento de poder”, tema de que trateremos adiante. Cf. BRUSOTTI, 2011.}. 
Essa preocupação se associa ao diagnóstico segundo o qual o ocidente sofreria de uma cultura de exaltação do sentimento (que o filósofo associa à cultura artística e religiosa cristã, como temos dito) e também de um fenômeno mais recente de “hipertrofia do sentido lógico”
\footnote{Cf. GT/NT, §13-18; HL/Co. Ext. II, §1, §9.}, 
uma combinação potencialmente explosiva, um risco à vida interior do sujeito. Na vizinhança desse tema, está o \textit{tropos} epicurista da superação da dor e do medo instilados pela superstição através do conhecimento e do cultivo de si
\footnote{Cf. MA/HH II – WS/AS, §16. Especialmente no segundo volume de \textit{Humano}, não só o epicurismo, mas as escolas socráticas, em geral, têm presença marcada, o que, é claro, está em sintonia com a agenda do livro, lembrando que essas escolas desenvolveram, cada uma a seu modo, o tema do cuidado de si, da desconfiança em relação à vida pública e da busca por um modo de vida mais “natural”. Infelizmente, não temos condições de desenvolver aqui o tema do traço helenista de \textit{Humano}.}. 
O filósofo então espera que o cultivo científico, enquanto capaz de criar um sentido de moderação no trato das crenças, possa reverberar também no trato com os afetos, e assim “resfriar a máquina” afetiva que chegou a todo vapor à modernidade. Dessa forma, o princípio de honestidade e moderação regeriam tanto a vida intelectual do espírito livre, quanto sua vida afetiva. 

O plano de experimento da liberdade de espírito se dá no âmbito da vida pessoal
\footnote{Há em \textit{Humano} um tom visível de desconfiança em relação à vida pública, que seria um espaço de “teatro” dos afetos, ambiente pouco favorável ao cultivo de moderação e honestidade intelectual que caracterizam o espírito livre. E este certamente não é representado como um “homem de ação”. O filósofo se mostra ciente, é claro, do grande desafio de repensar a vida pública, comum, sem a tutela de instituições tradicionais que então davam sinais de descrédito, como o cristianismo, de forma que por vezes chega a considerar a possibilidade de que a própria cultura científica tome para si a tarefa de criar novas metas culturais de alcance geral. No entanto, o experimento ético de liberdade de espírito não chega a ser uma prescrição universal, extensiva a todos os setores da sociedade: “mesmo que por vezes Nietzsche acene para a possibilidade de uma redefinição das estruturas culturais a partir da sobreposição dos valores epistêmicos aos valores morais, a posição mais comumente encontrada em seus escritos do período intermediário é a de encarar sua época como um momento de transição onde, na impossibilidade de efetivamente colher das ciências os valores norteadores da vida, resta como melhor opção o caminho da auto-experimentação e do conhecimento de si.” (SANTOS, 2011, p. 146). }. 
Há, é claro, um esforço filosófico em pensar como esse modo de vida será conduzido sem recorrer à tutela da ética cristã
\footnote{“Maquiavel e Montaigne são dois importantes precursores de Nietzsche em sua empreitada de pensar uma ética emancipada da eticidade cristã; Maquiavel fez para a moralidade pública o que Montaigne ousou na esfera da moralidade privada, ou seja, pensar uma eticidade emancipada da tutela do cristianismo.” (LOPES, 2013, p. 107).}, 
mas, por outro lado, está vetada a possibilidade de um simples retorno à “natureza”. O processo de autosseleção não tem como contrapartida a proposta de dar livre curso aos impulsos e afetos, até porque os impulsos e afetos atuais já passaram por um processo social de remodelagem. A capacidade intelectual e o temperamento individual é que desempenharão o papel decisivo na criação de uma dietética e de um arranjo pessoal de impulsos, hábitos e valores.

	Uma vez que a antinomia entre seleção social e natural teria feito do homem, em geral, um animal doente, é de se imaginar que a autosseleção, enquanto liberdade possível, seria do interesse de todos. Mas essa não parece ser a aposta de Nietzsche. Um argumento possível para essa restrição é a ideia de que o “meta-hábito” ou “instinto gregário” talvez tenha se tornado tão forte na maioria das pessoas que seu efeito seja incontornável
\footnote{“(...) identificamos aquilo que Nietzsche chama de instinto de rebanho como sendo um tipo de predominância impulsiva segundo a qual os indivíduos se comportariam primordialmente como função de um todo. Por conseguinte, a moral do desinteresse predominante entre o tipo de rebanho seria fundamentalmente algo pessoalmente proveitoso a certos indivíduos, nos quais certa inclinação a ser função é dominante.” (SANTOS, 2011, p. 21).}.
Por outro lado, a disposição à liberdade de espírito é vista como algo raro. 

	O entusiasmo pelo conhecimento e pela prática científica que se vê em \textit{Humano} parece se sustentar na ideia de que a ciência criou, enfim, um \textit{ethos} ou um padrão de disposição de espírito capaz de acolher os traços raros que caracterizam o espírito livre: sua disposição à dúvida; sua melhor aceitação da mudança de hábitos e opiniões; sua busca por uma normatividade que não se paute pela fé, mas pela exigência de “razões”; seu apreço pela honestidade e moderação como antídotos à exaltação e à precipitação do juízo; sua agenda não metafísica de investigação do mundo; sua visão não teocêntrica da ação e da moral.

Contudo, o modo de vida contemplativo representado pelo personagem “espírito livre” não parece se limitar ao modo de vida do profissional científico. Em \textit{Humano} vemos o momento de maior aproximação entre filosofia e ciência na obra de nietzschiana; o fato de o livro se destinar a lançar um programa de “filosofia histórica”, nos termos de Nietzsche, é a maior evidência disso. Ainda assim, nossa impressão é a de que conquanto a ciência exerça um papel fundamental na estruturação do programa filosófico de \textit{Humano}, esse é um papel \textit{instrumental}, e o programa é, afinal, um programa filosófico. A ciência é acionada como provedora de um \textit{método}, primeiro no sentido de modo (histórico) de visar os condicionantes biológicos e culturais que produziram a contemporaneidade nietzschiana, depois no sentido de meio de cultivo da disciplina de espírito. Essas duas aplicações do pensamento científico estão, afinal, a serviço do espírito livre, que é o sujeito do experimento propriamente filosófico de criar valores e modos de vida.

\chapter{Novas tensões na filosofia do espírito livre}
\label{cap2}

No capítulo anterior, buscamos destacar as linhas gerais do programa filosófico que Nietzsche traz a público com \textit{Humano, demasiado humano}, momento de sua obra em que se anuncia pela primeira vez uma agenda bastante clara de aproximação entre filosofia e ciência, sendo que os principais pontos dessa agenda seriam: o recurso à ciência como meio de informar uma visada histórica sobre o material biológico e cultural que constitui o humano; a iniciação na prática científica como meio de disciplinamento dos impulsos cognitivos e aquisição de “virtudes epistêmicas”; o uso de um referencial científico para delimitação do campo de investigação, isto é, uma agenda de produção de conhecimento sobre as “coisas próximas”, aquelas passíveis de investigação científica, e de rechaço à especulação no campo das chamadas “questões últimas” da metafísica; a aplicação desses dois efeitos do convívio com a ciência – disciplina do pensamento e conhecimento sobre as coisas próximas – no âmbito individual do cuidado e si e na formação de um modo de vida contemplativa. Uma vez que o propósito do presente trabalho consiste na comparação da agenda naturalista lançada em \textit{Humano} com sua reelaboração no igualmente programático \textit{Além de Bem e Mal}, consideramos necessário apontar as principais tensões e inovações filosóficas que vêm à tona no pensamento nietzschiano entre uma obra e outra. Apontamos tais inovações, enquanto objeto deste capítulo, sem a menor pretensão de oferecer uma análise completa e detalhada das reviravoltas e desdobramentos do (vasto) pensamento nietzschiano; estamos, acima de tudo, em busca de pistas que permitam compreender melhor que tipo de tensão, preocupação ou intuição filosófica atua como propulsora dessa revisão, isto é, faz com que o programa relativamente simples e inequívoco de \textit{Humano} se transfigure em uma obra de extrema complexidade e densidade como \textit{Além de Bem e Mal}.

\section{Sentimento de Poder e Paixão do Conhecimento }

O elemento que mais contribuiu para uma guinada no quadro conceitual apresentado em \textit{Humano} talvez seja o conceito de “sentimento de poder”. Este conceito ganha espaço na sequência dos dois volumes de \textit{Humano}, com \textit{Aurora} (1881); trata-se, a princípio, de uma tese psicológica que tem implicações descritivas e normativas no pensamento nietzschiano.

	A inserção do conceito de sentimento de poder na filosofia nietzschiana tem como efeito imediato embaralhar a narrativa quase-utilitarista desenvolvida em \textit{Humano}. Na verdade, neste livro encontra-se já uma ocorrência discreta da expressão
\footnote{No aforismo 104 lê-se: “(...) na medida em que há um \textit{prazer} na ação (sentimento do próprio poder, da intensidade da própria excitação), a ação ocorre para conservar o bem-estar do indivíduo, sob um ponto de vista similar ao da legítima defesa (...).” (MA/HH, §104). Note-se que aqui a noção de sentimento de poder aparece estritamente associada à noção de prazer. Oscar Santos observa que: “Desta forma, muito embora já seja notável a antecipação de sua disposição a substituir o prazer pura e simplesmente considerado pelo sentimento de poder [\textit{Gefühl der Macht}] enquanto fator motivacional determinante do agir humano, este primeiro posicionamento em favor da autoconservação reflete a então estreita relação de Nietzsche com uma perspectiva quase-utilitária de inspiração naturalista darwiniana e que será abandonada aos poucos no decorrer do período intermediário de sua obra. É preciso, portanto, ter em conta que a relação de Nietzsche com o utilitarismo não apresenta uma posição contínua ou fixa ao longo do desenvolvimento de sua obra, sendo justamente durante o período do espírito livre, mais especificamente nos dois volumes de \textit{Humano, Demasiado Humano}, que o pensador alemão se mostrará mais simpático a alguns de seus traços característicos.” (SANTOS, 2011, p. 34-35)}. 
No entanto, é a partir de \textit{Aurora} que o conceito de “sentimento de poder” ganhará um lugar de destaque, tornando-se até mais operatório que a noção de \textit{prazer} enquanto elemento motivacional da ação, e forçando a perspectiva da “autoconservação” para longe do horizonte filosófico nietzschiano. Isto porque, uma vez que o sentimento de poder passa a ser visto como meta comum dos impulsos em geral, o recurso à autoconservação na descrição das ações é suplantado pela ideia de que, na busca por incremento do sentimento de poder o organismo pode ser expor até mesmo à dor e a situações potencialmente autodestrutivas
\footnote{Santos aponta que: “ao buscar definir o sentimento de poder como a maior inclinação a partir da qual o ser humano estabelece seus valores, o que se exclui do processo não é o prazer, mas antes o fator utilitário – isso desde que se firme a independência entre as duas noções, ou seja, que se pense, ao contrário de utilitaristas clássicos como Stuart Mill, que aquilo que é mais prazeroso não seja necessariamente coincidente com o que é mais útil, e nem tampouco excludente em relação ao desprazer.” (SANTOS, 2011, p. 47.) Estamos de acordo com Santos no que diz respeito ao gradual afastamento de Nietzsche em relação ao quadro conceitual do utilitarismo clássico e à aproximação entre os fenômenos de prazer e desprazer no desenrolar de sua filosofia. No entanto, parece-nos que o vocabulário da “utilidade” tem ainda grande destaque em \textit{Aurora}; o que muda, talvez, é que na perspectiva de aumento do sentimento de poder o foco não está naquilo que é meramente útil à sobrevivência, mas naquilo que é útil à intensificação do sentimento, pensada sobretudo no plano da vivência individual. }. 
	
	O recurso à psicologia do sentimento de poder serve à explicação de fenômenos como a crueldade, a compaixão e o ascetismo, comportamentos aparentemente nocivos para os quais Nietzsche buscava uma explicação naturalista já em \textit{Humano}. Diluindo as fronteiras entre egoísmo e altruísmo, Nietzsche vê o compassivo como alguém que aumenta seu sentimento de poder ao ajudar ou apenas comparar-se com o objeto de sua compaixão
\footnote{“(...) O acidente do outro nos ofende, ele nos provaria nossa impotência, talvez nossa covardia, se não o socorrêssemos. Ou já traz consigo uma diminuição de nossa honra perante os outros ou nós mesmos. Ou no acidente e sofrimento do outro há uma indicação de perigo para nós; e já como sinal da vulnerabilidade e fragilidade humana podem ter efeito penoso sobre nós. Rechaçamos esse tipo de dor e de ofensa, e a ele respondemos com um ato de compaixão, em que pode haver uma sutil legítima defesa e mesmo vingança. (…) o prazer surge à visão de um contraste à nossa situação, à idéia de que podemos ajudar se quisermos, ao pensar no louvor e na gratidão, caso ajudássemos; surge da atividade mesma de auxílio, enquanto o ato é bem-sucedido e, como algo de êxito progressivo, em si mesmo dá alegria a quem o realiza; mas, sobretudo, do sentimento de que nossa ação põe termo a uma revoltante injustiça (o desafogo da revolta já reanima). Tudo isso e ainda coisas mais sutis, é 'compaixão' (…).”. \textit{M/A}, §133.}. 
No mesmo sentido, também o asceta, apesar da aparente debilidade voluntária de impulsos, poderia ser visto como exercendo poder sobre si mesmo, e assim obtendo seu modo particular de fruição
\footnote{“(...) O empenho por distinção acarreta \textit{para o outro} – especificando somente alguns degraus dessa longa escada – : tormentos, depois golpes, depois horror, espanto angustiado, surpresa, inveja, admiração, elevação, alegria, serenidadade, riso, derrisão, escárnio, mofa, aplicação de golpes, imposição de tormentos – aqui, no final da escada, encontra-se o asceta e o mártir; ele sente o mais alto prazer em suportar ele mesmo, como conseqüência de seu impulso por distinção, aquilo que sua contrapartida no primeiro degrau da escada, o \textit{bárbaro}, inflige a um outro, no qual e ante o qual quer se distinguir. (…) Pois a felicidade, concebida como o mais vivo sentimento de poder, foi talvez maior nas almas dos ascetas supersticiosos do que em qualquer outro lugar. (...)” M/A, §113. }. 
Já a crueldade é pensada como uma forma “festiva” de exercício de poder e como que um alívio para homens “em estado de guerra, numa comunidade pequena e sempre ameaçada, onde reina a mais severa moralidade” (\textit{M/A}, §18)
\footnote{Em outra passagem de \textit{Aurora}, lê-se: “(...) Oh, quanta supérflua crueldade e tortura animal teve origem nas religiões que inventaram o pecado! E nos homens que quiseram, com isso, ter a mais alta fruição do seu poder!” (M/A, §53).}.

Até mesmo o fenômeno de submissão à autoridade (de um chefe, de uma instituição, de uma tradição) poderia ser entendido como uma estratégia de intensificação do sentimento de poder, pois em alguns casos ela evitaria a “dispersão da vontade”, como explica Lopes (LOPES, 2008, p. 406). Assim, os diferentes modos de vida podem ser vistos como arranjos que efetivam diferentes graus de intensidade de sentimento de poder
\footnote{A respeito do desenvolvimento do conceito de sentimento de poder na filosofia nietzschiana, Lopes aponta que: “A recusa de uma distinção ontológica entre as diferentes formas de vida não deve impedir uma hierarquização, que remete a diferenças puramente gradativas e de intensidade na fruição do poder. Estes experimentos do início da década de 80, circunscritos ao âmbito da psicologia do sentimento de poder, preparam o terreno para experimentos mais ousados no âmbito da ontologia, com a introdução da hipótese da vontade de poder. Esta hipótese, que propõe uma extensão cosmológica dos resultados obtidos no campo dos fenômenos antropológicos, é anunciada pela primeira vez no \textit{Zaratustra} como um substituto para a concepção schopenhaueriana da vontade de viver e desenvolvida com algum detalhe apenas nos póstumos da segunda metade da década de 80.” (LOPES, 2008, p. 399.)}. 

A perspectiva do aumento do sentimento de poder, como motivação primária da ação, desestabiliza, por consequência, o imperativo de moderação dos afetos, que constitui parte central do aspecto normativo de \textit{Humano}, livro no qual a ciência era invocada na tarefa de “resfriar a máquina”. Marco Brusotti resume assim o ponto de \textit{Humano}: “Distenção, alívio, arrefecimento são, portanto, as tarefas. Esta tarefa espelha a atitude do livro como um todo: ele aspira a uma forma de vida que, ao mesmo tempo, paira sobre paixões sublimadas.” (BRUSOTTI, 2011, p. 41). Mais adiante, o autor mostra que o propósito de moderar os afetos por meio do pensamento científico não desaparece do horizonte filosófico nietzschiano, ao menos durante as obras do período intermediário; o elemento novo, após \textit{Humano}, é que o filósofo mostra-se mais cônscio do risco de que também o arrefecimento dos afetos atinja um ponto exagerado, acarretando num empobrecimento motivacional. A prática científica ainda é vista como um antídoto para a super exaltação dos afetos e o fanatismo, mas, por outro lado, aposta-se na possibilidade de que a própria ciência ofereça recursos alternativos para manutenção do tônus da vida interior.

É notável, em \textit{Aurora}, a preocupação em buscar novas formas de intensificação do sentimento de poder e, curiosamente, a ciência será novamente invocada aqui, desta vez em razão da aposta de que a própria \textit{paixão do conhecimento} possa servir de motor afetivo para o espírito livre. A paixão do conhecimento é a alternativa que se abre ao espírito livre uma vez que foram rechaçados os estimulantes da moralidade, tradicionalmente ancorados numa cultura metafísica, e que não está no horizonte filosófico nietzschiano a possibilidade de um “retorno à natureza” em termos rousseaunianos. Uma vez que, em \textit{Aurora}, o modo de vida contemplativa do espírito livre ganha em tônus motivacional por meio dessa paixão nova, a paixão do conhecimento, fica para trás o propósito de buscar “uma vida muito mais simples e mais pura de paixões” (MA/HH, §34). O engajamento passional com a busca por conhecimento, que tem a verdade como objeto de desejo, em última instância, é proposto num cenário de forte ceticismo quanto à possibilidade real de conquista da verdade, dada a radicalidade dos mecanismos de autoengano e simplificação que são, a um tempo, os próprios meios de conhecimento de que o humano dispõe (tema de que tratamos no capítulo anterior). Há, portanto, algo de trágico nesse modo de vida apaixonado pelo conhecimento,  o que confere a \textit{Aurora} um ambiente bastante diferente de \textit{Humano} enquanto projeto de vida contemplativa. Seguindo essa linha, Oscar Santos nota que:

\begin{quotation}
Neste sentido, podemos dizer que, em oposição à tranqüilidade da alma visada nos escritos de \textit{Humano, Demasiado Humano}, a vida dedicada ao conhecimento assume em \textit{Aurora} certo caráter heróico, uma vez que a paixão do conhecimento impõe ao indivíduo por ela tomado que continue sua investigação da realidade, mesmo que tenha consciência de seu parco avanço, da possibilidade do perecimento e do sofrimento que acompanha este tipo de renúncia aos valores da tradição. (SANTOS, 2011, p. 150).
\end{quotation}

No prefácio acrescentado a \textit{Humano} em 1886, Nietzsche creditará a presença de um imperativo de moderação no livro ao momento de convalescença por que então passava. Comentadores têm apontado, ainda, o impacto da leitura de Pascal como determinante dessa passagem de um tipo de eudaimonismo montaigniano para uma perspectiva heroica de intensificação da vida interior e “alargamento dos espaços da alma”\footnote{Cf. LOPES (2008), BRUSOTTI (2011).}. O modelo de intensificação que Nietzsche encontra em Pascal é aquele das “contrariedades”, pelas quais o filósofo francês é conhecido. Nietzsche identifica a fonte de energia espiritual de Pascal na comparação imaginária entre o “eu odiável” e o “Deus amável”. Exaptando o modelo pascaliano de contrariedade para o ambiente pós-metafísico do espírito livre, aposta na possibilidade de criação de um motor espiritual a partir da tensão entre a percepção da radicalidade do erro e o espírito científico de pesquisa da verdade (em última instância inalcançável).

Nietzsche propõe então um modelo de intensidade espiritual inspirado em Pascal que tenha como contrariedade motora não os polos ódio de si/amor de Deus, mas os polos paixão do conhecimento/necessidade de ilusão, ou ainda integridade intelectual/onipresença do erro. O modo de vida impulsionado por tal contrariedade consiste em buscar o conhecimento e avizinhar-se da ciência, embora o espírito livre esteja ciente da necessidade da ilusão e do erro, uma vez que estes estariam entranhados na história biológica e cultural da espécie humana e cumpririam ainda uma função pragmática. O funcionamento da contrariedade depende de que cada pólo acabe por reforçar o outro: assim, a prática científica traz à tona, sempre em maior grau, o fato da penetração do erro na vida, e por outro lado, a percepção da necessidade do erro traz a aceitação da paixão (no caso, a paixão pelo conhecimento) como instrumento de vitalidade. Como nota Marco Brusotti, 

\begin{quotation}
A polaridade não é nova, pois trata-se sempre ainda de autodesprezo e orgulho. A ciência infringe ao homem a conhecida 'ferida narcísica'”, isto é, a ciência traz reiteradamente à tona o fato de que o homem vive mergulhado em erro, principalmente no erro de achar-se superior aos outros animais, munido de uma razão metafísica, mais próximo do divino que do animal, etc. (BRUSOTTI, 2012, p. 45)
\footnote{Sobre o tema da “ferida narcísica” em Nietzsche ver, por exemplo, \textit{Aurora} §7: “\textit{Esse} tipo de sentimento do espaço, então, é reduzido cada vez mais pela ação da ciência: de modo que dela aprendemos a perceber a Terra como pequena, e o próprio sistema solar como simples ponto.” (M/A, §7).}.
\end{quotation}

Brusotti conclui: “A questão é, antes de tudo, se a ciência pode mediar para o homem uma nova autoestima, que nos permita suportar com 'mais orgulho' o 'estado de aviltamento humano'” (BRUSOTTI, 2012, p. 45). A ideia é que, na medida em que expõe a ignorância humana, a ciência poderia inspirar algo como um arranjo tenso entre altivez e insatisfação, curiosidade e desconfiança, criando por fim um mecanismo de propulsão espiritual
\footnote{“Este engajamento erótico gera na alma uma tensão interna que cria as condições optimais para o exercício da auto-superação. A distância que se estabelece entre a imagem daquilo que se venera e a imagem daquilo que se é tem um efeito disciplinador sobre a alma. Obter este efeito sem recorrer a meios ilusórios é justamente o que caracteriza a ambição de Nietzsche neste momento. A intensidade deste engajamento erótico permite à alma expandir indefinidamente seus espaços interiores. É possível propor experimentos de tal magnitude fora do ambiente do ideal ascético? O homem do conhecimento é capaz de cultivar esta atitude de reverência por si mesmo e por seu objeto?” (LOPES, 2008, p. 401-402).}. 
A virtude que guia o experimento de paixão pelo conhecimento é, novamente, honestidade ou integridade intelectual (\textit{Redlichkeit}).

Vale lembrar, ainda, que a confrontação com Pascal não se dá por acaso; ela é, afinal, uma confrontação com o cristianismo. O interesse por Pascal se insere no plano de buscar novas formas de tonicidade espiritual para a então nascente cultura pós-metafísica. Nietzsche reconhece em Pascal uma extrema acuidade de observação psicológica, que se faz notar na forma como o filósofo francês desvela os mecanismos de autoengano do “Eu” vaidoso (e portanto, segundo sua conta, odiável); por esta razão, Pascal é, para Nietzsche, também um expoente de integridade intelectual, embora visto como uma “vítima do cristianismo” por ter, afinal, sucumbido ao “sacrifício do intelecto” na medida em que aceitou o imperativo de acreditar em artigos da fé cristã que ele mesmo via como “absurdos”. Além disso, é do interesse de Nietzsche superar a aversão à individualidade, vista como a marca cristã do pensamento pascaliano, pois está no cerne da filosofia imoralista nietzschiana o interesse pelos grandes indivíduos, modelos na busca por um modo de vida individual
\footnote{Lembramos o comentário de Eduardo R. de Melo: “Ora, Nietzsche vê no cristianismo, pela sua rejeição ao 'eu', considerado sempre digno de ódio, uma recusa ao indivíduo, debilitando-o e anulando-o para que possa denunciar e enumerar o mal e o hostil, custoso, luxuoso da antiga forma de existência individual. Só assim, para o cristianismo, pode o homem das ações simpáticas, desinteressadas, de utilidade geral ser considerado o homem moral: o cristianismo cultua a filantropia e busca, com isso, a gestação de uma forma de vida mais econômica, menos perigosa, mais homogênea, mais unitária, o que só se encontra nos grandes corpos e seus membros. A reação ao outro e à sua imprevisibilidade mostra, agora, um espraiamento, tornando-se medo a toda individualidade e originalidade, porque ambas trazem consigo um elemento de instabilidade à ordem estabelecida. Percebe-se, portanto, uma correlação necessária entre, de um lado, negação do outro, da vida individual e do 'eu' e, de outro lado, a exigência de homogeneização da vida social. Consequentemente, estabelece-se uma contraposição entre, de um lado, o homem livre, individual, amoral, mau, porque imprevisto e imprevisível, e, de outro, o homem tradicional, moral, bom, porque confiável e costumeiro.” (MELO, 2004, p. 64).}.

Curiosamente, o tema da paixão do conhecimento passa a dividir um espaço cada vez maior com a arte; ciência e arte protagonizam o programa filosófico do próximo livro de Nietzsche, \textit{A Gaia Ciência} (1882). Neste livro, em que se anuncia a “morte de Deus”, encontra-se a mesma busca por “fontes energéticas culturais” não-religiosas, por assim dizer, e sem dúvidas, a ciência continua contando como um ingrediente de extrema importância nessa busca. 

	Quer dizer, tudo indica que em grande parte Nietzsche credita ainda à ciência os mesmos efeitos benéficos já mencionados em \textit{Humano}: ela serviria como fonte de conhecimentos sobre os mecanismos biológicos e culturais atuando na história humana e como plasmadora de uma disposição de espírito avessa à superstição, à submissão intelectual e ao fanatismo. Além disso, Nietzsche não parece abandonar o projeto de uma vida contemplativa na vizinhança da ciência. O diálogo com as ciências humanas e naturais reaparece em várias passagens do livro. A dúvida que parece pairar sobre esse projeto filosófico, por vezes, diz respeito à capacidade da própria paixão pelo conhecimento, por si só, manter o apelo desse modo de vida, sem que acabe por “tiranizar” os outros impulsos, ou levar a uma recaída na moral.
	
	Além disso, ao que parece, Nietzsche começa a duvidar de que a dedicação à disciplina do pensamento e à investigação científica seja suficiente, sedutora ou estável o bastante, para alavancar a mudança cultural que adviria da passagem para uma cultura pós-metafísica, até porquê essa dedicação envolve exigências não universalizáveis. O filósofo se mostra cioso da raridade dessa paixão e da raridade da virtude de integridade intelectual, de que o exercício da paixão do conhecimento depende. Ora, na verdade, paixão do conhecimento e honestidade intelectual são traços do espírito livre, sendo que desde sua concepção em \textit{Humano}, o espírito livre é pensado, por definição, como um tipo raro, um tipo de exceção. A idiossincrasia do espírito livre não chega a colocar um grande problema para \textit{Humano}, talvez, porque este é um livro dedicado ao próprio espírito livre, e no âmbito de sua vida pessoal. À medida em que se fortalece na filosofia de Nietzsche intenções de intervenção na cultura, num plano mais geral, o conflito entre tipo de exceção e tipo normal/moral ganha destaque enquanto um problema que tem implicações não só nas estratégias retóricas de que o filósofo se vale, mas no próprio rumo a que conduz sua filosofia
\footnote{Lopes nota que tem início aqui uma nova agenda filosófica: “Uma característica importante desta nova agenda filosófica é a retomada gradativa das pretensões políticas da vida contemplativa. Este processo de substituição da agenda filosófica é gradual, tendo seu ponto culminante em \textit{Além de Bem e Mal} e \textit{Para a Genealogia da Moral}. Mas não seria um exagero afirmar que este processo é deflagrado pela disposição de Nietzsche de rivalizar com o cristianismo de Pascal.” LOPES, 2008, p. 389.}.

\section{Um pacto de convivência e a arte do contrapeso}

Ao longo de \textit{Aurora} e \textit{A Gaia Ciência}, a definição do ser moral como aquele que vive como função da regra é reafirmada
\footnote{Assim, em \textit{Aurora} lê-se: “(...) a moralidade não é outra coisa (e, portanto, \textit{não mais}!) do que obediência a costumes, não importa quais sejam; mas costumes são a maneira \textit{tradicional} de agir e avaliar. Em coisas nas quais nenhuma tradição manda não existe moralidade; e quanto menos a vida é determinada pela tradição, tanto menor é o círculo da moralidade. O homem livre é não-moral, porque em tudo quer depender de si, não de uma tradição: em todos os estados originais da humanidade, 'mau' significa o mesmo que 'individual', 'livre', 'arbitrário', 'inusitado', 'inaudito', 'imprevisível' (...)”. (M/A, §9.) Na mesma direção: “Que o indivíduo estabelecesse seu próprio ideal e dele derivasse sua lei, seus amigos e seus direitos (…) Ser hostil a esse impulso para um ideal próprio: tal era, então, a lei de toda moralidade.” (FW/GC, §143).}, 
e se expande a investigação sobre os mecanismos psicológicos envolvidos tanto na submissão quanto no desvio à regra, por meio do vocabulário do poder (relações de comando e obediência). Assim, o tipo normal é apresentado como aquele que frui seu sentimento de poder ao tornar-se função de algo, e goza de boa consciência pela própria obrigatoriedade de sua submissão – sua injustiça (uma injustiça que Nietzsche reconhece como necessária, devido a uma série de mecanismos psicológicos) estaria na crença de que o dever a que se submete aplica-se incondicionalmente a todos (Cf. \textit{FW/GC}, §5). Já o tipo de exceção se caracterizaria por ser movido por paixões singulares, pelo que há de mais pessoal e caprichoso em termos de gosto e julgamento; a singularidade dos meios que aplica na busca de uma intensidade incomum de sentimento de poder está em dissonância com a facilidade sempre maior de seguir a regra, e faz com que o tipo de exceção assuma eventualmente grandes riscos.
	
	O contraste entre tipo normal e tipo de exceção torna patente o afastamento de Nietzsche em relação ao pensamento utilitarista; a perspectiva de mera conservação da vida passa a ser vista como típica de constituições menos intensas em sentimento de poder, e não como um dado universal do vivente. Nietzsche não nega, é claro, a existência de um modo de vida pautado pelo cálculo racional das vantagens que garantiriam a sobrevivência – este é o modo de vida comum, a regra. A astúcia da autoconservação faz com que o tipo normal realize sua vontade somente enquanto função de um órgão mais poderoso (um líder, uma tradição, religião, etc.); ao transferir, por assim dizer, a força volitiva de seus impulsos a uma regra exteriormente fixada, o tipo normal tem a impressão de agir de forma absolutamente necessária. Por outro lado, o filósofo chama atenção para a diferença de temperamento do “tipo nobre”, cuja disposição para a grande paixão faz com que não se oriente prioritariamente por um cálculo racional a serviço da conservação da vida
\footnote{Lopes resume o ponto: “A grande paixão, por sua vez, recusa o cálculo utilitário e o princípio da autoconservação como sinais de uma prudência excessivamente egoísta e vulgar. Nietzsche estabelece este contraste entre as naturezas nobres e vulgares nos primeiros aforismos de \textit{A Gaia Ciência}. A força derivada da submissão a um dever incondicional é contrastada com a força derivada da paixão. A primeira exige a anulação de toda escala pessoal de valores como forma de ocultar a própria fraqueza ou transmutá-la em força: o valor não é referido ao próprio querer, mas a uma autoridade qualquer, que pode ser a autoridade de uma pessoa, de uma instituição ou ainda de um preceito moral metafisicamente fundado. O importante é que a responsabilidade pelos valores seja transferida para outra instância que não a da própria vontade, e que esta instância esteja em condições de exigir para si uma adesão incondicional. Justamente por isso os indivíduos que pregam este tipo de submissão são os opositores naturais do ceticismo e do esclarecimento moral. O espírito forte, por sua vez, é aquele cuja força se expressa na afirmação de sua própria vontade, sem recorrer à ilusão de um fundamento último para o seu querer. Sua paixão é rara e demonstra a idiossincrasia de um gosto excessivamente singular. Entregue à sua paixão, o espírito nobre é incapaz de compreender que o restante da humanidade não esteja sujeito às mesmas leis que movem sua vida interior. Esta é sua forma de parcialidade, sua eterna injustiça.” (LOPES, 2008, p. 407).}.
Por desviar-se da lógica da conservação, o “tipo nobre” seria algo de difícil compreensão para o “tipo normal”; o tipo nobre, por sua vez, frui com tanta intensidade sua própria paixão que não pode entender que ela não seja também a paixão de todos. A incompreensão mútua entre os tipos é assim apresentada:

\begin{quotation}
A natureza vulgar se caracteriza por nunca perder de vista a sua vantagem e pelo fato de este pensamento de uma vantagem e finalidade ser até mais forte que os mais fortes impulsos nela existentes: não permitir que esses impulsos a desencaminhem para ações despropositadas – eis sua sabedoria e seu amor-próprio. Comparada a ela, a natureza superior é a \textit{mais insensata}: – pois o indivíduo nobre, magnânimo, que se sacrifica, sucumbe mesmo a seus instintos, e em seus melhores momentos a sua razão \textit{faz uma pausa}. Este possui alguns sentimentos de prazer e de desprazer tão fortes, que o intelecto tem de silenciar ou de servi-los: o coração lhe toma o lugar da cabeça e fala-se de 'paixão'. (…) A desrazão ou razão oblíqua da paixão é aquilo que o vulgar despreza no nobre, mais ainda quando esta se volta para objetos cujo valor lhe parece fantástico e arbitrário. (…) a natureza superior tem uma peculiar medida de valor. Além disso, ela geralmente acredita que \textit{não há} uma peculiar medida de valor em sua idiossincrasia do gosto, e estabelece seus valores e desvalores como aqueles de validade geral, tornando-se incompreensível e pouco prática. (FW/GC, §3)
\footnote{O livro V de GC traz uma crítica mais direta contra o utilitarismo evolucionista. Mas vale lembrar que esse livro foi acrescentado à obra em 1886, cerca de 4 anos após a redação dos primeiros aforismos de GC, e ao mesmo tempo em que Nietzsche trabalhava em \textit{Além de Bem e Mal}. Por esta razão, o livro V compartilha do tom característico de ABM, apresentando uma retórica mais agressiva e um uso mais incisivo de certos pressupostos ontológicos, ao que parece. A nosso ver, o que faz com que esse acréscimo não tenha um aspecto totalmente alienígena no interior do livro é o fato de que Nietzsche já havia trabalhado uma tipologia que opõe o tipo conservador ao tipo passional, que “sucumbe aos próprios instintos”. Este parece ser o contexto necessário, se não suficiente, para tal crítica a Darwin e outros evolucionistas, que Nietzsche desenvolverá mais adiante, expressando um ponto de vista que parece se manter até o final da obra nietzschiana: a de que os evolucionistas foram demasiado provincianos em sua compreensão do processo evolutivo, tendo introduzido sub-repticiamente os traços de sua própria moral numa descrição que pretende ser uma descrição geral dos processos vitais. No livro V, Nietzsche combate a ideia de que os seres vivos estejam constantemente engajados numa luta pela mera manutenção da vida própria: “Querer preservar a si mesmo é expressão de um estado indigente, de uma limitação do verdadeiro instinto fundamental da vida, que tende à \textit{expansão do poder} e, assim querendo, muitas vezes questiona e sacrifica a autoconservação.” (FW/GC, §349). Vale ressaltar que expressões como “verdadeiro instinto fundamental da vida” destoam do fenomenismo mais comumente encontrado nos outros livros de GC. Richardson aponta aqui uma má compreensão do sentido do evolucionismo darwiniano por parte de Nietzsche, lembrando que o que está em questão, segundo Darwin, é a reprodução e transmissão dos caracteres adquiridos, e não a mera “existência”. (RICHARDSON, 2004, p. 24.) Vale lembrar que Nietzsche provavelmente teve apenas um acesso de segunda mão à teoria de Darwin, mas teve contato direto com o texto de outros pensadores evolucionistas, como Herbert Spencer. De toda forma, parece-nos que Nietzsche leva sua filosofia a um ponto de vista próprio, que mantém o sentido básico de evolução enquanto “desenvolvimento” –  uma intuição que o filósofo credita, na verdade, a Hegel: “(...) pois sem Hegel não haveria Darwin. (…) Nós alemães, somos hegelianos, mesmo que nunca tivesse havido um Hegel, na medida em que (à diferença dos latinos) damos instintivamente ao vir-a-ser, ao desenvolvimento, um valor mais profundo e mais rico do que àquilo que 'é' – nós mal acreditamos que se justifique o conceito de 'ser' (...)”. (FW/GC, §357). Ao que nos parece, é essa intuição do mundo como vir-a-ser que Nietzsche retém de Hegel, mas o faz em termos naturalizados, com um sentido não-metafísico de necessidade causal, ficando afinal mais próximo do evolucionismo do que da “teodiceia do espírito” hegeliana, na maior parte de sua obra. Mas Nietzsche vai além do quadro conceitual evolucionista ao apontar a tendência à “expansão do poder” como característica básica de todos os impulsos, e desenvolver uma tipologia que representa diferentes graus e diferentes arranjos adaptativos na busca pelo poder.}.
\end{quotation}

Ao não se dar conta de sua própria idiossincrasia, o tipo nobre exerce também sua forma de injustiça. A própria exigência ou pressuposição de honestidade intelectual reaparece aqui como o tipo particular de injustiça que o filósofo assume em si:

\begin{quotation}
\textit{A consciência intelectual}. – Continuo tendo a mesma experiência e me rebelando igualmente sem cessar contra ela, não desejo acreditar nela, ainda que me seja palpável: \textit{a grande maioria das pessoas não tem consciência intelectual}; e freqüentemente quis me parecer que se alguém a exige, nas mais populosas cidades, acha-se tão só como no deserto. Cada qual olha para você com olhar estrangeiro e prossegue no uso da sua balança, chamando a isso de bom e àquilo de mau; ninguém se enrubesce, quando você dá a entender que os pesos não estão justos – tampouco há indignação contra você: talvez riam de sua dúvida. Quero dizer: \textit{a grande maioria} não acha desprezível acreditar isso ou aquilo e viver conforme tal crença, \textit{sem} antes haver se tornado consciente das últimas e mais seguras razões a favor ou contra ela, e sem mesmo se preocupar depois com tais razões – os mais talentosos homens e as mais nobres mulheres também fazem parte dessa grande maioria. Mas que significam bondade, finura e gênio para mim, quando a pessoa que tem essas virtudes tolera em si mesma sentimentos frouxos ao crer e julgar, quando a \textit{exigência de certeza} não constitui para ela o mais íntimo desejo e a mais profunda necessidade – o que distingue os homens superiores dos inferiores! Em algumas pessoas piedosas encontrei ódio à razão e isso me agradou nelas: ao menos se revelava assim a má consciência intelectual! Mas estar em meio a essa \textit{rerum concordia discors} [discordante concerto das coisas] e toda a maravilhosa incerteza e ambigüidade da existência e \textit{não interrogar}, não tremer de ânsia e gosto da interrogação, nem sequer odiar quem interroga, talvez até se divertindo levemente com este – isto é o que percebo como \textit{desprezível}, e tal percepção é o que busco primeiramente em cada indivíduo: – algum desatino está sempre a me convencer de que todo ser humano tem esta percepção, como ser humano. É minha espécie de injustiça. (\textit{FW/GC}, §2).
\end{quotation}

A injustiça do tipo nobre deve-se a que, ao dar ouvidos com tamanha intensidade a suas próprias exigências de refinamento no gosto e no juízo, não consegue entender que tais exigências não tenham apelo à maioria das pessoas, enquanto a injustiça do tipo normal está em ver o dever a que se sujeita como algo incondicionalmente extensivo a todos. Mas o filósofo, por sua vez, está ciente de que o tipo de normatividade epistêmica que valoriza não é extensivo a espaços mais amplos da sociedade. Como uma forma de mediação desse conflito, encontra-se em \textit{GC} algo como uma “divisão social do trabalho cultural”, em que o tipo normal assume a função de manter as convenções que tornam a vida comum praticável, e assim evita a irrupção da “loucura” em maior escala – enquanto o tipo de exceção, percebendo de forma sempre mais apurada o que há de insão nos artigos tradicionais de crença e avaliação, exercitando formas próprias de viver e avaliar, avança na criação de inovações culturais apenas gradualmente incorporadas num nível mais amplo. Essa espécie de “pacto de convivência” é firmado no célebre aforismo 76 de \textit{GC}, intitulado “O perigo maior”, em que conclama-se uma atitude, sempre renovada, de defesa da exceção, com a condição de que ela se contente justamente em ser exceção
\footnote{“\textit{O perigo maior}: – Não tivesse havido sempre um grande número de homens que vissem o disciplinar de sua mente – sua “racionalidade” – como seu orgulho, sua obrigação, sua virtude, que fossem ofendidos ou envergonhados por todas as fantasias e excessos do pensamento, enquanto amigos do “saudável bom senso”, há muito a humanidade teria perecido! Sobre ela pairava e continua pairando, como o perigo maior, a irrupção da \textit{loucura} – isto é, a irrupção do capricho no sentir, ver e ouvir, o gosto na indisciplina da mente, a alegria do “mau senso”. O oposto do mundo dos loucos não é a verdade e a certeza, mas a universalidade e obrigatoriedade de uma crença, em suma, o que não é capricho no julgamento. E o maior trabalho dos homens até hoje foi entrar em acordo acerca de muitas coisas e submeter-se a uma \textit{lei da concordância} – não importando se tais coisas são verdadeiras ou falsas. Esta é a disciplina da mente, que conservou a humanidade; – mas os impulsos contrários são ainda tão poderosos, que não se pode, no fundo, falar confiantemente no futuro da humanidade. A imagem das coisas se move e se desloca ininterruptamente, e, a partir de agora, talvez com rapidez maior do que nunca; sem cessar, precisamente os espíritos mais seletos se revoltam contra tal obrigatoriedade – os investigadores da \textit{verdade} em primeiro lugar! Continuamente essa crença, enquanto crença de todos, produz uma náusea e uma nova ânsia nas mentes mais refinadas: e já o ritmo lento que ela requer para os processos espirituais, a imitação da tartaruga que aí é reconhecida como norma, transforma artistas e poetas em apóstatas: – é nesses espíritos impacientes que irrompe um verdadeiro prazer na locuura, pois ela tem um ritmo tão alegre! Portanto, intelectos virtuosos são necessários – ah, usarei o termo mais inequívoco –, a \textit{estupidez virtuosa} é necessária, os inabaláveis metrônomos do espírito \textit{lento}, para que os fiéis da grande crença geral se mantenham juntos e continuem a sua dança: é uma necessidade de primeira ordem que aí comanda e exige. \textit{Nós, os outros, somos a exceção e o perigo} – necessitamos perenemente de defesa! – Bem, algo pode ser dito em favor da exceção, \textit{desde que ela nunca deseje se tornar regra}.” (FW/GC, §76.)}.
	
	Há uma implicação filosófica forte aqui: a primazia da normatividade epistêmica, característica do tipo de exceção no qual o filósofo se inclui, é vista como necessária e suficiente para a revisão e reavaliação das crenças e valores. O sujeito da honestidade intelectual proíbe a si mesmo a atitude de “acreditar isso ou aquilo e viver conforme tal crença, \textit{sem} antes haver se tornado consciente das últimas e mais seguras razões a favor ou contra ela”, mas a maioria das pessoas, não. Isto seria razão suficiente para o tipo de exceção jamais tentar impor seu grau de honestidade como regra, mas, além disso, a própria possibilidade de compreender e comunicar-se com aquilo que constitui a regra moral depende de que ela seja investigada não como um produto de normatividade epistêmica.
	
	Buscamos apontar uma mudança de ênfase que implica numa mudança de estratégia: enquanto as opiniões eram objeto de estudo privilegiado de \textit{Humano} (apesar de que neste livro Nietzsche já estava atento, é claro, aos motivos inconscientes e extra-cognitivos que atuam na formação e adesão às opiniões), em \textit{GC} aparece de forma mais incisiva a questão da anterioridade do \textit{gosto} sobre a opinião. Torna-se visível, portanto, uma mudança gradual na filosofia nietzschiana, que em \textit{Aurora} dedicava ainda uma atenção significativa ao processo de refutação das crenças, e parecia apostar mais confiantemente no poder do conhecimento de reformar os “espaços internos” da alma
\footnote{Veja-se, por exemplo, o aforismo de \textit{Aurora} intitulado “Refutação histórica como refutação definitiva”: M/A, §95.}. 
Ganha espaço em \textit{GC} a percepção de que o erro, conquanto epistemicamente acusável, pode encontrar justificação em termos pragmáticos; quer dizer, por mais infundado que um artigo de fé se mostre, pode ser o caso que ele de fato cumpra uma função vital na estruturação de uma comunidade, grupo ou pessoa – como um remédio que tem efeito benéfico mesmo sobre quem não conhece seu funcionamento (\textit{FW/GC}, §354). Nas palavras de Nietzsche: “Uma moral pode ter nascido \textit{de} um erro: ainda com esta percepção o problema de seu valor não chega a ser tocado.”. Frente a esta constatação, o filósofo capitula suas estratégias de confrontação com a moral tradicional. Lopes ressalta que: 

\begin{quotation}
Aqui vemos que o embate decisivo com o sistema da moralidade como uma moralidade particular, como uma medicina específica, se dá no âmbito propriamente normativo, ou seja, em torno da questão de qual valor deve ser atribuído aos seus valores. O embate fundamental com o sistema da moralidade ou com a moralidade hegemônica do ocidente deve se deslocar do exame das credenciais epistêmicas de seus pressupostos descritivos para um confronto direto com seus compromissos normativos. (LOPES, artigo a ser publicado).
\end{quotation}

Se o tipo de exceção assume a tarefa de cuidar da renovação da cultura, terá que encontrar outros meios de persuasão e comunicação, que não aqueles da argumentação ou exposição científica. A arte teria uma função a cumprir aqui justamente porque não se trata mais de uma questão de “opinião”, mas de uma questão de “gosto”: 

\begin{quotation}
\textit{Mudança de gosto}. – A mudança de gosto é mais importante que a de opiniões. Estas, com as provas, refutações e toda a mascarada intelectual, são apenas sintomas do gosto que mudou, e certamente \textit{não} aquilo pelo qual freqüentemente são tomadas, as causas dessa mudança. Como se transforma o gosto geral: quando indivíduos, poderosos e influentes, exprimem o seu \textit{hoc est ridiculum, hoc est absurdum} [isto é ridículo, isto é absurdo], ou seja, o juízo do seu gosto e desgosto, e o fazem valer tiranicamente: – com isso impõem a muitos uma obrigação, que gradualmente se torna o hábito de outros mais e, por fim, \textit{uma necessidade de todos}. Mas o motivo para que esses indivíduos sintam e “saboreiem” de outra forma se acha normalmente numa singularidade de seu modo de vida, sua alimentação, digestão, talvez numa maior ou menor quantidade de sais inorgânicos no sangue e no cérebro; em suma, na sua \textit{physis} [natureza]: mas eles têm a coragem de reconhecer a sua \textit{physis} e dar ouvido às exigências dela, ainda nos seus tons mais sutis: seus juízos estéticos e morais são esses “tons sutilíssimos” da \textit{physis}. (FW/GC, §39).
\end{quotation}

A ciência tanto mais divide espaço com a arte enquanto protagonistas de um projeto filosófico quanto mais se percebe que toda atividade cognitiva é estreitamente ligada aos processos estéticos. A \textit{honestidade intelectual} recebe ainda bastante atenção ao longo do livro, e, claro, enquanto traço de “naturezas raras” é objeto do maior interesse por parte de Nietzsche; a diferença é que a partir de então passa-se a enfatizar a congenialidade entre esse traço e aquele de uma sensibilidade apurada, uma maior disposição para perceber e expressar os “'tons sutilíssimos' da physis”
\footnote{Nietzsche assume, portanto, um ponto de vista naturalista na descrição das diferenças entre tipo normal e tipo de exceção; elas remeteriam, afinal, a diferenças de constituição e temperamento. Está plasmada, assim, uma tipologia nietzschiana baseada na oposição entre tipo normal, que se associa ao campo da regra, segurança, da convenção, conservação, estabilidade, do que é comum e, por outro lado, o tipo de exceção associado ao campo da raridade, da pessoalidade, da sutileza de distinções, maior afinidade com o caráter mutável de um mundo visto como profundo devir.}.

	É a tais tipos de exceção que se atribui a tarefa de criar novos paradigmas de gosto. Esse paradigma, que tem origem no tipo de exceção, é transmitido ao tipo normal como uma \textit{imposição} de poder, é o que se afirma nessa passagem. O que o aforismo não diz com todas as letras, e deve ser procurado em outras passagens do livro, é que já que se trata de uma imposição, esta não se dá no plano argumentativo, mas de uma intervenção no plano afetivo. Ao que tudo indica, esta seria uma das tarefas da arte: criar meios sedutores para plasmar e divulgar a mudança cultural
\footnote{“\textit{A música como intercessora}. – 'Tenho sede de um mestre na arte dos sons', disse um inovador ao seu discípulo, 'que de mim aprendesse os pensamentos e os falasse depois na sua linguagem: assim eu chegaria melhor aos ouvidos e corações dos homens. Através dos sons podemos coduzir os homens a qualquer erro e qualquer verdade: pois quem conseguiria \textit{refutar} um som?' – 'Então você gostaria de ser visto como irrefutável?', perguntou o discípulo. O inovador respondeu: 'Eu gostaria que o gérmen se tornasse árvore. Para que uma doutrina se torne árvore, é necessário que acreditem nela por um bom tempo; para que nela acreditem, ela tem de ser vista como irrefutável. A árvore requer temporais, dúvidas, vermes, maldade, a fim de revelar a natureza e a força de seu gérmen; que ela se parta, caso não seja forte o bastante! Mas um gérmen só se pode destruir – nunca refutar!'. (…).” (FW/GC, §106).}. 
Isto se faz notar na própria busca nietzschiana por novos recursos com que comunicar sua filosofia, que passa a contar cada vez mais com a alegoria, a anedota, a parábola, a diversificação de personagens conceituais, etc
\footnote{Nietzsche recorre a alegorias ou parábolas para expressar as ideias mais polêmicas de \textit{GC}: “morte de Deus”, “eterno retorno” e o personagem do Zaratustra. Cf. FW/GC, §125, §341, §342.}.
	
	Um outro uso da arte estaria em mediar o convívio do tipo de exceção com a própria intensidade de sua paixão. Em especial, interessa a Nietzsche que a arte sirva de contrapeso à paixão do conhecimento, de forma que o espírito livre não cultive uma exigência moral, de pretensões universalizantes, da honestidade intelecutal. Desde \textit{Humano} aposta-se na possibilidade de uma vida dedicada à busca do conhecimento e regida pelo princípio de honestidade intelectual em razão de esta representar justamente a possibilidade de superação do ponto de vista moral; porém, ao investigar os motivos que historicamente levaram os homens a dedicar-se à ciência, Nietzsche traz à tona a questão do risco de que também o modo de vida contemplativo se torne presa da moralização:

\begin{quotation}
\textit{Graças a três erros}. – A ciência foi promovida nos últimos séculos, em parte porque com ela e mediante ela se esperava compreender melhor a bondade e a sabedoria divinas – o motivo principal na alma dos grandes ingleses (como Newton) –, em parte porque se acreditava na absoluta utilidade do conhecimento, sobretudo na íntima ligação de moral, saber e felicidade – o motivo principal na alma dos grandes franceses (como Voltaire) –, em parte porque na ciência pensava-se ter e amar algo desinteressado, inócuo, bastante a si mesmo, verdadeiramente inocente, no qual os impulsos maus dos homens não teriam participação – o motivo principal na alma de Spinoza, que, como homem do conhecimento, sentia-se divino: – graças a três erros, portanto. (\textit{FW/GC}, 37)
\end{quotation}

A honestidade intelectual, ao tomar-se a si mesma como objeto de investigação, percebe a sua própria \textit{pudendo} origo, e aponta a condição paradoxal da busca por conhecimento. Novamente, tem-se aqui o desafio de sustentar uma forma não-ilusória de paixão, aceitando a constante tensão envolvida no cultivo de uma ciência que a todo tempo traz à luz os erros em que ela mesma se assentava. 

	\textit{GC} desenvolve a questão problemática de como a própria prática de honestidade intelectual leva à conclusão de que o erro penetra todas as camadas da vida e é condição do próprio conhecimento. Tem-se então que, enquanto fio condutor da paixão do conhecimento, a honestidade intelectual não pode, em última instância, insurgir-se contra o erro. Dessa forma, reconciliar-se, mesmo que momentaneamente, com a ilusão, o erro e o engano, por meio da arte, seria justamente a atitude necessária para continuar perseguindo o conhecimento mesmo. O compromisso com a honestidade intelectual, e com um modo de vida dedicado à paixão do conhecimento é reafirmado em \textit{GC}, no entanto ele só seria vivível na vizinhança da arte:

\begin{quotation}
\textit{Nossa derradeira gratidão para com a arte}. – Se não tivéssemos aprovado as artes e inventado essa espécie de culto do não-verdadeiro, a percepção da inverdade e mendacidade geral, que agora nos é dada pela ciência – da ilusão e do erro como condições da existência cognoscente e sensível –, seria intolerável para nós. A \textit{retidão} teria por conseqüência a náusea e o suicídio. Mas agora a nossa retidão tem uma força contrária, que nos ajuda a evitar conseqüências tais: a arte, como a \textit{boa} vontade de aparência. Não proibimos sempre que os nossos olhos arredondem, terminem o poema, por assim dizer: e então não é mais a eterna imperfeição, que carregamos pelo rio do vir-a-ser – então cremos carregar uma \textit{deusa} e ficamos orgulhosos e infantis com tal serviço. Como fenômeno estético a existência ainda nos é \textit{suportável}, e por meio da arte nos são dados olhos e mãos e, sobretudo, boa consciência, para \textit{poder} fazer de nós mesmos um tal fenômeno. Ocasionalmente precisamos descansar de nós mesmos, olhando-nos de cima e de longe e, de uma artística distância, rindo de nós ou chorando por nós; precisamos descobrir o \textit{herói} e também o \textit{tolo} que há em nossa paixão do conhecimento, precisamos nos alegrar com a nossa estupidez de vez em quando, para poder continuar nos alegrando com a nossa sabedoria! E justamente por sermos, no fundo, homens pesados e sérios, e antes pesos do que homens, nada nos faz tanto bem como o \textit{chapéu de bobo}: necessitamos dele diante de nós mesmos – necessitamos de toda arte exuberante, flutuante, dançante, zombeteira, infantil e venturosa, para não perdermos a \textit{liberdade de pairar acima das coisas}, que o nosso ideal exige de nós. Seria para nós um \textit{retrocesso} cair totalmente na moral, justamente com a nossa suscetível retidão, e, por causa das severas exigências que aí fazemos a nós mesmos, tornarmo-nos virtuosos monstros e espantalhos. Devemos também \textit{poder} ficar acima da \textit{moral}: e não só ficar em pé, com a angustiada rigidez de quem receia escorregar e cair a todo instante, mas também flutuar e brincar acima dela! Como poderíamos então nos privar da arte, assim como do tolo? – E, enquanto vocês tiverem alguma \textit{vergonha} de si mesmos, não serão ainda um de nós! (FW/GC, §107).
\end{quotation}

Significativamente, este aforismo encerra o segundo livro de \textit{GC}, e portanto é o último aforismo antes daquele que anuncia a morte de Deus. Essa invocação da arte como contrapeso da honestidade intelectual se liga imediatamente ao anúncio da passagem para uma era pós-metafísica, à qual se teria chegado por questões de honestidade intelectual, inclusive, uma vez que esta tornou inviável a adesão a certo tipo de crença. Está dado o contexto que impõe ainda uma derradeira finalidade à arte, segundo \textit{GC}, e esta tem a ver com tornar suportável a compreensão da existência oferecida pela ciência, que ao apontar os erros que condicionam a experiência ordinária, deixa entrever uma visão de mundo cada vez mais \textit{desumanizada}.

\section{Humano, inumano e além-do-humano}

Parece exagero atribuir a Nietzsche uma “filosofia da ciência”, e isto provavelmente tem a ver com o modo não-teorizante com que desenvolve sua filosofia. Embora não se encontre em sua obra uma teoria da ciência como aquelas que foram desenvolvidas no século XX, encontram-se, é claro, várias reflexões sobre o método, o funcionamento e o alcance das ideias científicas. Pode-se dizer com segurança que Nietzsche tem uma compreensão fundamentalmente pragmática da ciência, quer dizer, uma compreensão segundo a qual a ciência, operando uma sistematização dos mesmos “erros” que condicionam a experiência comum, descreve regularidades de modo a permitir um maior grau de intervenção na natureza
\footnote{Isto certamente se deve ao contato com autores como Friedrich A. Lange, nos anos de sua formação. Cf. LOPES, 2008.}. Assim, ela tanto projeta uma face humana à natureza (embora esteja atenta ao fato de que se trata de uma “projeção”, fato que não costumamos levar em conta na experiência comum) quanto maneja o mundo de forma a torná-lo humanamente mais “habitável”
\footnote{Não se pode exigir de Nietzsche, que desenvolveu sua filosofia no século XIX, uma percepção muito apurada do lado nefasto dessa manipulação científica do mundo, algo que é tema da filosofia de Theodor W. Adorno, por exemplo, e que é uma questão cada vez mais em pauta em tempos de mudanças climáticas, agroindústria, e proliferação de uma série de mazelas no modo citadino-industrializado de viver. De todo modo, a filosofia nietzschiana trata do poder instrumental da ciência em manipular a natureza com vistas a fins humanamente estipulados, que é a concepção baconiana de ciência.}. Por vias da ciência, enfim, humaniza-se o mundo:

\begin{quotation}
\textit{Causa e efeito}. – “Explicação”, dizemos; mas é “descrição” o que nos distingue de estágios anteriores do conhecimento e da ciência. Nós descrevemos melhor – e explicamos tão pouco quanto aqueles que nos precederam. Descobrimos múltiplas sucessões, ali onde o homem e pesquisador ingênuo de culturas anteriores via apenas duas coisas, “causa” e “efeito”, como se diz; aperfeiçoamos a imagem do devir, mas não fomos além dessa imagem, não vimos o que há por trás dela. Em cada caso, a série de “causas” se apresenta muito mais completa diante de nós, e podemos inferir: tal e tal coisa têm de suceder antes, para que venha essa outra – mas nada \textit{compreendemos} com isso. Em todo devir químico, por exemplo, a qualidade aparece como um “milagre”, agora como antes, e assim também todo deslocamento; ninguém “explicou” o empurrão. E como poderíamos explicar: operamos somente com coisas que não existem, com linhas, superfícies, corpos, átomos, tempos divisíveis, espaços divisíveis – como pode ser possível a explicação, se primeiro tornamos tudo \textit{imagem}, nossa imagem! Basta considerar a ciência a humanização mais fiel possível das coisas, aprendemos a nos descrever de modo cada vez mais preciso, ao descrever as coisas e sua sucessão. Causa e efeito: essa dualidade não existe provavelmente jamais – na verdade, temos diante de nós um \textit{continuum}, do qual isolamos algumas partes; assim como percebemos um movimento apenas como pontos isolados, isto é, não o vemos propriamente, mas o inferimos. A forma súbita com que muitos efeitos se destacam nos confunde; mas é uma subitaneidade que existe apenas para nós. Neste segundo de subitaneidade há um número infindável de processos que nos escapam. Um intelecto que visse causa e efeito como \textit{continuum}, e não, à nossa maneira, como arbitrário esfacelamento e divisão, que enxergasse o fluxo do acontecer – rejeitaria a noção de causa e efeito e negaria qualquer condicionalidade. (FW/GC, §112).
\end{quotation}

O que há de paradoxal aqui é que ao mesmo tempo em que a ciência opera com “coisas que não existem”, isto é, “com linhas, superfícies, corpos, átomos, tempos divisíveis, espaços divisíveis”, e com isso humaniza o mundo, é a própria ciência que leva afinal à conclusão mesma de que tais coisas não existem! Quer dizer, ao expor o caráter antropomórfico de nossa compreensão de mundo, a ciência nos oferece por assim dizer um negativo dos fatores que condicionam a experiência humana. Ela nos oferece, enfim, uma depuração dos fatores antropomorfizantes, apontando sua atuação e seus limites e, com isso, abre uma espécie de janela para se pensar o não-humano. Ela exige que as próprias categorias com que trabalha sejam postas em perspectiva; essa percepção trará à filosofia de Nietzsche um tom cada vez mais cético: por exemplo, se em \textit{Humano} e \textit{Aurora} o sentido de causalidade era constantemente invocado como uma forma de \textit{corrigir} formas menos rigorosas de pensar, em \textit{A Gaia Ciência} questiona-se o sentido de causalidade ele mesmo, apontando a ideia de que causa e efeito são tão somente uma forma de organizar o \textit{fluxo} ou \textit{continuum} da existência. Essa cautela científica quanto à humanização do mundo é expressa num belo aforismo de \textit{GC}:

\begin{quotation}
\textit{Guardemo-nos!} – Guardemo-nos de pensar que o mundo é um ser vivo. Para onde iria ele expandir-se? De que se alimentaria? Como poderia crescer e multiplicar-se? Sabemos aproximadamente o que é o orgânico; e o que há de indizivelmente derivado, tardio, raro, acidental, que percebemos somente na crosta da terra, deveríamos reinterpretá-lo como algo essencial, universal, eterno, como fazem os que chamam o universo de organismo? Isso me repugna. Guardemo-nos de crer também que o universo é uma máquina; certamente não foi construído com um objetivo, e usando a palavra “máquina” lhe conferimos demasiada honra. Guardemo-nos de pressupor absolutamente e em toda parte uma coisa tão bem realizada como os movimentos cíclicos dos nossos astros vizinhos; um olhar sobre a Via Láctea já nos leva a perguntar se lá não existem movimentos bem mais rudimentares e contraditórios, assim como astros de trajetória sempre retilínea e outras coisas semelhantes. A ordem astral em que vivemos é uma exceção; essa ordem e a considerável duração por ela determinada tornaram possível a exceção entre as exceções: a formação do elemento orgânico. O caráter geral do mundo, no entanto, é caos por toda a eternidade, não no sentido de ausência de necessidade, mas de ausência de ordem, divisão, forma, beleza, sabedoria e como quer que se chamem nossos antropomorfismos estéticos. Julgados a partir de nossa razão, os lances infelizes são a regra geral, as exceções não são o objetivo secreto e todo o aparelho repete sempre a sua toada, que não pode ser chamada de melodia – e, afinal, mesmo a expressão “lance infeliz” já é uma antropomorfização que implica uma censura. Mas como poderíamos nós censurar ou louvar o universo? Guardemo-nos de atribuir-lhe insensibilidade e falta de razão, ou o oposto disso; ele não é perfeito nem belo, nem nobre, e não quer tornar-se nada disso, ele absolutamente não procura imitar o homem! Ele não é absolutamente tocado por nenhum de nossos juízos estéticos e morais! Tampouco tem impulso de autoconservação, ou qualque impulso; e também não conhece leis. Guardemo-nos de dizer que há leis na natureza. Há apenas necessidades: não há ninguém que comande, ninguém que obedeça, ninguém que transgrida. Quando vocês souberem que não há propósitos, saberão também que não há acaso: pois apenas em relação a um mundo de propósitos tem sentido a palavra “acaso”. Guardemo-nos de dizer que a morte se opõe à vida. O que está vivo é apenas uma variedade daquilo que está morto, e uma variedade bastante rara. – Guardemo-nos de pensar que o mundo cria eternamente o novo. Não há substâncias que duram eternamente; a matéria é um erro tal como o deus dos eleatas. Mas quando deixaremos nossa cautela e nossa guarda: quando é que todas essas sombras de Deus não nos obscurecerão mais a vista? Quando teremos desdivinizado completamente a natureza? Quando poderemos começar a \textit{naturalizar} os seres humanos com uma pura natureza, de nova maneira descoberta e redimida? (FW/GC, §109).
\end{quotation}

Esse aforismo, que tem a forma de uma meditação, desloca um a um os antropomorfismos que mais recorrentemente projetamos sobre o mundo. Em particular, ele se volta à desconstrução da compreensão da totalidade do mundo como ser vivo (a interpretação estoica)\footnote{Cf. HAAR, 1998, p. 15.} e como máquina (sugerindo que a versão mecanicista é, afinal, apenas um recurso provisório de cálculo, sem o poder de dar a palavra definitiva sobre a existência). Ao negar que haja “leis” na natureza, pois não há “ninguém que comande e ninguém que obedeça”, o filósofo afasta o vocabulário intencionalista e o jurídico. Vale lembrar que a filosofia nietzschiana opera uma associação entre os conceitos de átomo, coisa e incondicionado, sendo que todos esses estariam presentes na projeção de um sujeito metafisicamente compreendido, que seria, afinal, o sujeito da atribuição de responsabilidade
\footnote{Tema de que tratamos na segunda seção do capítulo \ref{cap1}. Segundo Michel Foucault, Nietzsche teria apontado aqui uma dobra conceitual que seria a chave de compreensão do surgimento do “homem jurídico” (FOUCAULT, 2007).}. 
A ideia, novamente, é que tem-se uma visão coisificada e moralizante do sujeito humano da ação, e esta é projetada como uma visão coisificada e moralizante do mundo e da natureza.

Nietzsche sugere uma nova compreensão de mundo ao falar da natureza como “caos por toda eternidade”, do qual não se exclui o sentido de “necessidade”, apontando o caráter não-arbitrário do desenvolvimento natural. Note-se que dificilmente se chega a construir uma imagem com essas parcas pistas; elas formam um cenário desumanizado, um cenário por assim dizer “alienígena” (CARVALHO, 2013, p. 52). Vale lembrar que quando da redação de GC Nietzsche já havia travado contato com as teorias dinâmicas, como aquela do físico R. J. Boscovich, que propõe a substituição da noção de “átomo” por aquela de “força”; a respeito do interesse nietzschiano por tais teorias, Carvalho aponta que:

\begin{quotation}
Algumas razões podem ser fornecidas para explicar essa preferência: uma delas está no fato de que as teorias dinâmicas parecem superar em alguma medida os aspectos sensuais e antropomórficos que ele identifica, por exemplo, em conceitos da visão mecanicista, e ele julga que as teorias dinâmicas seriam, portanto, menos antropomorfizadas que as demais; em segundo lugar, pelo fato de que tais teorias abdicam da ideia de átomo como sustentáculo da força e da representação do mundo como uma espécie de agregado de átomos, substituindo-a pela imagem do mundo como um sistema relacional e dinâmico de ação e reação de múltiplas forças; em terceiro lugar, porque ela se encaixa melhor com a concepção esposada por Nietzsche de fluxo ou devir. (CARVALHO, 2013, p. 47.)
\end{quotation}

Nietzsche busca nas teorias dinâmicas formas menos antropomórficas de pensamento porque as formas tradicionais seriam ainda “sombras de Deus” (FW/GC, §109); deve-se manter em mente que o aforismo “\textit{Guardemo-nos}” está localizado logo após aquele que anuncia a morte de Deus. A associação que se traça aqui não deve passar despercebida: a passagem para uma era pós-metafísica é uma oportunidade de passagem pra uma era pós-humana. 

Foi o pensamento moral e metafísico que criou a ilusão de uma separação radical entre homem e natureza. Essa ilusão, no entanto, tem efeitos reais – dentre eles, o modo antinatural como tivemos que nos referir à natureza. Mas não só isto. Ele definiu um espaço humanamente habitável, definiu padrões e metas culturais. Nietzsche aponta um momento em que as ilusões que davam suporte a essa forma de pensar começam a ser percebidas como insustentáveis, embora sejam ainda  condicionantes aplicados à nossa autocompreensão e organização. Superar a cultura metafísica implica em rever tudo aquilo que foi tido como distintamente humano; e esta seria uma tarefa que se impõe na medida em que a ciência traz à tona as inconsistências internas a esses traços que configuraram o humano:

\begin{quotation}
\textit{Os quatro erros.} – O homem foi educado por seus erros: primeiro, ele sempre se viu apenas de modo incompleto; segundo, atribuiu-se características inventadas; terceiro, colocou-se numa falsa hierarquia, em relação aos animais e à natureza; quarto, inventou sempre novas tábuas de bens, vendo-as como eternas e absolutas por um certo tempo, de modo que ora este ora aquele impulso e estado humano se achou em primeiro lugar, e foi enobrecido em conseqüência de tal avaliação. Excluindo o efeito desses quatro erros, exclui-se também humanidade, humanismo e “dignidade humana”. (FW/GC, §115). 
\end{quotation}

Note-se que os erros aqui apontados como fundadores do sentido corrente de humanidade são eminentemente “erros” morais: eles consistem, primeiramente, em oferecer uma imagem edulcorada do homem, uma visão “incompleta”, que exclui do campo de investigação os traços terríveis da natureza humana; as características “inventadas” que ele se atribuiu são também, provavelmente, supostas características morais, como a posse de uma razão de ascendência divina
\footnote{Lembrando que o naturalismo nietzschiano consiste justamente em reconduzir os supostos fenômenos morais a fenômenos que podem ser descritos em termos não-morais. A suspeita que move o pensamento nietzschiano aqui, provavelmente é aquela sintetizada posteriormente em \textit{Além de Bem e Mal} na fórmula: “Não existem fenômenos morais, apenas uma interpretação moral dos fenômenos.” (JGB/ABM, §108).}. 
Essas caraterísticas “inventadas” seriam justamente aquelas aplicadas na construção de uma falsa hierarquia entre o homem e os outros animais, o restante da natureza; que esses traços tenham sido usados para estabelecer uma “hierarquia” mostra o investimento moral a eles aplicado, ou seja, esses traços foram vistos como não apenas distintos mas como atestados do caráter \textit{melhor} do homem perante o do restante da natureza; essa visão começa a cair por terra na medida em que a biologia força o reconhecimento de que os outros animais compartilham de algum grau de razão e consciência, por exemplo, ou na medida em que a história das culturas mostra que esse “melhor” foi concebido de forma radicalmente diferente por diferentes povos ao longo do tempo. O paradigma moral seria, sobretudo, incapaz de uma visão afirmativa da diversidade de “impulsos” que constituem o humano, na medida em que “enobrece” uma ou outra área do material impulsivo humano conforme uma avaliação pré-determinada.

A morte de Deus, signo da morte do paradigma moral metafísico, força uma reconsideração do que até então foi tido como os contornos do humano. A fiar-se no diagnóstico nietzschiano, a cultura ocidental se encontraria, então, num momento de inflexão que impõe o desafio de repensar categorias linguísticas, visões de mundo, atribuições de sentido e valor. Nietzsche acolhe esse desafio em sua filosofia. A novidade desse acontecimento e a radicalidade de seu alcance não oferecem uma base segura de previsão das formas culturais a serem adotadas daí em diante – além da suposição, algo evidente, de que se trata de uma tarefa de criação estética, investigativa, conceitual e valorativa. À medida que o ambiente metafísico sai do horizonte do pensamento, uma nova visada à história ganha corpo, e a agenda investigativa se redefine, como preparação necessária para a redefinição dos modos de vida e da criação cultural
\footnote{“\textit{Algo para homens trabalhadores.} – Quem hoje pretende estudar as coisas morais, abre para si um imenso campo de trabalho. Todas as espécies de paixões têm de ser examinadas individualmente, perseguidas através de tempos, povos, grandes e pequenos indivíduos: toda a sua razão, todas as suas valorações e clarificações das coisas devem ser trazidas à luz! Até o momento, nada daquilo que deu colorido à existência teve história: se não, onde está uma história do amor, da cupidez, da inveja, da consciência, da piedade, da crueldade? Mesmo uma história comparada do direito, ou apenas do castigo, falta inteiramente até aqui. Já se tomou por objeto de pesquisa as diferentes divisões do dia, as conseqüências de uma fixação regular do trabalho, das festas e do repouso? Conhece-se os efeitos morais dos alimentos? Existe uma filosofia da alimentação? (O alarido a favor e contra o vegetarianismo, que volta e meia reaparece, já mostra que ainda não há uma tal filosofia!) Já foram reunidas as experiências de vida comunitária, as experiências dos mosteiros, por exemplo? Já foi mostrada a dialética do casamento e da amizade? Os costumes dos eruditos, dos comerciantes, artistas e artesãos – já encontraram seus pensadores? Há tanto a pensar aqui! Tudo o que até agora os homens consideraram suas 'condições de existência', e toda a razão, paixão e crendice desta consideração – isto já foi pesquisado até o fim? Apenas a observação do crescimento diverso que tiveram e poderiam ter ainda os impulsos humanos, conforme os diversos climas morais, já significa trabalho em demasia para o homem mais trabalhador; gerações inteiras, gerações de eruditos a trabalhar conjuntamente e de modo planejado, serão necessárias para esgotar aqui o material e os pontos de vista. O mesmo vale para a demonstração dos motivos para a diferença de clima moral ('Por que brilha aqui este sol de um juízo moral e medida de valor fundamental – e ali aquele outro?'). E seria um novo trabalho estabelecer o caráter errôneo de todos esses motivos e toda a natureza do juízo moral até agora. Supondo que todos esses trabalhos fossem realizados, viria ao primeiro plano a questão mais espinhosa: se a ciência estaria em condições de \textit{oferecer} objetivos para a ação, após haver demonstrado que pode liquidá-los – então caberia uma experimentação que permitira a satisfação de toda espécie de heroísmo, séculos de experimentação, que poderia deixar na sombra todos os grandes trabalhos e sacrifícios da história até o momento. A ciência ainda não ergueu suas construções ciclópicas até hoje; também para isso chegará o tempo.” (FW/GC, §7).}.

Essa tarefa parece requerer uma reorientação dos recursos culturais disponíveis; tal reorientação já teria sido iniciada pela própria ciência, na medida em que esta segue uma agenda não teológica de investigação. A ciência, ao exercer o trabalho de descrição das condições altamente incorporadas da perspectiva humana destaca os contornos dessa perspectiva e portanto é capaz de criar uma tensão em torno da questão do é humano e do que é não-humano e essa tensão seria sumamente importante porque, para Nietzsche, a reinvenção do espaço do “humano” dependeria de uma confrontação com os aspectos não-humanos da espécie e da existência, aspectos que foram ocultados pelo paradigma moral. 

A perspectiva da ciência é entalhada pelo método, o que a faz uma forma ímpar de satisfação das exigências de rigor e retidão, exigências que simplesmente se tornaram um fato incontornável do pensamento moderno, segundo Nietzsche. Porém, não se trata de um trabalho puramente científico aquele de realizar a passagem para uma cultura pós-metafísica, cultura que a partir do \textit{Zaratustra} Nietzsche passa a associar ao ideal do “além-do-humano”. O trabalho propriamente filosófico consistiria em investigar e experimentar medidas de incorporação do não-humano na cultura, avançando a suspeita de que é também a partir do material terrível, agressivo, dessa natureza que a cultura encontrará novas fontes energéticas de criação. 

Promover o trânsito entre perspectivas; plasmar uma nova imagem do humano; tornar suportável a visão dos abismos a que se chega pela própria investigação científica; prover material criativo para delinear o além-do-humano que parece se aproximar – estas parecem ser razões do protagonismo que Nietzsche atribui à arte nesse novo cenário filosófico. Lopes alerta que “Não devemos, entretanto, concluir que Nietzsche fechou sobre si mesmo um círculo, retornando à metafísica de artista de sua juventude. A arte está agora decididamente no pólo da aparência. Por seu intermédio nenhuma porta é aberta para a essência do mundo.” (LOPES, 2008, p. 416). O que está em questão não é a reconciliação com os aspectos terríveis da existência por vias da ficção metafísica, como em \textit{O Nascimento da Tragédia}; está em questão uma reconciliação estética e conceitual com o caráter pós-metafísico do mundo que se abre à experiência humana.

Vários aforismos de \textit{GC} exploram as metáforas da navegação, do mar aberto, da exploração de novos territórios. O livro tem o tom de conclamação a novas empreitadas do espírito, em direção a uma reinvenção cultural que não parte do nada, mas carrega toda uma herança cultural da arte, da ciência e da religião. Que traços fundamentais dessa herança tenham se tornado anacrônicos é um fato que aumenta a dificuldade do desafio colocado. 

\textit{Assim Falou Zaratustra} parece se destinar a apontar uma encruzilhada cultural a que o ocidente chegou quando da morte de Deus – frente a essa encruzilhada a humanidade teria que ou aceitar desafio de criar novas formas de tonicidade espiritual num contexto pós-metafísico, e portanto lançar-se à empreitada de criação de novos modos de vida, novas metas e sentidos para a experiência humana, possibilidade que Nietzsche associa ao signo do “além-do-homem”; ou acomodar-se a uma existência sem perspectivas de auto-superação, e empenhar-se meramente na satisfação das necessidades animais mais primárias, possibilidade representada pelo “último homem”
\footnote{Cf. Z, Pr. 3-5. O “além-do-homem” (\textit{Übermensch}) aparece nessa obra sempre associado ao desafio de auto-superação num cenário pós morte de Deus. Seguindo essa dobra além-do-homem/além-de-Deus, Michel Foucault construiu, a seu modo, uma interpretação desse personagem conceitual: “(…) através de uma crítica filológica, através de uma certa forma de biologismo, Nietzsche reencontrou o ponto onde o homem e Deus pertencem um ao outro, onde a morte do segundo é sinônimo do desaparecimento do primeiro, e onde a promessa do super-homem significa, primeiramente e antes de tudo, a iminência da  morte do homem. Com isso, Nietzsche, propondo-nos esse futuro, ao mesmo tempo como termo e como tarefa, marca o limiar a partir do qual a filosofia contemporânea pode recomeçar a pensar; ele continuará sem dúvida, por muito tempo, a orientar seu curso. (…). Em nossos dias não se pode mais pensar senão no vazio do homem desaparecido. Pois esse vazio não escava uma carência; não prescreve uma lacuna a ser preenchida. Não é mais nem menos que o desdobrar de um espaço onde, enfim, é de novo possível pensar.” (FOUCAULT, 2007, p. 472-473). Vale lembrar que Foucault incorpora o ideal do além-do-homem em seu próprio programa filosófico de superação das categorias das ciências modernas; há aqui uma evidente diferença programática entre os dois filósofos: enquanto Foucault localiza a invenção de um sentido de humanidade no surgimento das ciências modernas, portanto nos últimos três séculos, e destina seus esforços filosóficos a uma crítica radical dos paradigmas fundadores dessas ciências, o programa genealógico de Nietzsche especula sobre os condicionantes que, desde a pré-história, teriam formatado a experiência humana, e toma por aliados nessa investigação justamente as ciências modernas que então se constituíam como campos de saber.}. Um aforismo do livro V de GC expressa bem as tensões que animam o pensamento nietzschiano nesse momento, e suas expectativas de que o desafio da criação pós-metafísica reoriente os rumos da filosofia:

\begin{quotation}
\textit{A grande saúde.} –  Nós, os novos, sem nome, de difícil compreensão, nós, rebentos prematuros de um futuro ainda não provado, nós necessitamos, para um novo fim, também de um novo meio, ou seja, de uma nova saúde, mais forte alerta alegre firme audaz que todas as saúdes até agora. Aquele cuja alma anseia haver experimentado o inteiro compasso dos valores e desejos até hoje existente e haver navegado as praias todas desse “Mediterrâneo” ideal, aquele que quer, mediante as aventuras da vivência mais sua, saber como sente um descobridor e conquistador do ideal, e também um artista, um santo, um legislador, um sábio, um erudito, um devoto, um adivinho, um divino excêntrico de outrora: para isso necessita mais e antes de tudo uma coisa, \textit{a grande saúde} – uma tal que não apenas se tem, mas constantemente se adquire e é preciso adquirir, pois sempre de novo se abandona e é preciso abandonar... E agora, após termos estado por largo tempo assim a caminho, nós, argonautas do ideal, mais corajosos talvez do que seria prudente, e com freqüência náufragos e sofridos, mas, como disse, mais sãos do que nos concederiam, perigosamente, sempre novamente sãos – quer nos parecer como se tivéssemos, como paga por isso, uma terra ainda desconhecida à nossa frente, cujos limites ainda ninguém divisou, um além de todos os cantos e quadrantes do ideal, um mundo tão opulento do que é belo, estranho, questionável, terrível, divino, que tanto nossa curiosidade como nossa sede de posse caem fora de si – ah, de modo que doravante nada nos poderá mais saciar!... Como poderíamos nós, após tais visões, e com tal voracidade de ciência e consciência, satisfazermo-nos com o \textit{homem atual}? É muito mau, porém inevitável, que olhemos suas mais dignas metas e esperanças com seriedade a custo mantida, e talvez sequer as olhemos mais... Um outro ideal corre à nossa frente, um ideal prodigioso, tentador, pleno de perigos, ao qual ninguém gostaríamos de levar a crer, porque a ninguém reconhecemos tão facilmente o \textit{direito a ele}: o ideal de um espírito que ingenuamente, ou seja, sem o ter querido, e por transbordante abundância e potência, brinca com tudo o que até aqui se chamou santo, bom, intocável, divino; para o qual o mais elevado, aquilo em que o povo encontra naturalmente sua medida de valor, já não significaria senão perigo, declínio, rebaixamento ou, no mínimo, distração, cegueira, momentâneo esqueceder de si; o ideal de bem-estar e bem-querer humano-sobre-humano, que com freqüência parecerá inumano, por exemplo, ao colocar-se ao lado de toda seriedade terrena até então, ao lado de toda a anterior solenidade em gesto, palavra, tom, olhar, moral e dever, como sua mais viva paródia involuntária – e com o qual, não obstante tudo, só então talvez se alce a \textit{grande seriedade}, a verdadeira interrogação seja colocada, o destino da alma dê a volta, o ponteiro avance, a tragédia \textit{comece}... (FW/GC, §382).
\end{quotation}

O tom desse aforismo expressa uma boa disposição para o desafio de passar a um novo estágio da cultura a partir das tensões presentes. Ou seja, o ritmo alegre e aventureiro do texto não deve nos distrair do fato de que o filósofo identifica elementos potencialmente explosivos na condição “atual do homem”, tensões suficientes para dar início a uma “tragédia”. Que tensões são essas? Embora Nietzsche seja mais sugestivo do que taxativo a este respeito, como é de seu costume, o aforismo permite que arrisquemos algumas interpretações. Pode-se pensar, primeiramente, que essas tensões dizem respeito ao modo de vida, ao ideal de “saúde”, a ser buscado pelo tipo de exceção, com quem o filósofo se identifica: “Nós, os novos, sem nome, de difícil compreensão, nós, rebentos prematuros de um futuro ainda não provado”; o termo usado para “um futuro ainda \textit{não provado}” é \textit{unbewiesenen}, mais próximo do sentido de “desconhecido”, “não sabido”, do que do sentido de “incerto”, “indemonstrado”. Os “rebentos prematuros” desse futuro teriam sido os primeiros a alertar para o fato de que o ocidente se encontra no limiar da passagem para uma cultura pós-metafísica, ou seja, os primeiros a atentarem para a perda de credibilidade dos ideais morais metafisicamente fundados e a dimensionar os possíveis efeitos desse fato na cultura como um todo. Pode-se tentar ler nas entrelinhas uma tensão política que aí se delineia: como se reorganizará a convivência entre os “de difícil compreensão” com o tipo normal “atual” agora que o fundamento da norma foi abalado? A tensão entre a percepção excepcional da abertura a um novo ideal versus o acomodamento da maioria, do “povo”, com formas de avaliação cada vez mais deslocadas de seu substrato, é sem dúvida o tema desta passagem: 

\begin{quotation}
o ideal de um espírito que ingenuamente, ou seja, sem o ter querido, e por transbordante abundância e potência, brinca com tudo o que até aqui se chamou santo, bom, intocável, divino; para o qual o mais elevado, aquilo em que o povo encontra naturalmente sua medida de valor, já não significaria senão perigo, declínio, rebaixamento ou, no mínimo, distração, cegueira, momentâneo esqueceder de si.
\end{quotation}

Como veremos, já o prólogo de \textit{Além de Bem e Mal} expressa o receio de que a humanidade acabe por se furtar ao desafio de enfrentar tais tensões. O tom belicoso desse livro deve ser ouvido como uma resposta ao temor de que, ao se aninhar numa zona de conforto tornada possível pela ilustração democrática, a humanidade perca uma oportunidade ímpar de reinvenção dos limites da experiência humana.

\chapter{Tentativas e Tentações Naturalistas – \textit{Além de Bem e Mal}}
\label{cap3}

\textit{Além de Bem e Mal} é certamente o mais denso dentre os escritos nietzschianos. O livro é tão audacioso quanto ao lugar a que pretende lançar a filosofia; tão cerrada é a confecção do texto; tão magistral sua forma de argumentação, construída num ritmo presto mas sem pressa; é tão quase invisível a linha argumentativa que costura os aforismos entre si, e que deve ser recuperada pelo leitor; há tantas sugestões que só serão retomadas muitos parágrafos adiante, ou que têm que ser recobradas alguns parágrafos atrás; irrupções de jocosidade e especulação científica, num tom que vai da comédia à tragédia ao comentário científico em poucos lances – que o livro como um todo deve ser lido como uma grande xarada, um “claro enigma”. Em geral, dentre os escritos nietzschianos, este é o que costuma ter um efeito mais dramático e chocante sobre os leitores, o que se deve principalmente ao conteúdo de crítica radical à moderna democracia que ele contém, além do hermetismo de sua composição. 

	A nosso ver, a interpretação que Laurence Lampert desenvolveu ao longo de seu soberbo  \textit{Nietzsche's Task}: an interpretation of \textit{Beyond Good and Evil} (2001) não deixa dúvidas de que se trata de uma obra esotérica. Seu sentido esotérico foi perseguido, por diferentes vias, por Maudemarrie Clark e David Dudrick em \textit{The soul of Nietzsche's} Beyond Good and Evil (2012), que sem dúvidas conta como uma contribuição das mais valiosas ao trabalho de interpretação dessa obra; trata-se de uma leitura extremamente atenta às interlocuções, jogos de palavra, sugestões, elementos retóricos, e à argumentação algo capciosa de \textit{ABM}. Clark e Dudrick oferecem, enfim,  uma exegese que, dentre os comentários até então produzidos sobre o livro, se destaca pela minúcia de sua abordagem
\footnote{Nossa leitura se guia por vários elementos dessa exegese aplicados de forma localizada, apenas, uma vez que não aderirmos à tese de que em \textit{ABM} Nietzsche havia, de forma semi-consciente, abandonado seu compromisso com a tese falsificacionista ou “teoria do erro”, um ponto fundamental da interpretação de Clark e Dudrick, e que discutimos no primeiro capítulo deste trabalho.}. 
O problema é que, ao tomar como ponto de partida a convicção sobre a esotericidade do texto, e assumirem a tarefa de escavar por sob sua face visível um sentido oculto, os intérpretes abrem espaço para a projeção de suas próprias palavras sobre o texto nietzschiano. O fato de este ser um texto cheio de sugestões e lacunas a serem preenchidas pelo leitor, deixa espaço para que o leitor as preencha com uma miríade de formas possíveis. No caso de Clark e Dudrick este problema é agravado, a nosso ver, pela intenção deliberada de tornar o pensamento nietzschiano mais adequado às expectativas de certa linhagem filosófica de língua inglesa.
	
	Ademais, mesmo com a contribuição desses importantes intérpretes, acreditamos que ainda há que se discutir muito sobre qual peso atribuir, no interior da obra nietzschiana como um todo, às investidas políticas expressas em \textit{ABM}, que soam tão agressivas à sensibilidade moderna, e que apenas parcialmente poderemos endereçar aqui
\footnote{Questões polêmicas listadas por Clark e Dudrick: “This includes, for instance, its derogatory comments about women (BGE 231-239), (…) a dream of philosophers who will 'create' or 'legislate' values (BGE 213); a denigration of ordinary human beings, who are said to exist and to be allowed to exist only for service and the general utility (BGE 61); and a criticism of religions for preserving too many of those who should perish (BGE 62).” (CLARK, DUDRICK, 2012, p. 4.). Ver a seção “Platonismo político e vida contemplativa no entorno de \textit{Além de Bem e Mal}” em LOPES, 2008, p. 420-444.}.

As anotações e correspondências de Nietzsche atestam que o filósofo pretendia, a princípio, compor \textit{Além de Bem e Mal} como uma reedição de \textit{Humano, demasiado humano}. Nietzsche teria inclusive tentado tirar de circulação os exemplares de \textit{Humano}
\footnote{Informação lançada de forma não documentada por LAMPERT, 2001, p. 5.}. O por quê dessa empreitada de reedição é ainda objeto de discussão entre os estudiosos – de todo modo, a intenção de “reescrita” lança uma luz sobre a questão do hermetismo do texto, apontando para a busca de uma nova forma de atingir ou engajar o leitor, ou talvez de selecionar um certo tipo de leitor. 

	Para perseguirmos as mudanças que o filósofo tencionava inscrever na sua filosofia com essa reescrita, chamamos atenção para uma primeira pista encontrada já na forma de batismo de cada livro: enquanto \textit{Humano} é apresentado como um livro “para espíritos livres”, \textit{Além de Bem e Mal} seria um “prelúdio a uma filosofia do futuro”. Os subtítulos deixam claro, portanto, que há uma diferença fundamental quanto ao alcance da intervenção a que cada livro se destina – \textit{HH} delimita com clareza o campo de aplicação do que o livro tem de prescritivo, ou seja, o campo do modo de vida tipo espírito livre. O subtítulo de \textit{ABM} por outro lado, um tanto mais sugestivo, ao mirar o lugar da filosofia no futuro anuncia um tipo de preparação, e assim faz supor uma nova situação que abrigaria não só um novo modo de fazer filosofia mas um novo posto para a filosofia no interior da cultura. Pode-se considerar, ainda, uma diferença em relação à direção de cada projeto filosófico: enquanto os primeiros aforismos de \textit{HH} definem um programa de “filosofia histórica”, sugerindo a intenção de resgatar informações e aplicar um certo método dos quais a filosofia até então teria carecido, a abertura de \textit{ABM} destina-se a apresentar uma “filosofia do futuro”, esboçando uma recapitulação do que a filosofia foi até então e uma projeção do que ela pode ser.
	
	O subtítulo de \textit{ABM} sugere, portanto, a definição de um público e de um modo filosófico: seu público deve incluir todos aqueles de algum modo engajados na criação cultural, embora o caráter esotérico do livro garanta que apenas parte desse público tenha acesso às camadas mais profundas do texto
\footnote{Parece haver uma razão para tal: \textit{ABM} certamente anuncia uma meta de intervenção na cultura, e sugere uma hierarquia de funções no cumprimento dessa tarefa – cientistas, artistas e talvez certo tipo de homens religiosos teriam seu papel a desempenhar, mas eles estariam a serviço do filósofo, que assume para si a tarefa mais alta de criação de valores. Pode-se aventar a ideia de que é na medida em que o leitor consegue se engajar nas diferentes camadas do texto que assumirá para si uma parte dessa tarefa cultural. Eis o platonismo político de \textit{ABM}, que soa um tanto exótico a leitores modernos.}. 
Na medida em que visa no “futuro” as novas metas que delimitariam o campo filosófico, Nietzsche, por assim dizer, pede licença para antecipar hipóteses e formas de questionar para as quais não se tem ainda uma resposta segura, chamando atenção para o caráter “tentativo e tentador” de seu pensamento (\textit{JGB/ABM}, §42). Já de início, o prólogo de \textit{ABM} anuncia um programa de filosofia intervencionista, que ensaia um salto a ser dado, e portanto avança, arrisca, dispara a seta a partir de questões polêmicas e dificilmente respondidas. Isto deixa entrever uma característica importante do livro: embora com \textit{ABM} Nietzsche dê prosseguimento à crítica à forma tradicional de filosofar por especulação metafísica, o caráter da empreitada que se propõe enfrentar parece demandar algum grau de especulação filosófica, e esta já é uma dissonância notável em relação a \textit{HH}. Há, no entanto, um claro paralelismo entre muitas das seções de \textit{ABM} com aquelas de \textit{HH}, enquanto as principais tensões que se desenvolveram na filosofia nietzschiana entre uma obra e outra – e que giram em torno dos conceitos de vontade de verdade, sentimento de poder e honestidade intelectual – são acionadas na elaboração de um programa filosófico marcadamente diferente daquele de HH. E embora a ciência cumpra ainda um papel fundamental nesse novo programa, o lugar que o filósofo lhe destina é hierarquicamente posicionado a serviço das ambições mais gerais da nova filosofia a ser perseguida.

Sem mais delongas, passemos ao prefácio lapidar que Nietzsche confeccionou para \textit{ABM} em junho de 1885, portanto antes da escrita do livro.

\section{Eros, erro, filosofia e ciência}

\begin{quotation}
SUPONDO QUE A VERDADE SEJA UMA MULHER – não seria bem fundada a suspeita de que todos os filósofos, na  medida em que foram dogmáticos, entenderam pouco de mulheres? De que a terrível seriedade, a desajeitada insistência com que até agora se aproximaram da verdade, foram meios inábeis e impróprios para conquistar uma dama? É certo que ela não se deixou conquistar – e hoje toda espécie de dogmatismo está de braços cruzados, triste e sem ânimo. Se é que ainda está de pé! Pois há os zombadores que afirmam que caiu, que todo dogmatismo está no chão, ou mesmo que está nas últimas. Falando seriamente, há boas razões para esperar que toda dogmatização em filosofia, não importando o ar solene e definitivo que tenha apresentado, não tenha sido mais que uma nobre infantilidade e coisa de iniciantes; e talvez esteja próximo o tempo em que se perceberá quão pouco bastava para constituir o alicerce das sublimes e absolutas construções filosofais que os dogmáticos ergueram – alguma superstição popular de um tempo imemorial (como a superstição da alma, que, como superstição do sujeito e do Eu, ainda hoje causa danos), talvez algum jogo de palavras, alguma sedução por parte da gramática, ou temerária generalização de fatos muito estreitos, muito pessoais, demasiado humanos. A filosofia dos dogmáticos foi, temos esperança, apenas uma promessa através dos milênios: assim como em época anterior a astrologia, a cujo serviço talvez se tenha aplicado mais dinheiro, trabalho, paciência, perspicácia do que para qualquer ciência verdadeira até agora: a ela e suas pretensões “supraterrenas” deve-se o grande estilo da arquitetura na Ásia e no Egito. Parece que todas as coisas grandes, para se inscrever no coração da humanidade com suas eternas exigências, tiveram primeiro que vagar pela Terra como figuras monstruosas e apavorantes: uma tal caricatura foi a filosofia dogmática, a doutrina vedanta na Ásia e o platonismo na Europa, por exemplo. Não sejamos ingratos para com eles, embora se deva admitir que o pior, mais persistente e perigoso dos erros até hoje foi um erro de dogmático: a invenção platônica do puro espírito e do bem em si. Mas agora que está superado, agora que a Europa respira novamente após o pesadelo, e pode ao menos gozar um sono mais sadio, somos nós, \textit{cuja tarefa é precisamente a vigília}, os herdeiros de toda a força engendrada no combate a esse erro. Certamente significou pôr a verdade de ponta-cabeça e negar a \textit{perspectiva}, a condição básica de toda vida, falar do espírito e do bem tal como fez Platão; sim, pode-se mesmo perguntar como médico: “De onde vem essa enfermidade no mais belo rebento da Antigüidade, em Platão: O malvado Sócrates o teria mesmo corrompido? Teria sido realmente Sócrates o corruptor da juventude? E teria então merecido a cicuta?”. – Mas a luta contra Platão, ou, para dizê-lo de modo mais simples e para o “povo”, a luta contra a pressão cristã-eclesiástica de milênios – pois cristianismo é platonismo para o “povo” – produziu na Europa uma magnífica tensão do espírito, como até então não havia na Terra: com um arco assim teso pode-se agora mirar nos alvos mais distantes. Sem dúvida o homem europeu sente essa tensão como uma miséria; e por duas vezes já se tentou em grande estilo distender o arco, a primeira com o jesuitismo, a segunda com a Ilustração democrática – a qual pôde realmente conseguir, com ajuda da liberdade de imprensa e da leitura de jornais, que o espírito não mais sentisse facilmente a si mesmo como “necessidade”! (Os alemães inventaram a pólvora – todo respeito! –, mas ficaram novamente quites: inventaram a imprensa.) Mas nós, que não somos jesuítas, nem democratas, nem mesmo alemães o bastante, nós, \textit{bons europeus} e espíritos livres, \textit{muito} livres, nós ainda as temos, toda a necessidade do espírito e toda a tensão do seu arco! E talvez também a seta, a tarefa e, quem sabe? \textit{A meta}... (JGB/ABM, Pr.). 
\end{quotation}

Supondo que a verdade seja uma mulher, Nietzsche se apresenta como um pretendente, e  assume o tom da disputa, do enfrentamento agônico com os outros pretendentes. A menção figurativa à invenção da pólvora imprime de início o tom belicoso característico do livro. Já o imaginário do arco remete a Heráclito, mas também a Eros, cujas armas são justamente o arco e a flecha
\footnote{Clark e Dudrick notam que o imaginário do arco aparece também na \textit{República}, em que é usado por Sócrates para estabelecer o argumento da tripartição da alma. Cf. CLARK, DUDRICK, 2012, p. 26.}. 
O filósofo que debocha da inaptidão dogmática para a conquista da verdade reclama as armas de Eros para um novo tipo de filósofo, “nós, \textit{bons europeus} e espíritos livres, \textit{muito} livres”; e com isto mostra que antes de desistir da conquista, mira no futuro a possibilidade de busca filosófica da verdade. Tudo indica que há aqui uma avaliação positiva da vontade de verdade, honestidade intelectual, paixão do conhecimento e afins, e que Nietzsche disputará com a tradição a conquista desses valores. A rivalidade com Platão ganha destaque. A acusação contra Platão enquanto autor do “pior, mais persistente e perigoso dos erros até hoje” não deve nos distrair do significado que há em colocar essa rivalidade no centro da disputa: o maior rival é aquele que detém algo que se deseja alcançar, ou superar – lembrando que Platão é o único filósofo diretamente mencionado no prefácio. A luta contra Platão, ou contra a adaptação cristã da cultura platônica, teria sido o que “produziu na Europa uma magnífica tensão do espírito, como até então não havia na Terra” – um livro que anuncia a ambição de disparar uma nova meta cultural a partir dessa tensão terá que se haver com esse ascendente.

O texto esboça uma história da filosofia extremamente condensada, que seria a história de um combate que se definiu na antiguidade, se estendendo até a moderna ilustração democrática. Quais são as partes combatentes? Somos apresentados, a princípio, à filosofia dogmática, mas não encontramos uma definição precisa e explícita do sentido de dogmatismo que se tem em vista, definição que tampouco se encontra ao longo do livro. A primeira pista que temos está na referência a Platão; conquanto este não seja exatamente apontado como um filósofo dogmático, Nietzsche conta como um erro tipicamente dogmático sua invenção “do puro espírito e do bem em si”. Clark e Dudrick apontam boas razões para que “dogmático”, neste caso, seja tomado segundo a acepção kantiana, qual seja:

\begin{quotation}
Procedimentos dogmáticos são [portanto] métodos completamente a priori, em oposição a métodos empíricos. “Dogmatismo” é o termo kantiano para aqueles sistemas filosóficos que \textit{acriticamente} pressupõem a capacidade da razão pura – a razão enquanto operando sem recorrer a informações obtidas pelos sentidos – de obter conhecimento substantivo da realidade. Dogmáticos portanto pressupõem a possibilidade de conhecer objetos que estão além dos limites da experiência. (CLARK, DUDRICK, 2012, p. 20).
\end{quotation}

Os autores alegam que é este o sentido corrente do termo na época em que Nietzsche escreveu \textit{ABM}, inclusive entre filósofos neo-kantianos como Afrikan Spir e Gustav Teichmüller, cujas obras Nietzsche lia ou relia quando da composição do livro
\footnote{Clark e Dudrick desenvolvem, ainda, uma comparação bastante eloquente entre o prólogo de \textit{ABM} e o prólogo da \textit{Crítica da Razão Pura}, apontando o paralelo entre a metáfora da verdade enquanto mulher, que se encontra em \textit{ABM}, e a metáfora da metafísica enquanto mulher, tema do prólogo da \textit{CRP}. Ainda, a afirmação de que a Europa “pode ao menos gozar um sono mais sadio” parece aludir ao “sono dogmático” de Kant. Os autores argumentam convincentemente a favor da ideia de que o prólogo de ABM pode ser lido como uma espécie de paródia do prólogo da CRP. Que Nietzsche tivesse o texto de Kant em mente conta a favor da tese de que é ao sentido kantiano de “dogmático” e “dogmatismo” que se refere. Cf. CLARK, DUDRICK, 2012, p. 13-22.}. 
A respeito do sentido de dogmatismo que está em jogo, e suas implicações na interpretação do texto em questão, Clark e Dudrick apontam ainda dois pontos importantes: primeiro, que há uma proximidade entre “dogmático” e “metafísico”, pois como explicam os autores, “Dogmatismo é a confiança acrítica na habilidade da razão de conhecer objetos a priori. Metafísica é o corpo de pretensos conhecimentos a priori dos objetos.” (CLARK, DUDRICK, 2012, p. 20), e por isto os dois termos seriam aplicáveis ao mesmo grupo de filósofos, sendo que estes se opõem, em linhas gerais, aos filósofos empiristas (CLARK, DUDRICK, 2012, p. 22); segundo, que Nietzsche usualmente acusa o dogmatismo de introduzir inadvertidamente um fator valorativo sob o pretexto de afirmações puramente epistemológicas ou ontológicas, que então buscam legitimar por meios algo desonestos, ou seja, para Nietzsche “Dogmáticos são aqueles que tentaram defender seus valores como verdades apelando para a possibilidade de conhecimento a priori” (CLARK, DUDRICK, 2012, p. 18). Ao que parece, é essa imposição valorativa, típica do dogmatismo, o que é criticado enquanto uma “temerária generalização de fatos muito estreitos, muito pessoais, demasiado humanos”.

O sentido de dogmatismo apontado por Clark e Dudrick parece se adequar bem à denúncia de Nietzsche contra Platão. Dessa forma, o conteúdo do “erro dogmático” de Platão consistiria em postular uma entidade metafísica, o “puro espírito”, capaz de inteligir uma entidade metafísica imbuída de sentido moral “o bem em si”. Seria justamente uma avaliação moral de mundo, fundada sobre um pressuposto metafísico, a herança platônica do cristianismo, seria esse o conteúdo do “platonismo para o povo”. Que dois detalhes não se percam de vista aí: o fato de Platão não ser apontado como um filósofo dogmático, mas apenas como autor de um axioma dogmático, e o fato de que Nietzsche se refira a esse conteúdo dogmático da filosofia platônica como um artigo destinado ao “povo”, ou seja, um artigo exotérico. Esses detalhes sugerem que Platão não seria, afinal, um dogmático.

	A suspeita que se levanta aqui, e que ecoa discretamente ao longo do livro, é a de que Platão na verdade não cria em sua própria invenção, mas necessitava dela como uma forma de instaurar uma ordem moral de convivência. O “puro espírito” e o “bem em si” seriam a \textit{pia fraus} de Platão, seu modo de fundar uma moral antigrega, segundo Nietzsche. 
	
	Isto dito, voltamos à questão da identificação das partes combatentes cuja luta, segundo Nietzsche teria legado à modernidade uma “magnífica tensão do espírito”. Por um lado temos, portanto, Platão, a tradição cristã e uma visão moral de mundo metafisicamente fundada. Quem são seus detratores? O texto do prefácio não parece suficiente para respondermos a essa questão. Lampert aponta como chave do enigma um trecho também enigmático do livro:

\begin{quotation}
Como podem ser maldosos os filósofos! Não conheço nada mais venenoso do que a piada que Epicuro fez às custas de Platão e os platônicos: chamou-os de \textit{dionysiokolakes}. Isto significa, literalmente e em primeiro lugar, “aduladores de Dionísio”, ou seja, clientes de tiranos e puxa-sacos servis; além de tudo quer dizer, porém, que “são todos atores, nada neles é autêntico” (pois \textit{dionysokolax} era uma denominação popular para ator). E neste outro sentido está realmente a malícia que Epicuro lançou contra Platão: aborrecia-lhe o modo grandioso, a \textit{mise-en-scène} em que Platão e seus discípulos eram entendidos – e de que Epicuro nada entendia! Ele, o velho mestre-escola de Samos, que se escondeu no seu jardinzinho de Atenas e escreveu trezentos livros, talvez – quem sabe? – por ambição e raiva de Platão? – Foram precisos cem anos para a Grécia descobrir quem fora Epicuro, esse deus do jardim. – Mas descobriu? (\textit{JGB/ABM}, §7).
\end{quotation}

Aqui se identifica, então, o rival de Platão: Epicuro, herdeiro do pensamento científico de Demócrito. O que mais incomodava Epicuro a respeito de Platão era o apelo demagógico de sua filosofia, a \textit{mise-en-scène} que tornou sua escola a mais influente do ocidente. A piada maliciosa de Epicuro, ao sugerir que a filosofia platônica obteve sucesso em razão de uma \textit{atuação}, reforça a suspeita de que a invenção platônica era, na verdade, uma \textit{pia fraus}, uma mentira com fins edificantes
\footnote{Essa suspeita é retomada num outro aforismo, que tem início com uma reflexão sobre a dificuldade de tradução do \textit{tempo} com que um autor expressa seu pensamento, e termina com uma reflexão sobre a “natureza esfíngica” de Platão: “nada me fez refletir mais sobre a reserva a natureza esfíngica de Platão do que esse \textit{petit fait} [pequeno fato], felizmente conservado, que sob o travesseiro do seu leito de morte não se encontrou nenhuma 'Bíblia', nada egípcio, pitagórico, platônico, – mas sim Aristófanes. Como poderia até mesmo um Platão suportar a vida – uma vida grega, a qual ele disse 'não' – sem um Aristófanes?” (JGB/ABM, §28). Segundo Nietzsche, Platão teria traduzido para a cultura grega um outro \textit{tempo} de pensamento, introduzindo um moralismo estrangeiro. A afinidade de Platão com Aristófanes, no entanto, sugere a possibilidade de que Platão, afinal, não fosse um platônico, de modo que o sucesso do platonismo teria dependido necessariamente de uma má compreensão deliberadamente induzida. Sua “natureza esfíngica”, teria talvez encoberto uma motivação filantrópica, que o fez avançar a tese do “puro espírito” e do “bem em si” como uma forma de redução de danos sociais. Cf. LAMPERT, 2001, p. 69.}. 
Teria sido precisamente uma intenção moralista o que levou Platão a rejeitar a cultura científica grega que herdou; segundo Lampert, o “dogmatismo platônico” teria suplantado a cultura helênica que o precedeu, rejeitando sua “sabedoria trágica”, devido a uma convicção “não quanto a sua superior veracidade mas quanto a sua superior segurança” (LAMPERT, 2001, p.4).

	Podemos agora avançar a leitura de que a “magnífica tensão do espírito” herdada pelos modernos teve início com um embate entre Platão e os pré-platônicos; a origem dessa tensão, portanto, estaria localizada no confronto entre uma visão científica e uma visão moral de mundo no interior do helenismo. Esse embate longínquo teria se expressado na modernidade como uma luta entre a “pressão eclesiástica” cristã e o renascimento científico na Europa. Com isto, começa a se esclarecer a razão de Nietzsche apostar que justamente a modernidade seria o momento em que a tensão entre as duas pulsões envolvidas nesse embate atinge seu ponto mais alto. Segundo Lampert:

\begin{quotation}
O esotericismo tradicional da filosofia, claramente estabelecido por Platão, acreditava na indispensabilidade da nobre mentira, ficções morais que direcionassem os medos e esperanças dos cidadãos para práticas decentes, de espírito público, por apelo a uma ordem moral e introdução de bens morais como fiadores punitivos e recompensadores da moral. “Isso agora acabou”, Nietzsche diz simplesmente (GC 357) – agora que Deus está morto, agora que a moderna ciência roubou do Platonismo qualquer respeitabilidade intelectual e deixou claro a todos que a humanidade vive seus medos e esperanças num universo silencioso, irresponsivo. (LAMPERT, 2001, p. 4).
\end{quotation}

Por que, então, se “isso agora acabou”, se a ciência retirou toda credibilidade da interpretação moral de mundo fundada numa crença metafísica, se houve afinal uma vitória tardia de Epicuro e dos fisiólogos gregos sobre a \textit{pia fraus} de Platão – por que seria justamente esse o momento de maior tensão?

	Não há qualquer indício de que Nietzsche tenha intenção de reabilitar os artigos de fé da filosofia dogmática; ao que tudo indica, não há espaço na filosofia do futuro para as noções de Deus, alma, puro espírito e bem em si conforme tradicionalmente compreendidas. Embora expresse seu rechaço a essas noções enquanto “erros” dogmáticos, Nietzsche alerta para que “Não sejamos ingratos para com eles”. Há aqui uma reivindicação de que o relativo triunfo da perspectiva naturalista não seja visto como o fim das tensões engendradas pela luta contra o dogmatismo. A crítica de Nietzsche ao jesuitismo refere-se precisamente à tentativa de “distender o arco” ao tornar o cristianismo mais palatável à sensibilidade moderna. Conquanto o triunfo da ciência teria permitido à Europa “gozar um sono mais sadio”, aos filósofos do futuro caberia a tarefa da “vigília”, isto, é eles devem estar atentos à possibilidade de que exista algo, algum valor, alguma função na herança legada pelo dogmatismo, algo a ser assimilado na manutenção da tensão necessária para o desenvolvimento de uma “filosofia do futuro”. O que há de valioso nessa herança não são os artigos de fé que produziu, mas o desafio de criar sentido e valor para a experiência humana no mundo, desafio tanto mais difícil frente à perspectiva científica, que apresenta um “universo silencioso e irresponsivo”.
	
	Por outro lado, a “inaptidão” dos filósofos dogmáticos para a conquista da verdade pode se referir a uma falta de virtuosismo metodológico no modo como estes construíram sua filosofia. Assim, a negligência em relação às informações empíricas (históricas, psicológicas, geopolíticas, etc.), e a falta de honestidade no modo de projetar seus valores do mundo podem muito bem ser uma “inaptidão” apenas contornável mediante o exercício no rigor do pensamento científico.

Dogmatismo e pensamento científico, as duas forças capazes de “retesar o arco”, teriam legado, cada uma delas, uma exigência e uma limitação que apontam a direção da “filosofia do futuro”, da qual Nietzsche certamente se vê como um precursor: por parte do dogmatismo, a necessidade de criar sentidos e valores de amplo alcance cultural; esta é limitada pela exigência de integridade intelectual que se tornou indispensável numa cultura científica, no interior da qual a possibilidade de “sacrifício do intelecto” em razão de fins edificantes perdeu qualquer apelo. Por outro lado, a atividade científica encontra seu limite exatamente na necessidade de criação de valores: não é a ciência que assumirá essa tarefa.

	Mas como é possível essa tarefa? Como é possível experimentar o mundo de forma valorativa mas não dogmática? Como é possível experimentar o mundo de forma valorativa se a ciência nos oferece a visão de um mundo “silencioso e irresponsivo”? Qualquer sentença valorativa nessas circunstâncias não seria uma \textit{pia fraus}, ou até mesmo uma ímpia falsificação? Ademais, qual é a tarefa normativa a ser assumida pelo filósofo uma vez que ele não mais é motivado a fundar uma ordem moral por razões filantrópicas? Em que solo a “filosofia do futuro” plantará seus valores, se ela renunciou ao terreno metafísico? 
	
	Perpassando essas questões tão profundas e tão desafiadoras, parece haver uma questão metodológica. Clark e Dudrick rejeitam a ideia de que Nietzsche seria um naturalista metodológico, isto é, um filósofo que reivindica para a filosofia os mesmos métodos que pautam a atividade científica (Cf. CLARK, DUDRICK, 2012, p. 9). Concordamos parcialmente com a leitura de Clark e Dudrick na medida em que entendemos que Nietzsche reivindica para sua “filosofia do futuro” uma série de aptidões e virtudes que não se reduzem às virtudes epistêmicas cultivadas na atividade científica, quer dizer, Nietzsche pode ser lido como um naturalista metodológico na medida em que isto não implique na exigência de que a filosofia deva seguir \textit{exclusivamente} os métodos científicos. Mas, ao que nos parece, é justamente porque a ciência cultiva um virtuosismo metodológico que ela é capaz de fazer frente ao dogmatismo filosófico, criticado pelo filósofo como um modo algo “desonesto” de assumir a tarefa filosófica de criação de valores
\footnote{Clark e Dudrick não consideram a possibilidade de que seja esse o papel da ciência; sua leitura parece ser prejudicada por entenderem que os “métodos” oferecidos pela ciência são essencialmente procedimentais, quantitativos. Eles parecem não levar em conta o sentido de “método” como disciplina do pensamento, “espírito científico”.}. 
Ao que tudo indica, é essa cautela metodológica o que permitiria evitar que a criação de valores recaia na “temerária generalização de fatos muito estreitos, muito pessoais, demasiado humanos”, que segundo Nietzsche tem sido uma marca do dogmatismo.

No caso da interpretação de Clark e Dudrick, a rejeição do naturalismo metodológico se deve ao compromisso dos autores com uma premissa questionável: a de que o campo da criação de valor seria radicalmente diferente do campo de produção de verdades científicas sobre o mundo. Segundo Clark e Dudrick, a filosofia nietzschiana opera com uma distinção entre o “espaço das causas” e o “espaço das razões”: enquanto o “espaço das causas” abarcaria os fenômenos passíveis de mensuração e quantificação científica, o “espaço das razões” estaria reservado para as experiências normativas, que se supõe eminentemente humanas – enquanto a investigação do espaço das causas resulta em conhecimento sobre o que algo é, no espaço das razões investiga-se o modo como algo \textit{faz sentido} ou como algo \textit{tem valor}. Um exemplo dado pelos autores: posso afirmar que as \textit{causas} de um bolo de aniversário que preparei são o trigo, o ovo, o açúcar, etc., por outro lado, a \textit{razão} de eu ter feito um bolo de aniversário remete a uma tradição cultural, à minha própria valorização dessa tradição, etc. Essa distinção tem implicações metodológicas das mais importantes: ela serve à legitimação de uma separação bem definida entre ciências naturais (operando no interior do espaço das causas) e ciências humanas (operando no interior do espaço das razões).

	A nosso ver, a leitura de Clark e Dudrick, conquanto argumentada quase sempre de forma muito contundente e  meticulosa, causa uma série de estranhamentos. Em primeiro lugar, porque o propósito de distinção estrita entre ciências naturais e humanas foi tema da \textit{Methodenstreit}, a querela metodológica que se desenrolou nas universidades alemãs ao longo de quase todo o século XIX, e da qual Nietzsche optou deliberadamente por não participar. Em seus escritos, não há qualquer indicação de que o filósofo considere que esta seja uma distinção fundamental, pelo contrário, em muitas ocorrências as ciências naturais e humanas são mencionadas conjuntamente, recebendo o mesmo tratamento; pode-se mencionar o interesse de Nietzsche pela psicofisiologia, que contaria como uma ciência híbrida e que teve entre seus colaboradores o fisiólogo e filósofo Hermann von Helmholtz, ou ainda, suas tentativas no sentido de exercer a filologia como um agregado “centáurico” de ciência, arte e filosofia, e até mesmo sua intenção anunciada de deslocar a filologia do posto de “história das ideias” para o de “história dos impulsos” (Cf. PORTER, 2000, p.61). Além disso, a distinção entre espaço de causas e espaço de razões serve à reivindicação deste último como um âmbito distintamente humano, e assim reintroduz na filosofia uma separação estrita entre mundo natural e mundo humano, separação tantas vezes combatida pelo filósofo. Não se encontra nos escritos qualquer alusão ao homem como “animal racional”; as ocorrências de “animal não fixado” ou “animal doente” sugerem antes uma diferença apenas de grau em maleabilidade (cultural). Essa suposta separação mantém o terreno da ética, da estética e da psicologia – tidos por distintivamente humanos – fora do alcance da investigação naturalista (Cf. CLARK, DUDRICK, 2012, p. 122), de forma que, ao que nos parece, ela pode ser igualmente criticada como uma “crença na oposição dos valores”, contra a qual Nietzsche se volta no primeiro aforismo de \textit{HH}, espelhado no segundo aforismo de \textit{ABM}; criar um território especial para essas áreas de atividades humana parece ser justamente o caso das tentativas de alojá-las no “seio do ser, do intransitório”, alegando que “nisso, e em mais nada deve estar sua causa!” (JGB/ABM, §2).
	
	Ao que nos parece, portanto, a interpretação de Clark e Dudrick introduz certas dicotomias na filosofia nietzschiana, enquanto o próprio Nietzsche declaradamente se esforçou para dissolvê-las. Os autores argumentam que Nietzsche travou contato com a distinção conceitual entre espaço das causas e espaço das razões através da obra de Spir. No entanto, fato é que ao longo de \textit{ABM} Nietzsche aborda  uma série de posicionamentos e conceitos próprios do repertório spiriano, e em geral o faz de forma muito crítica, mas nele não se encontra algo que se pareça com uma afirmação da distinção entre espaço das causas e das razões nesse sentido.

Essa distinção tem um papel estrutural na argumentação de Clark e Dudrick. Ao entender que o espaço das causas encerra o domínio dos fatos relativos ao mundo natural (cientificamente acessíveis), enquanto o espaço das razões circunscreve o âmbito dos fatos relativos a valores (a serem investigados via uma abordagem puramente normativa), os autores avançam a tese de que a “magnífica tensão do espírito”, que segundo Nietzsche seria o motor da filosofia, é gerada por um conflito entre “vontade de verdade”, a ser satisfeita por investigação empírica científica, e “vontade de valoração”, que encontra satisfação na atividade filosófica normativa. Note-se o quanto essa interpretação aproxima a filosofia de Nietzsche do projeto kantiano, tantas vezes criticado por se destinar a criar um lócus no qual os artigos de fé são inatacáveis.

	Essa leitura tem sido criticada porque não há qualquer ocorrência textual de “vontade de valoração”, embora haja um campo bastante amplo de ocorrências relativas à “vontade”: “vontade de verdade”, “vontade de saber” (\textit{Wille zur Wahrheit, Wille zum Wissen}, §1, §10, §24); “vontade de ilusão”, “vontade de inverdade”, “vontade de estupidez” (\textit{Wille zur Täuschung, Wille zum Unwahren, Wille zur Dummheit}, §2, §24, §59, 107) e “vontade de poder” (\textit{Wille zur Macht}, §9, §13, §22, §23, §36, §44, §51, §186, §198, §211, §227, §257, §259). Além disso, deve-se considerar que parte significativa da genealogia nietzschiana se destina a mostrar como a vontade de verdade nasce de exigências morais, de forma que essa vontade ela mesma é movida por um valor, o valor da verdade – um ponto para o qual Paul Katsafanas chamou atenção em sua resenha do livro, concluindo que: “Isto implica em que a 'magnífica tensão do espírito' entre vontade de verdade e vontade de valoração seria antes uma tensão no interior da vontade de valoração. Seria uma tensão entre valores particulares, ao invés de uma tensão entre valor e alguma outra coisa.” (KATSAFANAS, 2013, p. 4). 
	
	Pelo que foi dito, acreditamos que há dificuldades o bastante para ver como insatisfatória a tese dos autores de que a “magnífica tensão do espírito” em que Nietzsche deposita suas esperanças para uma filosofia do futuro seja pensada como o conflito entre uma vontade de verdade realizada no plano empírico e uma suposta vontade de valoração a ser satisfeita no âmbito normativo
\footnote{A nosso ver, os autores são levados a essas teses problemáticas em razão do tratamento que dispensam a dois pontos que nos parecem cruciais, isto é, a questão do sentido do conceito de valor, que optam por interpretar de forma não naturalizada, e a questão do erro, que descartam por entenderem que em \textit{ABM} Nietzsche rejeita a chamada “teoria do erro” ou “tese falsificacionista”, que discutimos no capítulo \ref{cap1}. Trataremos desses pontos mais adiante.}. 

Ao que nos parece, as duas grandes interpretações de \textit{ABM} que temos discutido, a interpretação de Lampert e a de Clark e Dudrick apontam para algo que de fato deve ser levado em conta na busca por compreender a “magnífica tensão do espírito” filosófico, embora haja alguma insatisfação com os termos que empregaram nessa busca. Lampert aponta as forças atuantes nesse \textit{ágon} filosófico como um embate entre pensamento científico, por um lado, e \textit{pia fraus} dogmática por outro – e a dificuldade aqui é que, para Lampert, não há dúvidas de que Nietzsche vê como inócua qualquer tentativa de \textit{pia fraus} no mundo pós-Esclarecimento, de forma que o leitor é levado à ideia de que apenas um dos lados desse conflito, o espírito de investigação científica, teria ainda algum \textit{tônus} de atuação. Já Clark e Dudrick, ao localizarem a tensão num embate entre vontade de verdade e vontade de valoração fazem um esforço hercúleo para neutralizar as tantas ocorrências em que Nietzsche lança a questão sobre o valor da vontade de verdade, sumariamente desconsideram as ocorrências em que esta é confrontada com seu oposto, a “vontade de inverdade”, e assim desconsideram um conteúdo importante da filosofia nietzschiana, que é o questionamento da associação imediata entre o verdadeiro e o bom, ou o questionamento da suposição de adequação entre a natureza humana e a verdade; a nosso ver, os autores são levados a negligenciar esse ponto devido a seu compromisso altamente problemático e fracamente argumentado com a ideia de que Nietzsche desfez seu compromisso com a teoria do erro. Por outro lado, os autores resguardam a vontade de valoração no âmbito do “espaço de razões”, à parte da atividade empírica de busca da verdade – como resultado, é difícil ver o que restaria ainda de \textit{tensão} entre essas duas esferas.

	Acreditamos que uma chave importante para essa questão se encontra no aforismo 6 de \textit{ABM}, que a nosso ver recebe um tratamento não inteiramente satisfatório por parte dos autores discutidos, especialmente por Clark e Dudrick. Trata-se de um trecho importante sobre as tensões que, para Nietzsche, caracterizam a própria atividade de filosofar:

\begin{quotation}
Gradualmente foi se revelando para mim o que toda grande filosofia foi até o momento: a confissão pessoal de seu autor, uma espécie de memórias involuntárias e inadvertidas; e também se tornou claro que as intenções morais (ou imorais) de toda filosofia constituíram sempre o germe a partir do qual cresceu a planta inteira. De fato, para explicar como surgiram as mais remotas afirmações metafísicas de um filósofo é bom (e sábio) se perguntar antes de tudo: a que moral isto (ele) quer chegar? Portanto não creio que um “impulso ao conhecimento” seja o pai da filosofia mas sim que um outro impulso, nesse ponto e em outros, tenha se utilizado do conhecimento (e do desconhecimento!) como um simples instrumento. Mas quem examinar os impulsos básicos do homem, para ver até que ponto eles aqui teriam atuado como gênios (ou demônios, ou duendes) \textit{inspiradores}, descobrirá que todos eles já fizeram filosofia alguma vez – e que cada um deles bem gostaria de se apresentar como finalidade última da existência e legítimo senhor dos outros impulsos. Pois todo impulso ambiciona dominar: e \textit{portanto} procura filosofar. – Entre os doutores, é certo, entre os homens verdadeiramente científicos, pode ser diferente – “melhor”, se quiserem –, nesse caso pode haver realmente algo como um impulso a conhecer, algum pequeno mecanismo autônomo que, uma vez acionado, põe-se a trabalhar animadamente, sem que qualquer outro impulso tenha participação essencial. Por isso os verdadeiros “interesses” do homem douto se acham normalmente em outra parte, talvez na família, na obtenção de dinheiro ou na política; quase não faz diferença se a sua pequenina máquina é empregada nesta ou naquela área da ciência, ou que o jovem e “esperançoso” trabalhador se transforme num bom filólogo, químico ou especialista em cogumelos: – ele não é \textit{caracterizado} pelo fato de se tornar isso ou aquilo. No filósofo, pelo contrário, absolutamente nada é impessoal; e particularmente a sua moral dá um decidido e decisivo testemunho \textit{de quem ele é} – isto é, da hierarquia em que se dispõem os impulsos mais íntimos da sua natureza. (JGB/ABM, §6). 
\end{quotation}

Esse aforismo é valioso para o tratamento da questão porque ele avança uma tese sobre a origem da filosofia, segundo a qual esta não remete a um “impulso ao conhecimento”, uma vez que este seria apenas instrumento de “um outro impulso”. A leitura de Clark e Dudrick sugere que esse “outro impulso” seria justamente o impulso ou vontade de valoração, mas isto simplesmente não parece se seguir do texto. Na verdade, a sequência do texto aponta que \textit{todos} os impulsos já quiseram alguma vez filosofar, e aí encontra-se uma definição do próprio ato de filosofar: os impulsos já quiseram filosofar porque bem gostariam de “se apresentar como finalidade última da existência e legítimo \textit{senhor} dos outros impulsos”. Há aqui uma identificação entre \textit{filosofia} e \textit{assenhoramento} (\textit{Herrschsucht})\footnote{Um ponto que não passou despercebido a Lampert. Cf. LAMPERT, 2001, p. 31.} – ou seja, uma identificação entre filosofia e pathos de dominação. Parece-nos notavelmente supérflua a introdução de uma “vontade de valoração” aqui; seria suficiente manter uma visão naturalizada dos valores, de forma que cada impulso seria expressão da valorização de um certo objeto ou resultado, sendo que os impulsos disputariam entre si a oportunidade de “fazer valer” seus valores. Assim, pode-se pensar o valor como contínuo aos impulsos, e a filosofia como atividade afirmativa do valor expresso pelos impulsos do filósofo
\footnote{Interpretação proposta por RICHARDSON, 2004.}. 

	Estamos de acordo com Clark e Dudrick, no entanto, ao criticarem a interpretação que Brian Leiter oferece a esta passagem. Leiter afirma que a questão sobre “como surgiram as mais remotas afirmações metafísicas de um filósofo” deve ser respondida por apelo aos impulsos em si, isto é, as construções filosóficas podem ser reduzidas ao tipo de impulso que caracteriza o filósofo seu autor. Mais precisamente: segundo Leiter, as crenças teóricas de uma pessoa podem ser explicadas por remissão a suas crenças morais, e suas crenças morais são explicadas pelos fatos naturais sobre o tipo de pessoa que se é, no caso, os impulsos seriam como que as unidades básicas desses fatos naturais (Cf. LEITER, 2002, p. 8). A questão é que Nietzsche conclui o aforismo com uma definição daquilo que o filósofo é, e nessa definição ele não é imediatamente identificado com seus impulsos, mas com a “hierarquia”, isto é, o arranjo “em que se dispõem os impulsos mais íntimos da sua natureza”. Clark e Dudrick parecem inteiramente corretos ao afirmar que a identidade pessoal não advém da soma de impulsos que constituem uma pessoa, mas de uma espécie de “arranjo político” que instaura uma certa ordem entre os impulsos. Este parece ser um conceito interessante para a definição de identidade pessoal, não só do filósofo.

Um pequeno detalhe parece ter passado despercebido aos intérpretes, no entanto: a afirmação de que “particularmente” no caso do filósofo, a moral dá testemunho de quem ele é. Se a moral pode ser pensada como um sistema de valores afirmados por uma pessoa, e se no caso do filósofo haveria um ajuste mais preciso entre a moral afirmada e a personalidade do filósofo, o que o filósofo é, isto sugere que no caso do filósofo há um exercício pessoal de projeção de sua própria hierarquia de valores num sistema moral – ao passo que, pode-se imaginar, a moral afirmada por uma pessoa que não é filósofa não dá um testemunho preciso de quem ela é porque ela não está pessoalmente engajada em projetar seus próprios valores na forma de uma moral, certamente porque recebeu sua moral de fontes exteriores. Juntando as peças: o que uma pessoa é se identifica com a ordem hierárquica de seus impulsos e respectivos valores; a moral de um filósofo é a expressão dos valores que caracterizam sua personalidade porque a filosofia é a atividade mesma de projeção e afirmação da própria personalidade, movida por um pathos de dominação. Essa definição do que a filosofia \textit{é} parece ser positivamente aceita e endossada por Nietzsche, o que não quer dizer que ele se contente em ser um continuador do que a filosofia \textit{foi} “até o momento: a confissão pessoal de seu autor, uma espécie de memórias involuntárias e inadvertidas”. Parece haver uma advertência clara aqui: conquanto a projeção da personalidade pode muito bem ser o aspecto fundamental da filosofia,  essa projeção não deve necessariamente ser feita de forma \textit{involuntária} ou \textit{inadvertida}, como foi no passado. Um filósofo que tenha cultivado virtudes epistêmicas pode passar a exigir maior honestidade na projeção de experiências e valores pessoais.

	Acreditamos que, nessa altura, podemos esboçar uma nova definição das forças atuantes na “magnífica tensão do espírito” capaz de engendrar uma filosofia do futuro: por um lado, o aspecto original e fundamental da atividade filosófica, ou seja, a afirmação valorativa daquilo que o filósofo tem de mais pessoal; por outro lado, as exigências epistêmicas de cautela, prudência, \textit{objetividade}
\footnote{Entendemos que, segundo o pensamento nietzschiano, o maior grau de objetividade é conferido pela maior capacidade de coligir e confrontar perspectivas. A objetividade “expande” os espaços da subjetividade, e se opõe a tudo o que é estreitamente pessoal, unilateral e provinciano. Nossa referência para essa suposição é uma série de considerações que se encontram na \textit{Genealogia da Moral}, e que não poderemos discutir aqui.}, 
exigências que pressionam o filósofo a não projetar sua moral de modo demasiado estreito, provinciano, desonesto ou temerário
\footnote{Causa-nos muita estranheza a ideia avançada por Clark e Dudrick segundo a qual o tema dos “pré-conceitos” dos filósofos, ao qual Nietzsche dedica todo o primeiro capítulo de \textit{ABM}, não chega de fato a ser um alvo da crítica nietzschiana, porque, segundo os autores, a “vontade de valoração” necessária à atividade filosófica faz com que os preconceitos sejam um ingrediente necessário de toda filosofia (Cf. CLARK, DUDRICK, 2012, p. 48). Conquanto nos pareça que Nietzsche considera que toda filosofia fatalmente e devidamente tem um ingrediente de afirmação pessoal de valor, para nós não faz o menor sentido a ideia de que afirmações pré-conceituosas, isto é, precipitadas, provincianas, estreitas, idiossincráticas, sejam vistas por Nietzsche com bons olhos. A afirmação contrária consiste em neutralizar o traço mais fundamental da crítica de Nietzsche ao modo tradicional de se fazer filosofia, sua mais severa advertência contra a desonestidade de motivação moralista dos filósofos. Helmut Heit nota que: “A ‘Vorurteil’ (prejudice) is neither necessarily false, as Lampert suggests, nor is it a valuation, as Clark and Dudrick want to convince us (42). My native speaker’s intuition and Grimm’s lexicon of 19$^{th}$ century German is clear about it: a Vor-Urteil is a judgment made \textit{before} any serious investigation and questioning of its conditions and justifications. Being uncritical about unproven judgments is embarrassing for honest philosophers and I therefore disagree that the term ‘prejudice’ 'refers not to something problematic that philosophers must overcome but to the values (or prejudgments) that are essential to all philosophy' (42).” (HEIT, 2012, p. 9). A nosso ver, a leitura de Clark e Dudrick é prejudicada aqui mais uma vez por desconsiderarem as afirmações de Nietzsche no sentido do valor da ciência enquanto meio de cultivo de uma consciência metódica, e reservarem seus esforços para a crítica à ideia de que Nietzsche rejeita a aplicação dos métodos (que entendem, ao que parece, no sentido de “procedimentos quantitativos”) das ciências naturais à filosofia.}. 
Seguindo essa linha, a “vontade de verdade” não exerceria oposição a uma suposta “vontade de valoração”, mas antes atuaria como um inibidor da tendência natural a assentir \textit{involuntariamente} e \textit{inadvertidamente} a qualquer crença que satisfaça qualquer impulso, tendência que segundo Nietzsche seria marca de uma natureza muito mais ajustada ao erro do que à verdade. Faz sentido pensar que esse fator inibidor, a vontade de verdade virtuosamente cultivada, colabora, por fim, pra manutenção de uma tensão \textit{erótica} na atividade filosófica: ela evita que o filósofo sucumba à tentação de recorrer a meios desonestos de afirmação de seus próprios valores, como o recurso a um suposto conhecimento a priori de realidades metafísicas, ela evita que o filósofo faça “violência” contra a verdade-mulher. Por outro lado, um espírito puramente objetivo, sem qualquer colorido pessoal, seria um fraco candidato ao erotismo filosófico – na verdade, ele não seria, em absoluto, um \textit{filósofo}. 

Ao que nos parece, essa leitura oferece algumas vantagens. Primeiro porque ela parece captar a razão de Nietzsche destinar uma função instrumental à ciência no programa da filosofia do futuro: a função de prover informações obtidas por métodos rigorosos e oferecer um espaço propedêutico para aquisição de virtudes epistêmicas, que contam como requisito dos mais importantes mas não como único requisito à atividade filosófica. Essa hierarquização entre filosofia e ciência é o tema que perpassa todo o capítulo 6 (parágrafos 204-213), que tem o título algo irônico de “Nós, os eruditos”. Nesse capítulo, Nietzsche expressa sua gratidão pelo homem científico justamente por seu “espírito objetivo”, pois “quem já não teria se cansado até à morte de tudo que é subjetivo e de sua maldita 'mesmicidade'!”, mas logo na sequência adverte que “afinal deve-se ter cautela também com a própria gratidão, e refrear o exagero com que ultimamente a renúncia e despersonalização do espírito é celebrada, como quase um fim em si (…).” (\textit{JGB/ABM}, §207). Esse “exagero” seria a razão do receio de Nietzsche sobre a possibilidade de que uma “funesta inversão hierárquica” (\textit{JGB/ABM}, §204) venha à tona, uma inversão que reduza a filosofia “a 'teoria do conhecimento', na realidade apenas um tímido epoquismo
\footnote{“Epoquismo” é cunhado por Nietzsche a partir do termo grego $\varepsilon\pi{o}\chi\acute{\eta}$, que se refere à “suspensão do juízo”. Ver nota 108 da tradução de PCS.} 
e doutrina de abstenção: uma filosofia que nunca transpõe o limiar e que \textit{recusa} penosamente o direito de entrar – é uma filosofia nas últimas, um final, uma agonia, algo que faz pena. Como poderia uma tal filosofia – \textit{dominar}?” (JGB/ABM, §204) – pergunta que parece se traduzir em “como poderia uma tal filosofia afinal \textit{filosofar}?”. Seguindo essa linha compreende-se melhor também o receio de Nietzsche em relação à ilustração democrática, isto é, o receio de que ela tenha um efeito totalmente aplainador sobre as subjetividades, inviabilizando o surgimento de exceções, de grandes personalidades, de \textit{filósofos}.

Entendendo a questão em termos da tensão entre o espírito objetivo da ciência e o modo tradicional de filosofar por projeção tirânica da personalidade, podemos oferecer também uma boa resposta de por quê a tarefa de “avaliação dos valores” deve ser assumida pelo filósofo, e não de modo meramente científico. Podemos responder essa questão diferentemente de Clark e Dudrick, sem recorrer à dicotomia entre “espaço das causas” e “espaço das razões”. Nossa proposta insiste em que os valores não chegam a constituir uma esfera especial a que só se tem acesso por vias normativas; de fato, os valores podem ser estudados por recurso a sua realidade histórica, biológica, política, geográfica, psicológica, etc. Mas só se pode de fato questionar o valor de um valor através de uma afirmação de um outro valor, e afirmação de valor é justamente a atividade filosófica de projeção da própria personalidade. A avaliação dos valores requer um engajamento \textit{pessoal}. 

	Tomemos um exemplo mencionado por Clark e Dudrick referente ao aforismo 373 de \textit{A Gaia Ciência}, que data da mesma época de composição de \textit{ABM}, e leva o título de \textit{“Ciência” como preconceito}. O primeiro alvo desse aforismo é o evolucionista Herbert Spencer, e sua tentativa de enxergar a história evolutiva como um trajeto de “conciliação final de 'egoísmo e altruísmo'”, o que para Nietzsche seria uma projeção inadvertida de um valor idiossincrático, além de quê, esse valor afeta o próprio Nietzsche como “repugnante”. Este ponto foi notado por Clark e Dudrick, que o comentam, a nosso ver, de forma acertada: “Ele [Spencer] não vê que os valores que constituem seu ponto de vista pode atingir outras pessoas como repugnantes porque ele não reconhece em absoluto que seu ponto de vista é constituído por valores (...)” (CLARK, DUDRICK, 2012, p. 120). No entanto, acreditamos que esta falha não precisa ser interpretada segundo os termos dos autores, como uma falha de Spencer em perceber o conflito entre “vontade de verdade” e “vontade de valoração”, ou um conflito entre “ver o mundo como ele é” e “ver o mundo como se gostaria que ele fosse”. Antes, nos parece ser o caso de um assentimento precipitado e insciente aos valores que, inescapavelmente, constituem toda perspectiva, inclusive a perspectiva do pesquisador; é esse sentido de precipitação que parece ser o núcleo da crítica nietzschiana: “a necessidade que deles faz pesquisadores, sua íntima antecipação e desejo de que as coisas sejam assim e assim, seus temores e esperanças, muito cedo já encontram paz e satisfação.” (FW/GC, §373).
	
	Nietzsche a seguir afirma que a mesma crítica se aplica aos “cientistas naturais materialistas”, na medida em que só admitem como justificável uma interpretação científica de mundo, “uma tal que admite contar, calcular, pesar, ver, pegar e não mais que isso”. Uma possibilidade que Clark e Dudrick parecem não considerar é que essa crítica não aponta a limitação da perspectiva naturalista em si – surpreendentemente, eles não chegam a abordar o fato de que a “ciência” criticada aparece sempre entre aspas. Entendemos que o que é apontado aqui é a limitação do mecanicismo dentre outras formas possíveis de abordagem científica do mundo: “Que a única interpretação justificável do mundo seja aquela em que \textit{vocês} são justificados, na qual se pode pesquisar e continuar trabalhando cientificamente no \textit{seu} sentido (– querem dizer, realmente, de modo \textit{mecanicista}?)” (\textit{FW/GC}, §373). Por que trabalhar cientificamente de modo mecanicista? A alternativa óbvia seria trabalhar com a hipótese de uma ontologia não materialista como a de Boscovich, uma ontologia \textit{relacional} que pensa o mundo em termos de relações entre pontos de força, que aborda o mundo físico mas não em termos do que se pode “pesar, ver, pegar”. 

Essa alternativa não é levada em consideração por Clark e Dudrick na interpretação desse aforismo porque eles insistem em que, conquanto o mecanicismo seja plenamente adequado como método investigativo do “espaço das causas”, ele não teria lugar no “espaço das razões”. Eles insistem em que apenas mediante essa distinção pode-se compreender a objeção seguinte de Nietzsche: “Mas um mundo essencialmente mecânico seria um mundo essencialmente \textit{desprovido de sentido}! Suponha-se que o \textit{valor} de uma música fosse apreciado de acordo com o quanto dela se pudesse contar, calcular, pôr em fórmulas – como seria absurda uma tal avaliação “científica” da música!”. Pode-se entender, no entanto, que uma análise quantitativa da música sequer chega a exaurir o campo de interpretações “científicas” possíveis, pois uma ciência da música poderia compreender, além da quantificação de compassos ou ondas sonoras, uma história dos estilos, dos compositores, dos temas, da semiologia, do impacto psicológico que produz, etc.; nesse caso se teria uma análise científica mas não mecanicista da música, uma análise que toma por objeto uma série de aspectos que segundo Clark e Dudrick constituem o “espaço das razões”. Ainda assim, a mais rigorosa e abrangente análise objetiva da música não bastaria para apreciar o \textit{valor} de uma música, não porque com ela não se chega a tocar o espaço das razões, mas porque o juízo de valor depende de uma \textit{relação pessoal} do ouvinte com o objeto musical. Um juízo sobre o valor da música admite que o ouvinte empregue todo o conhecimento que possui sobre as artes, mas acima de tudo, requer que ele engaje suas próprias vivências, emoções, gosto e tudo o há de mais “íntimo” em sua natureza. E, uma vez que se trabalhe sobre a hipótese de uma ontologia relacional em termos de forças, não haveria um abismo tão grande entre a análise científica da música e a experiência de avaliação – é sempre o caso de uma relação entre força e força, ou entre os diferentes impulsos do ouvinte, ou uma relação entre a “hierarquia” de valores de um ouvinte e a peça de música que aprecia. A diferença é que, conquanto a análise científica tenha um maior teor objetivo, a atividade avaliativa é fundamentalmente pessoal.

	A ciência pode contribuir muito para o “alargamento dos espaços da alma” do filósofo: ela pode ensinar a coligir perspectivas; ela favorece a exposição a situações que desafiam qualquer ranço de provincianismo e pré-conceito; ela oferece informações valiosas sobre a origem dos impulsos, seu papel na definição de valores e morais; ela cultiva uma espécie de “gosto epistêmico”, que não se satisfaz com a “mesmicidade” do puramente subjetivo. E por isso mesmo, a única coisa que a ciência não pode é ser \textit{pessoal} – ela deve ao menos trabalhar com o princípio regulativo de envolver o mínimo possível de fatores pessoais. 
	
	A “filosofia do futuro” seria criada, então, pela tensão entre, por um lado, as exigências de objetividade da vontade de verdade disciplinada, exigências que implicam em algum grau de “crueldade” contra o que o pensador tem de idiossincrático, arbitrário, provinciano – e, por outro lado, a necessidade vital de que o filósofo expresse afirmativamente seus valores, e que em virtude dessa necessidade tem que acessar aquilo que sua natureza tem de mais básico e até mesmo mais terrível
\footnote{Note-se que está em questão aqui uma espécie de teodicéia da civilização: foi a introjeção dos impulsos mais terríveis do homem que, ao dobrar-se sobre si mesmos e se negarem satisfação, o que “ampliou” os espaços da alma, criou uma disciplina do pensamento e um senso de objetividade; por outro lado, a aposta de Nietzsche é que a superação dos efeitos funestos desse processo civilizatório, que fez do homem o “animal doente” mas também o “animal  interessante”, depende de que esses impulsos terríveis sejam de certa forma redimidos, e encontrem uma remodelação plástica que os habilite a cumprir um papel na reinvenção da cultura. Este é o tema de \textit{Genealogia da Moral}, que foi escrito como complemento e ilustração de \textit{Além de Bem e Mal}, mas que não poderemos endereçar de forma mais completa aqui. De todo modo, o tema está claramente indicado em \textit{ABM}, por exemplo na passagem em que o filósofo sugere que os afetos “de ódio, inveja, cupidez, ânsia de domínio”, enquanto parte da “economia global da vida” devem ser realçados “se a vida é para ser realçada” (JGB/ABM, §23).}. 

Por fim, acreditamos que esse modelo se aplica com mais sucesso às tantas passagens de \textit{ABM} que se referem ao tema do “erro” como “condição da vida”, ou que questionam o valor da verdade para a vida. Clark e Dudrick se esforçam para empurrar a ideia de que tais ocorrências são meras armadilhas para leitores desatentos compromissados com uma leitura naturalista. A nosso ver, o lugar de destaque dessas passagens torna muito improvável que esta seja a única função que elas desempenham; de fato, o tema do erro, parece cumprir um papel fundamental na manutenção da “tensão do espírito” da nova filosofia. O principal argumento dos autores para rejeitar a ideia de que Nietzsche estaria comprometido com a tese do erro é de que o filósofo, na verdade, acata o princípio protagoriano do “homem medida de todas as coisas”, de modo que não haveria qualquer  referência exterior e não-humana de correção das crenças, não haveria por exemplo a exigência de adequação a uma “coisa-em-si”, e portanto não haveria por quê suspeitar sistematicamente que os fatores que condicionam o pensamento produzem uma imagem falsa (CLARK, DUDRICK, 2012, p. 55). Vejamos o trecho apontado pelos autores:

\begin{quotation}
(…) em sua maior parte, o pensamento consciente de um filósofo é secretamente guiado e colocado em certas trilhas pelos seus instintos. Por trás de toda lógica e de sua aparente soberania de movimentos existem valorações, ou, falando mais claramente, exigências fisiológicas para a preservação de uma determinada espécie de vida. Por exemplo, que o determinado tenha mais valor que o indeterminado, a aparência menos valor que a “verdade”: tais avaliações poderiam, não obstante a sua importância reguladora \textit{para nós}, ser apenas avaliações-de-fachada, um determinado tipo de \textit{naiserie} [tolice], tal como pode ser necessário justamente para a preservação de seres como nós. Supondo, claro, que não seja precisamente o homem a “medida de todas as coisas”... (\textit{JGB/ABM}, §3).
\end{quotation}

Clark e Dudrick argumentam que por “exigências fisiológicas”, neste contexto, deve-se entender “exigências psicofisiológicas” e que no trecho “preservação de uma determinada espécie de vida” deve-se entender “vida” não no sentido biológico mas no sentido normativo (CLARK, DUDRICK, 2012, p. 59). Até aí, estamos de acordo com os autores, mesmo que entendamos que no vocabulário nietzschiano não há um abismo entre vida no sentido biológico e vida no sentido normativo: pode-se entender  que “certo modo de vida” se caracteriza por valorações, e estas remetem à atividade impulsiva culturalmente modelada. No entanto, os autores argumentam que a intenção de Nietzsche consiste em levantar e enfim dirimir a suspeita sobre a incapacidade dessas exigências lógicas e valorativas de “conduzirem à verdade”; segundo os autores, tais exigências de fato conduzem à verdade porque não há um padrão exterior que possa falsificá-las, se aceita-se que “o homem é a medida de todas as coisas”. Helmut Heit (2012, p.6) apontou que se pode entender que o homem \textit{não} é a medida de todas as coisas em muitos sentidos, não só por referência a uma coisa-em-si. Uma possibilidade seria a comparação com os padrões psicofisiológicos de outros animais, que é o clássico argumento cético. Especialmente se entende-se a ocorrência de “vida” nessa passagem não no sentido biológico mas no sentido “normativo”, isto é, abrangendo fatos relativos a valores e normas, tanto mais razões se tem para pensar que as condições de vida do homem atual não é uma medida fixa – basta lembrar a definição de homem como “animal não fixado” (expressão que aliás ocorre em \textit{ABM}, §62). Se fosse o caso de os “instintos” que guiam secretamente o pensamento filosófico serem condições normativas seguramente fixas, não haveria motivo para Nietzsche levantar a suspeita falibilista contra a atividade filosófica, isto é, aludir à possibilidade de que ela esconda uma \textit{naiserie}, que parece ser o conteúdo crítico desse aforismo. Acreditamos que a interpretação de Clark e Dudrick compromete sua abordagem do aforismo seguinte:

\begin{quotation}
A falsidade de um juízo não chega a constituir, para nós, uma objeção contra ele; é talvez nesse ponto que a nossa nova linguagem soa mais estranha. A questão é em que medida ele promove ou conserva a vida, conserva ou até mesmo cultiva a espécie; e a nossa inclinação básica é afirmar que os juízos mais falsos (entre os quais os juízos sintéticos \textit{a priori}) nos são os mais indispensáveis, que, sem permitir a vigência das ficções lógicas, sem medir a realidade com o mundo puramente inventado do absoluto, do igual a si mesmo, o homem não poderia viver – que renunciar aos juízos falsos equivale a renunciar à vida, negar a vida. Reconhecer a inverdade como condição de vida: isto significa, sem dúvida, enfrentar de maneira perigosa os habituais sentimentos de valor; e uma filosofia que se atreve a fazê-lo se coloca, apenas por isso, além do bem e do mal. (\textit{JGB/ABM}, §4).
\end{quotation}

Esse aforismo, localizado imediatamente na sequência da afirmação sugestiva sobre o “homem medida”, parece indicar que justamente porque o homem não é a medida de todas coisas, que os juízos necessários à sua preservação podem ser falsos, e ainda assim não se poderia fazer objeção a eles, dado seu papel vital. Clark e Dudrick, no entanto, vêem como absolutamente contraditória a ideia de que se possa tomar um juízo por falso e ainda assim não lhe fazer objeção, eles alegam que essa possibilidade faria cair por terra todas as nossas práticas comunicativas. Argumentando conforme sua linha de interpretação esotérica, Clark e Dudrick buscam um outro sentido para compreender a “falsidade” desses juízos. Segundo eles, essa suposta “falsidade” se refere apenas à impossibilidade de correspondência com coisa-em-si, o que no entanto não tornaria tais juízos propriamente falsos, uma vez que a medida humana é que é a fiadora de sua correção. A se fiar nessa interpretação, portanto, Nietzsche estaria meramente reeditando um aspecto básico da filosofia kantiana, referente à impossibilidade de correspondência entre os juízos e a coisa-em-si; esse é um dos argumentos dos autores a favor de sua tese geral de que \textit{ABM}, como um todo, pode ser lido como um testamento de estrita filiação de Nietzsche a Kant. Clark e Dudrick chegam a reconhecer que, para Nietzsche, as condições de possibilidade dos juízos não são transcendentais no sentido kantiano porque elas estão sujeitas a revisão face a novos dados empiricamente descobertos (CLARK, DUDRICK, p. 85), mas parecem não se dar conta de que justamente aí se encontra o traço falibilista radical do pensamento nietzschiano. Ademais, se é esse todo o conteúdo da consideração de Nietzsche sobre a “falsidade” dos juízos, como querem os autores, teríamos razão para achar todo o gestuário envolvido na sua apresentação algo exagerado – é realmente esse o motivo de o filósofo falar uma “nova” e “estranha” linguagem? É isso o que coloca sua filosofia além de bem e mal? Isso certamente não levou a filosofia de Kant para além de bem e mal.

	Ao que nos parece, Nietzsche aqui se esforça por apontar uma situação realmente conflituosa, em que artigos sobre os quais temos boas razões para suspeitar que sejam “falsos”, que sejam “inverdade”, desempenham um papel estrutural na organização da vida. Não nos parece que uma consideração sobre a impossibilidade de correspondência metafísica seja o motivo dessa suspeita, lembrando que no capítulo \ref{cap1} argumentamos sobre as consequências de, já em \textit{Humano}, Nietzsche tomar o conceito de coisa-em-si por inócuo. As suspeitas sobre sua falsidade se devem à percepção das inconsistências internas a essas formas, se devem ao fato de que sua aplicação leva a paradoxos, etc. Hume mostrou, por exemplo, que a forma como empregamos a noção de causalidade não pode ser justificada por nenhum conhecimento empírico e, no entanto, ela é suficiente para o sucesso de nossas práticas cotidianas de atribuição causal. Acima de tudo, a necessidade de “medir a realidade com o mundo puramente inventado do absoluto, do igual a si mesmo”, por mais que sejamos constantemente contraditos pela mudança, parece ser um atestado de uma tendência natural ao erro – uma ideia que reaparece ao longo de todo o pensamento nietzschiano.
	
	O que há de interessante na leitura de Clark e Dudrick, para nós, é alertar sobre a forma elusiva da escrita nietzschiana. Conquanto tenhamos argumentado que há boas razões, certamente reconhecidas por Nietzsche, para não se tomar “o homem como medida de todas as coisas”, a forma como esse tropos é lançado claramente aponta para um tensionamento. Por tudo o que foi dito, na verdade, pode-se entender que o homem é e não é a medida de todas as coisas. Quer dizer, por mais que os conhecimentos produzidos por nossos mais altos padrões epistêmicos ofereçam boas razões para suspeitar da humana medida de todas as coisas (acima de tudo, suspeitar que essa seja uma medida fixa e segura!), por outro lado é simplesmente uma exigência prática que, pela maior parte da vida, na maioria dos assuntos ordinários, nós nos comportemos \textit{como se} o homem de fato fosse a medida. Com isto, renuncia-se à expectativa de que nossos padrões cognitivos imediatamente “conduzam à verdade”, o que no entanto não parece afetar nossas práticas de honestidade e veracidade: “Honestidade é uma condição pragmática para o sucesso da comunicação, mas a convicção de que a verdade é alcançável não. A suposição do falibilismo e da impossibilidade de provar a verdade não solapa nossas práticas razoáveis e úteis.” (HEIT, 2012, p. 11).

Nesse ponto, podemos endereçar novamente as questões que levantamos anteriormente sobre a possibilidade de correção dos valores: se a aceitação da “inverdade” como condição da vida coloca a filosofia nietzschiana “além do bem e do mal”, é porque essa aceitação certamente tem um impacto sobre o valor dos valores, e a possibilidade que imediatamente se apresenta é a de que um valor vale justamente na medida em que  “promove ou conserva a vida, conserva ou até mesmo cultiva a espécie”. Nesse sentido, é possível que haja casos em que eu simplesmente não tenha boas razões para fazer objeções a uma crença ética ou religiosa uma vez que ela conserve a vida, por mais que eu tenha boas razões para duvidar de sua procedência. Mas a filosofia nietzschiana não se torna assim um grande e inócuo relativismo? Se a vida está embebida em erro e isso é suficiente para a sua conservação, por que a verdade, afinal? Vale lembrar, primeiramente, que esta não é uma filosofia que se destina a uma posição acomodatícia; o modelo é sempre \textit{agonístico}, e seu horizonte é aquele da autossuperação. Perseguindo a ideia nietzschiana de que a própria vida é o lastro que confere valor aos valores, Georg Simmel chegou a uma formulação que se aproxima muito da altura do texto nietzschiano:

\begin{quotation}
Nietzsche recoloca na própria vida o fim que lhe confere sentido, um fim que havia sido tornado ilusório ao ser retirado dela. O modo mais radical de propor essa inversão foi afirmar que a própria elevação, a simples realização daquilo que a vida possui como possibilidades de intensificação, já contém todos os fins e valores vitais. Assim, cada estágio da existência humana deixou de encontrar seu fim em algo absoluto e definitivo; agora o encontra no estágio seguinte, no qual tudo o que no anterior estava iniciado se amplia e aumenta em eficácia e no qual, portanto, a vida se faz mais plena e mais rica, mais vida. O super-homem nietzschiano corresponde ao grau de superação conquistado pela humanidade em determinado momento; não é um fim último predeterminado que dá sentido à evolução; ao contrário, expressa que não é necessário nenhum fim desse tipo. A vida possui valor em si mesma, na superação de cada grau por outro mais pleno e mais perfeito. A vida – cujos conteúdos diversos são aspectos ou manifestações de um misterioso processo unitário – converteu-se na mais elevada instância. Como a vida é superação e fluxo constante, cada manifestação sua tem na seguinte a norma mais elevada que lhe confere sentido. É para chegar a ela que desenvolve suas forças. (SIMMEL, 2011, p. 17-18).
\end{quotation}

Perseguimos a sugestão de que também a verdade, e não só o erro, seja condição de elevação da vida, ao menos de um certo estágio ou tipo de vida. Ao que nos parece, há pelo menos duas razões para que não se entenda que Nietzsche renunciou completamente à possibilidade e ao valor da verdade para a vida. Primeiro, a sugestão de que “verdade” não se opõe a “inverdade”, e que é apenas um preconceito derivado de hábitos metafísicos de pensamento o que nos impede de considerar que a verdade pode surgir da inverdade (Cf. \textit{JGB/ABM}, §2). De acordo com a leitura naturalista que temos perseguido, isto equivaleria a dizer que conquanto os fatores que condicionam nossas práticas cognitivas nos tornem muito mais propensos ao erro que à verdade, é possível refiná-los e, no interior de suas tensões, atingir elevados \textit{graus} de verdade. Vale lembrar que conquanto o valor de um valor é fiado pelo grau em que ele é condição da vida, o mesmo não se pode dizer da verdade, quer dizer, algo não é necessariamente mais verdadeiro porque favorece a vida. Se Nietzsche afirma que a “falsidade” de um juízo não constitui uma objeção contra ele, é para nos lembrar mais adiante que tampouco se deve tomar “por verdadeira uma doutrina apenas porque ela torna feliz ou virtuoso” (JGB/ABM, §39). Mesmo assim, e esta é a segunda razão da ideia de que a verdade exerce alguma tensão sobre o valor dos valores, Nietzsche deixa muito clara sua valoração pessoal da verdade; conquanto os erros sejam em geral as condições mais comuns e mais básicas da vida, é possível que a verdade represente um valor \textit{maior}, também conservador e promotor da vida, ao menos de um certo tipo de vida
\footnote{Lembrando o aforismo de \textit{A Gaia Ciência} sobre a possibilidade de incorporação da verdade: “O pensador: eis agora o ser no qual o impulso para a verdade e os erros conservadores da vida travam sua primeira luta, depois que também o impulso à verdade \textit{provou} ser um poder conservador da vida. Ante a importância dessa luta, todo o resto é indiferente: a derradeira questão sobre as condições da vida é colocada, faz-se a primeira tentativa de responder a essa questão com o experimento. Até que ponto a verdade suporta ser incorporada? – eis a questão, eis o experimento.” (FW/GC, §110).  }. 
 De fato, a capacidade de suportar a verdade é apresentada em \textit{ABM} como uma medida de força de espírito, e inversamente, a necessidade de “vê-la diluída, edulcorada, encoberta, amortecida, falseada” (\textit{JGB/ABM}, §39).

	Se esse é o caso, se a filosofia nietzschiana expressa um interesse congênito por pesquisa da verdade e valoração pessoal, vejamos como Nietzsche se sai ao avançar a tese cardeal de \textit{Além de Bem e Mal}, a “vontade de poder”.

\section{A grande tentativa, a grande tentação -- vontade de poder}

\begin{quotation}
Supondo que nada seja “dado” como real, exceto nosso mundo de desejos e paixões, e que não possamos descer ou subir a nenhuma outra “realidade”, exceto à realidade de nossos impulsos – pois pensar é apenas a relação desses impulsos entre si –: não é lícito fazer a tentativa e colocar a questão de se isso que é dado não bastaria para compreender, a partir do que lhe é igual, também o chamado mundo mecânico (ou “material”)? Quero dizer, não como uma ilusão, uma “aparência”, uma “representação” (no sentido de Berkeley e Schopenhauer), mas como da mesma ordem de realidade que têm nossos afetos, na qual ainda esteja encerrado em poderosa unidade tudo o que então se ramifica e se configura no processo orgânico (e também se atenua e se debilita, como é razoável), como uma espécie de vida instintiva, em que todas as funções orgânicas, como auto-regulação, assimilação, nutrição, eliminação, metabolismo, se acham sinteticamente ligadas umas às outras – como uma \textit{forma prévia} da vida? – Afinal, não é apenas lícito fazer essa tentativa: é algo imposto pela consciência do \textit{método}. Não admitir várias espécies de causalidade enquanto não se leva ao limite extremo (– até ao absurdo, diria mesmo) a tentativa de se contentar com uma só: eis uma moral do método, à qual ninguém se pode subtrair hoje; – ela se dá “por definição”, como diria um matemático. A questão é, afinal, se reconhecemos a vontade realmente como \textit{atuante}, se acreditamos na causalidade da vontade: assim ocorrendo – e no fundo a crença nisso é justamente a nossa crença na causalidade mesma –, \textit{temos} então que fazer a tentativa de hipoteticamente ver a causalidade da vontade como a única. “Vontade”, é claro, só pode atuar sobre “vontade” – e não sobre “matéria” (sobre “nervos”, por exemplo –): em suma, é preciso arriscar a hipótese de que em toda parte onde se reconhecem “efeitos”, vontade atua sobre vontade – e de que todo acontecer mecânico, na medida em que nele age uma força, é justamente força de vontade, efeito da vontade. –  Supondo, finalmente, que se conseguisse explicar toda a nossa vida instintiva como a elaboração e ramificação de uma forma básica da vontade – a vontade de poder, como é \textit{minha tese} –; supondo que se pudesse reconduzir todas as funções orgânicas a essa vontade de poder, e nela se encontrasse também a solução para o problema da geração e nutrição – é um só problema –, então se obteria o direito de definir \textit{toda} força atuante, inequivocamente, como \textit{vontade de poder}. O mundo visto de dentro, o mundo definido e designado conforme o seu “caráter inteligível” – seria justamente “vontade de poder”, e nada mais. – (\textit{JGB/ABM}, §36).
\end{quotation}

Com este experimento de pensamento, Nietzsche avança a tese mais polêmica de \textit{Além de Bem e Mal}: a projeção analógica da vontade de poder aos fenômenos mecânicos e vitais. Essa tese causa verdadeira aversão em intérpretes como Clark e Dudrick porque eles entendem, não sem razão, que ela só contribui para a ridicularização de Nietzsche em certos círculos filosóficos contemporâneos (Cf. CLARK, DUDRICK, 2012, p. 5). Em razão disto, esses autores passam a buscar “segundas intenções” no texto nietzschiano, notadamente intenções que o tornem mais palatável à filosofia contemporânea de língua inglesa. No entanto, fato é que o cenário filosófico e científico do século XIX obviamente é marcado por outro “estado da arte” e por  exigências outras que o cenário contemporâneo, de modo que o reajuste anacrônico dos procedimentos nietzschianos não parece lá a leitura mais honesta.

	Nietzsche escreve em meio ao que Thomas Kuhn chamou de “crise de paradigma” da física, uma crise advinda das dificuldades envolvidas em se abordar fenômenos como a gravidade por recurso ao modelo mecanicista, para desgosto de Newton
\footnote{A insatisfação com o modelo mecanicista, que descreve o movimento em termos de impacto ou de atração e repulsão (no caso da gravidade), levou a uma crise de legitimidade do materialismo, e portanto ao questionamento do conceito de matéria ele mesmo. Essa crise só foi superada no começo do século XX, quando os experimentos de Jean Baptiste Perrin demonstraram a realidade do atomismo de partículas, e afastaram a ideia de que o átomo seria apenas uma ficção regulativa, ideia com a qual Nietzsche trabalhou. A nova teoria atômica que a partir daí se desenvolveu levou em consideração uma série de críticas anti-atomistas do século XIX, algumas das quais compartilhadas por Nietzsche: “O novo programa atomista superou o energeticismo como importante alternativa de pesquisa e incorporou não somente a teoria atômica clássica, mas também, muitas idéias e propostas de anti-atomistas do século XIX. O átomo não poderia ficar restrito aos fundamentos científicos e epistemológicos da mecânica clássica. Nesta situação não era mais possível aceitá-lo apenas como uma partícula material maciça e indivisível. Comentando sobre o átomo da Física Moderna, Bachelard lembra a importância de se evocar a história das suas imagens, segundo ele '[...]\textit{o átomo é exatamente a soma das críticas a que se submete a sua imagem primeira}'” (OKI, 2009, p. 1081). Daí em diante, toda filosofia que se pretendeu naturalista teve que responder às exigências do fisicalismo, o que, no entanto, não era o caso na época de Nietzsche.}. Essa crise abriu espaço para todo tipo de especulação no campo da física, e uma das alternativas que então surgiram foi o modelo de Boscovich, já mencionado, que opera em termos de \textit{pontos não extensos de força}, lembrando que esta é uma alternativa interessante para Nietzsche porque

\begin{quotation}
as teorias dinâmicas [como a de Boscovich] parecem superar em alguma medida os aspectos sensuais e antropomórficos que ele identifica, por exemplo, em conceitos da visão mecanicista, e ele julga que as teorias dinâmicas seriam, portanto, menos antropomorfizadas que as demais; em segundo lugar, pelo fato de que tais teorias abdicam da ideia de átomo como sustentáculo da força e da representação do mundo como uma espécie de agregado de átomos, substituindo-a pela imagem do mundo como um sistema relacional e dinâmico de ação e reação de múltiplas forças; em terceiro lugar, porque ela se encaixa melhor com a concepção esposada por Nietzsche de fluxo ou devir. (CARVALHO, 2013, p. 47.)
\end{quotation}

Contudo, em algumas passagens do livro V de \textit{A Gaia Ciência} Nietzsche expressa um certo incômodo com a visão de mundo suscitada pelo mecanicismo, pois ela é a visão de um mundo \textit{sem sentido} (Cf. FW/GC, §373), e embora o modelo dinâmico de Boscovich seja visto como mais interessante em muitos aspectos, também ele oferece a visão de um mundo sem sentido e, além de sem sentido, o mais desumanizado possível. O incômodo revela um receio quanto aos efeitos dessa visão de mundo sem sentido e desumanizado sobre os valores – a possibilidade de que essa visão leve a uma sistemática desvalorização dos valores, o niilismo. A ideia do mundo como vontade de poder, por outro lado, é altamente humanizada; ela introduz o vocabulário da intencionalidade, do comando e da obediência, ela parece atribuir não só inteligibilidade mas até mesmo uma espécie de atividade interpretativa aos entes que compõem o mundo orgânico e inorgânico. E ainda hoje os intérpretes vêem como intrigante o fato de Nietzsche ter avançado uma visão de mundo altamente antropomórfica pouco tempo depois de fazer a crítica radical dos antropomorfismos estéticos e morais que projetamos sobre o mundo (no aforismo 109 de \textit{A Gaia Ciência}, que discutimos no capítulo anterior, entre outros), mesmo que essa visão humanizada seja proposta em termos hipotéticos
\footnote{Lopes alerta que “vontade de poder” não deve ser compreendida exatamente como uma hipótese, termo que se refere a uma tese que pode ser empiricamente comprovada ou refutada. Ela seria antes uma “ficção regulativa”, proposta com “a clara consciência de sua falsidade”, de forma que o experimento de pensamento desenvolvido em \textit{ABM} §36 seria uma “tentativa de substituir ficções regulativas menos econômicas e inconscientemente assumidas como verdades de tipo transcendental por ficções regulativas mais econômicas e conscientemente postuladas.” (LOPES, 2008, p. 421)}. 
As primeiras impressões são de que ou esse fato aponta para uma incoerência grotesca, ou para uma mudança de paradigma filosófico, ou para algum tipo de enigma, sofisma, enfim.

	Aventamos a ideia de que tanto a crítica sistemática ao antropomorfísmo (\textit{GC}, §109) quanto a projeção do conceito antropomórfico de vontade de poder são experimentos que visam levar o pensamento ao limite, “ao absurdo”, talvez. Uma ontologia antropomórfica em termos de vontade de poder não é, claro, o desfecho necessário da meditação anti-antropomórfica levada a cabo em \textit{A Gaia Ciência}, mas essa meditação purificadora e prudencial é pré-condição do experimento de pensar a vontade de poder. Faz sentido?

\textit{GC} §109 é uma meditação, \textit{ABM} §36 argumentação \textit{ex-concessis}. Os dois experimentos têm direções inversas: \textit{GC} §109 tem início com uma reflexão cosmológica, e somente ao final dessa reflexão lança perguntas (não respondidas) sobre a natureza humana: “Quando teremos desdivinizado completamente a natureza? Quando poderemos começar a \textit{naturalizar} os seres humanos com uma pura natureza, de nova maneira descoberta e redimida?”; por outro lado, o aforismo 36 de \textit{Além de Bem e Mal} tem como ponto de partida um \textit{insight} razoavelmente desdivinizado sobre a natureza humana, ou melhor, sobre a natureza do pensamento, entendido como a relação dos impulsos entre si, e sua conclusão diz respeito à legitimidade de se avançar uma ficção regulativa de alcance cosmológico que define “\textit{toda} força atuante, inequivocamente, como \textit{vontade de poder}”. Não só a direção mas o movimento são invertidos: enquanto \textit{GC} §109 procede pela remoção sistemática das crenças mais comuns sobre o mundo, o procedimento de \textit{ABM} §36 consiste em partir do núcleo mínimo de uma crença indispensável
\footnote{Isto é, a crença na causalidade. Mas encontra-se aqui também a busca por um traço da vida interior humana desnudado de conotações morais e metafísicas.} 
e projetá-la analogicamente ao domínio mais amplo. \textit{GC} §109 vai do tudo ao mínimo, \textit{ABM} §36 vai do mínimo ao tudo. É possível, portanto, que não haja uma contradição entre esses dois movimentos, mas que o trabalho de remoção dos antropomorfismos (morais e estéticos inclusive) seja uma preparação de terreno necessária ao experimento que definirá um conteúdo mínimo (e moralmente deflacionado) a ser conscientemente reprojetado sobre essa visão de um universo indiferente, sem a pretensão de que o que é projetado tenha aderência imediata e definitiva sobre nossa forma de ver os fenômenos naturais.

	Em \textit{ABM} §36 Nietzsche toma como premissas algumas concessões de outros filósofos, marcadamente aquelas de Descartes, Leibniz e Schopenhauer sobre a acessibilidade da vida interior. Contudo, aqui elas aparecem reformuladas, pois Nietzsche já havia feito a crítica da noção de “certeza imediata” em \textit{ABM} §16, de forma que o que “nos é dado como real”, isto é, “nosso mundo de desejos e paixões” não deve ser pensado como autotransparente ao sujeito nem como resultante em uma unidade como o “eu penso” ou “eu quero”, mas de toda forma como uma experiência em primeira pessoa que pode ser analisada \textit{a posteriori}. A crítica de Nietzsche consiste em trazer à baila os pressupostos inscientes embutidos nas noções avançadas por esses filósofos: “aquele 'eu penso' pressupõe que eu \textit{compare} meu estado momentâneo com outros estados que em mim conheço, para determinar o que ele é: devido a essa referência retrospectiva a um 'saber' de outra parte, ele não tem para mim, de todo modo, nenhuma 'certeza' imediata” (\textit{JGB/ABM}, §16). No caso de Nietzsche, essa análise a posteriori leva à tese de que a realidade mesma do pensar é uma complexa relação entre impulsos. Do mesmo modo, a noção schopenhaueriana de vontade é criticada em \textit{ABM} §19, que insiste em que a vontade é um fenômeno muito mais complexo, pois além do querer envolve sentir, pensar e um “afeto de comando”. 

Com essa visão corrigida da vontade, Nietzsche pode avançar o argumento, e o próximo passo é a exigência de “economia de princípios”, a exigência occamiana. Este é um princípio neutro, imposto pela “moral do método”:  “Não admitir várias espécies de causalidade enquanto não se leva ao limite extremo (– até ao absurdo, diria mesmo) a tentativa de se contentar com uma só”. Esta exigência está acima de qualquer suspeita. Em seguida, Nietzsche lança a questão fundamental desse experimento: “A questão é, afinal, se reconhecemos a vontade realmente como \textit{atuante}, se acreditamos na causalidade da vontade”. A crença na causalidade da vontade, igualmente, já havia sido criticada em várias passagens anteriores. Nietzsche sem dúvidas rejeita a noção de \textit{causa sui} que está envolvida na crença tradicional na causalidade da vontade, ou seja, a crença de que “o querer \textit{basta} para agir”, a crença fundamental das morais do livre-arbírtrio. Mas negar que a vontade seja suficiente para engendrar a ação não implica em negar qualquer poder de atuação à vontade. A ideia é que a vontade será eficiente em algumas \textit{circunstâncias}; aliás, uma vez que a vontade é vista como um fenômeno complexo que envolve comando e obediência por parte dos impulsos, uma condição importante para a efetividade causal da vontade seria a existência de um pacto político bem sucedido entre os impulsos, garantindo que os impulsos comandados estejam dispostos a obedecer
\footnote{Este é um ponto para o qual Clark e Dudrick chamaram atenção: “(...) we invoke J. L. Mackie's gloss on causation: 'The statement 'A caused P' often claims that A was necessary and sufficient for \textit{P in the circumstances}' (Mackie 1965: 248, emphasis added). To say, for example, that the faulty brakes caused the accident is to say that the faulty brakes were necessary and sufficient for the accident \textit{given the circumstances} – given that the car was moving at a certain speed, for example. (…) The error of thinking that 'willing \textit{suffices} for action', then, is not that of thinking that willing causes action but of thinking that commanding an action is enough, all by itself, to bring the action about” (CLARK, DUDRICK, 2012, p. 193-194). Mais adiante, os autores apontam a relevância dessa correção nietzschiana da noção de causalidade da vontade: “Consider what happens if one thinks that willing does suffice for action. One thinks that one can affect the causal order of one's drives – hence what one does – simply by issuing commands and forgets that commanding can do nothing by itself, that commanding is effective only if one's drives exist in a 'well-constructed and happy commonwealth', hence that bringing about such a commonwealth of drives is what is necessary if one is to change one's life. Second, this provides an excuse for the mistaken belief in 'freedom of the will' in what Nietzsche calls the 'superlative metaphysical sense' (BGE 21). Ignoring that fact that willpower brings about action only if the commanded drives are willing to obey allows one to ignore all of the moral luck – the influence of 'the world, ancestors, chance and society' (BGE 21) – that goes into having one's drives exist as a 'well-constructed and happy commonwealth' (BGE 19) – and thus to believe that one has total causal responsibility for one's action.” (CLARK, DUDRICK, 2012, p. 194). } 

\begin{quotation}
(…) o querente acredita, com elevado grau de certeza, que vontade e ação sejam, de algum modo, a mesma coisa – ele atribui o êxito, a execução do querer, à vontade mesma, e com isso goza de um aumento da sensação de poder que todo êxito acarreta. “Livre-arbítrio” é a expressão para o multiforme estado de prazer do querente, que ordena e ao mesmo tempo se identifica com o executor da ordem – que, como tal, goza também do triunfo sobre as resistências, mas pensa consigo que foi sua vontade que as superou. Desse modo  o querente junta as sensações de prazer dos instrumentos executivos bem-sucedidos, as “subvontades” ou sub-almas – pois nosso corpo é apenas uma estrutura social de muitas almas – à sua sensação de prazer como aquele que ordena. \textit{L'effect c'est moi} [O efeito sou eu]: ocorre aqui o mesmo que em toda comunidade bem construída e feliz, a classe regente se identifica com os êxitos da comunidade. (…). (\textit{JGB/ABM}, §19).
\end{quotation}

Uma vez que o mecanismo psicológico que liga à causalidade da vontade a noção de \textit{causa sui} seja desvelado, corrige-se a crença na causalidade da vontade, que então pode ser reabilitada. Segundo essa visão reconstruída, a vontade é efetiva não enquanto ato isolado, mas enquanto se coloca numa relação bem sucedida com a comunidade de “subvontades” que constitui o corpo. É interessante reabilitar a crença na causalidade da vontade porque ela seria a essência mesma da crença na causalidade – e este é o argumento decisivo de \textit{ABM} §36. Ou mantemos a crença na causalidade da vontade, mesmo que reformada, ou renunciamos à crença em qualquer causalidade, o que é pragmaticamente inviável. Uma vez atingido o traço mínimo da crença na causalidade, o experimento de projetá-la a todos os fenômenos causais é não apenas “lícito”, é uma imposição da consciência metódica. Por fim, uma vez que vontade só pode atuar sobre vontade, e não sobre matéria, seria lícito conceber todo evento como efeito da atuação de vontade sobre vontade, e com isto Nietzsche avança uma versão invertida do princípio fisicalista de “fechamento causal do mundo”. É importante notar que o experimento não demonstra a realidade factual do mundo como vontade, mas sua necessidade psicológica e sua salubridade metodológica. Dessa forma, o experimento acaba por projetar uma face humana sobre o modelo dinâmico do mundo físico como pontos de força, também ele visto como apenas uma interpretação  provisória. Esse seria o mundo “visto de dentro”, o interior da perspectiva de quem olha.

	Quais são os ganhos desse experimento?
	
	O experimento oferece uma ficção regulativa destinada a contornar a insatisfação com os modelos de explicação causal pós-Hume. A filosofia humeana mostrou que nossas práticas de atribuição causal carecem de qualquer \textit{insight} sobre a natureza da relação que se estabelece entre causa e efeito, de forma que tudo o que podemos fazer é registrar padrões de regularidade, renunciando a compreender de fato o mecanismo causal que ali atua. \textit{ABM} §36 arrisca uma resposta sobre o tipo de força que atua em relações causais: elas seriam relações de comando e obediência entre vontades.

Outras passagens de \textit{ABM} parecem sugerir que Nietzsche estava interessado na noção de vontade de poder como recurso para se pensar o fator que confere \textit{organicidade} aos entes, fator que faz com que os seres vivos não sejam um mero agregado disforme de impulsos, mas apresentem-se como uma unidade. Em uma dessas ocorrências, “a própria vida” é definida como vontade de poder:

\begin{quotation}
Os fisiólogos deveriam refletir, antes de estabelecer o impulso de autoconservação como o impulso cardinal de um ser orgânico. Uma criatura viva quer antes de tudo \textit{dar vazão} a sua força – a própria vida é vontade de poder –: a autoconservação é apenas uma das indiretas, mais frequentes \textit{consequências} disso. – Em suma: nisso, como em tudo, cuidado com os princípios teleológicos \textit{supérfluos}! – um dos quais é o impulso de autoconservação (nós o devemos à inconsequência de Spinoza). Assim pede o método, que deve ser essencialmente economia de princípios. (\textit{JGB/ABM}, §13).
\end{quotation}

Clark e Dudrick alertam para a possibilidade de que o que esteja em jogo aqui seja, acima de tudo, a questão sobre o que faz com que algo seja um “ser orgânico”, lembrando que a organicidade de algo não se diz apenas de entes biológicos, mas também de organizações como instituições sociais, culturais, etc
\footnote{Cf. CLARK, DUDRICK, 2012, p. 217. No entanto, a interpretação dessa passagem proposta pelos autores acaba por retirar toda a ênfase do ser orgânico enquanto ente biológico, uma vez que as premissas gerais de sua leitura levam-nos a entender que o que confere unidade é um aspecto normativo da “alma”, ou do espírito. Com isto, eles parecem incorrer num erro contra o qual o próprio Nietzsche alertou em um fragmento póstumo: “Onde há uma certa unidade no agrupamento, sempre se postulou o \textit{espírito} como causa dessa coordenação: sem razão nenhuma. Por que deveria ser a ideia de um fato complexo uma das condições desse fato: Ou por que haveria de preceder a um fato concreto a sua representação? (…) Não há qualquer motivo para atribuir ao espírito a propriedade de organizar e sistematizar. (…) No conjunto do processo de adaptação e de sistematização, a consciência não desempenha nenhum papel.” (\textit{WM/VP}, §526 [1888]).}. 
A menção a Spinoza conta a favor dessa leitura, sugerindo que o que está em questão não é tanto a “luta pela sobrevivência”, mas a conservação dos organismos enquanto organismos, enquanto unidade organizada, tema da filosofia spinozana. A resposta arriscada nesse aforismo é de que “a própria vida é vontade de poder”; enquanto “impulso cardeal”, a vontade de poder seria um aspecto de todo impulso, de modo que a afirmação de que “Uma criatura viva quer antes de tudo dar vazão a sua força” seria apenas uma outra forma de dizer esse fato. É porque todo impulso quer \textit{dar vazão} a sua força, expressar sua vontade de poder, que os impulsos são capazes de estabelecer um “pacto político” e ganhar unidade enquanto corpo, no qual mesmo aqueles que não estão numa posição de comando terão oportunidade de se expressar. Na medida em que é a vontade de poder o aspecto pulsional que os leva a estabelecer esse pacto e organizar-se em uma unidade, a postulação de um “impulso de conservação” seria supérflua. A vontade de poder é o aspecto que lhes confere unidade na medida em que, sendo “da mesma ordem dos afetos”, nela estaria encerrado “em poderosa unidade tudo o que então se ramifica e se configura no processo orgânico (e também se atenua e se debilita, como é razoável), como uma espécie de vida instintiva, em que todas as funções orgânicas, como auto-regulação, assimilação, nutrição, eliminação, metabolismo, se acham sinteticamente ligadas umas às outras – como uma \textit{forma prévia} da vida” (\textit{JGB/ABM}, §36). Como o que está em questão é a \textit{organicidade} das estruturas e não apenas de estruturas biológicas, o modelo poderia servir para se pensar a vida e morte das culturas ou das etnias, por exemplo, traçando as relações de poder firmadas em seu interior, sua capacidade de intensificar a atuação das partes ou, por outro lado, desorganizar e fazer perecer.

	Ademais, os ganhos dos experimentos com vontade de poder são assim resumidos por Carvalho:

\begin{quotation}
Ao “definir toda força atuante, inequivocamente, como vontade de poder”, Nietzsche estaria concebendo-a como pontos de força dotados de um querer interno, na expectativa de que essa associação com a visão menos antropomorfizada da natureza enquanto pontos não extensos de força formulada por Boscovich pudesse contrabalançar o vocabulário altamente intencionalista e antropomórfico que a vontade de poder veicula. A pergunta pela razão que leva Nietzsche a antropomorfizar a visão científica dos centros de força reverte-se na compreensão de que a teoria dos centros de força mitiga o caráter sem dúvida antropomorfizante da vontade de poder. Além da vantagem de se reconhecer como interpretação e de produzir efeitos potencializadores, na medida em que não mais submete o homem a uma interpretação de mundo que leva à resignação contemplativa e à conformidade, mas que intensifica a sua capacidade criativa e interpretativa, a introdução do conceito procuraria, do ponto de vista teórico, dar conta da multiplicidade do real, do caráter processual do mundo do “devir”, superar a visão dualista natureza/cultura, além de ambicionar dar conta também de problemas mais pontuais, como o problema da ação à distância, reinterpretada a partir do modelo do mando e obediência. Tudo isso, sem dúvida, na forma de um ousado experimento de pensamento. (CARVALHO, 2013, p. 54).
\end{quotation}

Esse experimento portanto teria valor enquanto resposta provisória às exigências da consciência metódica, por um lado, e de nossas expectativas cognitivas, por outro (quer dizer, expectativas referentes à compreensão das relações causais e orgânicas). Ele satisfaria ainda expectativas valorativas, ao oferecer uma visão de mundo como abundância e dinamismo, uma visão estimulante mas não moralizante. Ele é, enfim, expressão de algo que o próprio Nietzsche toma por valioso, embora esteja longe de ser uma visão “edulcorada” da realidade
\footnote{Muito pelo contrário! Certamente há uma aposta de que o mundo visto como vontade de poder, portanto  à imagem de uma projeção humana, tenha maior apelo do que a simples visão de mundo como pontos de força, por exemplo; no entanto, a imagem de mundo como vontade de poder implica imediatamente no desafio de se levar em consideração alguns dos aspectos mais terríveis da existência: “Nietzsche’s earliest attitude towards the power drive was ironic and critical. And in \textit{Zarathustra}, where 'self-overcoming' is preached to the 'wisest,' they are told that their 'danger' lies not in the river of becoming, but in 'the will to power' itself. Before Freud, but after Hartmann’s \textit{Philosophy of the Unconscious} (1868), Nietzsche sought to look beneath or behind the masks of civility, to uncover the primitive drives and impulses of mankind—to expose 'the tiger' that is hidden within each of us and threatens to burst forth into overt or covert aggression for the sake of power. The hypothetical reduction of all drives or urges to a will to power as a psychological interpretation of natural mankind is not only not presented as a myth, but is clearly intended as an autopsy of the human psyche that discloses what is considered as a 'terrible truth,' one central to Nietzsche’s stark 'natural history of man,' his suspicious interpretation of \textit{homo natura}.” (STACK, 2005, p. 173.). Discutiremos a aplicação psicológica da doutrina da vontade de poder mais adiante.}. 
Mas, principalmente, com esse experimento avança uma apropriação crítica de sua herança filosófica. Os experimentos desenvolvidos com a noção de “vontade de poder” ao longo de \textit{ABM} retomam e retificam questões e noções muito caras à história da filosofia, às quais um seu herdeiro simplesmente não pode se furtar. Significativamente, após apresentar seu conceito de “vontade”, em \textit{ABM} §19, Nietzsche passa a uma consideração sobre a gênese dos conceitos filosóficos:

\begin{quotation}
Os conceitos filosóficos individuais não são algo fortuito e que se desenvolve por si, mas crescem em relação e em parentesco um com o outro; embora surjam de modo aparentemente repentino e arbitrário na história do pensamento, não deixam de pertencer a um sistema, assim como os membros da fauna de uma região terrestre – tudo isto se confirma também pelo fato de os mais diversos filósofos preencherem repetidamente um certo esquema básico de filosofias \textit{possíveis}. À mercê de um encanto invisível, tornam a descrever sempre a mesma órbita: embora se sintam independentes uns dos outros com sua vontade crítica ou sistemática, algo neles os conduz, alguma coisa os impele numa ordem definida, um após o outro – precisamente aquela inata e sistemática afinidade entre os conceitos. O seu pensamento, na realidade, não é tanto descoberta quanto reconhecimento, relembrança; retorno a uma primeva, longínqua morada perfeita da alma, de onde os conceitos um dia brotaram – neste sentido, filosofar é um atavismo de primeiríssima ordem. (\textit{JGB/ABM}, §20).
\end{quotation}

Essas considerações ocorrem justamente no intervalo entre dois aforismos dedicados ao tema da vontade, dando a impressão de que Nietzsche deseja a um tempo legitimar e apontar os limites de sua própria concepção da vontade. Na sequência do aforismo acima, o filósofo passa a considerações sobre como as línguas indo-europeias estruturam um certo modo de pensar e definem os contornos das filosofias possíveis no seu interior, “do mesmo modo que o caminho parece interditado a certas possibilidades outras de interpretação do mundo” (\textit{JGB/ABM}, §20). Com tais considerações não se prova coisa alguma a respeito da correção das crenças fundamentais a que somos induzidos devido a práticas linguísticas, mas prova-se a condicionalidade de todo pensamento, e se traz à luz certas expectativas epistêmicas incontornáveis, cuja satisfação, no entanto, leva renitentemente a suposições inconsistentes. Nesse sentido, a sintaxe do pensamento ocidental operaria “naturalmente” em termos de causalidade da vontade, e toda tentativa de refinamento desses termos em algum ponto esbarrará nos limites que constituem as condições vitais ou a “fisiologia” do pensamento. Ter este fato em mente é uma atitude epistemicamente sã, que expande a reserva falibilista do pensamento. A conclusão do aforismo faz menção à “superficialidade de Locke no tocante à origem das ideias”, menção cujo significado não poderemos explorar em detalhe aqui, mas que certamente aponta para a impossibilidade de que o pensamento  seja indefinidamente plasmável, que ele possa ser receptáculo de qualquer “verdade” empiricamente descoberta, independentemente das expectativas que o filósofo herdou por vias biológicas e culturais, e que operam nas camadas mais profundas do seu pensamento. Mesmo que busque contornar os riscos do dogmatismo, Nietzsche assume que sua filosofia tem \textit{terroir}. Sua resposta às tensões entre consciência metódica e projeção da personalidade nos parece no mínimo interessante.

	Não há como negar, entretanto, que a doutrina da vontade de poder soa um tanto exótica. Ela certamente coloca um desafio para qualquer intérprete. Segundo Clark e Dudrick, todo o experimento desenvolvido ao longo de \textit{ABM} §36 não passa de uma espécie de armadilha: o leitor que dela escapar poderá então compreender a inaplicabilidade do conceito de vontade de poder à totalidade dos fenômenos naturais. Para nós, não está nada claro que esse seja o sentido do elaborado experimento proposto por Nietzsche, mas além disso, a tese final defendida por eles nos parece altamente improvável: que na medida em que o experimento se destina a provar o naufrágio de toda tentativa de projeção de vontade de poder a outros domínios que não o “espaço das razões”, sua verdadeira função seria fundar uma psicologia não-naturalista (CLARK, DUDRICK, p. 243). 

Clark e Dudrick não são os únicos a resistir à interpretação cosmológica de vontade de poder. John Richardson, que busca uma interpretação profundamente naturalizada, admite que interpretar vontade de poder como um aspecto de todo mundo vivente e até mesmo de todo mundo inorgânico é a conotação dominante nos escritos nietzschianos, embora um tanto problemática; segundo Richardson, haveria uma interpretação “recessiva”, segundo a qual o poder seria não o aspecto fundamental mas a meta mais comum dos impulsos, por resultado da seleção natural. O intérprete admite, contudo, que essa interpretação não é suficiente para abordar grande parte dos escritos de Nietzsche. George Stack sugere que uma cosmologia em termos de vontade de poder seria algo como um mito moderno, um amálgama de arte, filosofia e ciência, que se destina principalmente a fins edificantes. Mas segundo Stack, o mesmo não pode ser dito das aplicações da doutrina da vontade de poder no campo da psicologia:

\begin{quotation}
A análise psicológica da luta por poder, por um “sentimento de poder”, dominação, por mais, por controle, foi uma peça central do pensamento de Nietzsche muito antes de ele apresentar uma interpretação “cosmológica” de toda a “efetividade” (\textit{Wirklichkeit}). (…) A interpretação psicológica decididamente não é derivada de um mito nem é ela mesma mítica: é uma interpretação perspectivista de nosso ser fundamental que é inteiramente independente da \textit{transferência} dessa concepção psicodinâmica ao domínio não-humano. (STACK, 2005, p. 184).
\end{quotation}

Temos discutido, no entanto, as investidas de Nietzsche no sentido de oferecer a doutrina da vontade de poder como um princípio metateórico com validade na física, na biologia, na psicologia –  nesse sentido ela teria valor como “ficção regulativa”, mas não exatamente como “mito”, como quer Stack. Além disso, a leitura de Stack sugere um retorno de Nietzsche à posição de juventude, em que estava comprometido com a ideia de que há certas necessidades metafísicas que são um dado antropológico inevitável, de forma que devemos assumir com boa consciência o “ponto de vista do ideal”, ou seja, a criação de ficções de cunho metafísico e função edificante, posição a que Nietzsche renunciou em \textit{HH}. Ou ainda, a interpretação de Stack parece aproximar a investida de Nietzsche a uma possível \textit{pia fraus}. 

	Na verdade, Stack perde de vista que o maior candidato a artifício com fins edificantes em \textit{ABM} não são as teses relativas à vontade de poder, mas a experiência algo mística do “eterno retorno”. Não há qualquer ocorrência de “eterno retorno” em \textit{ABM}, mas há uma alusão muito clara à ideia de eterno retorno em um aforismo que trata da superação do pessimismo. Nesse aforismo Nietzsche afirma que “quem, como eu” levou o pessimismo em consideração da forma mais profunda até enfim superá-lo seria capaz de abrigar um “ideal contrário”:  “o ideal do homem mais exuberante, mais vivo e mais afirmador do mundo, que não só aprendeu a resignar e suportar tudo o que existiu e é, mas deseja tê-lo novamente, \textit{tal como existiu e é}, por toda a eternidade, gritando incessantemente '\textit{da capo}' [do início]”. “Eterno retorno” é acima de tudo uma experiência pessoal possível para aquele que tem uma disposição afirmativa de si mesmo  “porque sempre necessita outra vez de si mesmo – e se faz necessário” e já que o “si mesmo” é apenas uma peça do todo, já que toda uma série de acasos se fez necessária para a sua existência, a afirmação do si mesmo é a afirmação do todo. A experiência do eterno retorno marca uma disposição erótica para com a vida. Significativamente, esse aforismo ocorre no capítulo que Nietzsche dedica a “A natureza religiosa” (\textit{Das Religiöse Wesen}), como indicação de uma experiência mística possível numa era que começa a se despedir das religiões
\footnote{Esse capítulo encerra uma série de considerações sobre como, por vezes paradoxalmente, não só o cristianismo, mas as filosofias que de alguma forma recorreram à religião, como a de Platão, puderam exercer uma forma de cultivo e revigoramento humano, tema que não teremos oportunidade de desenvolver aqui. Esse tema foi desenvolvido com maestria por Lampert (2000).}. 
Eterno retorno, no sentido explorado em \textit{ABM}, é uma experiência de desejo: não é uma \textit{pia fraus} porque não afirma qualquer coisa que possa ser verdade ou mentira.

Os experimentos com vontade de poder, por outro lado, têm outro tom e outra destinação. Não há muitas dúvidas quanto à afirmação de Stack sobre a anterioridade cronológica e filosófica das aplicações psicológicas de “vontade de poder”. Firmada sobre o vocabulário de comando e obediência, a tese da vontade de poder tem um aspecto essencialmente político. Conquanto Nietzsche inegavelmente tenha ensaiado uma expansão por analogia do campo psicológico aos domínios mais amplos – com uma retórica que não é aquela do mito, mas que emula a especulação científica – o conteúdo dessa analogia de fato parece surtir efeitos mais robustos ali onde nasceu, ou seja, no campo da interpretação das relações políticas entre impulsos (vida psíquica) e das relações políticas entre pessoas. Mas isto não exclui a possibilidade de que, uma vez disponível, Nietzsche se valha da conotação cosmológica da vontade de poder, que lhe permite pensar a conexão causal que atribuímos aos eventos não humanos em continuidade com o tipo de causalidade que reconhecemos como atuante no âmbito dos fenômenos humanos em sua interpretação não moralizada.

	Muitas das referências de Nietzsche à vontade de poder se destinam a apontar as “origens baixas” daquilo a que tradicionalmente se atribuiu os mais altos valores, ou seja, ele identifica uma ânsia por “domínio” ou “controle” atuando na base de comportamentos vistos como os mais sublimes, como a compaixão. Vontade de poder é um instrumento interpretativo usado para identificar a “fisiologia” dos mais diferentes arranjos entre impulsos, arranjos que fazem com que se encontre satisfação mesmo em comportamentos aparentemente antinaturais, como o ascetismo e a crueldade. Esse aspecto bárbaro ou “tirânico” da vontade de poder seria, no entanto, passível de refinamento: assim, a atividade filosófica, em si, é apresentada como a “mais espiritual vontade de poder” (\textit{JGB/ABM}, §9). 

Mas ao que parece, o aspecto psicológico da doutrina da vontade de poder encontra aplicação, sobretudo ao longo de \textit{ABM}, na busca por um novo modelo de atribuição de responsabilidade. No capítulo \ref{cap1}, discutimos a tese da “irresponsabilidade radical”, apresentada em \textit{HH}, que enquanto crítica da compreensão moral da ação (fundada na noção de livre-arbítrio) levaria à derrocada de qualquer base segura para atribuição de censura ou mérito. Em \textit{ABM}, Nietzsche dá sinais claros de haver revisto essa tese. É óbvio que não há qualquer interesse em reabilitar a noção de livre-arbítrio e todo o vocabulário da culpa e do pecado a que ela está associada, na verdade, o que se dá em \textit{ABM} é um esforço mais radical no sentido de pensar a atuação da vontade para além das dicotomias liberdade-determinismo. Nietzsche claramente recusa a ideia de liberdade absoluta da vontade, que associa à ideia de \textit{causa sui}, mas a partir de \textit{ABM} vê-se uma recusa também do seu oposto, a “vontade cativa”, e o filósofo passa a falar então de “vontade forte” e “vontade fraca”:

\begin{quotation}
A \textit{causa sui} [causa de si mesmo] é a maior autocontradição até agora imaginada, uma espécie de violentação e desnatureza lógica: mas o extravagante orgulho do homem conseguiu se enredar, de maneira profunda e terrível, precisamente nesse absurdo. O anseio de “livre-arbítrio”, na superlativa acepção metafísica que infelizmente persiste nos semi-educados, o anseio de carregar a responsabilidade última pelas próprias ações, dela desobrigando Deus, mundo, ancestrais, acaso, sociedade, é nada menos que o de ser justamente essa \textit{causa sui} e, com uma temeridade própria do barão de Münchhausen, arrancar-se pelos cabelos do pântano do nada em direção à existência. Supondo que alguém perceba a rústica singeleza desse famoso “livre-arbítrio” e o risque de sua mente, eu lhe peço que leve sua “ilustração” um pouco à frente e risque da cabeça também o contrário desse conceito-monstro: isto é, o “cativo-arbítrio”, que resulta em um abuso de causa e efeito. Não se deve \textit{coisificar} erroneamente “causa” e “efeito”, como fazem os pesquisadores da natureza (e quem, assim como eles, atualmente “naturaliza” no pensar –), conforme a tacanhez mecanicista dominante, que faz espremer e sacudir a causa, até que “produza o efeito”; deve-se utilizar a “causa”, o “efeito”, somente como puros \textit{conceitos}, isto é, como ficções convencionais para fins de designação, de entendimento, \textit{não} de explicação. No “em si” não existem “laços causais”, “necessidade”, “não-liberdade psicológica”, ali não segue “o efeito à causa”, não rege nenhuma “lei”. Somos nós apenas que criamos as causas, a sucessão, a reciprocidade, a relatividade, a coação, o número, a lei, a liberdade, o motivo, a finalidade; e ao introduzir e entremesclar nas coisas esse mundo de signos, como algo “em si”, agimos como sempre fizemos, ou seja, \textit{mitologicamente}. O “cativo-arbítrio” não passa de mitologia: na vida real há apenas vontades fortes e fracas. – É quase sempre um sintoma daquilo que falta nele próprio, quando um pensador sente em toda “conexão causal” e “necessidade psicológica” um quê de coação, exigência, obrigação de seguir, pressão, não-liberdade: estas são impressões delatoras – a pessoa se trai. E, se observei corretamente, em geral a “não-liberdade de arbítrio” é vista como problema por dois lados inteiramente opostos, mas sempe de maneria profundamente \textit{pessoal}: uns não querem por preço algum abandonar sua “responsabilidade”, a fé em si, o direito pessoal ao \textit{seu} mérito (as raças vaidosas estão deste lado –); os outros, pelo contrário, não desejam se responsabilizar por nada, ser culpados de nada, e, a partir de um autodesprezo interior, querem depositar o fardo de si mesmos em algum outro lugar. (…). (JGB/ABM, §21). 
\end{quotation}

Afirmar que a vontade é “cativa”, quer dizer, impotente frente a um mundo radicalmente determinista, seria um caso de “coisificar erroneamente 'causa' e 'efeito', como fazem os pesquisadores da natureza (e quem, assim como eles, atualmente 'naturaliza' no pensar –), conforme a tacanhez mecanicista dominante” (JGB/ABM,§21). A versão mecanicista seria, assim, uma versão espelhada da doutrina do livre-arbítrio: ambas exigem que, para ter qualquer poder de atuação, a vontade enquanto “causa” deveria agir de modo isolado e autossuficiente. A doutrina do livre-arbítrio atribui essa capacidade à vontade racionalmente deliberada, o mecanicismo a nega sumariamente. A doutrina do livre-arbítrio visa desobrigar “Deus, mundo, ancestrais, acaso, sociedade” pelos atos de vontade, a visão mecanicista faz recair todo o peso sobre esses fatores – contra as duas partes, é possível levantar a suspeita de que elas perdem de vista uma interação mais profunda entre esses diversos fatores. O vocabulário de Nietzsche aqui é agressivo: fala do que as coisas são “em si” e do que é a vontade “na vida real”
\footnote{“Em si” não precisa ser entendido no sentido de coisa-em-si; pode ter o sentido mais fraco de “objetivamente”, sentido com que era usado por vezes pelo próprio Kant.}. Uma forma de fazer frente a essas duas concepções “errôneas” de causalidade seria, claro, recorrer à causalidade da vontade tal como reformulada por Nietzsche. Richardson é um dos intérpretes que levaram essa possibilidade a sério, e comenta a respeito dessa passagem:

\begin{quotation}
Ambos os lados da disputa recortam o devir em sequências de causas e efeitos discretos, atomizam os processos na forma de partes completas em si mesmas. Eles separam a causa do ato de causar, o efeito do ser efetivado, quer os considerem como substâncias ou como “estado de coisas”. Mas quando vemos que as “partes” reais do mundo são processos de vontade definidos por suas relações de poder entre si, vemos que não há partes autossuficientes e que as coisas “condicionam” umas às outras de forma ainda mais penetrante que o determinismo havia suposto; nós aprendemos uma nova forma desse pensamento, um novo fatalismo. Mas ainda, nós também podemos ver como essas relações de poder de comando e obediência dão a base para um novo tipo de liberdade e responsabilidade, agora não mais como a herança comum de todos os sujeitos mas como uma forma ideal de comando que algumas vontades podem alcançar. (RICHARDSON, 1996, p. 211-212).
\end{quotation}

Uma vez que a vontade de poder é pensada como aspecto comum dos impulsos que lhes permite colocarem-se numa relação orgânica, a “vontade forte” seria o sintoma de que os diferentes impulsos conseguiram formar uma “comunidade bem construída e feliz” e, inversamente, a “vontade fraca” seria um sintoma de desagregação dos impulsos, de decadência
\footnote{Essa visão escaparia ao mecanicismo porque ela se dá em termos profundamente relacionais; não é nada óbvio, no entanto, que com isto Nietzsche renuncie a qualquer visão “naturalizada” da ação.}. 

	Na seção anterior, discutimos a ideia de que é a hierarquia estabelecida entre diferentes vontades de poder o que confere organicidade e identidade aos entes, sejam eles uma pessoa, uma instituição, uma cultura. Por isto mesmo, poder-se-ia abordar os processos de vida e morte de homens e culturas tomando como referência privilegiada o modo como seus diferentes impulsos formam uma composição hierárquica; a sensibilidade à detecção desse traço hierárquico seria algo como um “sexto sentido” desenvolvido pelo homem moderno. É esse tipo de sensibilidade que Nietzsche associa ao “sentido histórico”, entendido como “a capacidade de perceber rapidamente a hierarquia de valorações segundo as quais um povo, uma sociedade, um homem viveu, o 'instinto divinatório' para as relações entre essas valorações, para o relacionamento da autoridade dos valores com a autoridade das forças atuantes” (\textit{JGB/ABM}, §224). Nietzsche parece entender que o modo de composição hierárquica contribui decisivamente para a capacidade de intensificação ou degeneração da vida, de forma que seu estudo seria uma etapa importante do trabalho de auferição de valor a um homem ou sociedade. Observando a disposição hierárquica de uma organização, o historiador busca compreender de que modo essa organização pôde se firmar como uma “comunidade bem construída e feliz” ou, por outro lado, como em certas configurações essa comunidade se desagrega e definha. É por isso que grande parte do discurso político de \textit{ABM} se destina ao estabelecimento de hierarquias: hierarquias entre virtudes; entre áreas de atividade humana, como ciência e filosofia e até mesmo, como veremos adiante, hierarquias entre homem e mulher.

Já que haveria uma “fatalidade” que faz com que certas vontades sejam fortes ou fracas, o sentido de “responsabilidade” que se pode lhes atribuir deve ser repensado, mas o modelo abre a possibilidade de se atribuir diferentes \textit{valores} a diferentes composições orgânicas, segundo o grau de sua força. Com esse modelo retira-se o foco do sujeito metafísico tradicionalmente concebido, passível de atribuição de culpa, mas não se renuncia à atribuição de valor, e justamente isto, para Nietzsche, seria a marca da passagem para um estágio “extramoral” (Cf. JGB/ABM, §32). No entanto, o filósofo renuncia, em \textit{Além de Bem e Mal} à pretensão de estabelecer uma “justa medida” de hierarquização ou intensidade de vida, e por isto renuncia também à possibilidade de haver uma resposta pré-definida e evidente sobre o valor de cada composição pessoal e cultural. A atividade de atribuição de valor, novamente, depende de um ato pessoal
\footnote{Nesse belíssimo aforismo, \textit{ABM} §224, discute-se a ideia de que o pensamento histórico, por nutrir uma disposição à compreensão e apreciação dos mais diferentes arranjos possíveis, de que diferentes povos e diferentes épocas são testemunho, é incapaz de proferir um “Sim” ou “Não” definitivo, isto é, condenar ou redimir definitivamente a forma como tais povos ou épocas encontraram sua composição hierárquica própria, e por isto ele é incapaz também de oferecer uma medida fixa de avaliação: “'sentido histórico' significa quase que sentido e instinto para tudo, gosto e língua para tudo: no que logo se revela o seu caráter \textit{não-nobre}. (…) O tão definido Sim e Não do seu palato [de homens de culturas nobres], seu pronto desgosto, sua hesitante reserva face a tudo que lhes for estranho, seu horror à falta de gosto que há na curiosidade viva, e sobretudo aquela má vontade que toda cultura nobre e autossuficiente demonstra em admitir uma nobre cobiça, uma insatisfação com o que é seu e uma admiração do que é outro: tudo isso os predispõe negativamente até em face das melhores coisas do mundo, que não são sua propiredade e não poderiam es tornar sua presa – nenhum sentido é mais incompreensível para esses homens do que justamente o sentido histórico, com a sua servil curiosidade plebeia. (…) Nós, homens do 'sentido histórico': como tais temos nossas virtudes, não se pode negar – somos despretensiosos, desinteressados, modestos, bravos, plenos de autossuperação, de dedicação, muito gratos, muito pacientes e acolhedores – e com tudo isso não somos talvez 'de muito bom gosto'. (…) A \textit{medida} nos é estranha, confessemos a nós mesmos; a comichão que sentimos é a do infinito, imensurado. Como um ginete sobre o corcel em disparada, deixamos cair as rédeas ante o infinito, nós, homens modernos, semibárbaros; e temos a nossa bem-aventurança ali onde mais estamos – \textit{em perigo}.” (\textit{JGB/ABM}, §224).}.

A interpretação nietzschiana da causalidade da vontade oferece não apenas uma revisão dos conceitos tradicionais, mas também algumas teses sobre o por quê psicológico das teorias tradicionais da ação. Como vimos no aforismo §21 citado acima, essas teses se referem à relação  com o senso de “orgulho” e tipificam dois tipos de moralidade, uma moral “vaidosa”, que atribui à vontade um mérito despropositado, e uma moral do “autodesprezo”, expressão da incapacidade de reconhecer a atuação da própria vontade e do desejo de desimcumbir-se de qualquer responsabilidade. A reedição nietzschiana da vontade visa criar um novo senso de orgulho humano, capaz de “entender realistamente até onde se estende sua responsabilidade” (LAMPERT, 2000, p. 54), e isto a colocaria muito distante da \textit{pia fraus} ou de intenções meramente “edificantes”. Conquanto se tenha lançado uma nova luz sobre a compreensão da ação, a retórica nietzschiana dá a entender que esses avanços filosóficos têm ainda um caráter preparatório para o compromisso com uma nova “tarefa”, lembrando que \textit{ABM} é anunciado como um “prelúdio”. Essa tarefa é programaticamente apresentada em alguns aforismos, um deles destinado ao tema da interpretação da ação, cuja conclusão é: “A superação da moral, num certo sentido até mesmo a autossuperação da moral, inclusive: este poderia ser o nome para o longo e secreto lavor que ficou reservado para as mais finas e honestas, e também mais maliciosas consciências de hoje, na condição de ardentes pedras de toque da alma.” (\textit{JGB/ABM}, §32).  

A compreensão da ação que cria um novo “orgulho” para o homem é tarefa das mais “honestas” e mais “maliciosas” consciências. A interpretação da natureza humana em termos de vontade de poder é apresentada então ao mesmo tempo como mais verdadeira e como receptiva à atividade valorativa. Na seção seguinte abordaremos a questão de como o exercício de honestidade se liga à passagem para uma era extramoral. Cuidaremos também de uma questão que se impõe dada a tentativa de pensar a o homem em continuidade com o restante do mundo físico: como é possível viver conforme essa natureza?

\section{O texto natural}

\begin{quotation}
Por fim se considere que mesmo o homem do conhecimento, ao obrigar seu espírito a conhecer, \textit{contra} o pendor do espírito e também, com freqüência, os desejos de seu coração – isto é, a dizer Não, onde ele gostaria de aprovar, amar, adorar – atua como um artista e transfigurador da crueldade; tomar as coisas de modo radical e profundo já é uma violação, um querer-magoar a vontade fundamental do espírito, que incessantemente busca a aparência e a superfície – em todo querer-conhecer já existe uma gota de crueldade. (\textit{JGB/ABM}, §229).

(…) São palavras belas, solenes, reluzentes, tilintantes: honestidade, amor à verdade, amor à sabedoria, sacrifício pelo conhecimento, heroísmo do que é veraz – há algo nelas que faz subir o orgulho. Mas nós, eremitas e marmotas, há muito nos persuadimos, no fundo segredo de uma consciência eremita, que também essa digna pompa verbal é parte do velho enfeite-mentira, poeira e purpurina da inconsciente vaidade humana, e que também sob uma cor e uma pintura tão lisonjeiras deve ser reconhecido, uma vez mais, o terrível texto básico \textit{homo natura}. Retraduzir o homem de volta à natureza; triunfar sobre as muitas interpretações e conotações vaidosas e exaltadas que até o momento foram rabiscadas e pintadas sobre o eterno texto \textit{homo natura}; fazer com que no futuro o homem se coloque frente ao homem tal como hoje, endurecido na disciplina da ciência, já se coloca frente à \textit{outra} natureza, com intrépidos olhos de Édipo e ouvidos tapados como os de Ulisses, surdo às melodias dos velhos, metafísicos apanhadores de pássaros, que por muito tempo lhe sussurraram: “Você é mais! É superior! Tem outra origem!” – essa pode ser uma louca e estranha tarefa, mas é uma tarefa – quem o negaria: Por que a escolhemos, essa tarefa? Ou, perguntando de outro modo: “Por que conhecimento, afinal?”. Todos nos perguntarão isso. E nós, premidos desse modo, nós, que já nos fizemos mil vezes a mesma pergunta, jamais encontraremos resposta melhor que... (\textit{JGB/ABM}, §230)
\end{quotation}

Nesta que é uma das peças mais citadas e de mais difícil compreensão da obra nietzschiana encontramos a única ocorrência da promessa de “retraduzir o homem de volta à natureza”. Em nenhum outro lugar da obra publicada Nietzsche define sua tarefa exatamente nesses termos: embora até o fim Nietzsche se refira orgulhosamente a suas habilidades em filologia enquanto metáfora para a arte de rigorosa interpretação do mundo e de si mesmo, é somente em \textit{ABM} §230 que se encontra a menção ao “terrível texto básico \textit{homo natura}” ou “eterno texto \textit{homo natura}”. A ocorrência dessas expressões faz com que, dentre todos os aforismos programáticos da obra nietzschiana publicada, \textit{ABM} §230 soe particularmente ambicioso e enigmático. 
	
	A definição dessa “tarefa” é precedida por considerações sobre a honestidade e sua relação com a crueldade; na verdade, esse tema é retomado imediatamente do aforismo anterior, \textit{ABM} §229, que se refere à vontade de conhecimento como uma “violação” da “vontade fundamental do espírito”. \textit{ABM} §230 começa com um esclarecimento do que seria essa “vontade fundamental do espírito”, ou antes, aponta pra uma dualidade fundamental. “Esse imperioso algo a que o povo chama espírito” é apresentado primeiramente como algo que “tem a vontade de conduzir da multiplicidade à simplicidade, uma vontade restritiva, conjuntiva, sequiosa de domínio e realmente dominadora.” (\textit{JGB/ABM}, §230). Ao que tudo indica, esse traço fundamental do espírito seria expressão do “princípio de identidade”, que segundo Spir é a estrutura mais básica e indispensável de nosso aparato cognitivo, isto é, o princípio que faz com que vejamos o mundo como entidades simples e discretas, reconduzindo o múltiplo ao uno – esta tendência Nietzsche associa a um impulso de “dominação”. Ao longo do aforismo Nietzsche aponta, ao que parece, 7 desdobramentos dessa tendência primordial, referentes a necessidades e faculdades que seriam “as mesmas que os fisiólogos apresentam para tudo que vive, cresce e se multiplica”, quais sejam: “assimilar o novo ao antigo”; “simplificar o complexo”; “rejeitar ou ignorar  o inteiramente contraditório”; “enquadrar novas coisas em velhas divisões”; “não saber” – até aqui, predomina o signo da “assimilação”, cujo campo semântico remete ao sistema digestivo e metabólico, sentido que é reforçado pela sequência do aforismo, em que se encontra a ideia de que o grau de assimilação dependeria da “força digestiva” do espírito, uma vez que o “'espírito' se assemelha mais que tudo a um estômago”. Essa imagem leva a pensar que o espírito se “alimenta” de mundo, rejeita ou absorve conteúdos segundo suas capacidades e necessidades – rejeita certos dados instintivamente vistos como desnecessários ou potencialmente nocivos, “venenosos” talvez, os quais prefere ignorar, não saber, e outros conteúdos o espírito absorve e reconduz às funções mais profundamente incorporadas, predominando a preferência prudencial pelo que já é conhecido. Pode-se pensar que “enquadrar novas coisas em velhas divisões” seja uma estratégia de redução do gasto energético envolvido nesse processo; essa função também remete a um segundo sentido de “assimilar” enquanto “tornar similar”, que se liga ao impulso à simplificação, em que o traço dominador aparece mais claramente
\footnote{Vale lembrar que o vocabulário da assimilação é utilizado na sociologia para se referir aos processos de acolhimento e incorporação de grupos minoritários pelas culturas hegemônicas. }.

A metáfora do espírito como estômago aparece na sequência da apresentação dessas cinco primeiras funções cognitivas, e então se segue a referência a mais duas funções: “deixar-se iludir” e “iludir outros espíritos e disfarçar-se diante deles”. A primeira se ligaria a um desejo de fruição da própria “arbitrariedade”, de forma que o espírito chega a suspeitar que “as coisas \textit{não} são assim, de que apenas se convenciona que sejam assim” mas se satisfaz por “um gosto na incerteza e ambiguidade”. A segunda função expressa a atividade espiritual de criar máscaras, metamorfosear-se, e seria um mecanismo de proteção e esconderijo, além da fruição com a própria criatividade. Por razões de simplificação, podemos nos referir a essa primeira tendência do espírito, explorada por Nietzsche em termos tão ricos, como referentes à tendência à “falsificação”. O próprio Nietzsche as resume como expressão de  “vontade de aparência, de simplificação, de máscara, de manto, enfim, de superfície – pois toda superfície é um manto”. 

	Lampert captou muito bem as implicações dessa visada nietzschiana sobre a inclinação natural do espírito:

\begin{quotation}
A clarificação de Nietzsche da vontade básica do espírito destaca a naturalidade do costume, das crenças e das práticas que representam equivocadamente o mundo para dominá-lo. O antinatural é natural para o espírito da humanidade; ele naturalmente opõe as verdades palpáveis mas indigestas da natureza tal como a natureza animal da humanidade, que Nietzsche está inclinado a deixar escapar (229). A vontade básica do espírito humano o inclina poderosamente para o cosmético, para superfícies mentirosas, e a filosofia até aqui permitiu e incentivou essa inclinação natural. (LAMPERT, 2001, p. 228)
\end{quotation}

Ao que nos parece, Nietzsche de fato aponta para um certo acordo entre senso comum e boa parte da tradição filosófica no sentido de incentivar essa tendência tendência primordial do espírito, que no entanto encontraria uma contraposição no “homem do conhecimento”, cujo “pendor” estaria em “tomar e \textit{querer} tomar as coisas de modo profundo, plural, radical”, tendência que se identifica com “uma espécie de crueldade da consciência e do gosto intelectuais”. Essa seria uma disposição cruel do espírito, em primeiro lugar, por estabelecer-se fundamentalmente como resistência à sua própria tendência mais básica, que é a tendência à falsificação; isto é sugerido no aforismo anterior, \textit{ABM} §229, onde se reivindica a expansão do conceito de crueldade para abarcar não só o gozo com o sofrimento alheio, mas também o gozo com o próprio sofrimento. O impulso ao conhecimento se liga à crueldade na medida em que expõe o sujeito cognoscente a tudo aquilo que os mecanismos prudenciais mais incorporados o fazem evitar: o múltiplo, o novo, o desconhecido, a nuance, a diferença. O “homem do conhecimento” faz violência a si mesmo ao afastar-se dessa prudência e expor-se aos efeitos potencialmente corrosivos da verdade. Para Nietzsche, esse impulso ao conhecimento por vezes é visto como uma “extravagante honestidade”, termo que o próprio filósofo parece considerar ainda demasiado suave, apenas uma sutileza própria de homens “virtuosos e amáveis”. Nos termos do próprio Nietzsche, que nesse aforismo se dirige da forma mais direta possível, trata-se de crueldade, dureza e disciplina – disposições de espírito cultivadas pela “disciplina da ciência” e convocadas à tarefa de recuperação do “terrível texto básico \textit{homo natura}”. 

Nesse ponto começam os problemas de interpretação do aforismo. Como é possível que o homem, enquanto animal não fixado, chegue a dispor de um “eterno texto” natural? Se milênios de seleção social, sob vigência da cultura moral, atuaram sobre hábitos, instintos, práticas cognitivas, práticas sexuais e valores humanos, em que sentido se pode pensar num \textit{homo natura}? Se a atuação da seleção social fez do homem um palimpsesto, como recuperar o texto básico? O que haveria de atual e efetivo nesse texto original? Trata-se de um texto original e extraviado ou de um texto “fundamental”, no sentido de que ainda serve de alicerce para as camadas mais recentes e superficiais? O palimpsesto esconde uma natureza mais natural? O erro, a simplificação, a exaltação vaidosa com que se escreveu o palimpsesto não seriam afinal expressão dessa mesma natureza? A tarefa que se coloca é uma tarefa arqueológica em que se deve cavar por baixo das camadas mais recentes da “pintura”? Ou é mesmo uma tarefa de tradução, no sentido de trazer para uma linguagem mais primitiva a configuração atual do homem, ou trazer para uma linguagem mais atual as camadas mais primitivas que ainda atuam no homem? É possível, é desejável retornar a uma primeira natureza? É esse o sentido da tarefa?

Como não parece haver qualquer resposta direta e inequívoca a essas questões, vamos proceder por remoção, endereçando primeiro as interpretações sobre as quais o filósofo deseja “triunfar”: estas seriam “as muitas interpretações e conotações vaidosas e exaltadas que até o momento foram rabiscadas e pintadas sobre o eterno texto \textit{homo natura}” ou “melodias dos velhos, metafísicos apanhadores de pássaros, que por muito tempo lhe sussurraram: 'Você é mais! É superior! Tem outra origem!'”. A primeira referência que temos, portanto, são às interpretações morais de cunho metafísico que atribuíram ao homem uma origem ou uma destinação divinas. Valendo-nos das poucas pistas que temos aqui, passemos a uma breve consideração sobre a vaidade, como forma de melhor compreendermos a crítica às interpretações metafísicas.

	A vaidade, Nietzsche sugere em outro aforismo, consiste num mecanismo em que os seres “buscam criar de si uma opinião boa que eles mesmos não têm – e portanto não 'merecem' também –, e que no entanto passam a \textit{crer} posteriormente nessa boa opinião” (\textit{JGB/ABM}, §261). Ao longo desse aforismo, desenvolve-se a ideia de que a força que atua no vaidoso é na verdade um traço de submissão às valorações que lhe são atribuídas por outrem, uma transferência da capacidade de autoavaliação para uma outra pessoa ou instituição. No caso das “interpretações vaidosas” rabiscadas sobre o “texto \textit{homo natura}”, haveria, portanto, um caso de submissão aos valores criados por “velhos metafísicos, apanhadores de pássaros”, interpretações que, na verdade, contribuiram tanto para exaltar quanto para colocar sobre o homem o peso de todo um vocabulário do pecado, da culpa, da danação. 
	
	O vaidoso coloca-se numa situação de extrema suscetibilidade, ficando vulnerável a qualquer tipo de opinião e valoração, nem sempre positivas: 

\begin{quotation}
O vaidoso se alegra de cada opinião boa que ouve sobre si (independente de qualquer ponto de vista de utilidade, e também não considerando se é falsa ou verdadeira), assim como sofre de cada opinião ruim: pois ele se submete a ambas, ele se sente submetido a elas, por esse antigo instinto de submissão que nele irrompe. (\textit{JGB/ABM}, §261).
\end{quotation}

Uma vez que o impulso à submissão leva o vaidoso a desconsiderar os fatores de veracidade e utilidade das opiniões, ele se torna mais suscetível também a uma relação “exaltada” com essas opiniões. Para Nietzsche, o traço fundamental da vaidade, o “instinto de submissão”, denuncia sua origem “escrava”, por oposição ao ponto de vista do nobre, cujo ímpeto maior está em “só por si arrogar um valor e 'pensar bem' de si” (\textit{JGB/ABM}, §261). O filósofo aponta que com a ascensão da democracia, vista como uma ocasião de mistura entre moral escrava e nobre, há uma tendência à derrocada do comportamento vaidoso, de forma que se poderia afirmar que “a vaidade é um atavismo.” (\textit{JGB/ABM}, §261). A derrocada do comportamento vaidoso na modernidade é vista como um fato positivo, claro, uma vez que falta às opiniões vaidosas qualquer aporte de veracidade ou utilidade. Não há dúvidas de que Nietzsche vê esse momento como uma oportunidade para a construção de uma visão mais justa do humano
\footnote{A superação da submissão envolvida no comportamento vaidoso, por outro lado, ecoa o tema kantiano da maioridade, que na verdade reaparece com uma certa constância ao longo de \textit{ABM}, ver, por exemplo \textit{JGB/ABM} §34 e §35. Apesar de declarar guerra à ilustração democrática já no prólogo de \textit{ABM}, Nietzsche parece acalentar sua própria versão de uma aposta iluminista; a este respeito, ver o comentário de Lampert: “\textit{The New Enlightenment} is a prospective title often used in the notebooks containing materials that eventually appeared under the title \textit{Beyond Good and Evil} (for example, \textit{KSA} 11.26 [293, 298]; 27 [79]; 29 [40]). Nietzsche’s task is itself an enlightenment that carries forward the Greek enlightenment prior to the Platonic strategy that reconciled that enlightenment with popular prejudice.” (LAMPERT, 2001, p. 16).}.

	Às interpretações vaidosas e exaltadas, Nietzsche contrapõe sua própria interpretação, apresentada de forma mais direta que o usual, num aforismo cujo tom é carregado de crueldade:

\begin{quotation}
O homem, um animal complexo, mendaz, artificial, intransparente, e para os outros animais inquietante, menos pela força que pela astúcia e inteligência, inventou a boa consciência para chegar a fruir sua alma como algo \textit{simples}; e toda a moral é uma decidida e prolongada falsificação, em virtude da qual se torna possível a fruição do espetáculo da alma. Desse ponto de vista, o conceito de “arte” incluiria bem mais do que normalmente se crê. (\textit{JGB/ABM}, §291). 
\end{quotation}

A crueldade aqui está em trazer à tona sem qualquer recurso atenuante aquilo que mais contraria as interpretações vaidosas do homem: o fato de que o homem é um \textit{animal}. A caracterização de sua natureza animal como fundada em mendacidade, artificialidade, intransparência, traz a sugestão paradoxal de que há um fator natural que faz o homem querer se distinguir da natureza. Quanto à afirmação, um tanto irônica, de que o conceito de “arte” pode ser expandido para  incluir “bem mais do que normalmente se crê”, o contexto do aforismo leva a pensar que o que aí deve ser incluído é toda falsificação e autoengano que tornam o mundo humanamente habitável (satisfazendo a vontade fundamental do espírito), bem como a “boa consciência” com essa condição artificial. A afirmação é de que não só a arte, mas toda moral estaria a serviço desse impulso primordial à falsificação, a começar pela falsificação da realidade mais próxima, a realidade da própria alma humana. O conceito de “arte” nesse sentido pode incluir os artifícios da moral. A moral seria fundamentalmente um mecanismo de autoengano, de má compreensão da vida espiritual humana, uma tal que paradoxalmente permite fruir do “espetáculo da alma”. Por mais enviesada que seja, a visada moral é, afinal, um instrumento que permite ao homem cognição de si mesmo, e um tipo de cognição que guarda uma certa relação de afinidade com a “tendência fundamental do espírito” que discutimos acima. Mas isto não leva à ideia de que a moral, portanto, seria expressão de uma necessidade natural?

Essa questão fundamental já havia sido endereçada no primeiro capítulo do livro, em que Nietzsche aponta os paradoxos a que a filosofia estoica é levada em razão de tomar como princípio a possibilidade de viver “conforme a natureza”, frente ao qual o filósofo indaga: “Viver – isto não é precisamente querer ser diverso dessa natureza: Viver não é avaliar, preferir, ser injusto, ser limitado, querer ser diferente?” (\textit{JGB/ABM}, §9). No cenário que se delineia aqui vê-se que o homem, enquanto animal não fixado, precisa ajustar para si um mundo falsificado, fixar para si um ambiente antinatural, o que, por intrigante que seja, é para ele uma necessidade vital. Na verdade, Nietzsche parece apontar a necessidade de falsificar o mundo como compartilhada em algum grau por todo vivente, na medida mesma em que configura uma perspectiva e necessita \textit{organizar-se}. Mas enquanto animal não fixado, o homem teria desenvolvido mecanismos mais complexos de falsificação, e além de mais complexos, mecanismos não estabilizados, de forma que necessitaria viver constantemente sob tentativas de ajuste, dentre as quais se inclui a \textit{coerção} moral a essa natureza primeira, amorfa e potencialmente destrutiva.

O ponto da relação entre natureza e moral é finalmente desenvolvido num outro aforismo:

\begin{quotation}
Toda moral é, em contraposição ao \textit{laisser aller} [“deixar ir”], um pouco de tirania contra a “natureza”, e também contra a “razão”: mas isso ainda não constitui objeção a ela, caso contrário se teria de proibir sempre, a partir de alguma moral, toda espécie de tirania e desrazão. O essencial e inestimável em toda moral é o fato de ela ser uma demorada coerção (…). Mas o fato curioso é que tudo o que há e houve de liberdade, finura, dança, arrojo e segurança magistral sobre a Terra, seja no próprio pensar, seja no governar, ou no falar e convencer, tanto nas artes como nos costumes, desenvolveu-se apenas graças à “tirania de tais leis arbitrárias”; e, com toda a seriedade, não é pequena a probabilidade de que justamente isso seja “natureza” e “natural” – e \textit{não} aquele \textit{laisser aller}! (JGB/ABM, §188).
\end{quotation}

Estamos diante da afirmação da “probabilidade” de que a natureza humana se caracterize justamente por um impulso de distinguir-se da natureza, e de que o instrumento mais eficaz para a satisfação desse impulso foi precisamente a moral. As formulações a que se chega são certamente intrigantes: a moral seria a um tempo um pouco de “tirania contra a natureza”, o que não veta a probabilidade de que a tirania contra a natureza seja “natural”.

	Não se encontra aqui a proposta de naturalização dos conteúdos morais
\footnote{Se o mero fato de que o aspecto tirânico de toda moral “não chega a constituir uma objeção”, isto de forma alguma neutraliza as objeções de Nietzsche à “moral no sentido pejorativo”, ancorada em pressupostos metafísicos, com pretensões universalizantes. O conteúdo dessa moral é sumamente objetável.}, 
claro, mas a suspeita de que justamente o aspecto arbitrário e infundado desses conteúdos dá um testemunho do que o homem é: um animal que tem a cultura como parte de sua etologia, e que portanto vive a sobreposição e mesmo a contradição dos processos de seleção biológica e de seleção social, haja visto que a seleção social apresenta uma lógica própria e menos sedimentada que a seleção natural. Num certo sentido, no entanto, a submissão a leis arbitrárias parece ser apontada como apenas uma etapa na história do homem, ao menos na história do homem europeu; esta seria uma etapa que viabilizou o surgimento própria liberdade, até mesmo a liberdade “no próprio pensar”, e esta, por sua vez, pressiona o homem moderno a reinventar tais leis e avançar uma nova compreensão do material pulsional por tanto tempo coagido pela moral. Ou seja, o ponto de vista moral, e todas suas implicações práticas, seriam um dado irreversível da história ocidental, o que não quer dizer que não possa ser superado, ou mesmo “autossuperado”, como afirma Nietzsche. E isto, mais uma vez, reforça o sentido de maleabilidade que Nietzsche atribui à natureza humana.
	
	Na sequência do aforismo, Nietzsche elenca as principais exigências morais que marcaram essa história:

\begin{quotation}
A prolongada sujeição do espírito, a desconfiada coerção na comunicação dos pensamentos, a disciplina que se impôs o pensador, a fim de pensar sob uma diretriz eclesiástica ou cortesã ou com pressupostos aristotélicos, a duradoura vontade espiritual de interpretar todo acontecimento segundo um esquema cristão e redescobrir e justificar o Deus cristão em todo e qualquer acaso – tudo o que há de violento, arbitrário, duro, terrível e antirracional nisso revelou-se como o meio através do qual o espírito europeu viu disciplinada a sua força, sua inexorável curiosidade e sutil mobilidade: mesmo reconhecendo a quantidade insubstituível de força e espírito que aí teve de ser sufocada, suprimida e estragada (pois nisso, como em tudo, a natureza se mostra como é, em toda a sua magnificência pródiga e \textit{indiferente}, que nos revolta, mas que é nobre). (JGB/ABM, §188)
\end{quotation}

O que há de curioso nesse aforismo é que Nietzsche parte de uma consideração sobre uma série de acidentes e vicissitudes que determinaram o curso da cultura intelectual europeia
\footnote{Há aqui uma clara referência ao período escolástico.}, reconhecendo que estes estão igualmente sob o signo do “violento, arbitrário, duro, terrível e antirracional”, mas o filósofo parece encontrar nesses acidentes um indício do que a natureza \textit{é}. A primeira conclusão que Nietzsche parece extrair é que não há uma adequação entre a natureza humana e a verdade, principalmente, que a natureza humana não é autotransparente, e que não se deve esperar encontrar qualquer tipo de correção epistêmica entre os pressupostos da moral e a natureza; segundo, que por meio de exigências as mais arbitrárias como a de “interpretar todo acontecimento segundo um esquema cristão e redescobrir e justificar o Deus cristão em todo e qualquer acaso”, o espírito realmente atinge uma disciplina que lhe permite chegar a formas menos falsificadas; finalmente, que a natureza, em geral, “em sua magnificência pródiga e \textit{indiferente}” é tão vasta e maleável que todas essas tentativas de ajuste, por maior conteúdo falsificador que tenham, atingem algum grau de \textit{correção} adaptativa. Por outro lado, conquanto as exigências morais sejam sumamente arbitrárias, elas não deixam de ser sintomáticas do combate empreendido contra certas pulsões, e portanto são também uma via indireta para se abordar tais pulsões. Elas são, por assim dizer, um material valioso para o estudo filológico do “texto \textit{homo natura}”.

A partir daqui, podemos fazer algumas considerações sobre o procedimento “filológico” com que Nietzsche aborda a história do pensamento sobre o humano, e que resulta numa reinvenção dessa história. O retrato cruelmente traçado do animal humano em \textit{ABM} §291 parece ser uma refutação ponto por ponto dos principais \textit{tropos} que configuraram a visada tradicional metafísica. O filósofo contrapõe, primeiramente, a ideia da origem divina do homem à visão de sua animalidade; depois, inverte a visão da adequação entre a natureza e verdade, que levou a tradição a equacionar a verdade com o bem; e então, contra toda avaliação moral – espaço do humano para o qual se buscou uma fundamentação mais decididamente inquestionável, ancorada na “revelação” – levanta a suspeita de que justamente aí se vê a marca definitiva da “mendacidade” do animal humano; mas, afinal, o filósofo parece sugerir que a moral, enquanto instrumento dessa “mendacidade” e “artificialidade”, deixa entrever algo positivamente verdadeiro sobre a natureza humana. A história do pensamento ocidental produziu um texto sobre o humano, um texto que merece toda a nossa desconfiança, e mesmo assim, com referência a esse texto, Nietzsche parece tirar algumas conclusões sobre o que a natureza é. Uma vez que, segundo Nietzsche, a cultura metafísica colocou a “verdade de ponta-cabeça” (\textit{JGB/ABM}, Pr.), seu procedimento aqui consiste em colocar a cultura metafísica de ponta-cabeça. Que analogia podemos traçar entre este procedimento e o procedimento filológico?

Importante estudioso das obras filológicas de Nietzsche, James Porter oferece em \textit{Nietzsche and the Philology of the Future} (2000) uma análise eloquente da atividade de Nietzsche  e da tradição filológica em que se insere, que seria eminentemente uma escola de filologia cética. Essa escola toma como princípio metodológico a suspeita sistemática sobre a confiabilidade de todo e qualquer aspecto dos documentos: suspende-se em primeiro lugar o status de autenticidade de autoria de um testemunho, e então o status de pertinência do que é testemunhado. Aqui todos são culpados até que se prove o contrário, e as circunstâncias são altamente desfavoráveis a que se consiga provar qualquer coisa nesse sentido. Este é, de certa forma, o procedimento de “colocar de ponta-cabeça” as posições mais comumente aceitas pela tradição.

	A dúvida sistemática que rege a filologia converte-se, então, no caso de Nietzsche, numa dúvida quanto à própria possibilidade de fazer filologia (PORTER, 2000, p. 22). A crítica de Nietzsche se volta contra a tendência dos filólogos a fantasiarem a antiguidade ao mesmo tempo em que se aferram a “seu estimado positivismo histórico, sua fé ingênua na habilidade de reconstituir e captar o passado em sua condição original, sua afirmação temerária da própria capacidade de separar o espúrio do autêntico com alguma certeza” (PORTER, 2000, p. 9). Por mais rigorosos que sejam seus métodos, a filologia, para Nietzsche, deve renunciar à pretensão de atingir uma imagem pura, positiva, não enviesada da antiguidade. Nietzsche leva as dúvidas que pairam sobre a confiabilidade da filologia ao mais alto grau de tensão, de forma que Porter sugere que sua atividade consiste, na verdade, em uma “contra-filologia” (PORTER, 2000, p. 17), e que sua visão sobre a filologia, em geral, é de que não há para ela qualquer fundamento possível a não ser uma espécie de “vontade de filologia”, nos termos do intérprete (PORTER, 2000, p.9). Mas se a filologia parece definitivamente incapaz de reconstituir uma imagem verdadeira da antiguidade, por que filologia, afinal? 

Segundo Porter, não é possível compreender a atividade filológica de Nietzsche sem manter em mente seu caráter extemporâneo. A busca pelo passado empreendida pela filologia oferece uma oportunidade ímpar de compreensão da própria modernidade, e é portanto um material valioso para o tratamento de uma das questões com que Nietzsche supostamente mais se preocupou:

\begin{quotation}
Uma das preocupações centrais ao longo de sua carreira, e uma que será central para esse estudo, foi determinar as formas nas quais a filosofia e a filologia são sintomáticas de hábitos culturais modernos, ideologias, e imaginações. Nietzsche talvez tenha delineado o mais corrosivo dos modelos até agora disponíveis para exumar os espectros da sociedade ocidental, na forma de uma antropologia cultural. Uma preocupação incessante com a sintomatologia do “sujeito” moderno – seus transtornos, suas ilusões, e os signos de sua presença insuprimível – é, eu acredito, o que unifica a obra de Nietzsche mais do que qualquer questão particular. (PORTER, 2000, p. 4).
\end{quotation}

O caráter extemporâneo dessa visada lança a suspeita de que nossa reconstrução dos antigos é, por um lado, sintoma da condição moderna e, por outro, a consideração de que essa condição moderna é ela mesma um resultado da cultura antiga. A modernidade revela-se, assim, como produto de uma antiguidade, conquanto a visão que podemos ter desta permaneça fatalmente  nebulosa. Nas palavras de Porter: “A filologia não descobre o passado. Ela descobre o presente à luz da inesgotável futuridade do passado” (PORTER, 2000, p. 14). Sobre o título “filologia do futuro” (forjado pelo próprio Nietzsche para encabeçar um de seus projetos de juventude, muitos anos antes do programa de “filosofia do futuro” que temos discutido) Porter comenta: “O que quer que Nietzsche tenha pretendido dizer além disso com esse título, a filologia do futuro implicará em investigação do passado na medida em que afeta o presente” (PORTER, 2000, p.15)
\footnote{Estamos diante de uma antinomia que foi registrada por Nietzsche em um de seus cadernos de estudos: “a \textit{antiguidade} na verdade sempre foi entendida \textit{da perspectiva do presente} – e deveria o \textit{presente} agora ser entendido \textit{da perspectiva da antiguidade}?” (apud PORTER, 2000, p. 15).}. Como isto funciona?

\begin{quotation}
A meta da filologia, criticamente concebida, não é substituir uma representação mais adequada dos antigos mas trazer à tona as inadequações das representações que já temos. Os critérios de inadequação aqui não são objetivos, como seriam no caso da filologia convencional. Antes, eles são internos e sintomáticos. Inconsistências de argumento, traços de raciocínio projetivo, toques anacrônicos, motivos não expressos e possivelmente nem mesmo compreendidos – esses são os signos reveladores de uma imagem inautêntica e autoenganadora do passado. A crítica é em primeira instância cultural e psicológica, não epistemológica por natureza, e esse foco no sujeito ao invés do objeto de conhecimento é igualmente uma constante em Nietzsche mesmo além de seus dias na Basileia. (PORTER, 2000, p. 8)
\end{quotation}

Essa invectiva psicológica e culturalista é aplicada por Nietzsche em muitas camadas, na medida em que toma por objeto tanto o sujeito moderno da filologia quando os sujeitos antigos que constituem suas fontes. Porter lembra que uma das maiores contribuições de Nietzsche à filologia foi reabilitar a anedota como fonte interessante de conhecimento sobre a antiguidade. Por mais que seja absolutamente impossível verificar a factualidade de uma anedota, ela seria suficiente, ou até mesmo um objeto privilegiado, para a compreensão de como certa época via certo autor, que tipo de incômodo suas ideias ali despertavam, a que tipo de tema ou companhia ele era associado. 

	À medida em que a filologia vai esclarecendo suas próprias vicissitudes e arbitrariedades, ela pode chegar, de fato, a uma nova compreensão da antiguidade. E mesmo que tenha-se renunciado à expectativa de ver os gregos exatamente como eles se viam, ou pior, ver os gregos de um ponto de vista neutro a-histórico, pode-se ainda satisfazer a expectativa de superar alguns dos preconceitos dos filólogos e produzir uma imagem realmente nova dos antigos, e esta será igualmente portadora de uma \textit{nova} verdade sobre os modernos.
	
	Trazendo a discussão de volta para o problema do “texto \textit{homo natura}”, podemos agora avançar alguns pontos: o procedimento de Nietzsche consiste, muitas vezes, em partir da desconstrução cética da visão do humano que legada pelos “velhos metafísicos apanhadores de pássaros”, estando atento para os fatores do humano que essa visão mais se esforçou por encobrir, ou a que tipo de motivação não declarada ela pode servir. A tese fundamental do próprio Nietzsche, por outro lado, fundada no conceito de “vontade de poder” seria a peça chave, ao menos em \textit{ABM}, para a realização da tarefa de “retradução do homem de volta à natureza”; vontade de poder é o vocabulário básico do que Nietzsche anuncia como uma “nova” e “estranha” linguagem (\textit{JGB/ABM}, §4). Essa é a um tempo uma novíssima linguagem, uma reformulação nietzschiana da tradição moderna de que é herdeiro imediato, bem como de um tipo de especulação para a qual havia espaço na ciência de sua época; ela é, por outro lado, uma linguagem extemporânea, que busca recuperar algo encoberto pela velha linguagem metafísica, ou seja, algo da visão pré-socrática do homem integrado ao mundo da \textit{physis}, e do mundo concebido como conflito, segundo o modelo do agonismo grego.

A busca pelo “eterno texto \textit{homo natura}” dificilmente pode ser compreendida como a busca por um texto original
\footnote{Supor tal possibilidade seria algo como tentar estabelecer a chamada “questão homérica” apostando que se pode encontrar os manuscritos completos da \textit{Ilíada} e \textit{Odisseia} com assinatura e firma reconhecida do poeta.}. Frente a tudo o que temos dito, se há uma coisa que não se pode dizer do homem, em linguagem nietzschiana, é que ele é “eterno”. O uso da expressão aqui parece sinalizar a tentativa de escavar sob o texto atual os aspectos terríveis dessa natureza humana que, apesar de todo esforço da cultura moral metafísica, continuam atuando no homem e na cultura. 

	A interpretação proposta por Lampert a essa questão até certo ponto vai ao encontro da nossa:

\begin{quotation}
Nietzsche, manancial de esforços desconstrutivos para negar a fundamentalidade de qualquer texto, fala da natureza humana como um texto eterno. A natureza humana, produto de uma história evolutiva, é inerradicável por cosméticos, permanecendo presente como o que há de mais profundo independentemente do que é “rabiscado e pintado” sobre ela por escritores e artistas da humanidade a serviço da vontade básica do espírito. A vontade contrária opõe a vontade básica tornando-se senhora sobre essas interpretações, traçando a má interpretação moral de volta a seu texto pré-moral. (LAMPERT, 2001, p. 231).
\end{quotation}

O problema é que, na verdade, todas as indicações deixadas ao longo de \textit{Além de Bem e Mal} sugerem que a tarefa lançada por Nietzsche não consiste numa recondução à condição pré-moral, mas numa passagem para o extramoral. E isto tem a ver com um segundo ponto aparentemente inexato da interpretação de Lampert: ao que nos parece, pode-se dizer que a natureza humana permanece “inerradicável” \textit{apesar de} mas não \textit{independentemente} da ação “cosmética” das interpretações morais, uma vez que estas tiveram um papel efetivo no processo de seleção social, e deixaram suas marcas. O impulso mesmo que impele o filósofo a tomar para si essa tarefa seria um produto tanto da seleção natural quanto da seleção social; nos referimos, é claro, ao ímpeto cruel ao conhecimento.

Com isto retornamos finalmente à questão não respondida que encerra o muito comentado aforismo 230 de \textit{ABM}, sobre a vontade de conhecimento ela mesma. O aforismo que teve início com o estabelecimento do ponto de que o espírito tende fundamentalmente ao engano, à simplificação e à vontade de dominação, tendência contraposta por um exercício de cruel  honestidade, passa à edição da tarefa de recuperação do texto \textit{homo natura} e é concluído com a pergunta: “'Por que conhecimento, afinal?'. Todos nos perguntarão isso. E nós, premidos desse modo, nós, que já nos fizemos mil vezes a mesma pergunta, jamais encontraremos resposta melhor que...” (\textit{JGB/ABM}, §230). A resposta a esta questão fundamental, a questão sobre o valor da vontade de verdade, representada por Édipo e a Esfinge no primeiro aforismo do livro, parece se encontrar nas primeiras linhas do aforismo seguinte: “A aprendizagem nos transforma; faz como toda alimentação, que não apenas 'conserva' –: como bem sabe o fisiólogo.” (\textit{JGB/ABM}, §231). Ao retomar a metáfora da alimentação, Nietzsche indica que a resposta a essa questão na verdade já estava dada em \textit{ABM} §230: o espírito que se alimenta de conhecimento é capaz de transformar a vontade básica do espírito, e passa a assumir novas necessidades “fisiológicas”, um novo “gosto intelectual”. A busca pela verdade torna-se para ele uma necessidade vital, uma fatalidade. Uma alimentação que “não apenas 'conserva'” seria aquela que favorece a expansão da “vontade de poder”; finalmente, a sugestão aqui é de que o ímpeto cruel para a verdade é igualmente um ímpeto para a dominação.

	A sequência dessa consideração é um tanto surpreendente: “Mas no fundo de todos nós, 'lá em baixo', existe algo que não aprende, um granito de \textit{fatum} [destino] espiritual, de decisões e respostas predeterminadas a seletas perguntas predeterminadas.” (\textit{JGB/ABM}, §231). Segundo a interpretação de Lampert, neste ponto pouco convincente, a passagem faz referência à própria fatalidade da vontade de verdade (Cf. LAMPERT, 2001, p. 233). A nosso ver, Nietzsche aqui se refere a um ponto cego que há “em todos nós”, um ponto resistente às invectivas da crueldade do conhecimento, e que provavelmente tem a ver com a “fatalidade” da herança biológica e cultural de cada um. Com isto Nietzsche por assim dizer pede licença ao leitor para apresentar seu próprio ponto cego, e este, ainda mais surpreendentemente, diz respeito a suas considerações sobre “a mulher em si”, pois “Em todo problema cardinal fala um imutável 'sou eu'; sobre o homem e a mulher, por exemplo, um pensador não pode aprender diversamente, mas somente aprender até o fim – descobrir inteiramente o que nele está 'firmado' a esse respeito” (JGB/ABM, §231). Nietzsche então concede em expor sua visão sobre o tema “desde que já se saiba que são apenas verdades \textit{minhas}.” (\textit{JGB/ABM}, §231).

O que se segue é uma série de comentários prescritivos sobre a relação homem e mulher, que soam absolutamente agressivos à sensibilidade moderna. Estes comentários contêm, em primeiro lugar, críticas às tentativas de ilustração da mulher, mais precisamente, críticas às tentativas de ilustração \textit{sobre} a mulher. Nietzsche prescreve neste ponto uma aceitação da disposição fundamental do espírito a não saber, não saber sobre a mulher, sob o suposto risco de revelar-se o “eterno-tedioso da mulher” (\textit{JGB/ABM}, §232). Esta consideração expressa a intenção de impor um limite à investigação científica. Outras ocorrências bastante intrigantes incluem: “Nós, os homens, desejamos que a mulher não continue a se comprometer através do esclarecer: assim como foi cuidado e atenção masculina para com a mulher que a Igreja decretasse \textit{mulier taceat in eclesia}! [que a mulher se cale na igreja!].” (\textit{JGB/ABM}, §232); “Se a mulher fosse uma criatura pensante teria descoberto, cozinhando há milênios, os mais importantes fatos fisiológicos, e teria também aprendido a arte da cura!” (JGB/ABM, §234); “Equivocar-se no problema fundamental 'homem e mulher', nele negar o mais profundo antagonismo e a necessidade de uma tensão hostil, e sonhar talvez com direitos iguais, igual educação, reivindicações e deveres iguais: eis um sinal \textit{típico} de superficialidade” (\textit{JGB/ABM}, §238); “ele tem que conceber a mulher como posse, como propriedade a manter sob sete chaves, como algo destinado a servir e que só então se realiza” (JGB/ABM, §238); “Que a mulher ouse avançar quando já não se quer nem se cultiva o que há de amedrontador no homem, mais precisamente o \textit{homem} no homem, é algo de se esperar e também de compreender; o que dificilmente se compreende é que por isso mesmo a mulher – degenera.” (\textit{JGB/ABM}, §239); “Pensa-se inclusive, aqui e ali, em fazer das mulheres livres-pensadores e literatos: como se uma mulher sem religião não fosse, para um homem profundo e ateu, algo totalmente repugnante ou ridículo” (\textit{JGB/ABM}, §239).

Que o capítulo “Nossas virtudes”, em que Nietzsche anuncia sua tarefa de retraduzir o homem de volta à natureza seja encerrado com considerações tão deletérias, idiossincráticas e, como não dizer, moralistas, é um fato bastante perturbador
\footnote{Lampert tem uma interpretação bastante diferente desse ponto, talvez por ser bem mais simpático às colocações de Nietzsche a este respeito. Segundo o autor, Nietzsche não avança suas considerações sobre a mulher como uma mera idiossincrasia nem como algo alheio às considerações anteriores sobre o “texto \textit{homo natura}”. Ou seja, o autor toma literalmente as afirmações nietzschianas, e as interpreta como expressão de uma verdade sobre a natureza do homem e da mulher, verdade que serviria de fundamentação para uma espécie de “política sexual”.  Lampert chega mesmo a traçar um paralelo entre homem e mulher e as duas disposições do espírito, associando a mulher à vontade básica de engano e o homem à crueldade do conhecimento. Não acompanhamos o autor nessa interpretação porque parece-nos muito clara a modulação com que Nietzsche apresenta tais considerações: elas não são apresentadas como parte do texto natural, mas como parte daquilo que não se consegue traduzir para uma linguagem mais verdadeira, por se estar num terreno em que no máximo é possível dar um testemunho de si mesmo, dos limites sócio-histórico culturais que nos impedem de nos livrarmos de certas crenças e disposições. No entanto não duvidamos da afirmação de Lampert quanto à conotação política desse tema; é provável que Nietzsche de fato tivesse em mente algo como um plano de “política sexual”.}. 
O texto como um todo só se torna legível se não perdermos de vista as modulações com que essas considerações são apresentadas. 
	
	Toda a sequência de aforismos que encerram o capítulo é bastante peculiar. De \textit{ABM} §229, que trata do tema da crueldade do conhecimento até \textit{ABM} §239, que encerra as considerações programáticas sobre o papel da mulher na cultura europeia, passando por \textit{ABM} §230, que anuncia a tarefa de retradução do homem à natureza, vemos Nietzsche se expressar de modo muito mais pessoal que de uso. Toda essa sequência tem um tom confessional. E se é verdade que podemos afirmar, juntamente com  Porter (2000, p. 13) que o texto nietzschiano leva o leitor a uma profunda crise de identidade dada a dificuldade de estabelecer o peso que se deve conferir a suas muitas vozes, nesse trecho de \textit{Além de Bem e Mal} parece-nos que ouvimos um Nietzsche em sua própria voz, ou na voz mais pessoal possível. Mas essa confissão em voz própria tem dois aspectos: ela fala primeiramente da possibilidade de avançar efetivamente o campo do conhecimento sobre o humano, e então passa aos limites desse avanço, limites que apontam para a “grande estupidez que somos, para nosso \textit{fatum} espiritual, \textit{o que não aprende} 'lá embaixo'” (\textit{JGB/ABM}, §231). E é como uma “grande estupidez” que apresenta suas considerações sobre a mulher. Por mais que as afirmações de Nietzsche sobre a mulher sejam discutíveis, e de valor duvidoso, não se pode acusá-lo aqui de falta de honestidade. Enquanto o exercício de honestidade no conhecimento e no autoconhecimento seria uma expressão de crueldade, o reconhecimento da própria estupidez seria uma “gentileza” endereçada a si mesmo; ambos os movimentos são, na verdade, atos honestidade sobre as próprias capacidades e limites. E ainda, o filósofo parece atribuir valores igualmente relevantes tanto para nossa mais profunda crueldade no conhecer quanto a nossa “grande estupidez”, e papeis igualmente relevantes para ambos na reinvenção da cultura. Aqui vemos Nietzsche atingir seu próprio ideal do que é fazer filosofia, ao expressar sua própria “magnífica tensão do espírito”. É talvez por não renunciar a qualquer um dos lados dessa tensão que a filosofia nietzschiana se coloca além do ceticismo, além da crítica (\textit{Cf. JGB/ABM}, §208), além de bem e mal. É por isto que os filósofos do futuro, entre os quais Nietzsche certamente se inclui como precursor, “bem poderiam, ou mesmo mal poderiam, ser chamados de tentadores. Esta denominação mesma é, afinal, apenas uma tentativa e, se quiserem uma tentação.” (\textit{JGB/ABM}, §42)
\footnote{Significativamente, esse aforismo em que Nietzsche anuncia suas expectativas quanto à nova filosofia por ele inaugurada é imediatamente sucedido por uma reafirmação da vontade de verdade: “Serão novos amigos da 'verdade' esses filósofos vindouros? Muito provavelmente: pois até agora todos os filósofos amaram suas verdades. Mas com certeza não serão dogmáticos.” (\textit{JGB/ABM}, §43). A partir daí, o aforismo passa a uma consideração sobre o quanto há de \textit{pessoal} na busca pela verdade, e desvaloriza as tentativas de universalização das verdades filosóficas. As verdades do filósofo, o autoconhecimento e projeção de sua personalidade, incluem tanto aquilo que pode ser aprendido no exercício sempre algo cruel de buscar o conhecimento quanto o que não pode ser aprendido, o o ponto cego, o “grão de \textit{fatum}” que o filósofo nunca pode esclarecer completamente, mas que é levado em conta como uma peça das mais importantes no exercício de autoconhecimento. E como os limites entre aprendizagem e “estupidez” não se dão a conhecer de antemão, este será sempre um exercício de experimentação.}. 

Para apreciar algo da riqueza do programa filosófico lançado em \textit{Além de Bem e Mal} deve-se estar atento ao fato de que cada uma de suas grandes teses são investidas de modulações bastante diferentes. Sua principal tese talvez seja a “vontade de poder”, que é apresentada como um experimento de pensamento bastante rigoroso, com possíveis aplicações no campo científico, e talvez até mesmo como \textit{insight} fundamental de uma nova psicologia; \textit{at last but not least}, esta é uma tese que destina ao estudo das relações hierárquicas de dominação estabelecidas entre diferentes formas de vida, diferentes valorações, ou diferentes campos da atividade humana, como filosofia e ciência. Já o tema do “eterno retorno” se refere a uma espécie de vivência mística marcada por uma disposição erótica para com a vida, experiência afirmativa desejável numa época que a vida espiritual deixa de ser conduzida pelas religiões. A “retradução do homem à natureza” ou recuperação do “terrível texto básico \textit{homo natura}” é uma tarefa que o filósofo se coloca num momento de total honestidade quanto às possibilidades do conhecer. Cada um desses temas é apresentado com diferentes tonalidades, diferentes investidas retóricas e sinalizadores que alertam o leitor para seu peso no interior do programa filosófico em questão. Tais sinalizadores, contudo, sempre colocam um desafio de decifração, que força o leitor a engajar-se pessoalmente no texto e encontrar sua veia de entrada. \textit{Além de Bem e Mal} é um testemunho em grande estilo de que, numa época de conquistas científicas, no limiar da passagem para uma cultura pós-metafísica e extramoral, o filosofar é decididamente possível.

\chapter*{Considerações finais}
\addcontentsline{toc}{chapter}{Considerações finais} 

Ao longo de todo seu trajeto intelectual, Nietzsche nunca negligenciou a importância da ciência para a atividade filosófica – isto é, a importância não só das ciências modernas mas de todo o pensamento científico, o “espírito científico” que teria encontrado, por exemplo, nos fisiólogos gregos pré-socráticos.

	O filósofo alemão certamente foi um mestre da capacidade de reinventar a própria filosofia, e ao longo de sua profícua atividade filosófica experimentou com as mais diversas formas de pensamento e expressão, deixando para a posteridade uma obra riquíssima e variegada no que diz respeito a estilo e gêneros textuais, estratégias retóricas, estratégias de abordagem e reinterpretação da história do pensamento, alianças filosóficas, incorporação de novos resultados no campo das mais diversas ciências (história, biologia, física, teologia, psicofisiologia, etc.). Essa filosofia mutante, amante do múltiplo e da mudança, parece abrigar, contudo, algumas constantes, quais sejam: uma visão mobilista de mundo; o compromisso com uma tarefa de crítica cultural e construção da passagem para uma era extramoral; a visão da filosofia como atividade \textit{pessoal} de investigação, atribuição de sentido e valor, experimentação de modos de vida; a sensibilidade para detecção das origens morais do pensamento científico, de sua aplicação corretiva a certa tendência natural ao erro, e de suas contribuições para um modo extramoral de visar o mundo. Essas constantes são visíveis nos múltiplos arranjos que Nietzsche confere aos diferentes elementos de sua filosofia conforme se lança a diferentes experimentos de enfrentamento de suas questões cardeais, reinterpretação de sua herança filosófica e cultural, intervenção em diferentes âmbitos da vida cultural.
	
	Dessa forma, quer nos parecer que a ênfase no rastro moral do pensamento científico, a crítica ao “otimismo socrático” que se vê em algumas obras de juventude, revela uma atitude prudencial de Nietzsche quanto à dosagem desse ingrediente cognitivo no balanço geral da cultura. A partir de \textit{Humano, demasiado humano}, no entanto, a ciência é definitivamente assumida como uma aliada das mais importantes na tarefa de superação da moral. Isto porque a ciência traz à tona os aspectos do humano que a moral se esforçou por encobrir; porque ela contraria alguns pressupostos básicos da moral, mostrando, por exemplo, como a noção de livre-arbítrio se funda sobre uma ignorância sobre os diversos fatores que condicionam a ação; porque ela viabiliza uma nova narrativa sobre o humano, em que este é pensado em continuidade com o restante do mundo natural e porque ela é capaz de fazer frente à disposição do espírito a colocar-se numa relação não qualificada com as crenças morais, isto é, porque a ciência educa o espírito a não pensar em termos de “convicções”. Já em \textit{Humano} encontra-se o germe da narrativa que Nietzsche desenvolverá posteriormente em \textit{Genealogia da Moral}, segundo a qual a disciplina no pensamento – cujo cultivo se associa primeiramente às exigências morais no pensar e posteriormente assumido pelo \textit{métier} científico – engendra uma disposição de espírito virtuosa que afinal volta-se criticamente contra os pressupostos da moral.

Pode-se dizer com segurança que Nietzsche nunca identificou a figura do filósofo com aquela do moderno homem de ciência, em termos absolutos. Sua própria obra evidentemente é demasiado vasta e sofisticada para que seja lida apenas como um comentário à ciência ou como nada mais que um programa de abordagem científica dos problemas tradicionais da filosofia.

	O desenvolvimento da obra nietzschiana leva a caminhos que estão muito além do que se poderia prever mediante o programa estrito de filosofia histórica lançado em \textit{Humano}. O contato com novas fontes advindas do campo filosófico e científico, a reformulação de questões e prioridades, o aperfeiçoamento do estilo, a busca sempre mais desafiadora por sinais de vigor intelectual, afetivo e cultural: tudo isso, a nosso ver, conta para o alargamento do programa iniciado em Humano e enriquecimento geral da obra. \textit{Humano} e \textit{Além de Bem e Mal} expressam um juízo idêntico quanto à relação entre filosofia e ciência, em que esta é vista como um dos mais valiosos instrumentos à atividade filosófica, mas não como um fim em si mesma, e seguramente não como um fim que coincide com a finalidade da filosofia, destinada à tarefa de criação de valores e modos de vida. Ademais, uma vez que a filosofia se faz necessária justamente porque não há uma resposta pré-definida sobre o valor dos valores ou o valor dos modos de vida, a cada experimento nesse campo o instrumento científico encontrará diferentes aplicações.
	
	Conquanto Nietzsche pareça ater-se a uma mesma visão do que é a filosofia e do que é a ciência, as mudanças notáveis entre um programa como o de \textit{Humano} e o de \textit{Além de Bem e Mal} (e mesmo depois, com \textit{Ecce Homo}, por exemplo) podem ser efeitos da experimentação com os diversos modo de vida filosóficos possíveis. A cada tentativa de estabelecer um diferente modo de vida, e em razão deste, o filósofo rearranja uma dietética cultural, reorganiza a aplicação dos diferentes recursos culturais: arte, religião e ciência.

Em \textit{Humano}, a ciência é instrumento de cultivo de uma disposição de espírito salutar à atividade filosófica e também provedora de conhecimentos que o filósofo aplica no tratamento tanto das questões mais gerais referentes ao estudo do humano e das culturas, quanto das questões mais pessoais referentes ao cultivo de si. Em \textit{Além de Bem e Mal} a ciência é instrumento de um programa extremamente ambicioso, que envolve especulação científica; criação de valores cuja aplicação não se restringe ao círculo dos modos de vida filosóficos, mas que pretendem alcançar um impacto mais amplo sobre os diferentes círculos da cultura, e até mesmo, como sugerido em alguns aforismos, a ciência ali poderia ser um instrumento da produção planificada dos mais vigorosos modos de vida, realizados por “espécimens” de exceção, a ser orientada pelo filósofo. A este respeito, Lopes comenta que:

\begin{quotation}
Esta concepção absurdamente ambiciosa da ocupação do filósofo é apenas o ponto de culminância e, em certo sentido, a conclusão natural do primado que Nietzsche confere à vida contemplativa sobre a vida ativa, e à vida filosófica sobre a vida religiosa, artística e científica no mundo da cultura. Enquanto tese não se trata, a bem da verdade, de uma novidade que aparece apenas com o \textit{Zaratustra}. Nietzsche reivindica este primado desde \textit{Humano, demasiado Humano}. Mas nas obras do período intermediário Nietzsche não faz desta tese um programa propriamente político. Ele se contenta em confrontar estilos de vida e em atribuir supremacia ao estilo de vida contemplativo identificado com a filosofia. (LOPES, 2008, p. 424).
\end{quotation}

Se, afinal, \textit{Além de Bem e Mal} soa tão estranho ao ambiente de \textit{Humano}, isto certamente não se deve a qualquer rejeição da abordagem naturalista, mas à reformulação do que seja a “ocupação do filósofo”, sendo que a forma como esta é concebida em \textit{ABM} inviabiliza o ideal de buscar uma “vida muito mais simples e mais pura de paixões” (\textit{MA/HH}, §34) dedicada exclusivamente à pesquisa, à contemplação, ao autoconhecimento. Ademais, uma série de fatores contribui para que o programa de reforma dos afetos pautada pela moderação seja deixado pra trás. Esses fatores têm a ver com um certo heroísmo da paixão do conhecimento, que torna-se um tema da filosofia “intermediária” de Nietzsche a partir de \textit{Aurora}; têm a ver também com o desenvolvimento de uma sensibilidade cada vez mais aguçada do filósofo para a observação das relações de poder, que levam-no afinal a uma psicologia muito mais sofisticada que aquela oferecida pelo viés utilitarista, centrado nas noções de utilidade e prazer; por fim, o ideal de moderação cai por terra frente à percepção de que a visada histórica, receptiva à compreensão e valorização dos mais diversos arranjos entre impulsos trazidos à tona pelo estudo das culturas, leva o filósofo a renunciar a um ideal de “justa medida”.

Conquanto o programa de \textit{Humano} expresse uma associação mais estrita entre ciência e filosofia, \textit{Além de Bem e Mal} traz contribuições muito interessantes para o debate naturalista. Ao que nos parece, em \textit{ABM} Nietzsche se mostra de fato como um leitor bastante arrojado das ciências modernas; tendo há muito incorporado um certo \textit{pathos} científico, Nietzsche nessa obra não só opera com diversos resultados científicos como desenvolve algumas críticas a certas áreas da ciência e também se insere incisivamente no debate científico de sua época, avançando algumas teses de cunhagem própria. Todos esses artifícios naturalistas, no entanto, estão a serviço da crítica profundamente anti-moderna que leva a cabo em \textit{ABM}, e que é motivada, ao que tudo indica, pelo receio quanto à possibilidade de que a ilustração democrática tenha um efeito aplainador sobre as subjetividades, o que colocaria em risco a própria possibilidade de fazer filosofia. Esse tampouco parece ser um receio que vem à tona exclusivamente com \textit{ABM}, ele já pode ser identificado nas obras de juventude de Nietzsche. Se, por um lado, \textit{Humano} busca uma espécie de conciliação, um pacto de boa convivência entre modernidade e liberdade de espírito, por outro lado, o que há de mais característico em \textit{ABM} é que ali o filósofo mobiliza todos os seus esforços no sentido de intervir positivamente na cultura, engendrando uma tensão contrária que possa fazer frente aos efeitos anti-filosóficos a que o pensamento moderno está exposto.

	Nietzsche, o extemporâneo, realiza em sua obra todas as tensões envolvidas no desafio de se fazer filosofia na modernidade, o desafio de perseguir novas metas culturais através de uma apropriação crítica de toda a herança moderna, que inclui o pensamento científico como um dos artigos mais importantes e mais decisivos para definição dos rumos que então se abriam à cultura ocidental. A nosso ver, \textit{ABM} deve ser tomado como um testemunho valiosíssimo de como a filosofia é possível nessas condições, desde que seja tomado como um dos experimentos nesse campo mas não como a palavra final de Nietzsche sobre o modo de vida filosófico que se impõe na modernidade, se é que essa palavra é dada em alguma obra. E é justamente porque essa obra tão rica encerra diferentes experimentos de vida filosófica, porque nela se enfrentou de modo abissal mas sempre tão criativo a questão da extemporaneidade da filosofia, que ela continuará a ser revisitada por quem aceita o desafio de fazer filosofia num mundo pós-moderno.

\chapter*{Referências Bibliográficas}
\addcontentsline{toc}{chapter}{Referências Bibliográficas} 

\setlength{\parindent}{0cm}

ACAMPORA, C. D., Naturalism and Nietzsche's Moral Psychology. In PEARSON, K. A, (ed.) \textit{A Companion to Nietzsche}. Blackwell Publishing, 2006, p. 314-344.

ALMEIDA, R. M., \textit{Nietzsche e o Paradoxo}. São Paulo: Edições Loyola, 2005. 

BABICH, B., \textit{Nietzsche's philosophy of Science: Reflecting science on the grounds of art and life}. University of New York Press, 1994.

BORNEDAL, P., \textit{The Surface and the Abyss: Nietzsche as Philosopher of Mind and Knowledge.} Berlin/New York: De Gruyter, 2010.

BRUSOTTI, M., Tensão – um conceito para o grande e para o pequeno. In: Dissertatio, 33. Pelotas, UFPEL, 2011, p. 35-62.

CARVALHO, D., O silêncio da natureza e o barulho da moralidade: Nietzsche e o problema da antropomorfização. Revista Trágica: estudos sobre Nietzsche – 1º semestre de 2013 – Vol. 6 – nº 1, (p. 39-56).

CLARK, M. \textit{Nietzsche on Truth and Philosophy}. Cambridge: Cambridge University Press, 1990.

CLARK, M., DUDRICK, D., \textit{The soul of Nietzsche's Beyond Good and Evil}. Cambridge University Press, 2012.

COMTE, A., \textit{Opúsculos de filosofia social}. Tradução de Ivan Lins e João Francisco de Souza. Porto Alegre: Editora Globo, 1972. 

COX, C., \textit{Nietzsche – Naturalism and Interpretation}. University of California Press, 1999.

DANTO, A. \textit{Nietzsche as Philosopher}. New York: Columbia University Press, 1965.

DELEUZE, G., GUATTARI, F., \textit{O que é filosofia?}. Tradução de Bento Prado Jr. e Alberto Alonso Muñoz. Rio de Janeiro: Editora 34, 1992.

FREZATTI JR., W. A., \textit{Nietzsche contra Darwin}. Coleção Sendas \& Veredas; GEN/ Discurso Editorial/Editora UNIJUÍ, 2001.

FREZZATTI JR, W. A., \textit{A fisiologia de Nietzsche: a superação da dualidade cultura/biologia. 1}. ed. IJUÍ: Unijuí, 2006. 

FOUCAULT, M., \textit{As Palavras e as Coisas}. Tradução de Salma Tannus Muchail. São Paulo: Martins Fontes, 2007.

GIACOIA JR., O., \textit{Labirintos da Alma: Nietzsche e a autossupressão da moral}. Campinas: Editora da Universidade Estadual de Campinas UNICAMP, 1997.

GIACOIA JR., O., A autossupressão como catástrofe da consciência moral. \textit{Estudos Nietzsche} Curitiba, v. 1, n. 1, p. 73-128, jan./jun. 2010.

GREEN, M. S., \textit{Nietzsche and the Transcendental Tradition}. University of Illinois, 2002. 

HAAR, M., Vida e totalidade natural. Cadernos Nietzsche 5, p. 13-37, 1998.

HAN-PILE, B., Aspectos transcendentais, compromissos ontológicos e elementos naturalistas no pensamento de Nietzsche. Cadernos Nietzsche 29, vol. 1, 2011, p. 163-220. 

HEIT, H., Advancing the Ag\={o}n. Clark’s and Dudrick’s Esoteric Reading of \textit{Beyond Good and Evil}. (A ser publicado). 2012.

HUSSAIN, N., Nietzsche's Positivism. European Journal of Philosophy 12:3, p. 326–368. Blackwell Publishing Ltd., 2004.

HUSSAIN, N., Error theory and Fictionalism. In The Routledge Companion to Ethics. Edited by John Skorupski, p. 335-345. London and New York: Routledge, 2010.

JANAWAY, C. 2007. \textit{Beyond Selflessness: Reading Nietzsche’s Genealogy}. Oxford: Oxford University Press.

KAIL, P. J. E., Nietzsche and Hume – Naturalism and Explanation. In ACAMPORA, C. (ed.) \textit{The Journal of Nietzsche Studies} issue 37, 2009.

KATSAFANAS, P., Review on Maudemarie Clark and David Dudrick, \textit{The Soul of Nietzsche’s} Beyond Good and Evil. Publicação online, 2013.
Cambridge: Cambridge Univertsity Press, 2012.

LANGE, F. A, \textit{The History of Materialism and Criticism of its Present Importance}.  New York: Hacourt, Brace \& Co., 1925.

LAMPERT, L., \textit{Nietzsche's Task}: An Interpretation of \textit{Beyond Good and Evil}. Yale University Press, 2001.

LEITER, B. Morality in the pejorative sense: On the logic of Nietzsche's critique of morality. \textit{British Journal for the History of Philosophy }3 (1). p.113 – 145, 1995.

LEITER, Brian. \textit{Guidebook of Nietzsche on morality}. London: Routledge, 2002. 

LEITER, B. Nietzsche's Naturalism Reconsidered. In GEMES, K., RICHARDSON, J., \textit{The Oxford Handbook of Nietzsche}. Oxford University Press, 2013. p. 576-599.

LOPES, R. A, \textit{Ceticismo e vida contemplativa em Nietzsche}. Tese apresentada como requisito de obtenção de título de doutorado pela FAFICH – UFMG, 2008.

LOPES, R., “A almejada assimilação do materialismo”: Nietzsche e o debate naturalista na filosofia alemã da segunda metade do século XIX. Cadernos Nietzsche 29, vol. 2, 2011, pp. 309-352.

LOPES, R., Há espaço para uma concepção não moral da normatividade prática em Nietzsche? Notas sobre um debate em andamento. Cadernos Nietzsche 33, 2013.

LOPES, R., A catástrofe do humanismo: agonismo e perfeccionismo em Nietzsche. A ser publicado. 2013.

MATTIOLI, W., “Do idealismo transcendental ao naturalismo: um salto ontológico no tempo a partir de uma fenomenologia da representação”. Cadernos Nietzsche, 29: São Paulo, 2011. pp. 221-270.

MELO, E. R., \textit{Nietzsche e a Justiça}. São Paulo: Perspectiva, FAPESP, 2004.

MOORE, G., BROBJER, T. H., (ed.) \textit{Nietzsche and Science}. Ashgate, 2005.

NIETZSCHE, F., \textit{Kritische Studienausgabe}. Ed. Giorgio Colli and Mazzino Montinari. Berlin/New York, Walter de Gruyter, 1967-88.

NIETZSCHE, F., \textit{O Nascimento da Tragédia – ou helenismo e pessimismo}. Tradução, notas e posfácil de J. Guinsburg. São Paulo: Companhia das Letras, 2007.

NIETZSCHE, F., \textit{A filosofia na época trágica dos gregos}. Organização e tradução de Fernando R. de Moraes Barros. São Paulo: Hedra, 2008. 

NIETZSCHE, F., \textit{Schopenhauer as Educator}. In \textit{Untimely Meditations}. Traduzido por J.R. Hollingdale. Editado por Daniel Breazeale. New York: Cambridge University Press, 1997.

NIETZSCHE, F., \textit{Segunda Consideração Intempestiva: Da utilidade e desvantagem da história para a vida}. Tradução Marco Antônio Casanova. Rio de Janeiro: Relume Dumará, 2003.

NIETZSCHE, F. \textit{Humano, demasiado humano – um livro para espíritos livres}. Tradução, notas e posfácio de Paulo César de Souza. São Paulo: Companhia das Letras, 2005.

NIETZSCHE, F., \textit{Humano, demasiado humano II}. Tradução, notas e posfácio de Paulo César de Souza. São Paulo: Companhia das Letras, 2008.

NIETZSCHE, F., \textit{Aurora}. Tradução, notas e posfácio de Paulo César de Souza. São Paulo: Companhia das Letras, 2004.

NIETZSCHE, F., \textit{A Gaia Ciência}. Tradução, notas e posfácio de Paulo César de Souza. São Paulo: Companhia das Letras, 2001.

NIETZSCHE, F.,  \textit{Assim falou Zaratustra}. Tradução Mário da Silva. Rio de Janeiro: Civilização Brasileira, 2005.

NIETZSCHE, F., \textit{Além do Bem e do Mal – prelúdio a uma filosofia do futuro}. Tradução, notas e posfácio de Paulo César de Souza. São Paulo: Companhia das Letras, 2005. 

NIETZSCHE, F., \textit{Genealogia da Moral – Uma polêmica}. Tradução, notas e posfácio de Paulo César de Souza. São Paulo: Companhia das Letras, 1998.

NIETZSCHE, F., \textit{O Anticristo}. Tradução, notas e posfácio de Paulo César de Souza. São Paulo: Companhia das Letras, 2007.

NIETZSCHE, F., \textit{Crepúsculo dos Ídolos}. Tradução, notas e posfácio de Paulo César de Souza. São Paulo: Companhia das Letras, 2006.

NIETZSCHE, F., \textit{Ecce Homo – Como Alguém se Torna o que é}. Tradução, notas e posfácio de Paulo César de Souza. São Paulo: Companhia das Letras, 1995.

NIETZSCHE, F., \textit{A Vontade de Poder}. Tradução e notas Marcos Sinésio Pereira Fernandes, Francisco José Dias de Moraes; apresentação Gilvan Fogel. Rio de Janeiro: Contraponto, 2008.

OKI, M. C. M., Controvérsias sobre o atomismo no século XIX. Revista Química Nova Vol. 32, No. 4, 1072-1082, 2009.

PAOLIELLO, G. D., \textit{Nietzsche Como Pensador da História – Crítica e Apologia do Pensamento Histórico na Filosofia de Nietzsche}. Dissertação apresentada ao Programa de Pós-Graduação do Departamento de História da UFMG, como requisito parcial para obtenção do título de Mestre em História. UFMG: Belo Horizonte, 2009.

PIMENTA, O,  Nietzsche naturalista?. In: Assim falou Nietzsche, 1999, Ouro Preto, 1999. 

PORTER, J. I., \textit{Nietzsche's Philology of the Future}. Stanford: Stanford University Press, 2000.

RIBEIRO, N., \textit{Fernando Pessoa e Nietzsche – O pensamento da pluralidade}. Lisboa: Verbo, 2011.

RICHARDSON, J., \textit{Nietzsche's System}. New York: Oxford University Press, 1996.

RICHARDSON, J., \textit{Nietzsche's new Darwinism}. New York: Oxford University Press, 2004.

RICHARDSON, J., Nietzsche on Time and Becoming. In PEARSON, K. A, (ed.) \textit{A Companion to Nietzsche}. Blackwell Publishing, 2006, p. 208-229. 

SANTOS, O. A. R., \textit{Natureza e dinâmica de valores na filosofia do espírito livre de Nietzsche}. Dissertação de mestrado apresentada ao Programa de Pós-Graduação em Filosofia da FAFICH/UFMG, 2011.

SCHACHT, R. \textit{Nietzsche}. Routledge, 1983.

SIMMEL, G., Schopenhauer e Nietzsche. Tradução César Benjamin. Rio de Janeiro: Contraponto, 2011.

STACK, G., \textit{Nietzsche's Anthropic Circle: Man, Science and Myth}. University of Rochester Press, 2005.

VAN TONGEREN, P., \textit{A moral da crítica de Nietzsche à moral: estudo sobre} Para Além de Bem e Mal. Tradução Jorge Luiz Visenteiner; apresentação de Oswaldo Giacoia Jr. Curitiba: Champagnat, 2012.

% ----------------------------------------------------------
% ELEMENTOS PÓS-TEXTUAIS
% ----------------------------------------------------------
\postextual

% ----------------------------------------------------------
% Referências bibliográficas
% ----------------------------------------------------------
\bibliography{abntex2-modelo-references}

\end{document}
